% Simone Iovine's Differential Geometry Notes

\documentclass[12pt]{report}

\usepackage[utf8]{inputenc}
\usepackage[english]{babel}
\usepackage{datetime}
\usepackage{mathtools,amsthm,amssymb}
\usepackage{mathtools}
\usepackage{cases}
\usepackage{centernot}
\usepackage[makeroom]{cancel}
\usepackage{graphics,graphicx}
\graphicspath{ {C:/Users/simon/Desktop/Simo/"geometria differenziale"/images/} }
\usepackage[a4paper,width=170mm,top=25mm,bottom=25mm]{geometry}
\usepackage{fancyhdr}
\usepackage{setspace}
\usepackage{float}
\usepackage[svgnames]{xcolor}
\usepackage{tikz,pgfplots,tikz-3dplot}
%\usepackage{blindtext}
%\usepackage{pgfplots}
\usetikzlibrary{3d,calc,decorations.pathmorphing,patterns}
\usepackage{xfrac}
\usepackage{multirow,multicol}
\usepackage{physics}
\usepackage{xcolor}
\usepackage{tcolorbox}
\usepackage{enumitem}
\usepackage{bbm}
\usepackage[toc]{appendix}
\usepackage{parskip}
\usepackage{tikz-cd}
\usepackage{ifthen}

% symbols definition

\newcommand{\id}{\operatorname{id}}
\newcommand{\bigone}{\mathbbm{1}}
\newcommand{\der}{\operatorname{Der}}
\newcommand{\diag}{\operatorname{diag}}
\newcommand{\supp}{\operatorname{supp}}
\newcommand{\ob}{\operatorname{Ob}}
\newcommand{\mor}{\operatorname{Mor}}
\newcommand{\st}{\, \middle| \,}
\newcommand{\hatapp}{\; \hat{} \;}
\DeclarePairedDelimiter{\ceil}{\lceil}{\rceil}

%\newcommand{\ie}{i.e. \phantom{$ \!\! $}}

\newcommand{\N}{\mathbb{N}}
\newcommand{\Q}{\mathbb{Q}}
\newcommand{\Z}{\mathbb{Z}}
\newcommand{\R}{\mathbb{R}}
\newcommand{\C}{\mathbb{C}}
\newcommand{\K}{\mathbb{K}}
\newcommand{\T}{\mathbb{T}}
\renewcommand{\S}{\mathbb{S}}

\newcommand{\rp}[1]{\R\mathcal{P}^{#1}}
\newcommand{\B}{\mathcal{B}}

\newcommand{\PC}{\mathcal{PC}}
\newcommand{\PR}{\mathcal{PR}}
\newcommand{\VC}{\mathcal{VC}}
\newcommand{\VR}{\mathcal{VR}}

\newcommand{\g}{\mathfrak{g}}
\newcommand{\h}{\mathfrak{h}}
\newcommand{\so}{\mathfrak{so}}
\newcommand{\su}{\mathfrak{su}}

\newcommand{\notimplies}{\centernot\implies}
\newcommand{\E}{\exists \;}
\newcommand{\ps}{\mathcal{P}}

\newcommand{\hal}{\hspace{13px}}

\DeclareDocumentCommand\dpdv{}{\displaystyle\partialderivative}
\DeclareDocumentCommand\ddv{}{\displaystyle\derivative}

% map definition

\newcommand{\map}[5]{
	\begin{align}
		\begin{split}
			#1 : #2 &\to #3 \\
			#4 &\mapsto #5
		\end{split}
	\end{align}
}

% map w/o equation number

\newcommand{\maps}[5]{
	\begin{align*}
		\begin{split}
			#1 : #2 &\to #3 \\
			#4 &\mapsto #5
		\end{split}
	\end{align*}
}

% image definition

\newcommand{\img}[2]{
	\begin{figure}[H]
		\centering
		\includegraphics[width=#1\textwidth,keepaspectratio]{#2}
	\end{figure}
}

% diagram definition

% mind this link for why & -> \&:
% https://tex.stackexchange.com/questions/15093/single-ampersand-used-with-wrong-catcode-error-using-tikz-matrix-in-beamer

\newcommand{\diagr}[1]{
	\begin{figure}[H]
		\centering
		\begin{tikzcd}[ampersand replacement=\&]
			#1
		\end{tikzcd}
	\end{figure}
}

% side-by-side (two neighbouring boxes)

% \sbs{[width left box]}{[content left box]}{[width right box]}{[content right box]}

\newcommand{\sbs}[4]{
	\noindent\begin{minipage}[c]{#1\textwidth}
		#2
	\end{minipage}
	\begin{minipage}[c]{#3\textwidth}
		#4
	\end{minipage}
}

%

% exercise macro

\newcounter{solutions}
\setcounter{solutions}{42} % change this number from 42 to whatever to hide solutions

% \exer{[title]}{[label]}{[exercise text]}{[solution]}

\newcommand{\exer}[4]{
	\section{#1}\label{#2}
	
	\begin{tcolorbox}
		#3
	\end{tcolorbox}
	
	\ifthenelse{\thesolutions = 42}
	{#4}
	{}
	
	%
	
	\newpage
}

%

%\renewcommand{\contentsname}{Indice}

%\makeatletter
%\renewcommand{\@chapapp}{Capitolo}
%\makeatother

% environments names

\newtheorem{theorem}{Theorem}
\newtheorem{corollary}{Corollary}[theorem]
\newtheorem{lemma}[theorem]{Lemma}
\newtheorem*{remark}{Remark}
\newtheorem{definition}{Proposition}[theorem]

%\renewcommand*{\proofname}{Dimostrazione}
\renewcommand\qedsymbol{$\square$}

% \highlight[<colour>]{<stuff>}

\newcommand{\highlight}[2][yellow]{\mathchoice%
	{\colorbox{#1}{$\displaystyle#2$}}%
	{\colorbox{#1}{$\textstyle#2$}}%
	{\colorbox{#1}{$\scriptstyle#2$}}%
	{\colorbox{#1}{$\scriptscriptstyle#2$}}}%

\definecolor{hlc}{RGB}{204,241,202} % hlc = highlight colour

\newcommand{\hl}[1]{%
	\highlight[hlc]{#1}}

% blue links for footsnotes and stuff

\usepackage[
	bookmarksnumbered = true,
	linktocpage = true
]{hyperref}

\hypersetup{
	colorlinks = true,
	linkcolor = blue,
	anchorcolor = blue,
	citecolor = blue,
	filecolor = blue,
	urlcolor = blue
}

% page layout

\pagestyle{fancy}
\renewcommand{\chaptermark}[1]{ \markboth{#1}{} }

\fancyhead{}
\fancyhead[R]{Differential Geometry Notes}
\fancyhead[L]{\leftmark}
\fancyfoot{}
\fancyfoot[L]{Simone Iovine}
\fancyfoot[R]{\thepage}
\renewcommand{\headrulewidth}{0.4pt}
\renewcommand{\footrulewidth}{0.4pt}
\renewcommand{\footnoterule}{\vfill\kern -3pt \hrule width 0.4\columnwidth \kern 2.6pt}

% bibliography

\usepackage[
	backend = bibtex,
	style = nature,
	natbib = true,
	sorting = none,
	autocite = inline
]{biblatex}
% bibliography is sorted in the same order in which the articles are cited

\addbibresource{chapters/biblio} % imports bibliography file

% table of contents

\setcounter{tocdepth}{3}
%\setcounter{secnumdepth}{3}

% title page stuff

\title{\textbf{Differential Geometry Notes}}
\author{Simone Iovine}
\date{\today}

\begin{document}

\maketitle

%

\pagenumbering{roman}

%

\tableofcontents

%

\chapter*{Notes}
\addcontentsline{toc}{chapter}{Notes}
I seguenti appunti sono frutto della revisione di appunti presi durante le lezioni virtuali tenute dal professore Andrea Loi nell'A.A. 2020-2021 dal Dipartimento di Matematica nell'Università degli Studi di Cagliari. \\
Alcune definizioni sono prese dal libro \textit{Introduzione alla Topologia Generale} di Andrea Loi \autocite{loi}.\\
I testi adottati nel corso sono \textit{Introduction to Smooth Manifolds} di John M. Lee \cite{lee} e \textit{An Introduction to Manifolds} di Loring W. Tu \cite{tu}.

Sito docente: \url{https://loi.unica.it/geomdiff2021.html}


%

\chapter*{Notation}
\addcontentsline{toc}{chapter}{Notation}
\begin{table}[H]
	\sbs{0.5}{%
				\begin{tabular}{|c|p{0.65\linewidth}|}
					\hline
					\textbf{Simbolo} & \textbf{Significato} \\
					\hline
					\hline
					$ = $ & uguaglianza \\
					\hline
					$ \equiv $ & coincide \\
					\hline
					$ \{\dots\} $ & elementi di insieme \\
					\hline
					$ \E $ & esiste \\
					\hline
					$ \E ! $ & esiste ed è unico \\
					\hline
					$ \forall $ & qualsiasi \\
					\hline
					$ \in $ & appartenente \\
					\hline
					$ \implies $ & implica (sufficiente) \\
					\hline
					$ \impliedby $ & è implicato da (necessario) \\
					\hline
					$ \iff $ & se e solo se \\
					\hline
					$ \subset $ & contenuto \\
					\hline
					$ \subseteq $ & contenuto o uguale \\
					\hline
					$ \supset $ & contiene \\
					\hline
					$ \supseteq $ & contiene o uguale \\
					\hline
					$ \setminus $ & differenza (insiemi) \\
					\hline
					$ \cap $ & intersezione \\
					\hline
					$ \cup $ & unione \\
					\hline
					$ \emptyset $ & insieme vuoto \\
					\hline
					$ \sqcup $ & unione disgiunta \\
					\hline
					$ \ps(S) $ & insieme delle parti di $ S $ \\
					\hline
					$ \times $ & prodotto diretto \\
					\hline
					$ \oplus $ & somma diretta \\
					\hline
					$ \to $ & mappa \\
					\hline
					$ \mapsto $ & associa \\
					\hline
					$ \circ $ & composizione \\
					\hline
					$ \eval{f}_{U} $ & $ f $ valutata in $ U $ \\
					\hline
					$ \id $ & identità \\
					\hline
					$ \therefore $ & quindi \\
					\hline
					$ \because $ & poiché \\
					\hline
					$ \land $ & "e" logico \\
					\hline
					$ \lor $ & "o" logico \\
					\hline
					$ \infty $ & infinito \\
					\hline
					$ \mid $ & tale che \\
					\hline
					$ \sim $ & equivalenza \\
					\hline
					$ \sfrac{S}{\sim} $ & quoziente \\
					\hline
				\end{tabular}
				}
		{0.5}{%
				\begin{tabular}{|c|p{0.65\linewidth}|}
					\hline
					\textbf{Simbolo} & \textbf{Significato} \\
					\hline
					\hline
					$ \stackrel{iso}{\simeq} $ & isomorfismo \\
					\hline
					$ \stackrel{omeo}{\simeq} $ & omeomorfismo \\
					\hline
					$ \stackrel{diff}{\simeq} $ & diffeomorfismo \\
					\hline
					$ \stackrel{omo}{\simeq} $ & omomorfismo \\
					\hline
					$ \N $ & numeri naturali \\
					\hline
					$ \Z $ & numeri interi \\
					\hline
					$ \Q $ & numeri razionali \\
					\hline
					$ \R $ & numeri reali \\
					\hline
					$ \C $ & numeri complessi \\
					\hline
					$ \K $ & $ \R $ oppure $ \C $ \\
					\hline
					$ \T^{n} $ & toro $ n $-dimensionale \\
					\hline
					$ \S^{n} $ & sfera $ n $-dimensionale \\
					\hline
					$ \rp{n} $ & proiettivo reale $ n $-dimensionale \\
					\hline
					$ \B $ & base \\
					\hline
					$ \ev{v} $ & genera \\
					\hline
					$ \PC $ &  punti critici \\
					\hline
					$ \PR $ &  punti regolari \\
					\hline
					$ \VC $ & valori critici \\
					\hline
					$ \VR $ & valori regolari \\
					\hline
					$ \g $ & algebra di Lie (associata a $ G $) \\
					\hline
					$ \sum_{i=1}^{n} $ & sommatoria da $ 1 $ a $ n $ \\
					\hline
					$ \prod_{i=1}^{n} $ & produttoria da $ 1 $ a $ n $ \\
					\hline
					$ \norm{v} $ & modulo/norma di $ v $ \\
					\hline
					$ \det $ & determinante \\
					\hline
					$ \tr $ & traccia \\
					\hline
					$ \sbmqty{ a & b \\ c & d } $ & matrice \\
					\hline
					$ \smdet{ a & b \\ c & d } $ & determinante di matrice \\
					\hline
					$ \bigone $ & matrice unitaria/identità \\
					\hline
					$ \supp $ & supporto \\
					\hline
					$ \ob $ & oggetti (categoria) \\
					\hline
					$ \mor $ & morfismi (categoria) \\
					\hline
					$ \lceil v \rceil $ & funzione "soffitto" \\
					\hline
					i.e. & cioè (\textit{id est}) \\
					\hline
					e.g. & ad esempio (\textit{exempli gratia}) \\
					\hline
				\end{tabular}
				}
\end{table}


%

\newpage

%

\pagenumbering{arabic}

%

\chapter{Differential geometry in euclidean spaces}
\section{Smooth and real analytic functions}

\subsection{Smooth functions}

Let us consider $ \R^{n} $ with $ n \geqslant 1 $ and $ U \subset \R^{n} $ open, and let $ f : U \to \R $ be a function and $ p \in U $ a point: defining the $ k $th-order derivatives of $ f $ as

\begin{equation}
	\dfrac{\partial^{k} f}{\partial (x^{1})^{i_{1}} \cdots \partial (x^{k})^{i_{k}}} %
	\qq{where} \sum_{j=1}^{k} i_{j} = k \in \N
\end{equation}

we say that $ f \in C^{k} $ in $ p $ with $ k \in \N $ if the $ k $th-order derivatives of $ f $ exist and are continuous in $ p $. \\
Let $ k = 0 $, then

\begin{equation}
	f \in C^{0} \iff f \text{ continuous}
\end{equation}

Notation-wise:

\begin{itemize}
	\item $ f \in C^{k} $ in $ U $ if $ f \in C^{k} $ in $ p $ for all $ p \in U $
	
	\item $ f \in C^{\infty} $ or \textit{smooth} in $ p $ if $ f \in C^{k} $ in $ p $ for all $ k \in \N $
	
	\item $ f \in C^{\infty} $ or \textit{smooth} in $ U $ if $ f \in C^{\infty} $ for all $ p \in U $
\end{itemize}

therefore a function is denoted smooth if all its derivatives of any order exist and are finite. \\
In general, we consider functions defined not in $ \R $ but in $ \R^{n} $. \\
A function $ f : U \to \R^{n} $ with $ n \geqslant 1 $ and $ U \subset \R^{m} $ with $ m \geqslant 1 $ open is $ C^{k} $ in $ p $ if all its components $ f^{j} : U \to \R $ are $ f^{j} \in C^{k} $ in $ p $ with $ k \geqslant 0 $. In particular, $ f = (f^{1}, \dots, f^{m}) $ or $ f^{j} = \pi_{j} \circ f $ where $ \pi_{j} $ is the \textit{projection}

\map{\pi_{j}}
	{\R^{m}}{\R}
	{(x^{1}, \dots, x^{m})}{x^{j}}

for $ j = 1, \dots, m $. \\
A function $ f : U \to \R^{m} $ is:

\begin{itemize}
	\item $ C^{k} $ in $ U $ if $ f^{j} \in C^{k} $ in $ U $
	
	\item smooth in $ p \in U $ if $ f^{j} \in C^{\infty} $ in $ p $ for all $ j = 1, \dots, m $
	
	\item smooth in $ U $ if $ f^{j} \in C^{\infty} $ in $ U $ for all $ j = 1, \dots, m $
\end{itemize}

\subsubsection{\textit{Examples}}

\paragraph{1) Cubic root}

Let

\map{f}
	{\R}{\R}
	{x}{x^{1/3}}
%	
This function is continuous ($ f \in C^{0} $) and it is and homeomorphism\footnote{%
	Both the function and its inverse are continuous.%
} but $ f \notin C^{1} $ in the origin $ p = 0 $ because

\begin{equation}
	f' = \dfrac{x^{-2/3}}{3}
\end{equation}

which is not defined in the origin and therefore $ f \notin C^{1} (\R) $.

\paragraph{2) $ C^{1} $ function which is not $ C^{2} $}

Integrating $ f $ from the previous example, we obtain a $ C^{1} $ function which is not $ C^{2} $. Let $ g : \R \to \R $ with

\begin{equation}
	g (x) = \int_{0}^{x} f (t) \dd{t} = \dfrac{3 x^{2/3}}{4}
\end{equation}

from which $ g \in C^{1} (\R) $ but $ g \notin C^{2} (\R) $.

\paragraph{3) $ C^{k} $ function which is not $ C^{k+1} $}

See Exercise \ref{exer1-1}.

\subsection{Funzioni reali analitiche}

Sia il punto $ p \in \R^{n} $, un \textit{intorno} $ U $ di $ p $ è un aperto di $ \R^{n} $ che contiene $ p $. \\
Sia una funzione $ f : U \to \R $ con $ U \subset \R^{n} $ aperto, diremo che $ f $ è \textit{reale analitica} in $ p \in U $ se $ f $ coincide con il suo sviluppo di Taylor intorno a $ p $. Questo significa che se prendiamo una funzione $ f (x) $ con $ x = (x^{1}, \dots, x^{n}) $ e $ p = (p^{1}, \dots, p^{n}) $ abbiamo che
\begin{align}
	\begin{split}
		f (x) &= f (p) + \sum_{i=1}^{n} \pdv{f}{x_{i}} (p) (x^{i} - p^{i}) + \cdots + \dfrac{1}{k!} \sum_{i_{1},\dots,i_{k}=1}^{n} \dfrac{\partial^{k} f}{\partial x^{i_{1}} \cdots \partial x^{i_{k}}} (p) ( (x^{i_{1}} - p^{i_{1}}) \cdots (x^{i_{k}} - p^{i_{k}}) ) + \cdots \\
		&= f (p) + \sum_{k=1}^{\infty} \dfrac{1}{k!} \sum_{i_{1},\dots,i_{k}=1}^{n} \dfrac{\partial^{k} f}{\partial x^{i_{1}} \cdots \partial x^{i_{k}}} (p) \prod_{j=1}^{k} (x^{j} - p^{j})
	\end{split}
\end{align}

Se abbiamo una serie di potenze, possiamo derivarla termine a termine dunque, siccome una funzione reale analitica coincide con il suo sviluppo in serie di Taylor, è possibile derivarla ottenendo sempre una funzione continua con derivata continua. A questo punto $ f \in C^{\infty} $: questo segue dall'analisi in quanto le serie di potenze possono essere derivate un numero arbitrario di volte.

\subsubsection{\textit{Esempi}}

\paragraph{1) Seno}

La funzione $ f (x) = \sin(x) $ è liscia reale analitica e ha sviluppo di Taylor

\begin{equation}
	\sin(x) = x - \dfrac{x^{3}}{3!} + \dfrac{x^{5}}{5!} - \dots %
	= \sum_{j=0}^{\infty} (-1)^{j} \dfrac{x^{2j+1}}{(2j+1)!}
\end{equation}

Per calcolare la derivata possiamo derivare termine a termine lo sviluppo di Taylor

\begin{align}
	\begin{split}
		\dv{x} \sin(x) &= \dv{x} \sum_{j=0}^{\infty} (-1)^{j} \dfrac{x^{2j+1}}{(2j+1)!} \\
		&= \sum_{j=0}^{\infty} (-1)^{j} (2j+1) \dfrac{x^{2j}}{(2j+1)!} \\
		&= \sum_{j=0}^{\infty} (-1)^{j} \dfrac{x^{2j}}{(2j)!} \\
		&= \cos(x)
	\end{split}
\end{align}

\paragraph{2) Esponenziale}

Per trovare la derivata di $ f(x) = e^{x} $ ripetiamo lo stesso procedimento

\begin{equation}
	\dv{x} e^{x} = \dv{x} \sum_{j=0}^{\infty}\dfrac{x^{j}}{j!} %
	= \sum_{j=0}^{\infty} j \, \dfrac{x^{j-1}}{j!} %
	= \sum_{j=0}^{\infty} \dfrac{x^{j-1}}{(j-1)!} %
	= \sum_{n=-1}^{\infty} \dfrac{x^{n}}{n!} %
	= \sum_{n=0}^{\infty} \dfrac{x^{n}}{n!} %
	= e^{x}
\end{equation}

\paragraph{3) Funzione liscia non reale analitica}

Un esempio di funzione liscia ma non reale analitica è

\map{f}
	{\R}{\R}
	{x}{%
		\begin{cases}
			e^{-1/x^{2}}, & \text{se } x > 0 \\
			0, & \text{se } x \leqslant 0
		\end{cases}
		}

Per dimostrare che sia $ C^{0} $ dobbiamo verificare che

\begin{equation}
	\lim_{x \to 0} e^{-1/x^{2}} = 0
\end{equation}

Per dimostrare che sia liscia\footnote{%
	Vedi Esercizio \ref{exer1-2}.}

\begin{equation}
	\lim_{x \to 0} f' (x) = \lim_{x \to 0} \left(\dfrac{2}{x^{3}}\right) e^{-1/x^{2}} = 0
\end{equation}

Tutto questo ci dice che $ f \in C^{\infty}(\R) $. Se fosse anche reale analitica, dovrebbe coincidere con il suo sviluppo in serie di Taylor anche nell'origine, dunque

\begin{align}
	\begin{split}
		f(x) = \sum_{k=0}^{\infty} \pdv[k]{f}{x} (0) \, x^{k}
	\end{split}
\end{align}

ma $ f(x) $ nell'intorno di 0 è nulla solo per $ x \leqslant 0 $ mentre lo sviluppo di Taylor è sempre nullo: questa contraddizione porta a dire che, nonostante $ f \in C^{\infty}(\R) $, questa non è reale analitica, scritto anche come $ f \notin C^{\omega}(\R) $. \\
Un altro motivo per il quale $ f \notin C^{\omega}(\R) $ segue dal fatto che se $ f : U \to \R $ con $ U \in \R $ aperto è reale analitica e $ f = 0 $ in un aperto, allora $ f \equiv 0 $ ovunque\footnote{%
	Questa proprietà è valida anche se si considera una costante diversa da 0.%
}.

\section{Diffeomorfismi tra aperti di $ \R^{n} $}

Siano $ U, V \in \R^{n} $ aperti, diremo che $ f : U \to V $ è un \textit{diffeomorfismo} se è una bigezione\footnote{%
	Perciò è invertibile.%
}, $ f \in C^{\infty}(U) $ e la sua inversa $ g : V \to U $ è  $ g \in C^{\infty}(V) $. \\
Ad esempio, la funzione
%
\map{f}
	{\R}{\R}
	{x}{x^{3}}

è una bigezione liscia ma la sua inversa non è liscia, dunque $ f $ non è un diffeomorfismo. \\
Quando esiste un diffeomorfismo tra due aperti, si dice che questi sono \textit{diffeomorfi}, i.e. $ U $ e $ V $ sono diffeomorfi se esiste $ f : U \to V $ diffeomorfismo, in notazione $ U \simeq V $. \\

\begin{theorem}[Invarianza topologica della dimensione]
	Se $ U \subset \R^{n} $ e $ V \subset \R^{m} $ sono aperti omeomorfi allora $ n = m $.
\end{theorem}

\begin{theorem}[Invarianza differenziabile della dimensione]
	Se $ U \subset \R^{n} $ e $ V \subset \R^{m} $ sono aperti diffeomorfi allora $ n = m $\footnote{%
		Questo teorema implica quello di "Invarianza topologia della dimensione" in quanto la condizione di diffeomorfismo implica quella di omeomorfismo: una bigezione liscia con inversa liscia è una bigezione continua con inversa continua, poiché $ C^{\infty} \implies C^{0} $.%
	}.
\end{theorem}

È naturale verificare se gli spazi legati da omeomorfismi siano legati anche da diffeomorfismi. Ad esempio, abbiamo che i seguenti sottoinsiemi aperti di $ \R $ sono diffeomorfi tra loro\footnote{%
	Vedi Esercizio \ref{exer1-3}.%
}:

\begin{equation}
	(a,b) \simeq (c, + \infty) \simeq (- \infty, d) \simeq \R %
	\qcomma \forall a,b,c,d \in \R, \, a < b
\end{equation}

\subsection{Diffeomorfismo tra $ B_{\delta} (c) $ e $ \R^{n} $}

Indichiamo con $ B_{1} (0) $ la \textit{palla} di centro l'origine e raggio unitario, i.e.

\begin{equation}
	B_{1} (0) = \left\{ x \in \R^{n} \st \norm{x} \doteq \sqrt{\sum_{i=1}^{n} (x^{i})^{2}} < 1 \right\}
\end{equation}

Per $ n=1 $, $ B_{1} (0) \equiv (-1,1) \simeq \R $. \\
Definiamo

\map{f}
	{B_{1} (0)}{\R^{n}}
	{x}{\left( \dfrac{x^{1}}{\sqrt{1 - \norm{x}^{2}}}, \cdots, \dfrac{x^{n}}{\sqrt{1 - \norm{x}^{2}}} \right)}

questa applicazione è un diffeomorfismo. Per verificarlo, dobbiamo dimostrare che $ f $ sia un bigezione, $ f \in C^{\infty}(B_{1}(0)) $ e che $ f^{-1} \in C^{\infty}(\R^{n}) $. \\
L'inversa è

\map{f^{-1}}
	{\R^{n}}{B_{1} (0)}
	{x}{\left( \dfrac{x^{1}}{\sqrt{1 + \norm{x}^{2}}}, \cdots, \dfrac{x^{n}}{\sqrt{1 + \norm{x}^{2}}} \right)}

in quanto

\begin{equation}
	f \circ g = \id_{\R^{n}} \; \wedge \; g \circ f = \id_{B_{1} (0)}
\end{equation}

Perché sia $ f $ che $ f^{-1} $ siano lisce, dobbiamo verificare che ogni loro componente lo sia, il quale è verificato perché la derivata di una delle componenti di $ f $ ha al denominatore sempre

\begin{equation}
	\sqrt{1 - \norm{x}^{2}} \neq 0 \qcomma \forall x \in B_{1} (0)
\end{equation}

e lo stesso vale per la sua inversa

\begin{equation}
	\sqrt{1 + \norm{x}^{2}} \neq 0 \qcomma \forall x \in \R^{n}
\end{equation}

\begin{corollary}
	La palla di centro $ c $ e raggio $ \delta $ con $ c \in \R^{n} $ e $ \delta \geqslant 0 $ è diffeomorfa a $ \R^{n} $, i.e. $ B_{\delta} (c) \simeq B_{1} (0) \simeq \R^{n} $.
\end{corollary}

\begin{proof}
	Vedi Esercizio \ref{exer1-4}; la dimostrazione passa per il mostrare che le traslazioni (le quali sono lineari e affini) e le omotetie (scala di un fattore $ \delta $) siano diffeomorfismi.
\end{proof}

\sbs{0.3}{%
			Per praticità di notazione, chiamiamo $ h $ il diffeomorfismo $ B_{\delta} (c) \to \R^{n} $ definito sopra. Per far vedere come nasce questo diffeomorfismo, si può usare la costruzione geometrica a lato.
			}
	{0.8}{%
			\begin{figure}[H]
				\centering
				\begin{tikzpicture}[scale=1]
					
					\begin{axis}[
						axis lines=center,
						anchor=origin,
						axis equal=true,
						xlabel={$ (x^{1}, \dots, x^{n}) $},
						ylabel=$ x^{n+1} $,
						xticklabel=\empty,
						yticklabel=\empty,
						xlabel style = {anchor=south west},
						ylabel style = {anchor=south},
						xmax=2.5,
						xmin=-1.5,
						ymax=1.5,
						ymin=-0.5
						]
						
						\addplot[
						samples=1000, 
						color=black,
						]
						{1-sqrt(1-x^2)};
						
						\addplot[
						domain=0:2,
						samples=1000, 
						color=red,
						]
						{1-x/2};
						
						\addplot[
						samples=1000, 
						color=black,
						dashed,
						]
						coordinates {(1,0)(1,1)};
						
						\addplot[
						samples=1000, 
						color=black,
						dashed,
						]
						coordinates {(-1,0)(-1,1)};
						
						\addplot[
						color=red,
						mark=*,
						]
						coordinates {(0,1)};
						
						\addplot[
						color=red,
						mark=*,
						]
						coordinates {(2,0)};
						
						\addplot[
						color=blue,
						mark=*,
						]
						coordinates {(0,1)};
						
						\addplot[
						color=blue,
						mark=*,
						]
						coordinates {(2,0)};
						
					\end{axis}
					
				\end{tikzpicture}
				
			\end{figure}
			}

Consideriamo la semicalotta aperta in $ \R^{n+1} $ centrata in $ (0,\dots,0,1) $ di raggio $ 1 $:

\begin{equation}
	S = \left\{ (x^{1}, \dots, x^{n+1}) \in \R^{n+1} \st (x^{n+1})^{2} + \sum_{i=1}^{n} (x^{i})^{2} = 1 \, \wedge \, x^{n+1} < 1 \right\}
\end{equation}

La palla $ B_{1}(0) $ vive nella proiezione della semicalotta sull'iperpiano $ (x^{1}, \dots, x^{n}) $, definita come

\begin{equation}
	B_{1}(0) = \left\{ x \in \R^{n} \st \norm{x} < 1 \right\}
\end{equation}

Questa proiezione permette di costruire l'applicazione $ h $ in due passaggi: prima prendiamo un punto in $ B_{1}(0) $, lo proiettiamo su $ S $ e, con una proiezione stereografica, lo portiamo su $ \R^{n} $. La prima applicazione è $ f : B_{1}(0) \to S $ mentre la seconda $ g : S \to \R^{n} $, cioè la proiezione stereografica dal punto $ (0,\dots,0,1) $. Abbiamo dunque che $ g \circ f = h $. Le mappe sono

\begin{equation}
	f (x^{1},\dots,x^{n+1}) = \left( x^{1},\dots,x^{n},1-\sqrt{1-\norm{x}^{2}} \right)
\end{equation}

\begin{equation}
	g (x^{1},\dots,x^{n+1}) = \left( \dfrac{x^{1}}{1-x^{n+1}},\dots,\dfrac{x^{n}}{1-x^{n+1}},0 \right)
\end{equation}

da cui

\begin{align}
	\begin{split}
		h (x^{1},\dots,x^{n+1}) &= (g \circ f) (x^{1},\dots,x^{n+1}) \\
		&= g \left( x^{1},\dots,x^{n},1-\sqrt{1-\norm{x}^{2}} \right) \\
		&= \left( \dfrac{x^{1}}{\sqrt{1 - \norm{x}^{2}}}, \cdots, \dfrac{x^{n}}{\sqrt{1 - \norm{x}^{2}}}, 0 \right)
	\end{split}
\end{align}

A questo punto $ B_{1} (0) \simeq \R^{n} $: dal punto di vista della geometria differenziale, due oggetti diffeomorfi vengono considerati equivalenti\footnote{%
	In topologia, vale lo stesso ragionamento per oggetti omeomorfi.%
}.

\subsection{Teorema di Taylor con resto}

Una funzione reale analitica coincide con il suo sviluppo di Taylor. Per una funzione liscia questo non è detto: la coincidenza di una funzione liscia con il suo sviluppo di Taylor è data a meno di un \textit{resto}. Introduciamo ora il concetto di insieme stellato rispetto a un punto per definire il resto sopraccitato. \\
Un sottoinsieme aperto $ U \subset \R^{n} $ è \textit{stellato} rispetto a un punto $ p \in U $ se il segmento di retta che unisce $ p $ a qualsiasi $ x \in U $ è interamente contenuto in $ U $.

\begin{remark}
	Un insieme convesso è stellato rispetto a ogni suo punto.
\end{remark}

L'ipotesi che un sottoinsieme sia stellato è forte a livello globale ma sempre rispettata a livello locale, in quanto è sempre possibile trovare un aperto stellato rispetto a un punto all'interno di un insieme.

\begin{theorem}[Taylor con resto]\label{thm:taylor}
	Sia $ f : U \to \R $ con $ U \subset \R^{n} $ stellato rispetto a un punto $ p \in U $ e supponiamo $ f \in C^{\infty}(U) $, allora esistono $ n $ funzioni $ g_{i} \in C^{\infty}(U) $ per $ i = 1,\dots,n $ definite come
	
	\begin{equation}
		g_{i}(p) \doteq \pdv{f}{x^{i}} \, (p) \qcomma i=1,\dots,n
	\end{equation}

	tali che
	
	\begin{equation}
		f(x) = f(p) + \sum_{i=1}^{n} (x^{i}-p^{i}) \, g_{i}(x) \qcomma \forall x \in U
	\end{equation}	
\end{theorem}

\begin{proof}
	Consideriamo il segmento $ r $ che  unisce $ p $ a un punto $ x \in U $ con $ x $ fissato  arbitrariamente:
	
	\begin{equation}
		r=p+t(x-p) \qcomma t \in [0,1]
	\end{equation}
	
	Essendo $ U $ stellato rispetto a $ p $, possiamo valutare $ f $ in questo segmento (tutti i punti di $ r $ sono definiti in $ U $). Consideriamo fissi $ x $ e $ p $ e derivo $ f(r) $ rispetto a $ t $
	
	\begin{align}
		\begin{split}
			\dv{t} f(r) &= \dv{t} f(p+t(x-p)) \\
			&= \sum_{i=1}^{n} \pdv{f}{x^{i}} \, (p+t(x-p)) \, \left( \dv{r}{t} \right)^{i} \\
			&= \sum_{i=1}^{n} \pdv{f}{x^{i}} \, (p+t(x-p)) \, (x^{i}-p^{i})
		\end{split}
	\end{align}
	
	per la \textit{regola della catena}. \\
	Integrando rispetto a $ t $ nell'intervallo $ [0,1] $ otteniamo
	
	\begin{align}
		\begin{split}
			\int_{0}^{1} \dv{t} f(p+t(x-p)) \dd{t} &= \int_{0}^{1} \sum_{i=1}^{n} \pdv{f}{x^{i}} \, (p+t(x-p)) \, (x^{i}-p^{i}) \dd{t} \\
			f(x) - f(p) &= \sum_{i=1}^{n} (x^{i}-p^{i}) \int_{0}^{1} \pdv{f}{x^{i}} \, (p+t(x-p)) \dd{t}
		\end{split}
	\end{align}
	
	chiamando
	
	\begin{equation}
		g_{i}(x) \doteq \int_{0}^{1} \pdv{f}{x^{i}} \, (p+t(x-p)) \dd{t}
	\end{equation}
	
	si può scrivere
	
	\begin{equation}
		f(x)= f(p) + \sum_{i=1}^{n} (x^{i}-p^{i}) \, g_{i}(x)
	\end{equation}
	
	dove $ g_{i}(x) \in C^{\infty}(U) $ perché derivata parziale di una funzione liscia. \\
	Inoltre
	
	\begin{equation}
		g_{i}(p) = \int_{0}^{1} \pdv{f}{x^{i}} \, (p) \dd{t} = \pdv{f}{x^{i}} \, (p) \qcomma i=1,\dots,n
	\end{equation}
\end{proof}

Sia $ f : U \to \R $ con $ p $ corrispondente all'origine: per il teorema di Taylor con resto, sappiamo che esiste una funzione $ g_{1} \in C^{\infty}(U) $ tale che

\begin{equation}
	f(x) = f(0) + x \, g_{1}(x) \qq{con} g_{1}(0) = f'(0)
\end{equation}

Riapplicando il teorema a $ g_{1} $ (in quanto liscia), otteniamo

\begin{equation}
	g_{1}(x) = g_{1}(0) + x g_{2}(x) %
	\qquad %
	\begin{cases}
		g_{2} \in C^{\infty}(U) \\
		g_{2}(0) = g_{1}'(0)
	\end{cases}
\end{equation}

Per induzione

\begin{equation}
	g_{i}(x) = g_{i}(0) + x g_{i+1}(x) %
	\qquad %
	\begin{cases}
		g_{i+1} \in C^{\infty} (U) \\
		g_{i+1}(0) = g_{i}'(0)
	\end{cases} %
	\qquad %
	\forall i \geqslant 1
\end{equation}

Sostituendo in $ f $ tutte queste funzioni, si ottiene

\begin{align}
	\begin{split}
			f(x) &= f(0) + x g_{1}(x) \\
			&= f(0) + x g_{1}(0) + x^{2} g_{2}(x) \\
			&= f(0) + x g_{1}(0) + x^{2} g_{2}(0) + x^{3} g_{3}(x) \\
			& \;\, \vdots \\
			&= f(0) + x g_{1}(0) + \dots + x^{k} g_{k}(0) + x^{k+1} g_{k+1}(x)
	\end{split}
\end{align}

A questo punto si può definire

\begin{equation}
	g_{k}(0) = \dfrac{1}{k!} \pdv[k]{f}{x} \, (0) \doteq \dfrac{f^{(k)}(0)}{k!}
\end{equation}

da cui

\begin{equation}
	f(x) = f(0) + \sum_{k=1}^{i} \dfrac{x^{k}}{k!} f^{(k)} (0) + x^{i+1} g_{i+1}(x) \qcomma \forall i \in \N
\end{equation}

dove la prima parte coincide con lo sviluppo in serie di Taylor mentre l'ultimo termine indica il \textit{resto}. \\
Per esercizi sul resto, vedi Esercizi \ref{exer1-5} e \ref{exer1-6}.

\section{Vettori tangenti in $ \R^{n} $}

Preso un punto $ p \in \R^{n} $, lo \textit{spazio tangente} in quel punto viene chiamato $ T_{p}(\R^{n}) $. Lo spazio tangente a un punto $ p $ è l'insieme\footnote{%
	Formalmente, è uno spazio vettoriale con origine il punto $ p $.%
} di tutti i vettori che escono dal punto stesso. Essendo $ T_{p}(\R^{n}) \stackrel{iso}{\simeq} \R^{n} $, un elemento $ v \in T_{p}(\R^{n}) $ può dunque essere rappresentato come un \textit{vettore riga o colonna}

\begin{equation}
	\bmqty{ v^{1} & \cdots & v^{n} } \qquad \lor \qquad \bmqty{ v^{1} \\ \vdots \\ v^{n} }
\end{equation}

dove le $ v^{i} $ sono le componenti del vettore nella base canonica, i.e.

\begin{equation}
	v = \sum_{i=1}^{n} v^{i} e_{i}
\end{equation}

Per generalizzare questo concetto, considereremo gli elementi degli spazi tangenti non più come oggetti geometrici vettori ma come \textit{derivazioni}.

\subsection{Derivate direzionali}

Siano un'applicazione $ f : U \to \R $ con $ f \in C^{\infty}(\R^{n}) $, un punto $ p\in U \subset \R^{n} $ e un vettore $ v \in T_{p}(\R^{n}) $. Consideriamo la retta $ c(t) $ che passa per $ p $ con direzione $ v $, parametrizzata come

\begin{equation}
	c(t) = p + t v \qcomma t \in \R
\end{equation}

Definiamo la \textit{derivata direzionale} di $ f $ rispetto a $ v $ come

\begin{align}
	\begin{split}
		D_{v} f &\doteq \lim_{t \to 0} \dfrac{f(c(t)) - f(p)}{t} \\
		&= \left. \dv{t} f(c(t)) \right|_{t=0} \\
		&= \sum_{i=1}^{n} \pdv{f}{x^{i}} \, (p) \left( \dv{t} c(t) \right)^{i} \\
		&= \sum_{i=1}^{n} \pdv{f}{x^{i}} \, (p) \, v^{i}
	\end{split}
\end{align}

dove $ D_{v} f \in \R $ e $ v = \bmqty{ v^{1} & \cdots & v^{n} } $.

\sbs{0.4}{%
			\begin{remark}
				Sia un'applicazione $ g \in C^{\infty} (\R^{n}) $ tale che $ g : V \to \R $ con $ V \subset \R $ e $ V \cap U \neq \emptyset $. Se $ g \equiv f $ in un intorno $ W $ del punto $ p \in W \subset U \cap V $, allora la loro derivata direzionale è la stessa\footnotemark, i.e.
				
				\begin{equation}
					D_{v} g = D_{v} f \qcomma \forall p \in W
				\end{equation}
			\end{remark}
			}
	{0.6}{%
			\img{0.6}{img1}
			}
\footnotetext{%
	Questo perché il limite del rapporto incrementale nella definizione di $ D_{v} f $ dipende da un intorno arbitrariamente piccolo.%
}

Definiamo ora l'insieme di coppie

\begin{equation}
	X \doteq \left\{ (f,U) \st f \in C^{\infty}(U), \, U \text{ intorno di } p \in U \right\}
\end{equation}

Diremo che\footnote{%
	Il simbolo $ \sim $ indica una relazione di equivalenza.%
} per $ p \in W $

\begin{equation}
	(f,U) \sim (g,V) \iff \E W \subseteq U \cap V, \, W \ni p \mid f(q) = g(q) \qcomma \forall q \in W
\end{equation}

Questa è effettivamente una relazione di equivalenza in quanto riflessiva, simmetrica e transitiva. \\
Prendiamo dunque lo spazio quoziente\footnote{%
	Approfondiremo l'argomento degli spazi quoziente nella Sottosezione \ref{s-sec:quot}.%
}

\begin{equation}
	\sfrac{X}{\sim} \doteq C_{p}^{\infty}(\R^{n})
\end{equation}

dove un elemento $ [(f,U)] $ di questo spazio è chiamato \textit{germe} intorno al punto $ p $ ed è una \textit{classe di equivalenza} di coppie $ (f,U) $. A questo punto, $ C_{p}^{\infty}(\R^{n}) $ è l'insieme dei germi di funzioni lisce intorno a $ p $, i.e.

\begin{equation}
	C_{p}^{\infty}(\R^{n}) = \{ [(f,U)] \mid f : U \to \R, \, f \in C^{\infty}(\R^{n}), \, U \subset \R^{n} \}
\end{equation} 

Possiamo definire un'applicazione

\map{D_{v}}
	{C_{p}^{\infty}(\R^{n})}{\R}
	{[(f,U)]}{D_{v} f}

Questa applicazione è ben definita in quanto l'associazione di un germe di funzioni a un numero reale non dipende dal rappresentante scelto poiché

\begin{equation}
	(f,U) \sim (g,V) \implies D_{v} g = D_{v} f
\end{equation}

\subsubsection{\textit{Esempio}}

Siano le applicazioni
\begin{equation}
	\begin{cases}
		f(x) = \dfrac{1}{1-x}, & x \in \R \setminus \{1\} \\ \\
		g(x) = \displaystyle\sum_{j=1}^{+\infty} x^{j}, & x \in (-1,1)
	\end{cases}
\end{equation}

Nonostante in generale $ f \neq g $, nell'intorno $ (-1,1) $ di $ p=0 $ vale l'equivalenza per i germi

\begin{equation}
	(f,\R \setminus \{1\}) \sim (g,(-1,1))
\end{equation}

in altre parole, le classi di equivalenza

\begin{equation}
	[(f,\R \setminus \{1\})] = [(g,(-1,1))] \in C_{0}^{\infty}(\R)
\end{equation}

\subsubsection{Algebra su campo $ \K $}

Un'algebra $ A $ su un campo $ \K $ è una coppia $ (V,\cdot) $ con $ V $ spazio vettoriale su un campo\footnote{%
	Dunque con operazioni
	\begin{equation*}
		\begin{cases}
			a+b \in A, & \forall a,b \in A \\
			\lambda a \in A, & \forall \lambda \in \K
		\end{cases}
	\end{equation*}%
} $ \K $ e un'operazione binaria

\map{\cdot}
	{A \times A}{A}
	{(a,b)}{a \cdot b}

tale che soddisfi le condizioni

\begin{equation}
	\begin{cases}
		(a \cdot b) \cdot c = a \cdot (b \cdot c) & \text{associatività}\footnotemark \\
		\begin{split}
			(a + b) \cdot c = a \cdot c + b \cdot c \\
			c \cdot (a + b) = c \cdot a + c \cdot b
		\end{split} & \text{distributività} \\
	\lambda (a \cdot b) = (\lambda a) \cdot b = a \cdot (\lambda b) & \text{omogeneità}
	\end{cases}
\footnotetext{%
	In generale, non è necessaria l'associatività per definire un'algebra.%
}
\end{equation}

per qualsiasi $ a,b,c \in A $ e qualsiasi $ \lambda \in \K $. \\
Equivalentemente, un algebra su un campo $ \K $ può essere pensata come un anello\footnote{%
	Le proprietà di associatività e distributività sono sufficienti per renderla un anello.%
} $ (V,+,\cdot) $ il quale sia anche uno spazio vettoriale con aggiunta la proprietà di omogeneità.

\subsection{$ C_{p}^{\infty}(\R^{n}) $ come algebra su $ \R $}

Definiamo la somma

\begin{equation}
	[(f,U)] + [(g,V)] = [(f + g,U \cap V)] \qcomma [(f,U)], [(g,V)] \in C_{p}^{\infty}(\R^{n})
\end{equation}

Questa somma è ben definita in quanto, prendendo due rappresentanti qualunque di $ [(f,U)] $ e $ [(g,V)] $, esiste sempre un intorno in cui questa somma sia definita. \\
Allo stesso modo, definiamo il prodotto

\begin{equation}
	[(f,U)] \cdot [(g,V)] = [(f g,U \cap V)] \qcomma [(f,U)], [(g,V)] \in C_{p}^{\infty}(\R^{n})
\end{equation}

e la moltiplicazione per scalari

\begin{equation}
	\lambda [(f,U)] = [(\lambda f,U)] \qcomma \lambda \in \R, \, [(f,U)] \in C_{p}^{\infty}(\R^{n})
\end{equation}

Tutte queste operazioni sono ben definite e soddisfano tutte le proprietà di un'algebra perché, per funzioni lisce, la somma, il prodotto e la moltiplicazione soddisfano queste stesse proprietà. \\
A questo punto si può dire che $ C_{p}^{\infty}(\R^{n}) $ sia un'algebra su $ \R $. \\
Nonostante non sia necessario per un'algebra, $ C_{p}^{\infty}(\R^{n}) $ è anche commutativa e unitaria\footnote{%
	Vedi Esercizio \ref{exer1-7}.%
} su $ \R^{n} $.

\subsection{Derivazione puntuale di $ C_{p}^{\infty}(\R^{n}) $}

A questo punto, possiamo definire l'applicazione chiamata \textit{derivazione puntuale} dell'algebra $ C_{p}^{\infty}(\R^{n}) $:

\map{D}
	{C_{p}^{\infty}(\R^{n})}{\R}
	{[(f,U)]}{D_{v} f = \sum_{i=1}^{n} \pdv{f}{x^{i}} \, (p) \, v^{i}}

con $ p \in U \subset \R^{n} $ e $ v = (v^{1}, \dots, v^{n}) \in T_{p} (\R^{n}) $. \\
Questa applicazione possiede le seguenti proprietà:

\begin{enumerate}
	\item $ \R $-linearità\footnote{%
		Rispetto alla struttura di spazio vettoriale di $ C_{p}^{\infty}(\R^{n}) $.%
	}, i.e.
		%
		\begin{align}
			D ([(f,U)] + [(g,V)]) &= D ([(f,U)]) + D ([(g,V)]) \\
			D (\lambda [(f,U)]) &= \lambda D ([(f,U)])
		\end{align}
	
	\item soddisfa la \textit{regola di Leibniz}, i.e.
		%
		\begin{equation}
			D ([(f,U)] \cdot [(g,V)]) = D ([(f,U)]) \, g(p) + f(p) \, D ([(g,V)])
		\end{equation}
\end{enumerate}

\begin{proof}[Dimostrazione ($ \R $-linearità (somma))]
	%
	\begin{align}
		\begin{split}
			D ([(f,U)] + [(g,V)]) &= D ([(f+g,U \cap V)]) \\
			&= D_{v} (f+g) \\
			&= \sum_{j=1}^{n} \pdv{(f+g)}{x^{j}} \, (p) \, v^{j} \\
			&= \sum_{j=1}^{n} \pdv{f}{x^{j}} \, (p) \, v^{j} + \sum_{j=1}^{n} \pdv{g}{x^{j}} \, (p) \, v^{j} \\
			&= D_{v} f + D_{v} g \\
			&= D ([(f,U)]) + D ([(g,V)])
		\end{split}
	\end{align}

	per qualsiasi $ [(f,U)], [(g,V)] \in C_{p}^{\infty}(\R^{n}) $, qualsiasi $ p \in U \cap V \subset \R^{n} $ e qualsiasi $ v \in T_{p} (\R^{n}) $.
\end{proof}

\begin{proof}[Dimostrazione ($ \R $-linearità (moltiplicazione per scalare))]
	\begin{align}
		\begin{split}
			D (\lambda [(f,U)]) &= D ([(\lambda f,U)]) \\
			&= D_{v} (\lambda f) \\
			&= \sum_{j=1}^{n} \pdv{(\lambda f)}{x^{j}} \, (p) \, v^{j} \\
			&= \lambda \sum_{j=1}^{n} \pdv{f}{x^{j}} \, (p) \, v^{j} \\
			&= \lambda D ([(f,U)])
		\end{split}
	\end{align}

	per qualsiasi $ [(f,U)], [(g,V)] \in C_{p}^{\infty}(\R^{n}) $, qualsiasi $ \lambda \in \R $, qualsiasi $ p \in U \subset \R^{n} $ e qualsiasi $ v \in T_{p} (\R^{n}) $.
\end{proof}

\begin{proof}[Dimostrazione (Regola di Leibniz)]
	\begin{align}
		\begin{split}
			D ([(f,U)] \cdot [(g,V)]) &= D ([(f g,U \cap V)]) \\
			&= D_{v} (f g) \\
			&= \sum_{j=1}^{n} \pdv{(f g)}{x^{j}} \, (p) \, v^{j} \\
			&= \left( \sum_{j=1}^{n} \pdv{f}{x^{j}} \, (p) \, v^{j} \right) \, g(p) + f(p) \left( \, \sum_{j=1}^{n} \pdv{g}{x^{j}} \, (p) \, v^{j} \right) \\
			&= (D_{v} f) \, g(p) + f(p) \, (D_{v} g) \\
			&= D ([(f,U)]) \, g(p) + f(p) \, D ([(g,V)])
		\end{split}
	\end{align}

	per qualsiasi $ [(f,U)], [(g,V)] \in C_{p}^{\infty}(\R^{n}) $, qualsiasi $ p \in U \cap V \subset \R^{n} $ e qualsiasi $ v \in T_{p} (\R^{n}) $.
\end{proof}

La derivazione puntuale è quindi un modo per associare un numero reale a un germe di funzioni, soddisfacendo le proprietà definite sopra. \\
Indichiamo dunque l'insieme delle derivazioni puntuali di $ C_{p}^{\infty}(\R^{n}) $ come $ \der_{p}(C_{p}^{\infty}(\R^{n})) $, i.e.

\begin{equation}
	\der_{p}(C_{p}^{\infty}(\R^{n})) \doteq \left\{ D ([(f,U)]) = D_{v} f \doteq \sum_{j=1}^{n} \pdv{f}{x^{j}} \, (p) \, v^{j} \in \R \st [(f,U)] \in C_{p}^{\infty}(\R^{n}), v \in T_{p} (\R^{n}) \right\}
\end{equation}

\subsection{Isomorfismo tra $ T_{p}(\R^{n}) $ e $ \der_{p}(C_{p}^{\infty}(\R^{n})) $}

Definiamo l'applicazione

\map{\varphi}
	{T_{p}(\R^{n})}{\der_{p}(C_{p}^{\infty}(\R^{n}))}
	{v}{D_{v}}

questa associa il vettore $ v = (v^{1},\dots,v^{n}) \in T_{p}(\R^{n}) $ con $ p \in \R^{n} $ alla derivazione puntuale $ D_{v} $, la quale è a sua volta un'applicazione che associa la classe di equivalenza di germi di funzioni $ [(f,U)] $ alla derivata direzionale di $ f $ rispetto a $ v $, i.e.

\begin{align}
	D_{v} f = \sum_{j=1}^{n} \pdv{f}{x^{j}} \, (p) \, v^{j} \in \R
\end{align}

Possiamo usare lo stesso simbolo, i.e. $ D_{v} ([(f,U)]) = D_{v} f $, perché questa relazione vale per qualunque rappresentante della classe. \\
L'applicazione $ \varphi $ permette di considerare equivalentemente l'insieme delle derivazioni puntuali dell'algebra dei germi delle funzioni $ C_{p}^{\infty}(\R^{n}) $ e lo spazio tangente a un punto, in quanto questi due insiemi sono isomorfi tra loro tramite $ \varphi $ stessa. Utilizzare le derivazioni è utile perché per alcune varietà differenziabili non esiste una visualizzazione dello spazio tangente.

\begin{theorem}
	L'applicazione $ \varphi $ è un isomorfismo degli spazi vettoriali $ T_{p}(\R^{n}) $ e $ \der_{p}(C_{p}^{\infty}(\R^{n})) $, i.e. tramite $ \varphi $ si ha che
	
	\begin{equation}
		T_{p}(\R^{n}) \stackrel{iso.}{\simeq} \der_{p}(C_{p}^{\infty}(\R^{n}))
	\end{equation}
\end{theorem}

Per dimostrare questo teorema è necessario notare che gli elementi $ D_{i} \in \der_{p}(C_{p}^{\infty}(\R^{n})) $ costituiscono uno spazio vettoriale\footnote{%
	Vedi Esercizio \ref{exer1-8}.%
} con operazioni

\map{+}
	{\der_{p}(C_{p}^{\infty}(\R^{n})) \times \der_{p}(C_{p}^{\infty}(\R^{n}))}{\der_{p}(C_{p}^{\infty}(\R^{n}))}
	{(D_{v}, D_{w})}{D_{v} + D_{w}}
%
\map{\cdot}
	{\R \times \der_{p}(C_{p}^{\infty}(\R^{n}))}{\der_{p}(C_{p}^{\infty}(\R^{n}))}
	{(\lambda, D_{v})}{\lambda D_{v}}

Consideriamo ora la seguente preposizione:

\begin{definition}
	Le operazioni dello spazio vettoriale $ \der_{p}(C_{p}^{\infty}(\R^{n})) $ su $ \R $ (definite sopra) sono $ \R $-lineari e la somma soddisfa la regola di Leibniz, i.e.
	
	\begin{align}
		(D_{v} + D_{w}) ([(f,U)]) &= D_{v}([(f,U)]) + D_{w}([(f,U)]) \\
		D (\lambda [(f,U)]) &= \lambda D ([(f,U)]) = (\lambda D) ([(f,U)]) \\
		(D_{v} + D_{w}) ([(f,U)] \cdot [(g,V)]) &= (D_{v} + D_{w}) ([(f,U)]) \, g(p) + f(p) \, (D_{v} + D_{w}) ([(g,V)])
	\end{align}
	
	per qualsiasi $ D, D_{v}, D_{w} \in \der_{p}(C_{p}^{\infty}(\R^{n})) $ e qualsiasi $ \lambda \in \R $.
\end{definition}

\begin{proof}[Dimostrazione (Proposizione)]
	Per la $ \R $-linearità:
	
	\begin{align}
		\begin{split}
			(D_{v} + D_{w}) (\alpha [(f,U)] + \beta [(g,V)]) &= D_{v} (\alpha [(f,U)] + \beta [(g,V)]) + \\
			& \hspace{15px} + D_{w} (\alpha [(f,U)] + \beta [(g,V)]) \\
			&= \alpha D_{v} ([(f,U)]) + \beta D_{v} ([(g,V)]) + \\
			& \hspace{15px} + \,\alpha D_{w} ([(f,U)]) + \beta D_{w} ([(g,V)]) \\
			&= \alpha (D_{v} + D_{w}) ([(f,U)]) + \beta (D_{v} + D_{w}) ([(g,V)])
		\end{split}
	\end{align}

	per qualsiasi $ D_{v}, D_{w} \in \der_{p}(C_{p}^{\infty}(\R^{n})) $ e $ \alpha,\beta \in \R $. \\
	Per la regola di Leibniz:
	
	\begin{align}
		\begin{split}
			(D_{v} + D_{w}) ([(f,U)] \cdot [(g,V)]) &= D_{v} ([(f,U)] \cdot [(g,V)]) + D_{w} ([(f,U)] \cdot [(g,V)]) \\
			&= D_{v} ([(f,U)]) \, g(p) + f(p) \, D_{v} ([(g,V)]) + \\
			& \hspace{15px} + D_{w} ([(f,U)]) \, g(p) + f(p) \, D_{w} ([(g,V)]) \\
			&= (D_{v} + D_{w}) ([(f,U)]) \, g(p) + f(p) \, (D_{v} + D_{w}) ([(g,V)])
		\end{split}
	\end{align}

	per qualsiasi $ D_{v}, D_{w} \in \der_{p}(C_{p}^{\infty}(\R^{n})) $.
\end{proof}

\begin{proof}
	Per dimostrare che $ \varphi $ sia un isomorfismo è necessario dimostrare che $ \varphi $ sia $ \R $-lineare, iniettiva\footnote{%
		Un'applicazione $ f $ tra due insiemi $ A $ e $ B $ è \textit{iniettiva} se
		%
		\begin{equation*}
			\forall a_{1},a_{2} \in A \mid a_{1} \neq a_{2} \implies f(a_{1}) \neq f(a_{2})
		\end{equation*}%
	} e suriettiva\footnote{%
		Un'applicazione $ f $ tra due insiemi $ A $ e $ B $ è \textit{suriettiva} se
		%
		\begin{equation*}
			\forall b \in B, \, \E a \in A \mid f(a) = b
		\end{equation*}%
	}. \\
	Per l'$ \R $-linearità, sia $ [(f,U)] \in C_{p}^{\infty}(\R^{n}) $, possiamo scrivere
	
	\begin{align}
		\begin{split}
			D_{\alpha v + \beta w} ([(f,U)]) &= D_{\alpha v + \beta w} (f) \\
			&= \sum_{j=1}^{n} \pdv{f}{x^{j}} \, (p) \, (\alpha v^{j} + \beta w^{j}) \\
			&= \alpha \sum_{j=1}^{n} \pdv{f}{x^{j}} \, (p) \, v^{j} + \beta \sum_{j=1}^{n} \pdv{f}{x^{j}} \, (p) \, w^{j} \\
			&= \alpha D_{v} f + \beta D_{w} f \\
			&= \alpha D_{v} ([(f,U)]) + \beta D_{w} ([(f,U)])
		\end{split}
	\end{align}

	per qualsiasi $ \alpha, \beta \in \R $ e $ v, w \in T_{p} (\R^{n}) $. \\
	Da questo si ottiene che l'applicazione $ \varphi $ è $ \R $-lineare:
	
	\begin{align}
		\begin{split}
			\varphi (\alpha v + \beta w) &= D_{\alpha v + \beta w} \\
			&= \alpha D_{v} + \beta D_{w} \\
			&= \alpha \varphi (v) + \beta \varphi (w)
		\end{split}
	\end{align}

	per qualsiasi $ \alpha, \beta \in \R $ e $ v, w \in T_{p} (\R^{n}) $. \\
	Per l'iniettività, consideriamo il \textit{kernel}\footnote{%
		Il \textit{kernel} o nucleo di un'applicazione, indicato con $ \ker $, è l'insieme di tutti e soli gli elementi del dominio che hanno come immagine lo $ 0 $ del codominio. Nel caso considerato ora
		
		\begin{equation*}
			\ker(\varphi) = \left\{ v \in T_{p}(\R^{n}) \st \varphi(v) \equiv D_{v} = 0 \in \der_{p}(C_{p}^{\infty}(\R^{n})) \right\}
		\end{equation*}%
	} di $ \varphi $: se questo contiene solo l'elemento $ 0 $, inteso come
	
	\map{0}
		{C_{p}^{\infty}(\R^{n})}{\R}
		{[(f,U)]}{0}

	i.e. $ \ker(\varphi) = \{0\} $, allora $ \varphi $ è iniettiva\footnote{
		Questo vale perché $ \varphi $ è lineare (vedi Teorema della dimensione).%
	}. \\
	Siccome $ 0 $ associa un qualunque germe liscio $ [(f,U)] $ sempre a $ 0 \in \R $, possiamo scegliere l'applicazione
	
	\map{x^{j}}
		{\R^{n}}{\R}
		{(x^{1}, \dots, x^{n})}{x^{j}}

	per qualsiasi $ j=1,\dots,n $, la quale è una proiezione liscia dunque il germe che la contiene è liscio, i.e. $ [(x^{j},\R^{n})] \in C_{p}^{\infty}(\R^{n}) $. A questo punto
	
	\begin{align}
		\begin{split}
			0([(x^{j},\R^{n})]) &= D_{v} ([(x^{j},\R^{n})]) \\
			&= D_{v} (x^{j}) \\
			&= \sum_{i=1}^{n} \pdv{x^{j}}{x^{i}} \, (p) \, v^{i} \\
			&= \sum_{i=1}^{n} \delta^{ij} \, v^{i} \\
			&= v^{j}
		\end{split}
	\end{align}

	perciò
	
	\begin{equation}
		\begin{cases}
			0([(f,U)]) = 0 \in \R, & \forall [(f,U)] \in C_{p}^{\infty}(\R^{n}) \\ \\
			0([(x^{j},\R)]) = v^{j}
		\end{cases} %
		 \implies v^{j} = 0 \qcomma \forall j=1,\dots,n
	\end{equation}
	
	da cui
	
	\begin{equation}
		v \in \ker(\varphi) \iff v = 0 \in T_{p}(\R^{n})
	\end{equation}
	
	perciò $ \varphi $ è iniettiva. \\
	La suriettività implica che se si fissa una qualunque derivazione puntuale esiste un vettore nello spazio tangente che mandato tramite $ \varphi $ dà quella derivazione: in simboli

	\begin{equation}
		\forall D \in \der_{p} (C_{p}^{\infty}(\R^{n})), \, \E v \in T_{p}(\R^{n}) \, \mid \, \varphi(v) = D
	\end{equation}
	
	dove in generale $ \varphi(v) = D_{v} $, dunque dobbiamo trovare un $ v $ tale che faccia coincidere $ D = D_{v} $. \\
	Prima di farlo, enunciamo il seguente lemma:

	\begin{lemma}[Derivazione di costante]
		Siano $ D \in \der_{p} (C_{p}^{\infty}(\R^{n})) $ e la funzione costante
		
		\map{c}
			{\R^{n}}{\R}
			{x}{c}
	
		allora $ D([(c,\R^{n})]) = 0 $.
	\end{lemma}

	\begin{proof}[Dimostrazione (lemma)]
		\begin{align}
			\begin{split}
				D([(c,\R^{n})]) &= D([(1 c,\R^{n})]) \\
				&= c \, D([(1,\R^{n})]) \\
				&= c \, D([(1 \cdot 1,\R^{n})]) \\
				&= c \, ( D([(1,\R^{n})]) \, 1 + 1 \, D([(1,\R^{n})]) ) \\
				&= 2 c \, D([(1,\R^{n})]) \\
				&= 0
			\end{split}
		\end{align}
	\end{proof}

	A questo punto, due applicazioni sono uguali se e solo se coincidono per ogni punto del dominio, i.e.
	
	\begin{equation}
		D_{v} = D \iff D_{v}([(f,U)]) = D([(f,U)]) \qcomma \forall ([(f,U)]) \in (C_{p}^{\infty}(\R^{n}))
	\end{equation}

	Prendendo un dominio $ U $ stellato rispetto al punto $ p $, per il teorema di Taylor con resto
	
	\begin{equation}
		f(x) = f(p) + \sum_{i=1}^{n} (x^{i}-p^{i}) \, g_{i}(x) \qcomma \forall x \in U
	\end{equation}

	con
	
	\begin{equation}
		\left. g_{i} \in C^{\infty}(U) \st g_{i}(p) = \dfrac{\partial f}{\partial x^{i}} (p) \right. \qcomma i=1,\dots,n
	\end{equation}

	Sia $ v = (v^{1},\dots,v^{n}) \in T_{p (\R^{n})} $ definito come $ v^{j} = D([(x^{j},\R^{n})]) $ per $ j=1,\dots,n $. \\
	Ora applichiamo $ D $ a un qualunque germe liscio $ [(f,U)] $
	
	\begin{align}
		\begin{split}
			D ([(f,U)]) &= \cancelto{0}{D ([(f(p),\R^{n})])} + D \left(\left[\left( \sum_{i=1}^{n} (x^{i}-p^{i}) \, g_{i}(x) , U \right)\right]\right) \\
			&= \sum_{i=1}^{n} D ([( (x^{i}-p^{i}) \, g_{i}(x) , U )]) \\
			&= \sum_{i=1}^{n} ( D ([( (x^{i}-p^{i}), U )]) \, g_{i}(p) + \cancelto{0}{(p^{i}-p^{i})} \, D ([( g_{i}(x), U )]) ) \\
			&= \sum_{i=1}^{n} ( D ([( x^{i}, U )]) - \cancelto{0}{D ([( p^{i}, U )])} ) g_{i}(p) \\
			&= \sum_{i=1}^{n} D ([( x^{i}, U )]) \, \pdv{f}{x^{i}} \, (p) \\
			&= \sum_{i=1}^{n} \pdv{f}{x^{i}} \, (p) \, v^{i} \\
			&= D_{v} f \\
			&= D_{v} ([(f,U)])
		\end{split}
	\end{align}

	dunque $ D = D_{f} $ e perciò $ \varphi $ è anche suriettiva.
\end{proof}

Date queste proprietà di $ \varphi $, questa applicazione è un isomorfismo tra $ T_{p}(\R^{n}) $ e $ \der_{p}(C_{p}^{\infty}(\R^{n})) $, i.e.

\begin{equation}
	T_{p}(\R^{n}) \stackrel{iso.}{\simeq} \der_{p}(C_{p}^{\infty}(\R^{n}))
\end{equation}

\begin{corollary}
	\begin{equation}
		\dim ( T_{p}(\R^{n}) ) = n = \dim ( \der_{p}(C_{p}^{\infty}(\R^{n})) )
	\end{equation}
\end{corollary}

\subsection{Base canonica per $ \der_{p}(C_{p}^{\infty}(\R^{n})) $}

L'insieme delle $ n $-uple

\begin{equation}
	\left( \eval{ \pdv{x^{1}} }_{p}, \dots, \eval{ \pdv{x^{n}} }_{p}  \right)
\end{equation}

i cui elementi sono definiti come

\begin{equation}
	\eval{ \pdv{x^{j}} }_{p} ([(f,U)]) = \pdv{f}{x^{j}} \, (p) \qcomma \forall p \in U, \, j = 1, \dots, n
\end{equation}

forma una base per lo spazio $ \der_{p}(C_{p}^{\infty}(\R^{n})) $.

\begin{proof}
	Essendo $ T_{p}(\R^{n}) \simeq \der_{p}(C_{p}^{\infty}(\R^{n})) $, da cui
	
	\begin{equation}
		\dim ( \der_{p}(C_{p}^{\infty}(\R^{n})) ) = n
	\end{equation}
	
	se $ (e_{1},\dots,e_{n}) $ è la base canonica\footnote{%
		Con $ (e_{j})_{k} = \delta_{jk} $, e.g. $ e_{3} = (0,0,1,0,\dots,0) $.%
	} di $ T_{p}(\R^{n}) $, si ha che

	\begin{equation}
		\varphi(e_{i}) = D_{e_{i}} \qcomma \forall i=1,\dots,n
	\end{equation}

	i.e. un isomorfismo porta elementi di base in altrettanti elementi di base. \\
	Applicando questo a una qualunque funzione $ f \in C^{\infty} (\R^{n}) $
	
	\begin{equation}
		D_{e_{i}} (f) = \sum_{j=1}^{n} \dfrac{\partial f}{\partial x^{j}} (p) \, (e_{i})_{j} = \sum_{j=1}^{n} \dfrac{\partial f}{\partial x^{j}} (p) \, \delta_{ij} = \dfrac{\partial f}{\partial x^{i}} (p)
	\end{equation}
\end{proof}

\section{Campi di vettori su aperti di $ \R^{n} $}

Sia un aperto $ U \subset \R^{n} $ con $ n \geqslant 1 $, un \textit{campo di vettori} su $ U $ è un'applicazione

\map{X}
	{U}{\bigsqcup_{p \in U} T_{p}(\R^{n})}
	{p}{X_{p}}

dove il codominio è l'\textit{unione disgiunta}\footnote{%
	L'unione disgiunta equivale a un'unione in cui ogni insieme ha un indice diverso, e.g. l'insieme non connesso $ (0,1) \sqcup (0,1) $ è diverso da $ (0,1) \cup (0,1) = (0,1) $.%
} degli spazi di vettori tangenti in ogni punto di $ U $; inoltre $ T_{p}(\R^{n}) = T_{p}(U) $ in quanto le due algebre seguenti coincidono $ C_{p}^{\infty}(\R^{n}) = C_{p}^{\infty}(U) $ perché i germi delle funzioni sono definiti localmente, quindi non dipendono dall'aperto considerato. \\
Un elemento del campo di vettori può essere scritto in funzione della base canonica di $ T_{p}(\R^{n}) $

\begin{equation}
	X_{p} = \sum_{i=1}^{n} a^{i}(p) \eval{ \pdv{x^{i}} }_{p}
\end{equation}

dove $ a^{i}(p) \in \R $ con $ i=1,\dots,n $. In modo naturale, l'elemento $ X_{p} $ si identifica con l'$ n $-upla $ X_{p} = (a^{1}(p),\dots,a^{n}(p)) $ in quanto $ T_{p}(\R^{n}) \simeq \R^{n} $. \\
La notazione che indica che un elemento di una base genera uno spazio è la seguente:

\begin{equation}
	\ev{ \eval{ \pdv{x^{i}} }_{p} } = T_{p}(\R^{n})
\end{equation}

Il campo di vettori $ X $ (senza la valutazione in un punto $ p $) si scrive come

\begin{equation}
	X = \sum_{i=1}^{n} a^{i} \pdv{x^{i}}
\end{equation}

dove ora $ a^{i} $ è una funzione $ a^{i} : U \to \R $.

\subsection{Campi di vettori lisci}

Un campo di vettori

\begin{equation}
	X = \sum_{i=1}^{n} a^{i} \pdv{x^{i}}
\end{equation}

è $ C^{\infty} (U) $ (liscio o differenziabile) se le funzioni $ a^{i} $ sono lisce, i.e. $ a^{i} \in C^{\infty}(U) $ per qualsiasi $ i=1,\dots,n $. \\
L'insieme dei campi di vettori che rispettano questa prescrizione è chiamato $ \chi(U) $, i.e.

\begin{equation}
	\chi(U) = \left\{ X : U \to \bigsqcup_{p \in U} T_{p}(\R^{n}), \, X(p) = X_{p} = \sum_{i=1}^{n} a^{i}(p) \eval{ \pdv{x^{i}} }_{p} \st U \subset \R^{n}, \, a^{i} \in C^{\infty}(U) \right\}
\end{equation}

\subsubsection{\textit{Esempi}}

\paragraph{1)}

Il campo di vettori seguente è liscio perché qualunque derivata delle sue componenti non annulla mai il denominatore in quanto l'origine non è compresa nel dominio

\sbs{0.5}{%
			\map{X}
			{\R^{2} \setminus \{(0,0)\}}{T_{(x,y)}(\R^{2})}
			{(x,y)}{%
					- \dfrac{x}{\sqrt{x^{2}+y^{2}}} \pdv{x} - \dfrac{y}{\sqrt{x^{2}+y^{2}}} \pdv{y} \\
					& \hspace{15px} = \left( - \dfrac{x}{\sqrt{x^{2}+y^{2}}}, - \dfrac{y}{\sqrt{x^{2}+y^{2}}} \right)%
					}
		}
	{0.5}{%
			\img{0.75}{img2}
		}

\paragraph{2)}

Per lo stesso motivo dell'esempio precedente, il campo di vettori seguente è liscio

\sbs{0.5}{%
			\map{X}
				{\R^{2} \setminus \{(0,0)\}}{T_{(x,y)}(\R^{2})}
				{(x,y)}{%
						- \dfrac{y}{\sqrt{x^{2}+y^{2}}} \pdv{x} - \dfrac{x}{\sqrt{x^{2}+y^{2}}} \pdv{y} \\
						& \hspace{15px} = \left( - \dfrac{y}{\sqrt{x^{2}+y^{2}}}, - \dfrac{x}{\sqrt{x^{2}+y^{2}}} \right)%
						}
		}
	{0.5}{%
		\img{0.75}{img3}
	}

\subsection{Operazioni in $ \chi(U) $}

Si può definire la somma in $ \chi(U) $ come

\begin{equation}
	(X+Y)_{p} \doteq X_{p} + Y_{p} %
	\qcomma X,Y \in \chi(U), \, p \in U
\end{equation}

questo significa che, presi due campi di vettori su $ U $

\begin{equation}
	X = \sum_{i=1}^{n} a^{i} \pdv{x^{i}} %
	\qcomma Y = \sum_{i=1}^{n} b^{i} \pdv{x^{i}} %
	\qcomma a^{i},b^{i} \in C^{\infty}(U), \, \forall i = 1, \dots, n
\end{equation}

allora

\begin{equation}
	X + Y = \sum_{i=1}^{n} (a^{i}+b^{i}) \pdv{x^{i}} %
	\qcomma a^{i} + b^{i} \in C^{\infty}(U), \, \forall i = 1, \dots, n
\end{equation}

Si può definire anche la moltiplicazione per scalari come

\begin{equation}
	(\lambda X)_{p} \doteq \lambda X_{p} \qcomma \forall X \in \chi(U), \, \forall \lambda \in \R, \, \forall p \in U
\end{equation}

questo significa che, preso

\begin{equation}
	X = \sum_{i=1}^{n} a^{i} \pdv{x^{i}} %
	\qcomma a^{i} \in C^{\infty}(U), \, \forall i = 1, \dots, n
\end{equation}

allora

\begin{equation}
	\lambda X = \sum_{i=1}^{n} (\lambda a^{i}) \pdv{x^{i}} %
	\qcomma \lambda a^{i} \in C^{\infty}(U), \, \forall i = 1, \dots, n
\end{equation}

L'ultima operazione è quella di moltiplicazione di un campo di vettori per un'altra funzione

\begin{equation}
	(f X)_{p} \doteq f(p) X_{p} %
	\qcomma X \in \chi(U), \, f \in C^{\infty}(U)
\end{equation}

questo significa che

\begin{equation}
	f X = \sum_{i=1}^{n} (f a^{i}) \pdv{x^{i}} %
	\qcomma f a^{i} \in C^{\infty}(U), \, \forall i = 1, \dots, n
\end{equation}

Le prime due operazioni dotano l'insieme di $ \chi(U) $ della proprietà di spazio vettoriale.

\subsection{$ \chi(U) $ come $ C^{\infty}(U) $-modulo}

\subsubsection{$ \R $-modulo sinistro}

Sia $ R $ un anello commutativo unitario, un gruppo abeliano $ (A,+) $ è detto $ R $\textit{-modulo sinistro} se esiste un'applicazione
%
\map{\cdot}
	{R \times A}{A}
	{(r,a)}{r \cdot a}
%
tale che

\begin{equation}
	\begin{cases}
		1_{R} \cdot a = a \\
		r \cdot (s \cdot a) = (r s) \cdot a \\
		(r+s) \cdot a = r \cdot a + s \cdot a \\
		r \cdot (a+b) = r \cdot a + r \cdot b
	\end{cases} %
	\qquad \forall r, s \in R, \, \forall a, b \in A
\end{equation}

Queste proprietà valgono solo da \textit{sinistra}, potrebbero non valere se calcolate da destra.

\subsubsection{$ \R $-modulo destro}

Sia $ R $ un anello commutativo unitario, un gruppo abeliano $ (A,+) $ è detto $ R $\textit{-modulo destro} se esiste un'applicazione
%
\map{*}
	{A \times R}{A}
	{(a,r)}{a * r}
%	
tale che

\begin{equation}
	\begin{cases}
		a * 1_{R} = a \\
		(a*r)*s = a*(r s) \\
		a*(r+s) = a*r + a*s \\
		(a+b)*r = a*r + b*r
	\end{cases} %
	\qquad \forall r, s \in R, \, \forall a, b \in A
\end{equation}

Queste proprietà valgono solo da \textit{destra}, potrebbero non valere se calcolate da sinistra. \\ \\
%
Tramite queste definizioni, definiamo $ (A,+) $ un $ R $-modulo se è sia un $ R $-modulo sinistro che destro, i.e. $ \cdot \equiv * $.

\begin{remark}
	Se un gruppo $ A $ è un $ R $-modulo ed $ R $ è un campo $ \K $, allora $ A $ è uno spazio vettoriale in $ \K $.
\end{remark}

\subsubsection{Caso di $ \chi(U) $}

Essendo $ C^{\infty}(U) $ un anello commutativo unitario, per l'insieme dei campi di vettori lisci su $ U $ vale il seguente teorema:

\begin{theorem}\label{thm:chi-mod}
	$ (\chi(U),+) $ è un $ C^{\infty}(U) $-modulo.
\end{theorem}

\begin{proof}
	Per dimostrare che il gruppo abeliano $ (\chi(U),+) $ sia un $ C^{\infty}(U) $-modulo, è necessario dimostrare che $ (\chi(U),+) $ sia un $ C^{\infty}(U) $-modulo sinistro e destro per la moltiplicazione di un campo di vettori per una funzione
	
	\map{\cdot}
		{C^{\infty}(U) \times \chi(U)}{\chi(U)}
		{(f,X)}{f X}
	
	Devono dunque essere verificate le seguenti proprietà sia a sinistra che a destra:

	\begin{equation}
		\begin{cases}
			1_{C^{\infty}(U)} X = X \\
			f (g X) = (f g) X \\
			f (X+Y) = f X + f Y \\
			(f+g) X = f X + g X
		\end{cases} %
		\qquad \forall f,g \in C^{\infty}(U), \, \forall X,Y \in \chi(U)
	\end{equation}

	Siccome la moltiplicazione per funzione è commutativa\footnote{%
		Nonostante ciò, scriveremo la funzione sempre a sinistra dei campi, per notazione e per evitare di confonderla con la derivata di funzione rispetto a un campo di vettori (vedi sottosezione successiva).%
	}, è sufficiente dimostrare che $ (\chi(U),+) $ sia un $ C^{\infty}(U) $-modulo sinistro (o destro) per dimostrare che sia $ C^{\infty}(U) $-modulo\footnote{%
		Vedi Esercizio \ref{exer1-9}.%
	}.
\end{proof}

\subsection{Derivata di funzione rispetto a un campo di vettori}

I campi di vettori permettono di derivare funzioni: la loro azione è equivalente alla derivata direzionale di una funzione rispetto a un vettore. \\
Siano un campo di vettori liscio $ X \in \chi(U) $ con $ U \subset \R^{n} $ aperto e una funzione liscia $ f \in C^{\infty}(U) $. Definiamo la derivata della funzione $ f $ rispetto al campo di vettori $ X $ come $ X f \in C^{\infty}(U) $. La derivata puntuale è definita come

\begin{equation}
	(X f) (p) = X_{p} f \qcomma p \in U \subset \R^{n}
\end{equation}

Preso un campo

\begin{equation}
	X = \sum_{i=1}^{n} a^{i} \pdv{x^{i}}
\end{equation}

allora

\begin{equation}
	(X f) (p) = \left( \left( \sum_{i=1}^{n} a^{i} \pdv{x^{i}} \right) f \right)_{p} %
	= \sum_{i=1}^{n} a^{i} (p) \, \pdv{f}{x^{i}} \, (p)
\end{equation}

perciò

\map{X f}
	{U}{\R}
	{p}{\sum_{i=1}^{n} a^{i} (p) \, \pdv{f}{x^{i}} \, (p)}

Questa applicazione è $ C^{\infty}(U) $ perché lo è $ (X f) (p) $, la quale lo è a sua volta perché $ f \in C^{\infty}(U) $ e $ X \in \chi(U) $ in quanto $ a^{i} \in C^{\infty}(U) $. \\
Possiamo considerare l'applicazione

\map{X}
	{C^{\infty}(U)}{C^{\infty}(U)}
	{f}{X f}

ricordando che $ C^{\infty}(U) $, oltre a essere un anello commutativo unitario, è un'algebra sui reali, perciò l'applicazione $ X $ è $ \R $-lineare. Inoltre, siccome $ X_{p} \in \der_{p}(C_{p}^{\infty}(\R^{n})) $, i campi di vettori valutati in un punto soddisfano la regola di Leibniz:

\begin{align}
	X (f g) (p) = X_{p} (f g) = (X_{p} f) \, g(p) + f(p) \, (X_{p} g)
\end{align}

perciò anche l'applicazione $ X $ soddisfa la regola di Leibniz:

\begin{equation}
	X (f g) = (X f) \, g + f \, (X g)
\end{equation}

\subsubsection{Derivazione di un'algebra}

Sia $ A $ un'algebra su campo
	\footnote{%
		Ricordiamo che un'algebra $ A $ su campo $ \K $ è una coppia $ (V, \cdot) $ dove $ V $ è uno spazio vettoriale e l'operazione
		%
		\maps{\cdot}
			{A \times A}{A}
			{(a,b)}{a \cdot b}
		%
		soddisfa le proprietà
		%
		\begin{equation*}
			\begin{cases}
				\begin{split}
					(a + b) \cdot c = a \cdot c + b \cdot c \\
					c \cdot (a + b) = c \cdot a + c \cdot b
				\end{split} & \text{distributività} \\
				\lambda (a \cdot b) = (\lambda a) \cdot b = a \cdot (\lambda b) & \text{omogeneità}
			\end{cases} %
			\qquad \forall a,b,c \in A, \, \forall \lambda \in \K
		\end{equation*} %
} $ \K $, un'applicazione $ D : A \to A $ che sia $ \K $-lineare e tale che soddisfi la regola di Leibniz

\begin{equation}
	D (a \cdot b) = (D a) \cdot b + a \cdot (D b) \qcomma \forall a, b \in A
\end{equation}

è chiamata \textit{derivazione dell'algebra} $ A $. L'insieme di tutte le derivazioni di un'algebra $ A $ viene indicato come $ \der(A) $
\footnote{%
	Vedi Esercizi \ref{exer1-10} e \ref{exer1-11}.%
}.

\subsection{Campo di vettori liscio come derivazione dell'algebra $ C^{\infty}(U) $}

Possiamo vedere un campo di vettori come una derivazione di un'algebra, quindi definiamo un'applicazione

\map{\varphi}
	{\chi(U)}{\der(C^{\infty}(U))}
	{X}{\varphi(X)}

da cui

\begin{equation}
	\varphi(X) (f) \doteq X f \qcomma f \in C^{\infty}(U)
\end{equation}

Sia $ \chi(U) $ che $ \der(C^{\infty}(U)) $ sono $ C^{\infty}(U) $-moduli tramite l'applicazione

\map{\cdot}
	{C^{\infty}(U) \times \der(C^{\infty}(U))}{\der(C^{\infty}(U))}
	{(f,D)}{f D}

per la quale vale

\begin{equation}
	(f D) (g) = f (D g) \qcomma \forall g \in C^{\infty}(U)
\end{equation}

Inoltre $ \varphi $ è anche $ C^{\infty}(U) $-lineare:

\begin{equation}
	\varphi(f X + g Y) = f \, \varphi(X) + g \, \varphi(Y) \qcomma \forall f, g \in C^{\infty}(U), \, \forall X, Y \in \chi(U)
\end{equation}

Dimostreremo per le varietà differenziabili\footnote{%
	Vedi Sotto-sottosezione \ref{ss-sec:iso-chi-der-man}.%
} che $ \varphi $ è un isomorfismo di $ C^{\infty}(U) $-moduli, i.e. $ \chi(U) \simeq \der(C^{\infty}(U)) $. \\
Tramite questo isomorfismo, si possono identificare i campi di vettori lisci con le derivazioni dell'algebra delle funzioni lisce, analogamente a come lo spazio tangente a un punto di $ \R^{n} $ si può identificare con le derivazioni puntuali dell'algebra dei germi delle funzioni in quel punto, i.e. $ T_{p}(\R^{n}) \simeq \der_{p}(C_{p}^{\infty}(\R^{n})) $.


%

\chapter{Differential manifolds}
%\section{Varietà topologiche e differenziabili}

\subsection{Spazi topologici}

Una topologia $ \tau $ su un insieme non vuoto $ X $ è una classe non vuota di sottoinsiemi di $ X $, i.e. $ \tau \subset \ps(X) $ (dove $ \ps(X) $ indica l'insieme delle parti di $ X $), per la quale valgono:

\begin{enumerate}
	\item $ \emptyset, X \in \tau $
	
	\item L'unione di un numero qualunque di insiemi di $ \tau $ appartiene a $ \tau $
	
	\item L'intersezione di due insiemi qualunque di $ \tau $ appartiene a $ \tau $
\end{enumerate}

La coppia $ (X, \tau) $ è detta \textit{spazio topologico}; in questo testo, indicheremo con $ X $ lo spazio topologico $ (X, \tau) $, sottintendendo la topologia $ \tau $ quando questa è chiara dal contesto.

\subsection{Varietà topologiche}

Sia $ X $ uno spazio topologico, diremo che $ X $ è una \textit{varietà topologica} se sono soddisfatte le seguenti condizioni:

\begin{itemize}
	\item $ X $ è $ T_{2} $ o \textit{di Hausdorff}, i.e. dati due punti distinti di $ X $ è sempre possibile trovare due intorni disgiunti che contengano i due punti;
	
	\item $ X $ è $ N_{2} $ o \textit{a base numerabile}, i.e. esiste una base per la topologia di $ X $ che sia numerabile\footnote{%
		Un insieme è numerabile se possiede la stessa cardinalità dell'insieme dei numeri naturali $ \N $; in notazione per una base $ \B $, $ \# \B = \# \N = \aleph_{0} $.%
	};
	
	\item $ X $ è \textit{localmente euclideo}, i.e. 
		\begin{equation}
			\forall p \in X, \, \E \text{aperto } U \ni p \wedge \E \varphi : U \to \R^{n} \, \mid \, \varphi : U \to \varphi(U) \subset \R^{n} \text{ omeomorfismo}
		\end{equation}
		dunque localmente ogni aperto di $ X $ è omeomorfo a un aperto di $ \R^{n} $.
\end{itemize}

La coppia $ (U,\varphi) $ è chiamata \textit{carta intorno a} $ p $; se $ \varphi(p) = 0 $, la carta è detta \textit{centrata} in $ p $. \\
Diremo che, se lo spazio topologico è localmente omeomorfo a $ \R^{n} $, la sua dimensione è pari a quella di quest'ultimo, i.e. $ \dim (X) = n $.

\sbs{0.55}{%
			\begin{remark}
				Possiamo sempre assumere che $ \varphi(U) \equiv \R^{n} $. Questo perché all'interno dell'immagine di $ U $ in $ \R^{n} $ (i.e. $ \varphi(U) $) è sempre possibile trovare una palla $ B_{\delta}(0) $ centrata nell'origine con raggio $ \delta $ opportuno per la quale esista sempre un diffeomorfismo (e quindi un omeomorfismo) $ h : B_{\delta}(0) \to \R^{n} $. \\
				Sia la carta $ (V,\psi) $ con $ V = \varphi^{-1}(B_{\delta}(0)) $ aperto e $ \psi = h \circ \eval{\varphi}_{V} $ in cui $ \psi : V \to \R^{n} $ è un omeomorfismo: da questo si ottiene che, cambiando $ U $ e $ \varphi $ se necessario, si può sempre assumere che l'immagine della carta sia $ \R^{n} $, dunque $ \psi(V) = \R^{n} $.
			\end{remark}
		}
	{0.4}{%
			\img{1}{img4}
		}

Il vantaggio di avere una carta su uno spazio topologico è quello di poterlo ricondurre, almeno localmente, a $ \R^{n} $.

\begin{remark}
	Se $ (U,\varphi) $ è una carta su $ X $ centrata in $ p \in U $, allora
	
	\begin{gather}
		\varphi(p) = (0, \dots, 0) \in \R^{n} \\
		\varphi(q) = (\varphi^{1}(q), \dots, \varphi^{n}(q)) \in \R^{n} \qcomma \forall q \in U
	\end{gather}

	dove $ \varphi^{i} : U \to \R $.
\end{remark}

\begin{remark}
	Il teorema di invarianza topologica della dimensione, i.e.
	
	\begin{equation}
		U \simeq V \implies \dim V = \dim U
	\end{equation}

	rende ben definita la dimensione di una varietà topologica.
\end{remark}

È importante precisare che le tre condizioni che determinano uno spazio topologico sono indipendenti tra di loro, dunque due di queste non implicano necessariamente la terza.

\subsubsection{Unione disgiunta di spazi topologici}

Siano una famiglia di spazi topologici $ \{A_{j}\}_{j \in J} $ e l'unione disgiunta

\begin{equation}
		A = \bigsqcup_{j \in J} A_{j} \equiv \bigcup_{j \in J} A_{j} \times \{j\}
\end{equation}

Definiamo la topologia dell'unione disgiunta\footnote{%
	Vedi Esercizio \ref{BONUS2-1}.%
} su $ A $ dichiarando $ U $ aperto in $ A $ se e solo se $ U \cap A_{j} $ è aperto in $ A_{j} $ per qualsiasi $ j \in J $.

\subsubsection{\textit{Esempi}}

\paragraph{1) Spazio $ N_{2} + T_{2} $ ma non localmente euclideo} 

Sia l'insieme degli assi coordinati

\begin{equation}
	X = \{ (x,y) \in \R^{2} \mid x = 0 \lor y = 0 \} \cup \{ (0,0) \}
\end{equation}

questo spazio è $ N_{2} + T_{2} $ in quanto sottospazio di $ \R^{2} $ il quale ha queste proprietà\footnote{%
	Le proprietà topologiche $ N_{2} $ e $ T_{2} $ sono \textit{ereditarie} e dunque un sottospazio di uno spazio topologico che le possiede avrà anch'esso queste proprietà.%
}, ma non è localmente euclideo nell'origine. Perché lo sia, deve essere possibile prendere un intorno dell'origine $ U $ il quale sia omeomorfo a $ \R^{n} $, ma se si toglie un punto da entrambi gli spazi, questi dovrebbero essere ancora omeomorfi. Nello specifico togliamo l'origine (supponendo che l'origine di $ U $ vada nell'origine di $ \R^{n} $) quindi dovremmo avere $ U \setminus \{(0,0)\} \simeq \R^{n} \setminus \{0\} $:

\sbs{0.5}{%
			\begin{itemize}
				\setlength\itemsep{2em}
				
				\item per $ n=1 $, lo spazio $ \R^{n} \setminus \{0\} $ è composto da due componenti connesse
				
				\item per $ n \geqslant 2 $, $ \R^{n} \setminus \{0\} $ è connesso
			\end{itemize}
		}
	{0.45}{%
			\img{1}{img5}
		}
	
Questo implica che $ \R^{n} \setminus \{0\} $ non è omeomorfo a $ U \setminus \{(0,0)\} $ in quanto quest'ultimo è formato da quattro componenti connesse e, dalla topologia, un omeomorfismo tra due spazi induce una bigezione tra le parti connesse.

\paragraph{2) Spazio $ T_{2} $ e localmente euclideo ma non $ N_{2} $}

Consideriamo l'unione disgiunta

\begin{equation}
	X = \bigsqcup (0,1) \equiv \bigcup_{j \in J} (0,1) \times \{j\}
\end{equation}

con $ J $ non numerabile. \\
Questo spazio non è $ N_{2} $ perché non esiste una base numerabile per infinite coppie non numerabili. \\
Consideriamo due punti qualunque di $ X $: questi stanno entrambi nello stesso $ (0,1) $, il quale è di Hausdorff, oppure stanno in due intervalli diversi, i quali sono due intervalli disgiunti che contengono i due punti; in entrambi casi, $ X $ è $ T_{2} $. \\
Lo spazio $ X $ è localmente euclideo perché un suo punto $ p $ sarà contenuto in uno degli intervalli $ (0,1) \times \{j_{0}\} $ il quale è omeomorfo a $ \R $.

\paragraph{3) Spazio $ N_{2} $ e localmente euclideo ma non $ T_{2} $ (retta con due origini)}

Sia l'insieme

\begin{equation}
	S = \R \setminus \{0\} \cup \{A\} \cup \{B\} \qcomma A,B \notin \R
\end{equation}

Per associare a questo spazio una topologia, possiamo trovare una famiglia di sottoinsiemi che faccia da base per la topologia (generata da questa famiglia). Siano $ c,d \in \R^{+} $ e l'insieme

\begin{equation}
	\B = \left\{ (a,b) \subset S \st a<b, \, 0 \notin (a,b) \right\} \cup \left\{ I_{A}(c,d) \right\} \cup \left\{ I_{B}(c,d) \right\} \subset \ps(S)
\end{equation}

dove

\begin{gather}
	I_{A}(c,d) = (-c,0) \cup \{A\} \cup (0,d) \\
	I_{B}(c,d) = (-c,0) \cup \{B\} \cup (0,d)
\end{gather}

Se l'intersezione di elementi di $ \B $ può essere scritta come unioni di elementi di $ \B $, allora esiste un'unica topologia $ \tau_{\B} $ generata da $ \B $ che ha questa come base. \\
L'intersezione tra intervalli di tipo $ (a,b) $ produce intervalli dello stesso tipo; quella tra intervalli $ (a,b) $ e $ I_{A}(c,d) $ produce ancora un intervallo euclideo che appartiene alla famiglia; per i casi

\begin{equation}
	I_{K}(c,d) \cap I_{K}(c',d') = %
		\begin{cases}
			I_{K}(c',d), & c' > c \wedge d > d' \\
			I_{K}(c,d'), & c > c \wedge d' > d \\
			I_{K}(c',d'), & c' > c \wedge d' > d \\
			I_{K}(c,d), & c > c \wedge d > d'
		\end{cases} %
	\qq{per} K = A, B
\end{equation}

\begin{equation}
	I_{A}(c,d) \cap I_{B}(c',d') = %
	\begin{cases}
		(- c',0) \cup (0,d), & c' > c \wedge d > d' \\
		(- c,0) \cup (0,d'), & c > c \wedge d' > d \\
		(- c',0) \cup (0,d'), & c' > c \wedge d' > d \\
		(- c,0) \cup (0,d), & c > c \wedge d > d'
	\end{cases}
\end{equation}

il risultato è l'unione di elementi di $ \B $. \\
Vogliamo dunque mostrare che lo spazio topologico $ (S,\tau_{\B}) $ sia $ N_{2} $ e localmente euclideo ma non $ T_{2} $. \\
Se consideriamo la topologia di $ S $ ristretta\footnote{%
	$ \R \setminus \{0\} \subset S $ perché una base per $ \tau_{eucl.} $ è $ \B_{\R \setminus \{0\}} = \left\{ (a,b) \st a<b, \, 0 \notin (a,b) \right\} \subset \B $.%
} a $ \R \setminus \{0\} \subset S $, otteniamo che $ \tau_{\eval{\B}_{\R \setminus \{0\}}} = \tau_{eucl.} $. Possiamo quindi trovare una famiglia numerabile tale che ogni intervallo della forma $ (a,b) $ possa essere scritto come unione di questa famiglia numerabile proprio perché $ \R \setminus \{0\} $ è $ N_{2} $, i.e.

\begin{equation}
	(a,b) = \bigcup_{r,s \in \Q} (r,s)
\end{equation}

dove $ \Q $ indica i numeri razionali\footnote{%
	Ricordiamo che $ \# \Q = \# \N = \aleph_{0} $.%
} e $ 0 \notin (r,s) $; analogamente

\begin{equation}
	I_{K}(c,d) = \bigcup_{r,s \in \Q} I_{K}(r,s) \qcomma K = A, B
\end{equation}

A questo punto, prendendo $ r, s \in \Q^{+} $, possiamo considerare la famiglia

\begin{equation}
	\mathcal{C} = \left\{ (t,u) \st t,u \in \Q \, \wedge \, t < u \, \wedge \, 0 \notin (r,s) \right\} \cup \left\{ I_{A}(r,s) \right\} \cup \left\{ I_{B}(r,s) \right\}
\end{equation}

la quale è una base numerabile per $ S $ e dunque $ S $ è $ N_{2} $. \\
Per mostrare che $ S $ sia localmente euclideo costruiamo l'applicazione

\map{\varphi}
	{I_{K}(c,d)}{(-c,d) \subset \R}
	{x}{x \\
	K &\mapsto 0}

la quale è un omeomorfismo, perché è una bigezione continua. Per la continuità, prendiamo un intervallo $ (e,f) \subset (-c,d) $ con $ 0 \notin (e,f) $ la cui controimmagine è sé stesso, i.e. $ \varphi^{-1}((e,f)) = (e,f) \in \tau_{\B} $; se invece consideriamo $ (-e,f) $ con $ e,f>0 $, la sua controimmagine sarà $ \varphi^{-1}((-e,f)) = I_{K}(e,f) \in \tau_{\B} $ il quale è un aperto. Per mostrare che $ \varphi $ sia effettivamente continua dobbiamo verificare che la sua inversa $ \varphi^{-1} $ sia anch'essa continua, ma questo è equivalente a mostrare che $ \varphi $ sia aperta\footnote{%
	Un'applicazione aperta associa aperti ad aperti.%
}: sia che si prenda un intervallo $ (e,f) $ oppure $ I_{K}(e,f) $, si ottiene comunque come immagine un aperto, perciò è continua e lo spazio è localmente euclideo. \\
Infine, troviamo che $ S $ non è $ T_{2} $ perché due aperti di questo spazio che contengono $ A $ e $ B $ rispettivamente si devono necessariamente intersecare, i.e. vale sempre $ I_{A}(c,d) \cap I_{B}(e,f) \neq \emptyset $ per qualsiasi $ c, d, e, f \in \R $.

\subsection{Atlanti}

\subsubsection{Atlanti topologici}

Sia $ M $ uno spazio topologico, un \textit{atlante topologico su} $ M $ è una famiglia di carte $ \{(U_{\alpha},\varphi_{\alpha})\}_{\alpha \in A} $ tali che $ U_{\alpha} $ siano sottoinsiemi aperti di $ M $, $ \varphi_{\alpha} $ siano omeomorfismi da $ U_{\alpha} $ a un sottoinsieme aperto di uno spazio euclideo e

\begin{equation}
	\bigcup_{\alpha \in A} U_{\alpha} = M
\end{equation}

i.e. la famiglia delle carte è un \textit{ricoprimento} dello spazio topologico. Essendo il codominio di ogni carta uno spazio euclideo $ \R^{n} $, $ M $ è una varietà topologica, per cui vale $ \dim M = n $.

\subsubsection{Compatibilità tra carte}

\sbs{0.55}{%
			Due carte $ (U,\varphi) $ e $ (V,\psi) $ di uno spazio topologico $ M $ sono $ C^{\infty} $\textit{-compatibili} se le composizioni\footnotemark
			
			\begin{gather}
				\varphi \circ \psi^{-1} : \psi(U \cap V) \to \varphi(U \cap V) \\
				\psi \circ \varphi^{-1} : \varphi(U \cap V) \to \psi(U \cap V)
			\end{gather}
			
			sono lisce.
		}
	{0.4}{%
			\img{0.9}{img6}
		}
	
	\footnotetext{%
		Queste composizioni sono chiamate \textit{applicazioni di transizione}, dal fatto che collegano due carte, oppure \textit{cambi di carte}.
	}

\begin{remark}
	La $ C^{\infty} $-compatibilità è automaticamente verificata se $ U \cap V = \emptyset $.
\end{remark}

\begin{remark}
	La proprietà di $ C^{\infty} $-compatibilità è riflessiva e simmetrica ma non transitiva:
	
	\begin{itemize}
		\item Se $ (U,\varphi) $ è $ C^{\infty} $-compatibile con $ (V,\psi) $ allora $ (V,\psi) $ è $ C^{\infty} $-compatibile con $ (U,\varphi) $;
		
		\item $ (U,\varphi) $ è $ C^{\infty} $-compatibile con $ (U,\varphi) $;
		
		\item Se $ (U_{1},\varphi_{1}) $ è $ C^{\infty} $-compatibile con $ (U_{2},\varphi_{2}) $ e se $ (U_{2},\varphi_{2}) $ è $ C^{\infty} $-compatibile con $ (U_{3},\varphi_{3}) $, non è detto che $ (U_{1},\varphi_{1}) $ e $ (U_{3},\varphi_{3}) $ siano $ C^{\infty} $-compatibili.
	\end{itemize}
\end{remark}

Definiamo le intersezioni tra aperti

\begin{equation}
	\begin{cases}
		U_{ij} \doteq U_{i} \cap U_{j} \\
		U_{ijk} \doteq U_{i} \cap U_{j} \cap U_{k}
	\end{cases} %
	\qquad i,j,k=1,2,3
\end{equation}

Per le prime due condizioni della transitività, le seguenti applicazioni sono lisce

\begin{gather}
	\varphi_{1} \circ \varphi_{2}^{-1} : \varphi_{2}(U_{12}) \to \varphi_{1}(U_{12}) \\
	\varphi_{2} \circ \varphi_{1}^{-1} : \varphi_{1}(U_{12}) \to \varphi_{2}(U_{12}) \\
	\varphi_{2} \circ \varphi_{3}^{-1} : \varphi_{3}(U_{23}) \to \varphi_{2}(U_{23}) \\
	\varphi_{3} \circ \varphi_{2}^{-1} : \varphi_{2}(U_{23}) \to \varphi_{3}(U_{23})
\end{gather}

e, se la $ C^{\infty} $-compatibilità fosse transitiva, dovrebbero esserlo anche

\begin{gather}
	\varphi_{1} \circ \varphi_{3}^{-1} : \varphi_{3}(U_{13}) \to \varphi_{1}(U_{13}) \\
	\varphi_{3} \circ \varphi_{1}^{-1} : \varphi_{1}(U_{13}) \to \varphi_{3}(U_{13})
\end{gather}

Scrivendo

\begin{equation}
	\varphi_{3} \circ \varphi_{1}^{-1} = \varphi_{3} \circ (\varphi_{2}^{-1} \circ \varphi_{2}) \circ \varphi_{1}^{-1} %
	= (\varphi_{3} \circ \varphi_{2}^{-1}) \circ (\varphi_{2} \circ \varphi_{1}^{-1})
\end{equation}

l'applicazione $ \varphi_{3} \circ \varphi_{2}^{-1} $ è $ C^{\infty}(\varphi_{2}(U_{23})) $ e $ \varphi_{2} \circ \varphi_{1}^{-1} $ è $ C^{\infty}(\varphi_{1}(U_{12})) $ dunque $ \varphi_{3} \circ \varphi_{1}^{-1} $ è sicuramente $ C^{\infty}(\varphi_{1}(U_{123})) $ ma non sappiamo a priori il suo comportamento in $ \varphi_{1}(U_{13} \setminus U_{123}) $, dunque non possiamo dire che $ \varphi_{3} \circ \varphi_{1}^{-1} $ sia $ C^{\infty}(\varphi_{1}(U_{13})) $ e perciò la $ C^{\infty} $-compatibilità non è transitiva. \\ \\
%
Sia $ \mathfrak{U} = \{(U_{\alpha},\varphi_{\alpha})\}_{\alpha \in A} $ un atlante topologico di uno spazio $ M $, diremo che $ (V,\psi) $ carta di $ M $ è $ C^{\infty} $-compatibile con $ \mathfrak{U} $ se $ (V,\psi) $ è $ C^{\infty} $-compatibile con $ (U_{\alpha},\varphi_{\alpha}) $ per qualsiasi $ \alpha \in A $.

\begin{definition}
	Sia $ \mathfrak{U} $ un atlante topologico su $ M $, se due carte $ (V,\psi) $ e $ (W,\sigma) $ di $ M $ sono $ C^{\infty} $-compatibili con $ \mathfrak{U} $ allora lo sono tra loro, i.e. $ (V,\psi) $ è $ C^{\infty} $-compatibile con $ (W,\sigma) $.
\end{definition}

\begin{proof}
	Vogliamo dimostrare che le due applicazioni
	
	\begin{gather}
		\psi \circ \sigma^{-1} : \varphi(V \cap W) \to \psi(V \cap W) \\
		\sigma \circ \psi^{-1} : \psi(V \cap W) \to \sigma(V \cap W)
	\end{gather}

	siano lisce. \\
	Prendiamo $ \sigma \circ \psi^{-1} $: questa è liscia se lo è in $ \psi(p) \in \psi(V \cap W) $ per qualsiasi $ p \in V \cap W \subset M $. \\
	Essendo l'atlante $ \mathfrak{U} $ un ricoprimento di $ M $, esisterà una carta $ (U_{\alpha},\varphi_{\alpha}) \ni p $ perciò con $ p \in U_{\alpha} \cap V \cap W $: potendo scrivere $ \sigma \circ \psi^{-1} = (\sigma \circ \varphi_{\alpha}^{-1}) \circ (\varphi_{\alpha} \circ \psi^{-1}) $, questa è automaticamente liscia in $ \psi(U_{\alpha} \cap V \cap W) $ in quanto, per ipotesi, lo sono i cambi di carte che la compongono. \\
	Questo conclude la dimostrazione perché abbiamo trovato un aperto che contiene il punto, i.e. $ U_{\alpha} \cap V \cap W \ni p $ per $ p $ arbitrario, in cui l'applicazione sia liscia\footnote{%
		La proprietà di un'applicazione di essere liscia ha carattere locale.%
	}
\end{proof}

\subsection{Varietà differenziabili}

Un atlante topologico $ \mathfrak{M} = \{(U_{\alpha},\varphi_{\alpha})\}_{\alpha \in A} $ è detto \textit{differenziabile} (liscio o $ C^{\infty} $) se ogni sua carta è $ C^{\infty} $-compatibile con ogni altra carta al suo interno. \\
Diremo che un atlante differenziabile $ \mathfrak{M} $ è \textit{massimale} se per qualunque altro atlante differenziabile $ \mathfrak{M}' $ di $ M $ si ha che

\begin{equation}
	\mathfrak{M} \subset \mathfrak{M}' \implies \mathfrak{M} = \mathfrak{M}'
\end{equation}

Dato $ M $ spazio topologico, una \textit{struttura differenziabile} su $ M $ è la scelta di un atlante massimale per $ M $. \\
Una \textit{varietà differenziabile} è una varietà topologica con una struttura differenziabile $ \mathfrak{M} $. \\
Diciamo che una varietà differenziabile ha dimensione $ n $ se questa è la dimensione della varietà topologica sottostante\footnote{%
	Vale il teorema di invarianza della dimensione differenziabile.%
}. \\ \\
%
Qualunque sia l'atlante differenziabile su una varietà topologica è sempre possibile trovare un atlante massimale unico che lo contiene, i.e.

\begin{definition}
	Sia $ \mathfrak{U} $ un atlante differenziabile su uno spazio topologico $ M $, allora esiste ed è unica la struttura differenziale (o atlante massimale) $ \mathfrak{M} $ su $ M $ tale che $ \mathfrak{U} \subset \mathfrak{M} $.
\end{definition}

\begin{proof}
	Consideriamo la famiglia $ \mathfrak{M} = \mathfrak{U} \cup \{(V_{i},\psi_{i})\}_{i \in I} $ dove $ \mathfrak{U} $ è un atlante differenziabile e $ \{(V_{i},\psi_{i})\}_{i \in I} $ è un insieme di carte $ C^{\infty} $-compatibili con l'atlante. Per la proposizione precedente, $ \mathfrak{M} $ è un atlante differenziabile di $ M $ perché, se tutte le carte $ \{(V_{i},\psi_{i})\}_{i \in I} $ sono compatibili con l'atlante allora sono compatibili tra di loro, dunque $ \mathfrak{M} \supset \mathfrak{U} $. \\
	Per dimostrare che $ \mathfrak{M} $ sia un atlante massimale, prendiamo $ \mathfrak{M}' $ atlante differenziabile di $ M $ tale che $ \mathfrak{M} \subset \mathfrak{M}' $: se $ \mathfrak{M}' \supset \mathfrak{M} $ allora $ \mathfrak{M}' \supset \mathfrak{U} $, quindi tutte le carte di $ \mathfrak{M}' $ sono $ C^{\infty} $-compatibili con $ \mathfrak{U} $, perciò per definizione $ \mathfrak{M} \supset \mathfrak{M}' $, dimostrando che $ \mathfrak{M}' = \mathfrak{M} $ e quindi che $ \mathfrak{M} $ sia massimale. \\
	Per dimostrare che $ \mathfrak{M} $ sia l'unico atlante massimale tale che $ \mathfrak{M} \supset \mathfrak{U} $, prendiamo $ \mathfrak{M}' $ un altro atlante differenziabile massimale di $ M $ tale che $ \mathfrak{M}' \supset \mathfrak{U} $, allora tutte le carte di $ \mathfrak{M}' $ sono $ C^{\infty} $-compatibili con $ \mathfrak{U} $ e quindi, per definizione, $ \mathfrak{M}' \supset \mathfrak{M} $ ma, siccome $ \mathfrak{M}' $ è massimale, questo implica che $ \mathfrak{M} = \mathfrak{M}' $.
\end{proof}

\begin{remark}
	Siano $ M $ una varietà differenziabile, $ (U_{\beta},\varphi_{\beta}) $ una carta della struttura differenziabile e $ V \subset U_{\beta} $ un aperto di $ M $. Allora $ (V,\varphi) $, dove $ \varphi $ è la restrizione della funzione $ \varphi_{\beta} $ all'aperto $ V $, i.e. $ \varphi = \eval{\varphi_{\beta}}_{V} $, è una carta per $ M $ per la struttura differenziabile fissata.
\end{remark}

Infatti, se prendiamo l'atlante massimale differenziabile $ \mathfrak{M} = \{(U_{\alpha},\varphi_{\alpha})\} $ e consideriamo i cambi di carte

\begin{gather}
	\varphi \circ \varphi_{\alpha}^{-1} : \varphi_{\alpha}(U_{\alpha} \cap V) \to \varphi(U_{\alpha} \cap V) \\
	\varphi_{\alpha} \circ \varphi^{-1} : \varphi(U_{\alpha} \cap V) \to \varphi_{\alpha}(U_{\alpha} \cap V)
\end{gather}

vediamo che queste sono $ C^{\infty} $-compatibili con l'atlante perché

\begin{align}
	\varphi \circ \varphi_{\alpha}^{-1} = \eval{\varphi_{\beta} \circ \varphi_{\alpha}^{-1}}_{\varphi_{\alpha}(U_{\alpha} \cap V)} \\
	\varphi_{\alpha} \circ \varphi^{-1} = \eval{\varphi_{\alpha} \circ \varphi_{\beta}^{-1}}_{\varphi_{\beta}(U_{\alpha} \cap V)}
\end{align}

sono restrizioni a un aperto di funzioni lisce, perciò rendono $ C^{\infty} $-compatibili le carte, che dunque appartengono all'atlante massimale. \\
In particolare, se $ U $ è un aperto qualunque di $ M $ e $ (U_{\beta},\varphi_{\beta}) $ è una carta della struttura differenziabile, allora $ \left( U \cap U_{\beta}, \eval{\varphi_{\beta}}_{U \cap U_{\beta}} \right) $ è una carta nella struttura differenziabile.

\subsubsection{\textit{Esempi}}

\paragraph{0) Spazio euclideo}

Sia la carta $ (\R^{n},\id_{\R^{n}}) $: questa rappresenta un atlante differenziabile per $ \R^{n} $ che contiene un'unica carta. Essendoci un atlante differenziale per lo spazio, esisterà un'unica struttura differenziabile su $ \R^{n} $ di cui $ (\R^{n},\id_{\R^{n}}) $ è l'unica carta.

\paragraph{1) Aperto di varietà differenziabile}

Siano $ M $ una varietà differenziabile e $ U \subset M $ un suo aperto, allora $ U $ è una varietà differenziabile, con struttura indotta da $ M $, tale che $ \dim (U) = \dim (M) $. Se $ \mathfrak{U} = \{(U_{\alpha},\varphi_{\alpha})\} $ è un atlante differenziabile su $ M $ (massimale o no), allora $ \mathfrak{V} = \{(U_{\alpha} \cap U,\left. \varphi_{\alpha} \right|_{U_{\alpha} \cap U})\} $ è un atlante differenziabile per $ U $.

\paragraph{2) Varietà 0-dimensionali}

Essendo $ \R^{0} = \{0\} $, siccome deve esistere un omeomorfismo dall'intorno del punto a $ 0 $, questo intorno sarà costituito dal solo punto. Una varietà 0-dimensionale sarà dunque costituita da un insieme di punti con \textit{topologia discreta}, i.e. ogni punto è un aperto per la topologia. Uno spazio discreto è sempre $ N_{2}+T_{2} $. La condizione di $ C^{\infty} $-compatibilità è automatica perché tutti gli aperti sono disgiunti.

\paragraph{3) Curve}

Per le curve differenziabili, $ \dim (M) = 1 $. Vale il seguente teorema

\begin{theorem}
	\begin{equation}
		M \text{ curva connessa } \implies M \stackrel{diff}{\simeq} \R \, \lor \, M \stackrel{diff}{\simeq} \S^{1}
	\end{equation}

	cioè, se $ M $ è una curva connessa, allora è diffeomorfa\footnote{%
		Il concetto di diffeomorfismo tra varietà differenziabili viene introdotto nella Sottosezione \ref{s-sec:diff}.%
	} alla retta reale $ \R $ oppure al cerchio unitario $ \S^{1} $.
\end{theorem}

\paragraph{4) Superfici}

Per le superfici differenziabili, $ \dim (M) = 2 $. Le \textit{parametrizzazioni} $ X $ di curve e superfici (compatte e connesse) sono l'inverso delle carte $ \varphi = X^{-1} $, i.e per la parametrizzazione di una superficie
%
\map{X}
	{U \subset \R^{2}}{\R^{3}}
	{(u,v)}{(x(u,v),y(u,v),z(u,v))}

e per la carta $ (X(u,v),\varphi) $.

\paragraph{5) Grafico di una funzione}

Siano un aperto $ A \subset \R^{n} $ e un'applicazione liscia $ f : A \to \R^{m} $.

\sbs{0.55}{%
			 Definiamo il \textit{grafico di} $ f $ come
			
			\begin{equation}
				\Gamma(f) \doteq \left\{ (x,f(x)) \st x \in A \right\} \subset \R^{n} \times \R^{m} = \R^{n+m}
			\end{equation}
		
			Si può dotare $ \Gamma(f) $ di una struttura di varietà differenziabile di dimensione $ n $. \\
			Il grafico $ \Gamma(f) \subset \R^{n+m} $ è $ N_{2}+T_{2} $ (perché lo è $ \R^{n+m} $) dunque è una varietà topologica. Considerando l'omeomorfismo
			
			\map{\varphi}%
			{\Gamma(f)}{A}%
			{(x,f(x))}{x}
		}
	{0.45}{%
			\img{1}{img7}
		}


possiamo costruire un atlante costituito da un'unica carta $ (\Gamma(f),\varphi) $, il quale implica l'esistenza di un atlante massimale che lo contenga. L'applicazione $ \varphi $ è un omeomorfismo perché è continua e la sua inversa è continua in quanto $ f $ è liscia.

\paragraph{6) Gruppo lineare}

Ricordiamo che è possibile identificare l'insieme delle matrici quadrate con lo spazio euclideo $ M_{n}(\R) = \R^{n^{2}} $ mappando le righe delle matrici in ordine per creare un vettore di $ n^{2} $ componenti. \\
Sia l'insieme delle matrici con determinante non nullo

\begin{equation}
	GL_{n}(\R) = \{ A \in M_{n}(\R) \mid \det(A) \neq 0 \} \subset M_{n}(\R)
\end{equation}

Con la topologia indotta da $ \R^{n^{2}} $, lo spazio $ GL_{n}(\R) $ è $ N_{2} + T_{2} $. Inoltre, $ GL_{n}(\R) $ è un aperto di $ M_{n}(\R) $ perché complementare dell'insieme chiuso

\begin{equation}
	 M_{n}(\R) \setminus GL_{n}(\R) = \{ A \in M_{n}(\R) \mid \det(A) = 0 \}
\end{equation}

in quanto il determinante è un'applicazione continua dunque la controimmagine del punto $ 0 $ (il quale è chiuso in $ \R $) è un chiuso. Essendo quindi un aperto di una varietà, $ GL_{n}(\R) $ è una varietà differenziabile con l'unica carta $ (GL_{n}(\R),\id_{GL_{n}(\R)}) $; questa varietà ha $ \dim(GL_{n}(\R)) = n^{2} $.

\paragraph{7) Cerchio unitario $ \S^{1} $}\label{example:diff-man-s1}

Il cerchio unitario $ \S^{1} $

\begin{equation}
	\S^{1} = \left\{ (x,y) \in \R^{2} \st x^{2} + y^{2} = 1 \right\} \subset \R^{2}
\end{equation}

con la topologia indotta da $ \R^{2} $ (quindi $ N_{2} + T_{2} $) può essere dotato di una struttura differenziabile costituita da quattro carte prendendo gli aperti\footnote{%
	Questi insiemi sono aperti in $ \S^{1} $ perché intersezioni di $ \S^{1} $ (aperto in sé stesso) con aperti di $ \R^{n} $, e.g. il semipiano superiore o quello inferiore.}

\begin{gather}
	U_{1} = \left\{ (x,y) \in \S^{1} \st y > 0 \right\} \\
	U_{2} = \left\{ (x,y) \in \S^{1} \st y < 0 \right\} \\
	U_{3} = \left\{ (x,y) \in \S^{1} \st x > 0 \right\} \\
	U_{4} = \left\{ (x,y) \in \S^{1} \st x < 0 \right\}
\end{gather}

i quali sono un ricoprimento del cerchio unitario, i.e.

\begin{equation}
	\S^{1} = \bigcup_{k=1}^{4} U_{k}
\end{equation}

Definiamo ora quattro funzioni che costituiranno, insieme agli aperti sopra definiti, un atlante per lo spazio

\sbs{0.45}{%
			\map{\varphi_{1}}
				{U_{1}}{(-1,1)_{x} \subset \R}
				{(x,y)}{x}
			
			\map{\varphi_{2}}
				{U_{2}}{(-1,1)_{x} \subset \R}
				{(x,y)}{x}
			
			\map{\varphi_{3}}
				{U_{3}}{(-1,1)_{y} \subset \R}
				{(x,y)}{y}
			
			\map{\varphi_{4}}
				{U_{4}}{(-1,1)_{y} \subset \R}
				{(x,y)}{y}
		}
	{0.52}{%
			\img{1}{img8}
		}

Vogliamo mostrare che $ \{(U_{i},\varphi_{i})\}_{i=1,\dots,4} $ sia un atlante differenziabile in modo tale da dotare $ \S^{1} $ di un atlante massimale e quindi renderlo una varietà differenziabile. \\
Dimostriamo prima che le $ \varphi_{i} $ siano omeomorfismi: sono continue (e anche lisce) perché sono restrizioni a una delle componenti e le inverse

\sbs{0.5}{%
			\map{\varphi_{1}^{-1}}
				{(-1,1)_{x}}{U_{1}}
				{x}{(x,\sqrt{1-x^{2}})}
			
			\map{\varphi_{2}^{-1}}
				{(-1,1)_{x}}{U_{2}}
				{x}{(x,-\sqrt{1-x^{2}})}
			}
	{0.5}{%
			\map{\varphi_{3}^{-1}}
				{(-1,1)_{y}}{U_{3}}
				{y}{(\sqrt{1-y^{2}},y)}
			
			\map{\varphi_{4}^{-1}}
				{(-1,1)_{y}}{U_{4}}
				{y}{(-\sqrt{1-y^{2}},y)}
			}

sono anch'esse continue (e lisce); questo dimostra che lo spazio è una varietà topologica. \\
Per dimostrare che sia una struttura differenziabile, prendiamo due carte arbitrarie e ne facciamo la composizione per controllare la loro $ C^{\infty} $-compatibilità:

\begin{gather}
	\varphi_{3} \circ \varphi_{1}^{-1} : \varphi_{1}(U_{1} \cap U_{3}) \to \varphi_{3}(U_{1} \cap U_{3}) \\
	\varphi_{1} \circ \varphi_{3}^{-1} : \varphi_{3}(U_{1} \cap U_{3}) \to \varphi_{1}(U_{1} \cap U_{3})
\end{gather}

dove

\begin{equation}
	U_{1} \cap U_{3} = \left\{ (x,y) \in \S^{1} \st x>0 \, \wedge \, y>0  \right\}
\end{equation}

e dunque

\begin{gather}
	\varphi_{1}(U_{1} \cap U_{3}) = (0,1)_{x} \\
	\varphi_{3}(U_{1} \cap U_{3}) = (0,1)_{y}
\end{gather}

L'azione delle composizioni è la seguente

\begin{gather}
	(\varphi_{3} \circ \varphi_{1}^{-1}) (x) = \varphi_{3} (x,\sqrt{1-x^{2}})) = \sqrt{1-x^{2}} \\
	(\varphi_{1} \circ \varphi_{3}^{-1}) (y) = \varphi_{1} (\sqrt{1-y^{2}},y)) = \sqrt{1-y^{2}}
\end{gather}

Queste applicazioni sono lisce in quanto le loro derivate non divergono mai perché i denominatori non si annullano mai negli intervalli dei domini. \\
Il ragionamento è analogo per tutte le altre composizioni, dunque $ \{(U_{i},\varphi_{i})\}_{i=1,\dots,4} $ è un atlante differenziabile per $ \S^{1} $.

\paragraph{8) Sfera unitaria $ \S^{2} $}\label{example:diff-man-unit-sph}

La sfera unitaria $ \S^{2} $ è una varietà topologica inclusa in $ \R^{3} $ (che induce la topologia e le proprietà di $ N_{2}+T_{2} $). \\
Gli aperti che ricoprono lo spazio sono i seguenti

\begin{align}
	U_{1} = \left\{ (x,y,z) \in \S^{2} \st z>0 \right\} \\
	U_{2} = \left\{ (x,y,z) \in \S^{2} \st z<0 \right\} \\
	U_{3} = \left\{ (x,y,z) \in \S^{2} \st y>0 \right\} \\
	U_{4} = \left\{ (x,y,z) \in \S^{2} \st y<0 \right\} \\
	U_{5} = \left\{ (x,y,z) \in \S^{2} \st x>0 \right\} \\
	U_{6} = \left\{ (x,y,z) \in \S^{2} \st x<0 \right\}
\end{align}

Prendendo il generico aperto

\begin{equation}
	D_{ij} \doteq \left\{ (i,j) \in \R^{2} \st i^{2}+j^{2}<1 \right\} \subset \R^{2} \qcomma i,j = x,y,z
\end{equation}

definiamo dunque le seguenti applicazioni per formare le carte $ \{(U_{i},\varphi_{i})\}_{i=1,\dots,6} $

\sbs{0.45}{%
			\map{\varphi_{1}}
				{U_{1}}{D_{xy}}
				{(x,y,z)}{(x,y)}
			
			\map{\varphi_{2}}
				{U_{2}}{D_{xy}}
				{(x,y,z)}{(x,y)}
			
			\map{\varphi_{3}}
				{U_{3}}{D_{xz}}
				{(x,y,z)}{(x,z)}
			
			\map{\varphi_{4}}
				{U_{4}}{D_{xz}}
				{(x,y,z)}{(x,z)}
			
			\map{\varphi_{5}}
				{U_{5}}{D_{yz}}
				{(x,y,z)}{(y,z)}
			
			\map{\varphi_{6}}
				{U_{6}}{D_{yz}}
				{(x,y,z)}{(y,z)}
		}
		{0.52}{%
			Ad esempio, l'applicazione $ \varphi_{1} $ è rappresentata di seguito:
			\img{1}{img9}
		}

Le applicazioni inverse sono

\sbs{0.5}{%
			\map{\varphi_{1}^{-1}}
				{D_{xy}}{U_{1} \subset \R^{2}}
				{(x,y)}{(x,y,\sqrt{1-x^{2}-y^{2}})}
			
			\map{\varphi_{2}^{-1}}
				{D_{xy}}{U_{2} \subset \R^{2}}
				{(x,y)}{(x,y,-\sqrt{1-x^{2}-y^{2}})}
			
			\map{\varphi_{3}^{-1}}
				{D_{xz}}{U_{3} \subset \R^{2}}
				{(x,z)}{(x,\sqrt{1-x^{2}-z^{2}},z)}
			}
	{0.5}{%
			\map{\varphi_{4}^{-1}}
				{D_{xz}}{U_{4} \subset \R^{2}}
				{(x,z)}{(x,-\sqrt{1-x^{2}-z^{2}},z)}
			
			\map{\varphi_{5}^{-1}}
				{D_{yz}}{U_{5} \subset \R^{2}}
				{(y,z)}{(\sqrt{1-y^{2}-z^{2}},y,z)}
			
			\map{\varphi_{6}^{-1}}
				{D_{yz}}{U_{6} \subset \R^{2}}
				{(y,z)}{(-\sqrt{1-y^{2}-z^{2}},y,z)}
			}

Ad esempio, verifichiamo che le carte $ (U_{1},\varphi_{1}) $ e $ (U_{4},\varphi_{4}) $ siano compatibili:

\begin{gather}
	\varphi_{1} \circ \varphi_{4}^{-1} : \varphi_{4}(U_{1} \cap U_{4}) \to \varphi_{1}(U_{1} \cap U_{4}) \\
	\varphi_{4} \circ \varphi_{1}^{-1} : \varphi_{1}(U_{1} \cap U_{4}) \to \varphi_{4}(U_{1} \cap U_{4})
\end{gather}

dove

\begin{equation}
	U_{1} \cap U_{4} = \left\{ (x,y,z) \in \S^{2} \st z>0 \, \wedge \, y<0 \right\}
\end{equation}

dunque

\begin{gather}
	(\varphi_{1} \circ \varphi_{4}^{-1})(x,z) = \varphi_{1} \left( x,-\sqrt{1-x^{2}-z^{2}},z \right) = \left( x,-\sqrt{1-x^{2}-z^{2}} \right) \\
	(\varphi_{4} \circ \varphi_{1}^{-1})(x,y) = \varphi_{4} \left( x,y,\sqrt{1-x^{2}-y^{2}} \right) = \left( x,\sqrt{1-x^{2}-y^{2}} \right)
\end{gather}

e analogamente per le altre composizioni. Tutte queste applicazioni e le loro inverse sono lisce e dunque continue per lo stesso motivo dato per il cerchio unitario. \\
Tutto ciò dota lo spazio di un atlante differenziabile e dunque la sfera unitaria $ \S^{2} $ di una struttura differenziabile\footnote{%
	Vedi Esercizio \ref{exer2-1}.%
}.

\paragraph{9) Sfera unitaria $ \S^{n} $ con proiezione stereografica}

La sfera unitaria $ \S^{n} $

\begin{equation}
	\S^{n} = \left\{ x = (x^{1},\dots,x^{n+1}) \in \R^{n+1} \st \norm{x}^{2} \doteq \sum_{i=1}^{n+1} (x^{i})^{2} = 1 \right\} \subset \R^{n+1}
\end{equation}

ha la topologia indotta da $ \R^{n+1} $ (quindi è $ N_{2}+T_{2} $). \\
Siano i poli $ N = (0,\dots,0,1) $ e $ S = (0,\dots,0,-1) $ da cui i due aperti\footnote{%
	Sono aperti perché complementari di un chiuso, i.e. i punti ai poli.%
} per le carte

\begin{gather}
	U_{N} = \S^{n} \setminus \{N\} \\
	U_{S} = \S^{n} \setminus \{S\}
\end{gather}

Costruiamo ora le applicazioni per la proiezione stereografica che si ottengono considerando una retta per uno dei poli e il punto da proiettare, e l'intersezione della retta stessa con il piano $ \R^{n} \times \{0\} = \R^{n} $:

\sbs{0.55}{%
			\map{\pi_{N}}
				{U_{N}}{\R^{n} \times \{0\} = \R^{n}}
				{(x^{1},\dots,x^{n+1})}{\left( \dfrac{2x^{1}}{1-x^{n+1}},\dots,\dfrac{2x^{n}}{1-x^{n+1}} \right)}
			
			\map{\pi_{S}}
				{U_{S}}{\R^{n} \times \{0\} = \R^{n}}
				{(x^{1},\dots,x^{n+1})}{\left( \dfrac{2x^{1}}{1+x^{n+1}},\dots,\dfrac{2x^{n}}{1+x^{n+1}} \right)}
		}
	{0.45}{%
			\img{1}{img10}
		}

Le inverse delle applicazioni sono

\map{\pi_{N}^{-1}}
	{\R^{n}}{U_{N}}
	{(x^{1},\dots,x^{n})}{\left( \dfrac{x^{1}}{1+\norm{x}^{2}},\dots,\dfrac{x^{n}}{1+\norm{x}^{2}},\dfrac{1-\norm{x}^{2}}{1+\norm{x}^{2}} \right)}

\map{\pi_{S}^{-1}}
	{\R^{n}}{U_{S}}
	{(x^{1},\dots,x^{n})}{\left( \dfrac{x^{1}}{1+\norm{x}^{2}},\dots,\dfrac{x^{n}}{1+\norm{x}^{2}},\dfrac{\norm{x}^{2}-1}{1+\norm{x}^{2}} \right)}

Sia le funzioni che le loro inverse sono lisce e sono omeomorfismi. \\
Per dotare $ \S^{n} $ di una struttura differenziabile, prendendo l'atlante massimale che include queste due carte, è necessario verificare che i cambi di carte siano lisci

\begin{gather}
	\pi_{N} \circ \pi_{S}^{-1} : \pi_{S}(U_{N} \cap U_{S}) \to \pi_{N}(U_{N} \cap U_{S}) \\
	\pi_{S} \circ \pi_{N}^{-1} : \pi_{N}(U_{N} \cap U_{S}) \to \pi_{S}(U_{N} \cap U_{S})
\end{gather}

dove $ U_{S} \cap U_{N} = \S^{n} \setminus \{N,S\} $ e quindi

\begin{equation}
	\pi_{N}(U_{S} \cap U_{N}) = \pi_{S}(U_{S} \cap U_{N}) = \R^{n} \setminus \{0\}
\end{equation}

A questo punto i cambi di carte si riducono all'applicazione identità e sono dunque lisce. \\
Considerando l'uguaglianza

\begin{equation}
	\dfrac{ 2 \left( \dfrac{x^{k}}{1 + \norm{x}^{2}} \right) }{ 1 - \dfrac{\norm{x}^{2} - 1}{1 + \norm{x}^{2}} } = \dfrac{2 x^{k}}{1 + \norm{x}^{2}} \left( \dfrac{ 1 + \norm{x}^{2} - \norm{x}^{2} + 1 }{ 1 + \norm{x}^{2} } \right)^{-1} %
	= \dfrac{2 x^{k}}{1 + \norm{x}^{2}} \left( \dfrac{1 + \norm{x}^{2}}{2} \right) %
	= x^{k}
\end{equation}

otteniamo

\begin{align}
	\begin{split}
		(\pi_{N} \circ \pi_{S}^{-1}) (x^{1},\dots,x^{n}) &= \pi_{N} (\pi_{S}^{-1} (x^{1},\dots,x^{n})) \\
		&= \pi_{N} \left( \dfrac{x^{1}}{1+\norm{x}^{2}},\dots,\dfrac{x^{n}}{1+\norm{x}^{2}},\dfrac{\norm{x}^{2}-1}{1+\norm{x}^{2}} \right) \\
		&= \left( \dfrac{ 2 \left( \dfrac{x^{1}}{1 + \norm{x}^{2}} \right) }{ 1 - \dfrac{\norm{x}^{2} - 1}{1 + \norm{x}^{2}} }, \dots, \dfrac{ 2 \left( \dfrac{x^{n}}{1 + \norm{x}^{2}} \right) }{ 1 - \dfrac{\norm{x}^{2} - 1}{1 + \norm{x}^{2}} } \right) \\
		&= (x^{1},\dots,x^{n})
	\end{split}
\end{align}

e analogamente per $ \pi_{S} \circ \pi_{N}^{-1} $. A questo punto

\begin{equation}
	\pi_{N} \circ \pi_{S}^{-1} = \pi_{S} \circ \pi_{N}^{-1} %
	= \id_{\R^{n} \setminus \{0\}}
\end{equation}

Queste composizioni sono lisce dunque le carte $ (U_{N},\pi_{N}) $ e $ (U_{S},\pi_{S}) $ sono $ C^{\infty} $-compatibili perciò $ \{(U_{N},\pi_{N}),(U_{S},\pi_{S})\} $ è un atlante differenziabile su $ \S^{n} $, la quale diventa una varietà differenziabile\footnote{%
	Vedi Esercizio \ref{exer2-2}.%
}.

\paragraph{10) Prodotto di varietà differenziabili}

Prendiamo due varietà differenziabili $ M $ ed $ N $ e consideriamo il prodotto cartesiano $ M \times N $ con \textit{topologia prodotto}, la quale rende $ M \times N $ ancora $ N_{2}+T_{2} $. La dimensione del prodotto di due varietà è

\begin{equation}
	\dim (M \times N) = \dim(M) + \dim(N) = m + n
\end{equation}

Siano $ \mathfrak{U} = \{(U_{\alpha},\varphi_{\alpha})\}_{\alpha \in A} $ un atlante differenziabile di $ M $ e $ \mathfrak{V} = \{(V_{\beta},\psi_{\beta})\}_{\beta \in B} $ un atlante differenziabile di $ N $. Sia un atlante per $ M \times N $

\begin{equation}
	\mathfrak{W} = \{(U_{\alpha} \times V_{\beta},\varphi_{\alpha} \times \psi_{\beta})\}_{\alpha \in A, \, \beta \in B}
\end{equation}

dove al variare di $ \alpha $ e $ \beta $ per l'aperto $ U_{\alpha} \times V_{\beta} \subset M \times N $ si ottiene un ricoprimento per lo spazio prodotto, i.e.

\begin{equation}
	M \times N = \bigcup_{\substack{ \alpha \in A \\ \beta \in B }} (U_{\alpha} \times V_{\beta})
\end{equation}

Le applicazioni

\map{\varphi_{\alpha} \times \psi_{\beta}}
	{U_{\alpha} \times V_{\beta}}{\varphi_{\alpha}(U_{\alpha}) \times \psi_{\beta}(V_{\beta}) \subset \R^{m} \times \R^{n} = \R^{m+n}}
	{(x,y)}{(\varphi_{\alpha}(x),\psi_{\beta}(y))}

sono continue (il prodotto di applicazioni continue è continuo) perché omeomorfismi e le inverse sono il prodotto delle inverse

\begin{equation}
	(\varphi_{\alpha} \times \psi_{\beta})^{-1} = \varphi_{\alpha}^{-1} \times \psi_{\beta}^{-1} : \varphi_{\alpha}(U_{\alpha}) \times \psi_{\beta}(V_{\beta}) \to U_{\alpha} \times V_{\beta}
\end{equation}

anch'esse continue perché inverse di omeomorfismi. \\
Questo rende $ \mathfrak{W} $ un atlante topologico; perché sia un atlante differenziabile, verifichiamo che le carte siano $ C^{\infty} $-compatibili controllando che i cambi di carte siano lisci

\begin{equation}
	(\varphi_{\alpha} \times \psi_{\beta}) \circ (\varphi_{\rho} \times \psi_{\sigma})^{-1} : (\varphi_{\rho} \times \psi_{\sigma})(U_{\rho} \times V_{\sigma}) \to (\varphi_{\alpha} \times \psi_{\beta})(U_{\alpha} \times V_{\beta})
\end{equation}

alternativamente possiamo scriverli come

\begin{equation}
	(\varphi_{\alpha} \circ \varphi_{\rho}^{-1}) \times (\psi_{\beta} \circ \psi_{\sigma})^{-1} : \varphi_{\rho}(U_{\rho}) \times \psi_{\sigma}(V_{\sigma}) \to \varphi_{\alpha}(U_{\alpha}) \times \psi_{\beta}(V_{\beta})
\end{equation}

Le composizioni $ \varphi_{\alpha} \circ \varphi_{\rho}^{-1} $ e $ \psi_{\beta} \circ \psi_{\sigma} $ sono lisce perché ognuna di queste prova la compatibilità delle carte in $ M $ e in $ N $ rispettivamente e dunque il loro prodotto è ancora liscio. Questi ragionamenti portano all'esistenza di una struttura differenziale per $ M \times N $, la quale diventa dunque una varietà differenziabile. \\
Esempi di superfici differenziabili risultato di prodotti tra varietà differenziabili sono il toro bidimensionale $ \T^{2} = \S^{1} \times \S^{1} $ e il cilindro infinito $ \mathcal{C} = \S^{1} \times \R $. Più in generale, anche il toro $ n $-dimensionale è una varietà differenziabile

\begin{align}
	\T^{n} = \prod^{n} \S^{1}
\end{align}

dove la produttoria implica, in questo caso, la topologia prodotto $ \times $.

\subsection{Spazi quoziente dal punto di vista topologico}\label{s-sec:quot}

Sia $ S $ uno spazio topologico e $ \sim $ una relazione di equivalenza su $ S $: lo \textit{spazio quoziente} $ \sfrac{S}{\sim} $ è l'insieme delle classi di equivalenza degli elementi di $ S $. Dalla relazione di equivalenza deriva una proiezione sullo spazio quoziente, la quale mappa un elemento $ x \in S $ nella classe di equivalenza $ [x] $, i.e.

\map{\pi}
	{S}{\sfrac{S}{\sim}}
	{x}{[x]}
	
Definiamo una topologia su $ \sfrac{S}{\sim} $ definendo gli aperti nel quoziente:

\begin{equation}
	U \text{ aperto in } \sfrac{S}{\sim} %
	\iff %
	U = \emptyset \, \lor \, U = \sfrac{S}{\sim} \, \lor \, \pi^{-1}(U) \text{ aperto in } S
\end{equation}

i.e. $ U $ è aperto se e solo se la sua controimmagine tramite la proiezione è aperta in $ S $. \\
Per verificare che questa sia una topologia è necessario che l'intersezione di due aperti e l'unione arbitraria di aperti siano ancora aperti nello spazio: queste proprietà sono soddisfatte perché rispettate da $ \pi^{-1} $

\begin{gather}
	\pi^{-1}(U \cap V) = \pi^{-1}(U) \cap \pi^{-1}(V) \qcomma \forall U, V \subset S \text{ aperti} \\
	\pi^{-1}\left( \bigcup_{j \in J} V_{j} \right) = \bigcup_{j \in J} \pi^{-1}(V_{j}) \qcomma \forall V_{j} \subset S \text{ aperti}
\end{gather}

Questa topologia è chiamata \textit{topologia quoziente} su $ \sfrac{S}{\sim} $.

\begin{remark}
	Una volta fissata la topologia quoziente, la proiezione $ \pi : S \to \sfrac{S}{\sim} $ è continua, perché la controimmagine di un aperto è aperta\footnote{%
		Questo perché nella definizione della topologie del quoziente, un insieme è aperto se e solo se la sua controimmagine tramite la proiezione è aperta.%
	}.
\end{remark}

Definiamo l'applicazione $ f : S \to Y $ tra spazi topologici: questa applicazione si dice \textit{costante sulle classi di equivalenza} se, presi due elementi $ x_{1}, x_{2} \in S $, si ha che

\begin{equation}
	x_{1} \sim x_{2} \implies f(x_{1}) = f(x_{2})
\end{equation}

Questa proprietà permette di considerare una seconda applicazione

\map{\tilde{f}}
	{\sfrac{S}{\sim}}{Y}
	{[x]}{f(x)}

la quale è ben definita in quanto il rappresentante della classe $ [x] $ che si prende per calcolare $ f(x) $ non influisce al fine del calcolo.

\sbs{0.5}{%
			Il diagramma a lato permette di derivare la seguente proprietà:
			
			\begin{equation}
				f = \tilde{f} \circ \pi
			\end{equation}
			}
	{0.5}{%
			\diagr{%
					S \arrow[rr, "f"] \arrow[dd, "\pi"']       \&  \& Y \\
					\&  \&   \\
					\sfrac{S}{\sim} \arrow[rruu, "\tilde{f}"'] \&  \&  
					}
			}

\begin{definition}
	$ f $ è continua $ \iff $ $ \tilde{f} $ è continua
\end{definition}

\begin{proof}[Dimostrazione ($ \impliedby $)]
	Per ipotesi  $ \tilde{f} $ è continua: siccome anche la proiezione $ \pi $ è continua, $ f = \tilde{f} \circ \pi $ è continua in quanto la composizione di funzioni continue è ancora continua.
\end{proof}

\begin{proof}[Dimostrazione ($ \implies $)]
	Sia $ V $ un aperto di $ Y $: prendiamo la controimmagine di $ V $ tramite la funzione continua $ f $
	
	\begin{equation}
			f^{-1}(V) = (\tilde{f} \circ \pi)^{-1}(V) = \pi^{-1}(\tilde{f}^{-1}(V))
	\end{equation}

	Siccome $ f $ è continua, $ f^{-1}(V) $ è aperto e dunque anche $ \pi^{-1}(\tilde{f}^{-1}(V)) $ è aperto. Per definizione di topologia quoziente, anche $ \tilde{f}^{-1}(V) $ è aperto in $ S $: questa è la definizione di applicazione continua dunque $ \tilde{f} $ è continua.
\end{proof}

\subsubsection{\textit{Esempi}}

\paragraph{1) Spazi ottenuti tramite quozienti di insiemi}

Sia $ S $ uno spazio topologico e $ A $ un suo qualunque sottoinsieme. Definiamo la relazione di equivalenza $ \sim $ su $ S $ come

\begin{equation}
	\begin{cases}
		x \sim x & \qq*{se} x \in S \setminus A \\
		x \sim y & \qq*{se} x,y \in A
	\end{cases}
\end{equation}

cioè ogni punto di $ S \setminus A $ è identificato con sé stesso mentre tutti i punti di $ A $ sono identificati tra loro\footnote{%
	In altre parole, tutti i punti di $ S \setminus A $ vanno in sé stessi mentre tutti i punti di $ A $ vanno in uno solo.%
}. \\
Scriviamo lo spazio quoziente rispetto a questa relazione di equivalenza come $ \sfrac{S}{\sim} = \sfrac{S}{A} $, chiamato \textit{spazio ottenuto da} $ S $ \textit{tramite} $ A $.

\paragraph{2) $ \S^{1} $ come quoziente}

Siano gli insiemi $ I = [0,1] \subset \R $ e $ A = \{0,1\} $. 

\sbs{0.49}{%
			Consideriamo il quoziente $ \sfrac{I}{\{0,1\}} $: utilizzando la stessa relazione di equivalenza $ \sim $ dell'esempio precedente, si può identificare lo spazio quoziente $ \sfrac{I}{\{0,1\}} $ con un cerchio, in quanto tutti i punti di $ I $ vengono identificati con sé stessi mentre gli estremi $ \{0,1\} $ si identificano uno con l'altro.
			}
	{0.5}{%
			\img{1}{img12}
			}
		
Dimostriamo ora che i due spazi menzionati sopra siano omeomorfi, i.e. $ \sfrac{I}{\{0,1\}} \simeq \S^{1} $. Sia l'applicazione continua

\map{f}%
	{I}{\S^{1} \subset \R^{2}}%
	{x}{e^{2 i \pi x} = (\cos(2 \pi x), \sin(2 \pi x))}

questa è costante sulle classi di equivalenza di $ \sfrac{I}{\{0,1\}} $. Tramite questa applicazione, possiamo costruire il diagramma

\diagr{%
		I \arrow[rr, "f"] \arrow[dd, "\pi"']        \&  \& \S^{1} \\
		\&  \&        \\
		{\sfrac{I}{\{0,1\}}} \arrow[rruu, "\tilde{f}"'] \&  \&       
		}

il quale implica l'esistenza della funzione $ \tilde{f} $, la quale è continua perché $ f $ è continua. Questa applicazione è anche iniettiva e suriettiva: perché sia identificata come un omeomorfismo, si utilizza il seguente lemma

\begin{lemma}[dell'applicazione chiusa]\label{lemma:clos-app}
	
	Sia $ f : X \to Y $ un'applicazione continua da uno spazio compatto $ X $ a uno spazio $ T_{2} $ o di Hausdorff $ Y $, allora valgono i seguenti fatti:
	
	\begin{enumerate}
		\item l'applicazione $ f $ è chiusa;
		
		\item se $ f $ è una bigezione allora è un omeomorfismo;
		
		\item se $ f $ è iniettiva allora è un embedding topologico\footnote{%
			L'embedding topologico verrà affrontato nella Sottosezione \ref{s-sec:img-smooth-app}.%
		}.
	\end{enumerate}
\end{lemma}

Per la dimostrazione dei fatti usati nelle dimostrazioni successive, vedi \cite{Loi}.

\begin{proof}[Dimostrazione (1)]
	Sia un sottoinsieme chiuso $ C \subset X $, siccome è un sottoinsieme dello spazio compatto $ X $, $ C $ è compatto. L'immagine di un compatto tramite un'applicazione continua, i.e. $ f(C) $, è compatto. Essendo $ f(C) \subset Y $ un sottoinsieme compatto dello spazio di Hausdorff $ Y $, questo è chiuso, dunque $ f $ è chiusa.
\end{proof}

\begin{proof}[Dimostrazione (2)]
	Una bigezione continua tra due spazi topologici è un omeomorfismo se e solo se è aperta (risp. chiusa) dunque, per la dimostrazione precedente, $ f $ è un omeomorfismo in quanto chiusa.
\end{proof}

\begin{proof}[Dimostrazione (3)]
	Se l'applicazione $ f $ è iniettiva, allora $ f : X \to f(X) $ è una bigezione continua, dunque un omeomorfismo per la dimostrazione precedente: siccome $ X \stackrel{omeo}{\simeq} f(X) \subset Y $, l'applicazione $ f $ è un embedding topologico.
\end{proof}

Lo spazio quoziente $ \sfrac{I}{\{0,1\}} $ è uno spazio compatto perché quoziente dello spazio compatto $ I $ (quindi immagine di un compatto tramite un'applicazione continua, i.e. $ \pi $) ed $ \S^{1} $ è $ T_{2} $ o di Hausdorff perché incluso in $ \R^{2} $.

\paragraph{3) Quoziente non $ T_{2} $}

Siano $ S = \R $ e $ A = (0,+\infty) $ e consideriamo il quoziente $ \sfrac{S}{A} = \sfrac{\R}{A} $. Lo spazio quoziente non è di Hausdorff perché, se lo fosse, ogni suo punto sarebbe un chiuso cioè $ \sfrac{\R}{A} $ sarebbe\footnote{%
	Per un qualunque spazio, se questo è $ T_{2} $ o di Hausdorff allora è $ T_{1} $ e anche $ T_{0} $, i.e. $ T_{2} \implies T_{1} \implies T_{0} $.%
} $ T_{1} $. \\
Considerando la classe di equivalenza $ [A] \in \sfrac{\R}{A} $, questa è chiusa se la sua controimmagine tramite la proiezione è chiusa ma

\begin{equation}
	\pi^{-1} ([A]) = (0,+\infty)
\end{equation}

il quale è un aperto perciò $ [A] $ non è chiuso e lo spazio quoziente non è di Hausdorff.

\paragraph{4) Quoziente non $ N_{2} $}

Siano $ S = \R $ e $ A = \Q $: è possibile dimostrare che $ \sfrac{\R}{\Q} $ non è $ N_{2} $ e nemmeno $ N_{1} $.

\subsubsection{Relazione di equivalenza aperta}

Sia $ S $ uno spazio topologico e $ \sim $ una relazione di equivalenza su $ S $. Diremo che $ \sim $ è aperta se la proiezione sul quoziente $ \pi : S \to \sfrac{S}{\sim} $ è aperta.

\subsubsection{\textit{Esempio}}

Siano $ S = \R $ e $ A = \{-1,1\} $ e consideriamo $ \sfrac{S}{A} = \sfrac{\R}{\{-1,1\}} $. L'applicazione $ \pi : \R \to \sfrac{\R}{\{-1,1\}} $ non è aperta perché, per esempio, $ \pi((0,2)) $ non è aperto in $ \sfrac{\R}{\{-1,1\}} $ poiché la sua controimmagine è pari a

\begin{equation}
	\pi^{-1}(\pi((0,2))) = (0,2) \cup \{-1\}
\end{equation}

in quanto i punti $ \{-1,1\} $ sono equivalenti, ma questo insieme non è aperto in $ \R $.

\subsubsection{Proprietà $ N_{2} $ e $ T_{2} $ dei quozienti}

\begin{definition}\label{prop:n2-open-quot}
	Se lo spazio topologico $ S $ è $ N_{2} $ e la relazione di equivalenza $ \sim $ è aperta allora il quoziente $ \sfrac{S}{\sim} $ è $ N_{2} $.
\end{definition}

\begin{proof}
	Per dimostrare questa proposizione è necessario costruire una base numerabile per $ \sfrac{S}{\sim} $ sapendo che ne esiste una in $ S $. \\
	Sia $ \{B_{n}\}_{n \in \N} $ una base numerabile per $ S $. Consideriamo l'immagine $ \pi(B_{n}) $ per qualsiasi $ n \in \N $: questa è aperta in $ \sfrac{S}{\sim} $ in quanto $ \sim $ è aperta (e dunque anche $ \pi $ è aperta); inoltre è parte di una famiglia numerabile. A questo punto, $ \{\pi(B_{n})\}_{n \in \N} $ è una base numerabile per il quoziente $ \sfrac{S}{\sim} $: sia $ U $ aperto in $ \sfrac{S}{\sim} $, allora $ \pi^{-1}(U) $ è aperta in $ S $ e questo implica che
	
	\begin{align}
		\begin{split}
			\pi^{-1}(U) &= \bigcup_{j} B_{j} \\
			\pi(\pi^{-1}(U)) &= \pi \left( \bigcup_{j} B_{j} \right) \\
			U &= \bigcup_{j} \pi(B_{j})
		\end{split}
	\end{align}
\end{proof}

\begin{definition}\label{prop:diag-t2}
	Sia $ S $ uno spazio topologico. Lo spazio
	
	\begin{equation}
		R = \left\{ (x,y) \in S \times S \st x \sim y \right\} \subset S \times S
	\end{equation}

	è chiuso rispetto alla topologia prodotto in $ S \times S $ se e solo se il quoziente $ \sfrac{S}{\sim} $ è $ T_{2} $.
\end{definition}

\begin{proof}
	Supponiamo che il quoziente $ \sfrac{S}{\sim} $ sia $ T_{2} $ e dimostriamo che $ R $ è chiuso in $ S \times S $. \\
	Siano $ S \times S \setminus R $ il complementare dello spazio $ R $ e il punto $ (x,y) \in (S \times S) \setminus R $, per il quale $ x \nsim y $ dunque $ [x] \neq [y] $ o alternativamente $ \pi(x) \neq \pi(y) $. Essendo $ \sfrac{S}{\sim} $ di Hausdorff, abbiamo che
	
	\begin{equation}
		\E U, V \subset \sfrac{S}{\sim} \text{ aperti} \, \mid \, U \ni [x] \wedge V \ni [y] \wedge U \cap V = \emptyset
	\end{equation}
	
	Siccome la proiezione $ \pi $ è aperta, gli insiemi $ \pi^{-1}(U) \ni x $ e $ \pi^{-1}(V) \ni y $ sono aperti in $ S $. Considerando la topologia prodotto, abbiamo che l'insieme $ \pi^{-1}(U) \times \pi^{-1}(V) \ni (x,y) $ è aperto in $ S \times S $ perché prodotto di due aperti. Dato che $ U $ e $ V $ sono disgiunti, $ \pi^{-1}(U) \times \pi^{-1}(V) \subset S \times S \setminus R $, perché altrimenti ci sarebbero dei punti di $ \pi^{-1}(U) \times \pi^{-1}(V) $ che andrebbero a finire in punti equivalenti in $ S \times S $. Questo implica che $ S \times S \setminus R $ è aperto e, siccome $ S \times S $ è anch'esso aperto perché prodotto di spazi aperti, $ R $ è chiuso in quanto complementare di $ S \times S $. \\
	L'implicazione opposta della proposizione si ottiene invertendo le implicazioni di questa dimostrazione.
\end{proof}

Considerando la proposizione riportata sopra, nel caso in cui la relazione di equivalenza $ \sim $ sia l'uguaglianza $ = $ (da cui $ \sfrac{S}{\sim} = S $), otteniamo il seguente corollario

\begin{corollary}
	Uno spazio topologico $ S $ è $ T_{2} $ se e solo se l'\textit{insieme diagonale}
	
	\begin{equation}
		\Delta = \left\{ (x,x) \st x \in S \right\}
	\end{equation}
	
	è chiuso in $ S \times S $.
\end{corollary}

\subsection{Proiettivo reale come varietà differenziabile}

Il proiettivo reale può essere pensato come una varietà "astratta" in quanto non è un sottoinsieme di uno spazio euclideo o un prodotto di varietà. \\
Consideriamo lo spazio $ \R^{n+1} \setminus \{0\} $ con $ n \geqslant 0 $ e tutte le rette di $ \R^{n+1} \setminus \{0\} $ passanti per l'origine, i.e. tutti i sottospazi vettoriali di $ \R^{n+1} $ di dimensione unitaria. Ognuna di queste rette è determinata dall'origine e da un suo punto $ x $, i cui multipli daranno tutti i punti della retta stessa. \\
Presi due punti $ x, y \in \R^{n+1} \setminus \{0\} $, possiamo  quindi individuare la seguente condizione di equivalenza

\begin{equation}
	x \sim y \, \iff \, \E \lambda \in \R \setminus \{0\} \, \mid \, y = \lambda x
\end{equation}

ciò significa che due punti sono equivalenti se e solo se sono proporzionali (e dunque appartengono alla stessa retta). \\
Il \textit{proiettivo reale} è dunque definito tramite il quoziente

\begin{equation}
	\rp{n} = \dfrac{\R^{n+1} \setminus \{0\}}{\sim}
\end{equation}

con la topologia quoziente indotta dalla relazione di equivalenza $ \sim $. \\
Vogliamo ora dimostrare che $ \rp{n} $ sia una varietà differenziabile compatta e connessa di dimensione $ n $.

\subsubsection{Omeomorfismo tra $ \rp{n} $ e $ \sfrac{\S^{n}}{\sim_{a}} $}\label{ss-sec:homeo-rp-qsph}

Iniziamo col dimostrare la compattezza e la connessione di $ \rp{n} $ mediante il suo omeomorfismo con la sfera quozientata.

\begin{definition}
	Lo spazio proiettivo reale è omeomorfo alla sfera $ n $-dimensionale quozientata alla relazione di equivalenza \textit{antipodale} $ \sim_{a} $ (due punti della sfera sono equivalenti se opposti), i.e.
	
	\begin{equation}
		\rp{n} \simeq \sfrac{\S^{n}}{\sim_{a}}
	\end{equation}

	con la relazione di equivalenza antipodale definita come
	
	\begin{equation}
		x \sim_{a} y \iff y = \pm x
	\end{equation}
\end{definition}

\begin{proof}
	Consideriamo innanzitutto i seguenti diagrammi di domini e mappe:
	
	\diagr{%
			\R^{n+1} \setminus \{0\} \arrow[rrrr, "f", bend left] \arrow[rrd, "g"] \arrow[rrddd, "\pi", bend right] \&  \&                                               \&  \& \sfrac{\S^{n}}{\sim_{a}} \&  \& x \arrow[rrrr, "f", maps to, bend left] \arrow[rrd, "g", maps to] \arrow[rrddd, "\pi", maps to, bend right] \&  \&                                                                              \&  \& {[x]_{a}} \\
			\&  \& \S^{n} \arrow[dd, "h"] \arrow[rru, "\pi_{a}"] \&  \&                          \&  \&                                                                                                             \&  \& \dfrac{x}{\norm{x}} \arrow[rru, "\pi_{a}", maps to] \arrow[dd, "h", maps to] \&  \&           \\
			\&  \&                                               \&  \&                          \&  \&                                                                                                             \&  \&                                                                              \&  \&           \\
			\&  \& \rp{n} \arrow[rruuu, "\tilde{f}", bend right] \&  \&                          \&  \&                                                                                                             \&  \& {[x]} \arrow[rruuu, "\tilde{f}", maps to, bend right]                        \&  \&          
			}

	Perché $ \rp{n} \simeq \sfrac{\S^{n}}{\sim_{a}} $ è necessario che $ \tilde{f} $ sia un omeomorfismo, i.e. una bigezione continua con inversa continua. \\
	Prendiamo l'applicazione
	
	\map{g}
		{\R^{n+1} \setminus \{0\}}{\S^{n}}
		{x}{\dfrac{x}{\norm{x}}}

	Se consideriamo un qualunque rappresentante di $ \R^{n+1} \setminus \{0\} $, questo viene mappato tramite $ g $ in $ \pm x $, perciò prendiamo anche l'applicazione
	
	\map{f}
		{\R^{n+1} \setminus \{0\}}{\sfrac{\S^{n}}{\sim_{a}}}
		{x}{[x]_{a}}

	la quale considera la classe dei punti antipodali nella sfera $ \S^{n} $. L'applicazione $ \tilde{f} $ che va dal proiettivo reale al quoziente è dunque ben definita tramite
	
	\begin{equation}
		\tilde{f}([x]) = [g(x)]_{a} \qq{in quanto} \tilde{f} \circ \pi = \pi_{a} \circ g
	\end{equation}

	Per mostrare che $ \tilde{f} $ sia continua, dimostriamo che $ f $ lo sia: siccome vale la relazione $ f = \pi_{a} \circ g $, $ f $ è dunque continua, poiché le componenti di $ g $ sono continue e la proiezione sul quoziente è continua. \\
	L'ultimo passo è mostrare che $ \tilde{f} $ sia una bigezione\footnote{%
		Non possiamo utilizzare il lemma dell'applicazione chiusa perché altrimenti dovremmo utilizzare come assunzione che $ \rp{n} $ sia compatto, ma questa è la tesi della dimostrazione.%
	}: per fare ciò prendiamo la sua inversa

	\map{\tilde{f}^{-1}}
		{\sfrac{\S^{n}}{\sim_{a}}}{\rp{n}}
		{[x]_{a}}{[x]}

	la quale è continua perché composizione di applicazioni continue
	
	\begin{equation}
		\tilde{f}^{-1} = (\pi_{a} \circ g \circ \pi^{-1})^{-1} = \pi \circ g^{-1} \circ \pi_{a}^{-1}
	\end{equation}

	Analogamente, possiamo dire che sia continua perché due punti antipodali appartengono obbligatoriamente alla stessa retta, i.e.
	
	\begin{equation}
		p,q \in [x]_{a} \implies p,q \in [x]
	\end{equation}

	Avendo dimostrato che $ \tilde{f} $ è una bigezione\footnote{%
		Essendo $ f^{-1} $ inversa sinistra e inversa destra di $ f $, quest'ultima è rispettivamente iniettiva e suriettiva, dunque una bigezione.%
	} continua, otteniamo che $ \rp{n} \simeq \sfrac{\S^{n}}{\sim_{a}} $.
\end{proof}

\begin{corollary}
	$ \rp{n} $ è compatto e connesso, in quanto lo è la sfera\footnote{%
		La sfera $ \S^{n} $ è connessa perché la sfera è densa nella sfera meno un punto (tramite proiezione stereografica in $ \R^{n} $) e la chiusura di uno spazio connesso è ancora connessa; per quanto riguarda la compattezza, il \textit{teorema di Heine-Borel} afferma che un sottoinsieme di $ \R^{n} $ è compatto se e solo se è chiuso e limitato, e.g. la sfera.%
	} $ \S^{n} $ e la topologia quoziente preserva queste proprietà.
\end{corollary}

\subsubsection{\textit{Esempi}}

Non è possibile rappresentare $ \rp{n} $ tramite esempi nello spazio tridimensionale per $ n > 2 $, dunque mostreremo solo i casi per $ n = 1 $ e  $ n = 2 $.

\paragraph{1) Proiettivo reale unidimensionale}

Se prendiamo $ n = 1 $, $ \rp{1} \simeq \sfrac{\S^{1}}{\sim_{a}} $: si può identificare $ \sfrac{\S^{1}}{\sim_{a}} $ con l'intervallo $ I $ (omeomorfo alla semicirconferenza) quozientato i due punti estremi $ \{A,B\} $, i.e. $ \sfrac{\S^{1}}{\sim_{a}} \simeq \sfrac{I}{\{A,B\}} $, in modo tale che si prenda solo uno dei due punti antipodali.

\paragraph{2) Proiettivo reale bidimensionale}

Se prendiamo $ n = 2 $, $ \rp{2} \simeq \sfrac{\S^{2}}{\sim_{a}} $. Per visualizzarlo, prendiamo la calotta superiore di $ \S^{2} $, la quale è omeomorfa al disco unitario $ D_{1} $. Il disco unitario, pieno della proiezione della semicalotta in cui sono identificati i punti antipodali della circonferenza, rappresenta dunque il proiettivo reale $ \rp{2} $.

\subsubsection{Proprietà di $ \rp{n} $}

Per dimostrare che $ \rp{n} $ sia una varietà differenziabile dobbiamo prima dimostrare che sia una varietà topologica, dunque $ N_{2}+T_{2} $ con un atlante topologico, e poi far vedere che i cambi di carta siano lisci.

\paragraph{Proprietà $ N_{2} $}

La dimostrazione che il proiettivo reale sia $ N_{2} $ passa per la proiezione

\begin{equation}
	\pi : \R^{n+1} \setminus \{0\} \to \dfrac{\R^{n+1} \setminus \{0\}}{\sim} = \rp{n}
\end{equation}

in quanto questa applicazione preserva la proprietà di numerabilità $ N_{2} $: se lo spazio di partenza $ \R^{n+1} \setminus \{0\} $ è $ N_{2} $ (lo è perché sottoinsieme di $ \R^{n+1} $, il quale è $ N_{2} $) allora, se l'applicazione $ \pi $ è aperta\footnote{%
	Equivalentemente, se la relazione di equivalenza $ \sim $ è aperta.%
}, lo è anche il proiettivo reale. \\
Perché $ \pi $ sia aperta, questa deve mandare aperti in aperti: sia l'aperto $ V \subset \R^{n+1} \setminus \{0\} $ e consideriamo l'insieme $ \pi(V) $. Nella topologia quoziente, l'immagine $ \pi(V) $ è aperta se la sua controimmagine $ \pi^{-1}(\pi(V)) $ è aperta: $ \pi(x) $ con $ x \in V $ è l'identificazione di tutti i punti che appartengono alla stessa retta di $ x $; prendendo $ \pi^{-1}(\pi(x)) $ otteniamo non solo il punto $ x $ ma tutti i punti che appartengono alla retta identificata da $ x $ stesso, dunque considerando tutti i punti dell'aperto $ V $ otteniamo \\
%
\sbs{0.5}{%
			\begin{equation}
				\pi^{-1}(\pi(V)) = \bigcup_{\lambda \in \R \setminus \{0\}} \{\lambda V\}
			\end{equation}
			
			che corrisponde al cono di rette passanti per l'origine individuato dai punti dell'aperto $ V $.
			}
	{0.5}{%
			\img{0.8}{img15}
			}

Fissando $ \lambda $, l'insieme

\begin{equation}
	\lambda V = \left\{ \lambda x \st x \in V \right\} \subset \R^{n+1} \setminus \{0\}
\end{equation}

è aperto perché omeomorfo a $ V $ tramite l'applicazione continua con inversa continua

\sbs{0.5}{%
			\map{\varphi_{\lambda}}
				{V}{\lambda V}
				{x}{\lambda x}
			}
	{0.5}{%
			\map{\varphi_{\lambda}^{-1}}
				{\lambda V}{V}
				{\lambda x}{x}
			}

Essendo $ \lambda V $ aperto in $ \R^{n+1} \setminus \{0\} $, $ \pi^{-1}(\pi(V)) $ è aperto e dunque la proiezione $ \pi $ è aperta. \\
Se lo spazio di partenza $ \R^{n+1} \setminus \{0\} $ è $ N_{2} $ e la relazione di equivalenza $ \sim $ è aperta, allora lo spazio quoziente $ \rp{n} $ è $ N_{2} $.

\paragraph{Proprietà $ T_{2} $}

Sapendo che la proiezione $ \pi $ è aperta, possiamo usare la Proposizione \ref{prop:diag-t2}, i.e. l'insieme

\begin{equation}
	R = \left\{ (x,y) \in \R^{n+1} \setminus \{0\} \times \R^{n+1} \setminus \{0\} \st y=\lambda x, \, \lambda \in \R \setminus \{0\} \right\} \subset \R^{n+1} \setminus \{0\} \times \R^{n+1} \setminus \{0\}
\end{equation}

è chiuso in $ \R^{n+1} \setminus \{0\} \times \R^{n+1} \setminus \{0\} $ se e solo se il proiettivo reale è $ T_{2} $. \\
Per dimostrare che $ R $ sia chiuso, consideriamo l'applicazione

\map{\varphi}
	{\R^{n+1} \setminus \{0\} \times \R^{n+1} \setminus \{0\}}{M_{n+1,2}(\R) = \R^{2(n+1)}}
	{(x,y)}{\bmqty{x & y}}

che associa alle coppie di vettori\footnote{%
	Possiamo parlare di vettori invece che di punti perché $ T_{p}(\R^{n}) = \R^{n} $.%
} $ (x,y) $ le matrici con $ (n+1) $ righe e $ 2 $ colonne costituite dai vettori $ x $ e $ y $ posti in colonne affiancate, i.e.

\begin{equation}
	(x,y) = \left( \bmqty{x^{1} \\ \vdots \\ x^{n+1}} , \, \bmqty{y^{1} \\ \vdots \\ y^{n+1}} \right) %
	\quad \stackrel{\varphi}{\longmapsto} \quad %
	\bmqty{x & y} = \bmqty{x^{1} & y^{1} \\ \vdots & \vdots \\ x^{n+1} & y^{n+1}}
\end{equation}

L'applicazione $ \varphi $ è un omeomorfismo perché la topologia prodotto di spazi euclidei coincide con la topologia euclidea, i.e. $ \R^{n} \times \R^{m} \simeq \R^{n+m} $ e quindi sono topologicamente equivalenti (ogni aperto di uno è aperto dell'altro). \\
A questo punto, dimostrando che l'immagine di $ R $ tramite $ \varphi $ è un chiuso in $ M_{n+1,2}(\R) $, otterremo che il proiettivo reale è di Hausdorff. L'insieme in questione è il seguente

\begin{equation}
	\varphi(R) = \left\{ \bmqty{x & y} \in M_{n+1,2}(\R) \st \rank \left( \bmqty{x & y} \right) = 1 \right\}
\end{equation}

La condizione per cui il rango delle matrici di questo insieme sia unitario è dovuta al fatto che queste matrici hanno le due colonne proporzionali tra loro, in quanto $ y= \lambda x $ con $ \lambda \in \R $.

\begin{lemma}\label{lemma:clos-matrix}
	L'insieme delle matrici reali di rango minore o uguale a $ r $
	
	\begin{equation}
		X_{r}(n,m) = \left\{ A \in M_{n,m}(\R) \st \rank(A) \leqslant r \right\} %
		\qcomma r \leqslant \min\{n,m\}
	\end{equation}

	è chiuso in $ M_{n,m}(\R) = \R^{n m} $.
\end{lemma}

\begin{proof}
	Consideriamo l'applicazione
	
	\map{f}
		{M_{n,m}(\R)}{\R^{d}}
		{A}{(m_{11}, \dots, m_{nm})}

	che mappa le matrici $ A \in M_{n,m}(\R) $ in $ d $-uple\footnote{%
		Il numero totale di minori di ordine $ r $ per una matrice $ A \in M_{n,m}(\R) $ è $ d = \binom{n}{r+1} \binom{m}{r+1} $.%
	} di minori $ m_{ij} \doteq \det(A_{ij}) $ di $ A $ di ordine $ r + 1 $, dove le $ A_{ij} $ sono le sottomatrici ricavate da $ A $ rimuovendo la $ i $-esima riga e la $ j $-esima colonna. Questa applicazione è continua perché composizione di due applicazioni continue: l'applicazione che rimuove righe e colonne alla matrice $ A $ e il determinante di questa stessa matrice. \\
	Una matrice $ A $ ha rango $ \rank(A) \leqslant r $ se e solo se tutti i minori di ordine $ r+1 $ sono nulli, dunque l'insieme $ X_{r}(n,m) $ è la controimmagine di $ 0 \in \R^{d} $ tramite l'applicazione continua $ f $: essendo $ 0 $ chiuso in $ \R^{d} $ poiché $ \R^{d} $ è di Hausdorff, allora anche la sua controimmagine tramite l'applicazione continua $ f $, i.e. $ f^{-1}(0) = X_{r}(n,m) $, è chiusa in $ M_{n,m}(\R) $.
\end{proof}

Seguendo questo ragionamento, possiamo pensare a $ \varphi(R) $ come l'intersezione dell'immagine dello spazio di Hausdorff $ \R^{n+1} \setminus \{0\} \times \R^{n+1} \setminus \{0\} $ tramite $ \varphi $ con l'insieme delle matrici a due colonne di rango unitario $ X_{1}(n+1,2) $, il quale è un chiuso, i.e.

\begin{equation}
	\varphi(R) = X_{1}(n+1,2) \cap \varphi(\R^{n+1} \setminus \{0\} \times \R^{n+1} \setminus \{0\})
\end{equation}

Essendo dunque $ \varphi(R) $ intersezione di un chiuso con lo spazio d'interesse, questo rende $ \varphi(R) $ chiuso in $ \varphi(\R^{n+1} \setminus \{0\} \times \R^{n+1} \setminus \{0\}) $, dunque $ R $ è chiuso in $ \R^{n+1} \setminus \{0\} \times \R^{n+1} \setminus \{0\} $ in quanto $ \varphi $ è un omeomorfismo: questo prova che $ \rp{n} $ sia $ T_{2} $.

\paragraph{Altante per $ \rp{n} $}

Costruiamo ora un atlante topologico per il proiettivo reale. \\
Siano gli $ n+1 $ insiemi

\begin{equation}
	U_{i} = \{ [(a^{0},\dots,a^{n})] = \pi(a^{0},\dots,a^{n}) \in \rp{n} \mid a^{i} \neq 0 \} \subset \rp{n} \qcomma i=0,\dots,n
\end{equation}

i cui elementi sono le classi di $ n+1 $-uple con componente $ i $-esima diversa da zero. Si può anche pensare a questi insiemi come $ U_{i} = \pi(V_{i}) $, dove

\begin{equation}
	V_{i} = \{ (a^{0},\dots,a^{n}) \in \R^{n+1} \setminus \{0\} \mid a^{i} \neq 0 \} \subset \R^{n+1} \setminus \{0\} \qcomma i=0,\dots,n
\end{equation}

Gli insiemi $ U_{i} $ sono aperti in $ \rp{n} $ perché la proiezione $ \pi $ è aperta e gli insiemi $ V_{i} $ sono aperti in $ \R^{n+1} \setminus \{0\} $. Questi insiemi sono un ricoprimento per $ \rp{n} $, i.e.

\begin{equation}
	\bigcup_{i=0}^{n} U_{i} = \rp{n}
\end{equation}

in quanto
%
\begin{equation}
	\bigcup_{i=0}^{n} V_{i} = \R^{n+1} \setminus \{0\}
\end{equation}

perché ogni punto di $ \R^{n+1} \setminus \{0\} $ avrà almeno una coordinata diversa da zero e, conseguentemente, la sua proiezione su $ \rp{n} $ avrà anch'essa almeno una coordinata diversa da zero. \\
Siano le applicazioni

\map{\varphi_{i}}
	{U_{i}}{\R^{n}}
	{[(a^{0},\dots,a^{n})]}{\left( \dfrac{a^{0}}{a^{i}}, \dfrac{a^{1}}{a^{i}}, \dots, \dfrac{a^{i-1}}{a^{i}}, \dfrac{a^{i+1}}{a^{i}}, \dots, \dfrac{a^{n}}{a^{i}} \right)}

con $ i=0,\dots,n $, cioè eliminiamo l'$ i $-esima componente dal vettore nell'immagine e dividiamo tutte le altre per $ a^{i} $. Queste applicazioni sono ben definite in quanto la proporzionalità tra gli elementi della classe di equivalenza $ [(a^{0},\dots,a^{n})] $ viene assorbita nei rapporti tra le componenti nell'immagine\footnote{%
	Gli elementi all'interno di una stessa classe appartengono tutti allo stesso spazio vettoriale di dimensione unitaria quindi sono proporzionali gli uni agli altri.%
}. \\
Essendo l'unione degli insiemi $ U_{i} $ un ricoprimento per $ \rp{n} $, mostrando che le $ \varphi_{i} $ sono degli omeomorfismi, dimostriamo che $ \mathfrak{M} = \{(U_{i},\varphi_{i})\} $ è un atlante topologico per $ \rp{n} $. \\
Queste applicazioni fanno parte dei diagrammi

\diagr{%
		V_{i} \arrow[rr, "\psi_{i}"] \arrow[dd, "\pi"'] \&  \& \R^{n} \\
		\&  \&        \\
		U_{i} \arrow[rruu, "\varphi_{i}"']                 \&  \&       
		}

dove

\map{\psi_{i}}
	{V_{i}}{\R^{n}}
	{(a^{0},\dots,a^{n})}{\left( \dfrac{a^{0}}{a^{i}}, \dfrac{a^{1}}{a^{i}}, \dots, \dfrac{a^{i-1}}{a^{i}}, \dfrac{a^{i+1}}{a^{i}}, \dots, \dfrac{a^{n}}{a^{i}}, \right)}

con $ i=0,\dots,n $, le quali sono continue perché tutte le loro componenti sono continue. A questo punto, le $ \varphi_{i} $ rendono commutativo questo diagramma (le $ \psi_{i} $ sono costanti sulle classi di equivalenza) perciò sono continue. \\
Le applicazioni inverse (anch'esse continue) sono

\map{\varphi_{i}^{-1}}
	{\R^{n}}{U_{i}}
	{(x^{1},\dots,x^{n})}{[(x^{1},\dots,x^{i},1,x^{i+1},\dots,x^{n})]}

siccome è presente la componente $ i $-esima uguale a 1, la classe $ [(x^{1},\dots,x^{i},1,x^{i+1},\dots,x^{n})] $ appartiene a $ \rp{n} $ anche se tutte le altre coordinate sono nulle. Queste applicazioni sono le inverse di $ \varphi_{i} $ perché

\begin{equation}
		(\varphi_{i} \circ \varphi_{i}^{-1})(x^{1},\dots,x^{n}) = \varphi_{i}([(x^{1},\dots,x^{i},1,x^{i+1},\dots,x^{n})]) %
		= (x^{1},\dots,x^{n})
\end{equation}

\begin{align}
	\begin{split}
		(\varphi^{-1}_{i} \circ \varphi_{i})([(a^{0},\dots,a^{n})]) &= \varphi_{i}^{-1} \left( \dfrac{a^{0}}{a^{i}}, \dfrac{a^{1}}{a^{i}}, \dots, \dfrac{a^{i-1}}{a^{i}}, \dfrac{a^{i+1}}{a^{i}}, \dots, \dfrac{a^{n}}{a^{i}} \right) \\
		&= \left[ \left( \dfrac{a^{0}}{a^{i}}, \dfrac{a^{1}}{a^{i}}, \dots, \dfrac{a^{i-1}}{a^{i}}, 1, \dfrac{a^{i+1}}{a^{i}}, \dots, \dfrac{a^{n}}{a^{i}} \right) \right] \\
		&= \left[ a^{i}\left( \dfrac{a^{0}}{a^{i}}, \dfrac{a^{1}}{a^{i}}, \dots, \dfrac{a^{i-1}}{a^{i}}, 1, \dfrac{a^{i+1}}{a^{i}}, \dots, \dfrac{a^{n}}{a^{i}} \right) \right] \\
		&= [(a^{0},\dots,a^{i-1},a^{i},a^{i+1},\dots,a^{n})] \\
		&= [(a^{0},\dots,a^{n})]
	\end{split}
\end{align}

Dunque abbiamo dimostrato che le $ \varphi_{i} $ sono omeomorfismi e quindi $ \mathfrak{M} = \{(U_{i},\varphi_{i})\}_{i=0,\dots,n} $ è un atlante topologico per $ \rp{n} $. \\
Dimostriamo ora che $ \mathfrak{M} $ è un atlante differenziabile di $ \rp{n} $: per fare ciò dobbiamo far vedere che le sue carte siano $ C^{\infty} $-compatibili tra loro. Siano $ i,j=0,\dots,n $ con $ i<j $ e consideriamo i cambi di carta

\begin{gather}
	\varphi_{i} \circ \varphi_{j}^{-1} : \varphi_{j}(U_{i} \cap U_{j}) \to \varphi_{i}(U_{i} \cap U_{j}) \\
	\varphi_{j} \circ \varphi_{i}^{-1} : \varphi_{i}(U_{i} \cap U_{j}) \to \varphi_{j}(U_{i} \cap U_{j})
\end{gather}

dove un elemento in $ U_{i} \cap U_{j} $ ha sia la $ i $-esima che la $ j $-esima componente diversa da zero. \\
Dunque

\begin{align}
	\begin{split}
		(\varphi_{i} \circ \varphi_{j}^{-1})(x^{1},\dots,x^{n}) &= \varphi_{i}([(x^{1},\dots,x^{j},1,x^{j+1},\dots,x^{n})]) \\
		&= \left( \dfrac{x^{1}}{x^{i+1}}, \dots, \dfrac{x^{i}}{x^{i+1}}, \dfrac{x^{i+2}}{x^{i+1}}, \dots, \dfrac{x^{j}}{x^{i+1}}, \dfrac{1}{x^{i+1}}, \dfrac{x^{j+1}}{x^{i+1}}, \dots, \dfrac{x^{n}}{x^{i+1}} \right)
	\end{split}
\end{align}

\begin{align}
	\begin{split}
		(\varphi_{j} \circ \varphi_{i}^{-1})(x^{1},\dots,x^{n}) &= \varphi_{j}([(x^{1},\dots,x^{i},1,x^{i+1},\dots,x^{n})]) \\
		&= \left( \dfrac{x^{1}}{x^{i}}, \dots, \dfrac{x^{i-1}}{x^{j}}, \dfrac{1}{x^{j}}, \dfrac{x^{i}}{x^{j}}, \dots, \dfrac{x^{j-1}}{x^{j}}, \dfrac{x^{j+1}}{x^{j}}, \dots, \dfrac{x^{n}}{x^{j}} \right)
	\end{split}
\end{align}

entrambi i cambi di carta sono lisci quindi $ \mathfrak{M} $ è un atlante differenziabile: questo dimostra l'esistenza di un atlante differenziabile massimale il che rende $ \rp{n} $ una varietà differenziabile.

\subsection{Grassmanniana come varietà differenziabile}

Siano $ k,n \in \N $ con $ k \leqslant n $, definiamo la \textit{grassmanniana} $ G(k,n) $ come l'insieme di tutti i sottospazi vettoriali di $ \R^{n} $ di dimensione $ k $. \\
Essendo $ \rp{n} $ l'insieme di tutti i sottospazi vettoriali di dimensione unitaria (i.e. rette) di $ \R^{n+1} $, abbiamo la seguente uguaglianza
%
\begin{equation}
	G(1,n+1) = \rp{n}
\end{equation}

dunque la grassmanniana può essere pensata come la generalizzazione del proiettivo reale. \\
Vogliamo ora dotare la grassmanniana di una struttura di varietà differenziabile, in modo analogo al proiettivo reale.

\begin{remark}
	Un $ k $-spazio vettoriale\footnote{%
		Cioè uno spazio vettoriale di dimensione $ k $.%
	} $ S $ di $ \R^{n} $ è determinato da $ k $ vettori $ \{a_{1},\dots,a_{k}\} $ linearmente indipendenti di $ \R^{n} $ o, alternativamente, da una matrice $ A = \bmqty{ a_{1} & \cdots & a_{k} } \in M_{n,k}(\R^{n}) $ di rango $ k $, con colonne i $ k $ vettori linearmente indipendenti citati sopra.
\end{remark}

Due matrici $ A $ e $ B $ rappresentano lo stesso $ k $-spazio vettoriale se e solo se le colonne di una sono linearmente dipendenti dalle colonne dell'altra, i.e. $ B = A g $ dove $ g \in GL_{k}(\R) $ ($ g $ è invertibile), perciò le matrici $ A $ e $ B $ sono legate da una trasformazione lineare (detta anche \textit{cambio di base}) nello spazio generato dai vettori colonna che compongono le matrici. \\
Siano l'insieme delle matrici di $ n $ righe e $ k $ colonne di rango $ k $

\begin{equation}
	F(k,n) = \left\{ A \in M_{n,k}(\R) \st \rank(A) = k \right\} \subset M_{n,k}(\R) = \R^{nk}
\end{equation}

e la relazione di equivalenza $ \sim $ definita come

\begin{equation}
	A \sim B \, \iff \, \E g \in GL_{k}(\R) \, \mid \, B = A g
\end{equation}

Possiamo dunque affermare che

\begin{equation}
	G(k,n) = \sfrac{F(k,n)}{\sim}
\end{equation}

Per ogni $ k $-spazio $ S $ esiste dunque una classe di equivalenza $ [A] $ di tutte le matrici legate tra loro da una trasformazione lineare le cui colonne generano lo spazio stesso. Tra la classe di equivalenza e lo spazio è presente una bigezione. \\
L'intento è di dimostrare che $ G(k,n) $ sia una varietà differenziabile di dimensione

\begin{equation}
	\dim (G(k,n)) = k (n-k)
\end{equation}

Come primo passo, associamo la topologia quoziente a $ G(k,n) $: l'insieme $ F(k,n) $ è un aperto in quanto il suo complementare

\begin{equation}
	M_{n,k}(\R) \setminus F(k,n) = \left\{ A \in M_{n,k}(\R) \st \rank(A) \leqslant k - 1 \right\} = X_{k-1} (n,k)
\end{equation}

è chiuso per il Lemma \ref{lemma:clos-matrix}. \\
La grassmanniana è uno spazio topologico connesso e compatto, per la dimostrazione vedi Esercizio \ref{exer2-5}.

\paragraph{Proprietà $ N_{2} $}

Per dimostrare che lo spazio $ G(k,n) $ sia $ N_{2} $, usiamo la Proposizione \ref{prop:n2-open-quot}: $ F(k,n) $ è $ N_{2} $ in quanto sottospazio di $ M_{n,k}(\R) $ e la proiezione $ \pi : F(k,n) \to G(k,n) $ è aperta (equivalentemente $ \sim $ è aperta).
La dimostrazione che $ \pi $ sia aperta è analoga a quella per il proiettivo: preso un aperto, la sua immagine deve essere aperta, il quale si dimostra facendo vedere che la controimmagine di quest'ultima sia aperta. \\
Sia $ V $ un aperto di $ F(k,n) $, possiamo scrivere

\begin{equation}
	\pi^{-1}(\pi(V)) = \bigcup_{g \in GL_{k}(\R)} V g
\end{equation}

dove

\begin{equation}
	V g = \{ A g \in M_{n,k}(\R) \mid A \in V \wedge g \in GL_{k}(\R) \}
\end{equation}

Prendiamo l'applicazione

\map{\varphi_{g}}
	{F(k,n)}{F(k,n)}
	{A}{A g}

questo è un omeomorfismo dunque $ \varphi_{g}(V) = V g $ è aperto perché $ V $ è aperto. Questo dimostra che $ \pi^{-1}(\pi(V)) $ è aperto in quanto unione di aperti e quindi $ G(k,n) $ è $ N_{2} $.

\paragraph{Proprietà $ T_{2} $}

Per dimostrare che lo spazio $ G(k,n) $ sia $ T_{2} $, usiamo la Proposizione \ref{prop:diag-t2}: sia l'insieme

\begin{equation}
	R = \{ (A,B) \in F(k,n) \times F(k,n) \mid B = A g, \, g \in GL_{k}(\R) \} \subset F(k,n) \times F(k,n)
\end{equation}

per dimostrare che sia chiuso, consideriamo l'applicazione

\map{\varphi}
	{M_{n,k}(\R) \times M_{n,k}(\R)}{M_{n,2k}(\R)}
	{(A,B)}{\bmqty{A & B}}

cioè $ \varphi $ "affianca" due matrici rendendole un'unica matrice con il doppio delle colonne.

\sbs{0.35}{%
			Essendo $ \varphi $ un omeomorfismo, per dimostrare che $ R $ sia chiuso in $ F(k,n) \times F(k,n) $ basta dimostrare che $ \varphi(R) $ sia chiuso in $ \varphi(F(k,n) \times F(k,n)) $.
			}
	{0.6}{%
			\img{0.9}{img17}
			}

Dal Lemma \ref{lemma:clos-matrix} prendiamo la definizione del seguente insieme

\begin{equation}
	X_{r}(n,m) = \left\{ A \in M_{n,m}(\R) \st \rank(A) \leqslant r \right\} %
	\qcomma r \leqslant \min\{n,m\}
\end{equation}

dunque l'immagine di $ R $ tramite $ \varphi $ è pari a

\begin{equation}
	\varphi(R) = \{ A \in M_{n,2k}(\R) \mid \rank(A) \leqslant k \} = \varphi(F(k,n) \times F(k,n)) \cap X_{k}(n,2k)
\end{equation}

in quanto $ \varphi(R) $ è l'insieme delle matrici $ n \times 2k $ le quali hanno rango al massimo uguale a $ k $, in quanto formate da due matrici di rango $ k $ legate tra loro da $ g $. Essendo $ X_{k}(n,2k) $ chiuso, allora anche $ \varphi(R) $ è chiuso in $ \varphi(F(k,n) \times F(k,n)) $ e dunque $ R $ è chiuso in $ F(k,n) \times F(k,n) $. Questo dimostra che $ G(k,n) $ è $ T_{2} $.

\subsubsection{Calcolo esplicito per $ G(2,4) $}

A titolo di esempio, mostriamo ora che $ G(k,n) $ è una varietà differenziabile per $ k=2 $ e $ n=4 $ (il ragionamento è analogo per altri valori di $ k $ e $ n $); in questo caso $ \dim(G(2,4)) = 2(4-2) = 4 $. \\
Siano gli aperti di $ G(2,4) $

\begin{equation}
	U_{ij} = \left\{ [A] \in G(2,4) \st A_{ij} \in GL_{2}(\R) \right\}
\end{equation}

i.e. $ A_{ij} $ è invertibile, dove $ A_{ij} $ è la matrice quadrata $ 2 \times 2 $ che si ottiene considerando le righe $ i $ e $ j $ di $ A $, i.e.

\begin{equation}
	[A] = \left[ \bmqty{ a_{11} & a_{12} \\ a_{21} & a_{22} \\ a_{31} & a_{32} \\ a_{41} & a_{42} } \right] %
	\quad \longrightarrow \quad %
	\begin{cases}
		A_{12} = \bmqty{ a_{11} & a_{12} \\ a_{21} & a_{22} } & A_{23} = \bmqty{ a_{21} & a_{22} \\ a_{31} & a_{32} } \\ \\
		%
		A_{13} = \bmqty{ a_{11} & a_{12} \\ a_{31} & a_{32} } & A_{24} = \bmqty{ a_{21} & a_{22} \\ a_{41} & a_{42} } \\ \\
		%
		A_{14} = \bmqty{ a_{11} & a_{12} \\ a_{41} & a_{42} } & A_{34} = \bmqty{ a_{31} & a_{32} \\ a_{41} & a_{42} }
	\end{cases}
\end{equation}

Considerando $ B = A g $ (i.e. $ A \sim B $), abbiamo che $ A_{ij} $ è invertibile se e solo se $ B_{ij} $ è invertibile

\begin{gather}
	A g = B \nonumber \\
	\bmqty{ a_{11} & a_{12} \\ a_{21} & a_{22} \\ a_{31} & a_{32} \\ a_{41} & a_{42} } \bmqty{ g_{11} & g_{12} \\ g_{21} & g_{22} } %
	= \bmqty{ a_{11} g_{11} + a_{12} g_{21} & a_{11} g_{12} + a_{12} g_{22} \\ a_{21} g_{11} + a_{22} g_{21} & a_{21} g_{12} + a_{22} g_{22} \\ a_{31} g_{11} + a_{32} g_{21} & a_{31} g_{12} + a_{32} g_{22} \\ a_{41} g_{11} + a_{42} g_{21} & a_{41} g_{12} + a_{42} g_{22} } %
	= \bmqty{ b_{11} & b_{12} \\ b_{21} & b_{22} \\ b_{31} & b_{32} \\ b_{41} & b_{42} }
\end{gather}

perciò $ A_{ij} g = B_{ij} $ e dunque la scelta del rappresentante di $ [A] $ è irrilevante. \\
Gli insiemi $ U_{ij} $ sono aperti in quanto immagini di aperti $ V_{ij} $ tramite la proiezione $ U_{ij} = \pi(V_{ij}) $, dove 

\begin{equation}
	V_{ij} = \left\{ A \in F(2,4) \st A_{ij} \in GL_{2}(\R) \right\}
\end{equation}

è aperto in $ F(2,4) $ perché complementare di un chiuso. \\
Inoltre abbiamo che l'unione degli $ U_{ij} $ è un ricoprimento di $ G(2,4) $

\begin{equation}
	G(2,4) = \bigcup_{\substack{ i,j = 1 \\ i < j }}^{4} U_{ij}
\end{equation}

in quanto l'unione degli $ V_{ij} $ è un ricoprimento di $ F(2,4) $

\begin{equation}
	F(2,4) = \bigcup_{\substack{ i,j = 1 \\ i < j }}^{4} V_{ij}
\end{equation}

perché una matrice, per essere di rango $ k = 2 $ e quindi appartenere a $ F(2,4) $, deve avere i suoi minori $ m_{ij} = \det(V_{ij}) $ di ordine 2 diversi da zero. \\
Per costruire le carte dell'atlante, consideriamo l'applicazione

\map{\varphi_{12}}
	{U_{12}}{M_{2}(\R) = \R^{4}}
	{[A]}{A_{34} (A_{12})^{-1}}

la quale è ben definita in quanto, se considerassimo un rappresentante diverso $ B = A g $, avremmo che $ B_{34} = A_{34} g $ e $ B_{12} = A_{12} g $, da cui

\begin{align}
	\begin{split}
		A_{34} (A_{12})^{-1} &= B_{34} \, g \, (B_{12} \, g)^{-1} \\
		&= B_{34} \, g \, g^{-1} \, (B_{12})^{-1} \\
		&= B_{34} (B_{12})^{-1}
	\end{split}
\end{align}

Le altre applicazioni sono analoghe:

\map{\varphi_{ij}}
	{U_{ij}}{M_{2}(\R) = \R^{4}}
	{[A]}{A_{lm} (A_{ij})^{-1}}

dove $ \{i,j,l,m\}=\{1,2,3,4\} $. \\
Verifichiamo che le $ \varphi_{ij} $ siano omeomorfismi prendendo come esempio $ \varphi_{12} $: dimostriamo che questa è continua e con inversa continua. La continuità è data dal seguente diagramma

\diagr{%
		V_{12} \arrow[rr, "\psi_{12}"] \arrow[dd, "\pi"'] \&  \& \R^{4} \\
		\&  \&        \\
		U_{12} \arrow[rruu, "\varphi_{12}"']                 \&  \&       
		}

dove l'applicazione

\map{\psi_{12}}
	{U_{12}}{M_{2}(\R) = \R^{4}}
	{A}{A_{34} (A_{12})^{-1}}

è continua e costante sulle classi di equivalenza. \\
L'inversa di $ \varphi_{12} $ è

\map{(\varphi_{12})^{-1}}
	{M_{2}(\R) = \R^{4}}{U_{12}}
	{\bmqty{ c_{11} & c_{12} \\ c_{21} & c_{22} }}{\left[ \bmqty{ 1 & 0 \\ 0 & 1 \\ c_{11} & c_{12} \\ c_{21} & c_{22} } \right]}

poiché

\begin{gather}
	\varphi_{12} \circ (\varphi_{12})^{-1} = \id_{\R^{4}} \\
	(\varphi_{12})^{-1} \circ \varphi_{12} = \id_{U_{12}}
\end{gather}

in quanto\footnote{%
	Il simbolo $ \bigone_{n \times m} $ indica la matrice identità di $ M_{n,m}(\R) $.%
}

\begin{gather}
	(\varphi_{12} \circ (\varphi_{12})^{-1})(A) = \varphi_{12} \left( \bmqty{ \bigone_{2 \times 2} \\ A } \right) = A (\bigone_{2 \times 2})^{-1} = A \\ \\
	%
	((\varphi_{12})^{-1} \circ \varphi_{12})([A]) = (\varphi_{12})^{-1}(A_{34} (A_{12})^{-1}) = \left[ \bmqty{ \bigone_{2 \times 2} \\ A_{34} (A_{12})^{-1} } \right] = \left[ \bmqty{ A_{12} \\ A_{34} } \right] = [A]
\end{gather}

L'inversa è continua perché composizione di funzioni continue

\begin{equation}
	(\varphi_{12})^{-1} = (\psi_{12} \circ \pi^{-1})^{-1} = \pi \circ (\psi_{12})^{-1}
\end{equation}

dove

\map{(\psi_{12})^{-1}}
	{\R^{4}}{\R^{8}}
	{\bmqty{ c_{11} & c_{12} \\ c_{21} & c_{22} }}{\bmqty{ 1 & 0 \\ 0 & 1 \\ c_{11} & c_{12} \\ c_{21} & c_{22} }}

Questo mostra l'esistenza di un atlante topologico $ \mathfrak{M} = \{(U_{ij},\varphi_{ij})\}_{i,j=1,\dots,4} $ per $ G(2,4) $: affinché questo atlante sia differenziabile, prendiamo ad esempio i cambi di carte relativi a $ \varphi_{12} $ e $ \varphi_{13} $

\begin{gather}
	\varphi_{12} \circ (\varphi_{13})^{-1} : \varphi_{13} (U_{12} \cap U_{13}) \to \varphi_{12} (U_{12} \cap U_{13}) \\
	\varphi_{13} \circ (\varphi_{12})^{-1} : \varphi_{12} (U_{12} \cap U_{13}) \to \varphi_{13} (U_{12} \cap U_{13})
\end{gather}

e mostriamo che siano lisci:

\begin{align}
	\begin{split}
		(\varphi_{13} \circ (\varphi_{12})^{-1}) (C) &= (\varphi_{13} \circ (\varphi_{12})^{-1}) \left( \bmqty{ c_{11} & c_{12} \\ c_{21} & c_{22} } \right) \\ \\
		&= \varphi_{13} \left( \left[ \bmqty{ 1 & 0 \\ 0 & 1 \\ c_{11} & c_{12} \\ c_{21} & c_{22} } \right] \right) \\ \\
		&= \bmqty{ 0 & 1 \\ c_{21} & c_{22} } \bmqty{ 1 & 0 \\ c_{11} & c_{12} }^{-1} \\ \\
		&= \bmqty{ 0 & 1 \\ c_{21} & c_{22} } \dfrac{1}{c_{12}} \bmqty{ c_{12} & 0 \\ -c_{11} & 1 } \\ \\
		&= \dfrac{1}{c_{12}} \bmqty{ -c_{11} & 1 \\ c_{12} c_{21} - c_{11} c_{22} & c_{22} } \\ \\
		&= \dfrac{1}{c_{12}} \bmqty{ -c_{11} & 1 \\ -\det(C) & c_{22} }
	\end{split}
\end{align}

dove $ c_{12} \neq 0 $ in quanto $ \det(C_{13}) = c_{12} $ e questa deve essere invertibile, abbiamo dunque

\map{\varphi_{13} \circ (\varphi_{12})^{-1}}
	{\R^{4}}{\R^{4}}
	{(c_{11}, c_{12}, c_{21}, c_{22})}{\dfrac{1}{c_{12}} (-c_{11}, 1, -\det(C), c_{22})}

la quale ha ogni componente liscia, proprio perché $ c_{12} \neq 0 $; il cambio di carte inverso $ \varphi_{12} \circ (\varphi_{13})^{-1} $ produce lo stesso risultato. Il ragionamento è analogo per gli altri cambi di carte: questo dimostra che $ \mathfrak{M} = \{(U_{ij},\varphi_{ij})\}_{i,j=1,\dots,4} $ è un atlante differenziabile per $ G(2,4) $ e dunque esiste un atlante massimale che lo contiene, il quale dota $ G(2,4) $ di struttura di varietà differenziabile.

\section{Funzioni lisce su e tra varietà differenziabili}

\subsection{Funzioni lisce su varietà}\label{s-sec:smooth-app-on-man}

Siano una varietà differenziabile $ M $ di dimensione $ n $ e una funzione $ f : M \to \R $, diremo che $ f $ è $ C^{\infty} $ (liscia o differenziabile) in un punto $ p \in M $ se esiste una carta $ (U,\varphi) $ intorno a $ p $ tale che la funzione $ f \circ \varphi^{-1} : \varphi(U) \to \R $, con $ \varphi(U) \subset \R^{n} $, sia liscia.

\img{0.6}{img19}

\begin{remark}
	$ (U,\varphi) $ è una carta nella struttura differenziabile di $ M $ (l'atlante è fissato).
\end{remark}

La definizione non dipende dalla carta scelta intorno al punto $ p $: se $ (V,\psi) $ è un'altra carta intorno a $ p $, allora

\begin{equation}
	f \circ \psi^{-1} = \eval{ (f \circ \varphi^{-1}) \circ (\varphi \circ \psi^{-1}) }_{\psi(U \cap V)} : \psi(U \cap V) \to \R
\end{equation}

dove $ \varphi \circ \psi^{-1} \in C^{\infty} $ perché le carte a cui appartengono sono $ C^{\infty} $-compatibili tra loro e $ f \circ \varphi^{-1} \in C^{\infty} $ per ipotesi, dunque $ \eval{ (f \circ \varphi^{-1})(\varphi \circ \psi^{-1}) }_{\psi(U \cap V)} \in C^{\infty} $ anche se ristretta a $ \psi(U \cap V) \ni p $, in quanto basta far vedere che esista un aperto opportuno che contenga il punto in cui la funzione sia liscia (condizione locale). \\
Una funzione $ f : M \to \R $ è liscia se lo per ogni punto $ p \in M $.

\begin{remark}
	Se $ f : M \to \R $ è una funzione liscia allora $ f $ è continua. Infatti $ \eval{f}_{\varphi(U)} = f \circ \varphi^{-1} \circ \varphi $ dove $ \varphi $ è continua perché omeomorfismo e $ f \circ \varphi^{-1} $ è continua in quanto liscia.
\end{remark}

Indicheremo con $ C^{\infty}(M) $ l'insieme delle funzioni lisce su $ M $ mentre con $ C^{0}(M) $ le funzioni continue.

\begin{definition}
	Sia $ M $ una varietà differenziabile con atlante massimale $ \mathfrak{M} $ e sia un atlante differenziabile $ \mathfrak{U} \subset \mathfrak{M} $, allora $ f : M \to \R $ è liscia se e solo se per qualunque carta $ (U,\varphi) \in \mathfrak{U} $ si ha che $ f \circ \varphi^{-1} : \varphi(U) \to \R $ sia liscia.
\end{definition}

\begin{proof}[Dimostrazione ($ \implies $)]
	Se $ f \in C^{\infty}(M) $ allora esiste una carta $ (V,\psi) \in \mathfrak{M} $ dell'atlante massimale tale che $ f \circ \psi^{-1} : \psi(V) \to \R $ sia liscia in $ p $: siccome la definizione di funzione liscia non dipende dalla carta scelta all'interno della struttura differenziabile, possiamo prendere anche una carta $ (U,\varphi) \in \mathfrak{U} $ e dunque $ f \circ \varphi^{-1} : \varphi(U) \to \R $ è ancora liscia.
\end{proof}

\begin{proof}[Dimostrazione ($ \impliedby $)]
	Siccome gli insiemi delle carte dell'atlante $ \mathfrak{U} $ sono un ricoprimento di $ M $, prendendo il punto $ p \in M $, esisterà una carta $ (U,\varphi) \ni p $ dell'atlante tale che, per ipotesi, $ f \circ \varphi^{-1} : \varphi(U) \to \R $ sia liscia, dunque abbiamo trovato una carta in cui è liscia perciò $ f $ è liscia in $ p $.
\end{proof}

Riassumendo, presa una varietà differenziabile $ M $ e un atlante differenziabile arbitrario $ \mathfrak{U} \subset \mathfrak{M} $ contenuto nella struttura differenziabile, allora $ f : M \to \R $ è liscia se e solo se per ogni carta $ (U,\varphi) \in \mathfrak{U} $ abbiamo che $ f \circ \varphi^{-1} : \varphi(U) \to \R $ sia liscia.

\begin{remark}
	La restrizione di una funzione liscia su varietà è ancora una funzione liscia\footnote{%
		Vedi Esercizio \ref{exer2-7}.%
	}.
\end{remark}

\subsection{Funzioni lisce tra varietà}

Sia $ F : N \to M $ una funzione tra varietà differenziabili $ N $ e $ M $ di dimensione rispettivamente $ n $ e $ m $. Diremo che la funzione $ F $ è $ C^{\infty} $ (liscia o differenziabile) in un punto $ p \in N $ se è continua ed esistono una carta $ (U,\varphi) $ di $ N $ intorno a $ p $ e una carta $ (V,\psi) $ di $ M $ intorno a $ F(p) $ tali che la composizione

\begin{equation}
	\psi \circ \eval{F}_{U} \circ \varphi^{-1} : \varphi(F^{-1}(V) \cap U) \to \R^{m}
\end{equation}

con $ \varphi(F^{-1}(V) \cap U) \subset \R^{n} $, sia liscia in $ \varphi(p) $. Possiamo scrivere $ \eval{F}_{U} \equiv F $ in quanto agisce prima $ \varphi^{-1} $ che restringe il dominio all'aperto $ U $; consideriamo l'aperto (in quanto $ F $ è continua) $ F^{-1}(V) \cap U \subset U $ in quanto non possiamo sapere se $ F(U) \subset V $.

\img{0.85}{img20}

La definizione non dipende dalle carte scelte, infatti se $ (U',\varphi') $ è una carta di $ N $ intorno a $ p $ e $ (V',\psi') $ è una carta di $ M $ intorno a $ F(p) $, allora

\begin{equation}
	\psi' \circ F \circ (\varphi')^{-1} = %
	\eval{ \left( (\psi' \circ \psi^{-1}) \circ (\psi \circ F \circ \varphi^{-1}) \circ (\varphi \circ (\varphi')^{-1}) \right) }_{ \varphi'(F^{-1}(V \cap V') \cap U \cap U') }
\end{equation}

dove $ \varphi'(F^{-1}(V \cap V') \cap U \cap U') \ni \varphi'(p) $, è liscia perché lo è $ \psi \circ F \circ \varphi^{-1} $ per ipotesi e lo sono $ \psi' \circ \psi^{-1} $ e $ \varphi \circ (\varphi')^{-1} $ perché cambi di carte di una struttura differenziabile. \\
Una funzione $ F : N \to M $ è liscia se lo è in ogni punto $ p \in N $.

\begin{remark}
	Nel caso in cui lo spazio ambiente sia tutto $ \R^{n} $, i.e. $ F : N \to \R^{n} $, dove la struttura differenziale dello spazio euclideo $ \R^{n} $ è data dall'atlante massimale che contiene $ (\R^{n},\id_{\R^{n}}) $, la funzione $ F $ è liscia in $ p $ se e solo se è liscia l'applicazione
	
	\begin{equation}
		\id_{\R^{n}} \circ F \circ \varphi^{-1} = F \circ \varphi^{-1} : \varphi(U) \to \R^{n}
	\end{equation}

	Per $ n = 1 $, questa è la definizione di funzione $ C^{\infty}(N) $.
\end{remark}

\begin{remark}
	Le inclusioni
	
	\map{i_{q_{0}}}
		{M}{M \times N}
		{p}{(p,q_{0})}
	
	sono funzioni lisce tra varietà\footnote{%
		Vedi Esercizio \ref{exer2-6}.%
	}.
\end{remark}

\begin{definition}
	Siano $ N $ e $ M $ varietà differenziabili rispettivamente di dimensione  $ n $ e $ m $ e con strutture differenziabili $ \mathfrak{N} $ e $ \mathfrak{M} $ e siano $ \mathfrak{U} \subset \mathfrak{M} $ e $ \mathfrak{V} \subset \mathfrak{N} $ due strutture differenziali rispettivamente in N e M. Sia $ F : N \to M $ un'applicazione, allora $ F \in C^{\infty} $ in $ p $ se e solo se $ F $ è continua e si ha che
	
	\begin{equation}
		\psi' \circ F \circ (\varphi')^{-1} : \varphi'(F^{-1}(V') \cap U') \to \R^{m} \qcomma \forall (U',\varphi') \in \mathfrak{U}, \, \forall (V',\psi') \in \mathfrak{V}
	\end{equation}

	è liscia in $ \varphi'(p) \in \varphi'(F^{-1}(V') \cap U') $.
\end{definition}

\begin{proof}[Dimostrazione ($ \implies $)]
	Siano $ F \in C^{\infty}(N) $ e $ p \in N $, allora esistono delle carte $ (U,\varphi) \in \mathfrak{M} $ e $ (V,\psi) \in \mathfrak{N} $ che contengono rispettivamente $ p $ e $ F(p) $ tali che $ \psi \circ F \circ \varphi^{-1} $ sia liscia in $ \varphi(p) $. Ma, dato che la definizione della proprietà di essere liscia non dipende dalle carte scelte, questo significa che $ \psi' \circ F \circ (\varphi')^{-1} $ è liscia in $ \varphi'(p) $. La dimostrazione si conclude per il fatto che $ p $ sia arbitrario.
\end{proof}

\begin{proof}[Dimostrazione ($ \impliedby $)]
	La dimostrazione dell'implicazione inversa è immediata in quanto le carte degli atlanti sono un ricoprimento degli spazi.
\end{proof}

Riassumendo, prese due varietà differenziabili $ N $ ed $ M $ e due atlanti differenziabili arbitrari $ \mathfrak{U} \subset \mathfrak{N} $ e $ \mathfrak{V} \subset \mathfrak{M} $ contenuti nelle rispettive strutture differenziabili, abbiamo che un'applicazione $ F : N \to M $ è liscia se e solo se è continua e $ \psi \circ F \circ \varphi^{-1} : \varphi(F^{-1}(V) \cap U) \to \R^{m} $ è liscia per qualsiasi $ (U,\varphi) \in \mathfrak{U} $ e qualsiasi $ (V,\psi) \in \mathfrak{V} $.

\begin{definition}[Composizione di funzioni lisce tra varietà]
	Siano $ M $, $ N $ e $ P $ varietà differenziabili, ed $ F : N \to M $ e $ G : M \to P $ due funzioni continue e lisce, allora la composizione continua $ G \circ F $ è liscia.
\end{definition}

\begin{proof}
	Siano $ (U,\varphi) $ una carta di $ N $ intorno a $ p $ e $ (W,\sigma) $ una carta di $ P $ intorno a $ G(F(p)) $. Dobbiamo dimostrare che la funzione $ \sigma \circ G \circ F \circ \varphi^{-1} $ sia liscia nel punto
	
	\begin{equation}
		\varphi(p) \in \varphi((G \circ F)^{-1}(W) \cap U) = \varphi(F^{-1}(G^{-1}(W)) \cap U)
	\end{equation}
	
	Sia $ (V,\psi) $ una carta intorno a $ F(p) $, allora consideriamo la funzione
	
	\begin{equation}
		(\sigma \circ G \circ \psi) \circ (\psi^{-1} \circ F \circ \varphi^{-1}) = \sigma \circ G \circ F \circ \varphi^{-1}
	\end{equation}

	la quale è liscia, in quanto composizione di applicazioni lisce (i.e. $ \sigma \circ G \circ \psi $ e $ \psi^{-1} \circ F \circ \varphi^{-1} $), in $ \varphi(F^{-1}(G^{-1}(W) \cap V) \cap U) \ni \varphi(p) $.
\end{proof}

\begin{definition}[Componenti]\label{prop:map-comp}
	Siano un'applicazione $ F : N \to \R^{n} $ e le proiezioni
	
	\map{r^{i}}
		{\R^{n}}{\R}
		{(x^{1},\dots,x^{n})}{x^{i}}

	con $ i=1,\dots,n $. \\
	L'applicazione $ F $ è liscia se e solo se tutte le $ r^{i} \circ F \doteq F^{i} $ sono lisce.
\end{definition}

\begin{proof}[Dimostrazione ($ \implies $)]
	Se $ F $ è liscia allora lo sono anche tutte le sue componenti $ F^{i} $.
\end{proof}

\begin{proof}[Dimostrazione ($ \impliedby $)]
	Supponiamo che le $ r^{i} \circ F \doteq F^{i} : N \to \R $ siano lisce per $ i=1,\dots,n $. Per qualsiasi $ p \in N $ abbiamo che $ F^{i} \circ \varphi^{-1} : \varphi(U) \to \R $ è liscia se scegliamo una carta $ (U,\varphi) \ni p $, dove $ \varphi(p) \subset \R^{n} $. A questo punto, la funzione
	
	\map{F \circ \varphi^{-1}}
		{\varphi(U)}{\R^{n}}
		{q}{(F^{1} \circ \varphi^{-1}, \dots, F^{n} \circ \varphi^{-1})(q)}

	è liscia in $ \varphi(p) $, dunque $ F $ è liscia in $ p $ per definizione, poiché
	
	\begin{equation}
		F \circ \varphi^{-1} \equiv \id_{\R^{n}} \circ F \circ \varphi^{-1}
	\end{equation}
\end{proof}

\subsubsection{\textit{Esempio}}

Consideriamo l'applicazione continua

\map{F}
	{\R}{\S^{1}}
	{t}{(\cos 2 \pi t, \sin 2 \pi t)}

dove con $ \R $ e $ \S^{1} $ si intendono gli spazi con le loro strutture differenziabili fissate, i.e. le relative varietà differenziabili. \\
Fissiamo gli atlanti $ (\R,\id_{\R}) $ e $ \{(U_{i},\varphi_{i})\}_{i=1,\dots,4} $ dove gli aperti $ U_{i} $ sono semicirconferenze di $ \S^{1} $ e le $ \varphi_{i} $ sono le proiezioni di queste sugli assi\footnote{%
	Vedi Esempio \ref{example:diff-man-s1}.%
}, i.e.

\begin{gather}
	U_{1} = \{ (x,y) \in \S^{1} \mid y>0 \} \\
	U_{2} = \{ (x,y) \in \S^{1} \mid y<0 \} \\
	U_{3} = \{ (x,y) \in \S^{1} \mid x>0 \} \\
	U_{4} = \{ (x,y) \in \S^{1} \mid x<0 \}
\end{gather}

\sbs{0.5}{%
			\map{\varphi_{1}}
				{U_{1}}{(-1,1)_{x} \subset \R^{2}}
				{(x,y)}{x}
			
			\map{\varphi_{2}}
				{U_{2}}{(-1,1)_{x} \subset \R^{2}}
				{(x,y)}{x}
			}
	{0.5}{%
			\map{\varphi_{3}}
				{U_{3}}{(-1,1)_{y} \subset \R^{2}}
				{(x,y)}{y}
			
			\map{\varphi_{4}}
				{U_{4}}{(-1,1)_{y} \subset \R^{2}}
				{(x,y)}{y}
			}

Per far vedere che $ F $ sia un'applicazione liscia tra varietà, dobbiamo verificare che le composizioni $ \varphi_{i} \circ F \circ \id_{\R} = \varphi_{i} \circ F $ siano lisce per qualsiasi $ i=1,\dots,4 $: possiamo scrivere che l'insieme $ \varphi(F^{-1}(V) \cap U) $ della definizione corrisponde a $ \id_{\R}(F^{-1}(U_{i}) \cap \R) = F^{-1}(U_{i}) \subset \R $ per questo caso, da cui le composizioni

\sbs{0.5}{%
			\map{\varphi_{1} \circ F}
				{F^{-1}(U_{1})}{\R^{2}}
				{t}{\cos(2 \pi t)}
			
			\map{\varphi_{2} \circ F}
				{F^{-1}(U_{2})}{\R^{2}}
				{t}{\cos(2 \pi t)}
			}
	{0.5}{%
			\map{\varphi_{3} \circ F}
				{F^{-1}(U_{3})}{\R^{2}}
				{t}{\sin(2 \pi t)}
			
			\map{\varphi_{4} \circ F}
				{F^{-1}(U_{4})}{\R^{2}}
				{t}{\sin(2 \pi t)}
			}

le quali sono tutte lisce perciò lo è anche $ F $.

\subsection{Diffeomorfismi tra varietà}\label{s-sec:diff}

Siano $ N $ e $ M $ due varietà differenziabili, un'applicazione $ F : N \to M $ è un \textit{diffeomorfismo} se

\begin{itemize}
	\item $ F \in C^{\infty} $
	
	\item $ F $ è invertibile
	
	\item $ F^{-1} \in C^{\infty} $
\end{itemize}

Diremo che due varietà $ N $ e $ M $ sono \textit{diffeomorfe} se esiste un diffeomorfismo $ F : N \to M $ che le collega; se due varietà $ N $ e $ M $ sono diffeomorfe, scriveremo $ M \simeq N $. \\
Sia $ V $ la classe di tutte le varietà differenziabili, allora

\begin{equation}
	M \sim N \iff M \simeq N
\end{equation}

definisce una relazione di equivalenza su $ V $.

\begin{theorem}
	Una varietà differenziabile di dimensione 1 connessa (curva differenziabili) è diffeomorfa a $ \S^{1} $ (compatta) oppure a $ \R $ (non compatta).
\end{theorem}

\sbs{0.55}{
	\begin{theorem}
		Una varietà differenziabile di dimensione 2 (superficie differenziabile) compatta e connessa è diffeomorfa a $ \Sigma_{g} $, i.e. una superficie con $ g $ \textit{buchi}. La superficie differenziabile $ \Sigma_{g} $ è una generalizzazione del toro (il quale ha un solo buco), in quanto $ \T^{2} \simeq \Sigma_{1} $.
	\end{theorem}
}
{0.45}{
	\img{0.8}{img21}
}

\begin{remark}
	Non ha senso classificare le superfici non compatte in quanto ogni spazio $ \R^{2} \setminus U $ con $ U $ chiuso in $ \R^{2} $ è una varietà differenziabile e i chiusi non sono classificabili.
\end{remark}

Data una varietà topologica $ M $, vogliamo ora determinare quante strutture differenziabili siano compatibili con $ M $ a meno di diffeomorfismi:

\begin{itemize}
	\item Se $ \dim(M) < 4 $ allora esiste una sola struttura differenziale su $ M $
	
	\item Se $ \dim(M) > 4 $ allora esiste un numero finito di strutture differenziabili su $ M $
\end{itemize}

Ad esempio, sull'ipersfera $ \S^{7} $ esistono 28 strutture differenziabili non diffeomorfe tra loro.

\begin{remark}
	Non si sa se su $ \S^{4} $ esista un numero finito o infinito di strutture differenziabili.
\end{remark}

\begin{remark}
	Esistono varietà topologiche che non ammettono strutture differenziabili.
\end{remark}

\subsubsection{\textit{Esempi}}

\paragraph{1) Due atlanti per $ \R $}

Sia $ \R $ la varietà differenziabile con atlante $ (\R,\id_{\R}) $ e $ \R' $ la stessa varietà topologica con atlante differenziabile $ (\R,\psi) $, dove

\map{\psi}
	{\R}{\R}
	{x}{x^{\sfrac{1}{3}}}

Queste due strutture differenziali sono diverse: le loro carte non sono $ C^{\infty} $-compatibili, in quanto il seguente cambio di carte è liscio

\map{\id_{\R} \circ \, \psi^{-1} = \psi^{-1}}
	{\R}{\R}
	{x}{x^{3}}

ma l'inverso non lo è
	
\map{\psi \circ (\id_{\R})^{-1} = \psi \circ \id_{\R} = \psi}
	{\R}{\R}
	{x}{x^{\sfrac{1}{3}}}

Però le varietà differenziabili sono diffeomorfe tramite l'applicazione tra varietà continua con inversa anch'essa continua

\sbs{0.5}{%
			\map{F}
				{\R}{\R'}
				{x}{x^{3}}
			}
	{0.5}{%
			\map{F^{-1}}
				{\R'}{\R}
				{x}{x^{\sfrac{1}{3}}}
			}

Per dimostrare che $ F $ e $ F^{-1} $ siano $ C^{\infty} $ tra varietà, controlliamo la composizione tra carte\footnote{%
	È importante notare che l'applicazione è o meno liscia in base alla composizione delle carte e non in base alla sola derivazione dell'immagine: questo differenzia la definizione di funzione liscia e funzione liscia \textit{tra varietà}.%
}

\map{\psi \circ F \circ (\id_{\R})^{-1} = \psi \circ F = \id_{\R}}
	{\R}{\R}
	{x}{x}
	
\map{\id_{\R} \circ F^{-1} \circ \psi^{-1} = F^{-1} \circ \psi^{-1} = \id_{\R'}}
	{\R'}{\R'}
	{x}{x}

le quali sono entrambe lisce e dunque $ \R \simeq \R' $. \\
L'identità da $ \R $ in $ \R' $ non è un diffeomorfismo in quanto

\begin{equation}
	(\psi \circ \id \circ (\id_{\R})^{-1})(x) = \psi(x) = x^{\sfrac{1}{3}}
\end{equation}

non è liscia.

\paragraph{2) $ \S^{n} $ come varietà differenziale di dimensione $ d > n+1 $}

Solitamente consideriamo $ \S^{n} $ all'interno di $ \R^{n+1} $ e dunque con la struttura di topologica e differenziale di dimensione $ n+1 $. Nonostante sia controintuitivo, possiamo però considerare il cerchio unitario $ \S^{1} $ come una varietà topologica di dimensione\footnote{%
	Numero arbitrario.%
} 307 non compatta: oltre alla topologia considerata in precedenza (quella che viene normalmente chiamata "standard"), $ \S^{1} $ è in bigezione con $ \R^{307} $, dunque possiamo portare la struttura di varietà topologica standard di $ \R^{307} $ in $ \S^{1} $ ($ \R^{307} $ non è compatto quindi non lo sarà nemmeno $ \S^{1} $). Lo stesso discorso vale se consideriamo la proprietà di varietà differenziale.

\subsection{Carte e diffeomorfismi}

\begin{definition}
	Siano $ N $ una varietà differenziabile di dimensione $ n $ e $ (U,\varphi) $ una sua carta, allora l'applicazione $ \varphi : U \to \varphi(U) $ è un diffeomorfismo, dove la struttura differenziabile di $ U $ proviene da quella di $ N $ mentre quella di $ \varphi(U) $ da $ \R^{n} $.
\end{definition}

\begin{proof}
	Osserviamo che $ (U,\varphi) $ è un'atlante differenziabile per $ U $ e $ (\varphi(U),\id_{\R^{n}}) $ è un atlante differenziabile per $ \varphi(U) $. Perché $ \varphi $ sia un diffeomorfismo è necessario che sia liscia, invertibile e con inversa liscia: per la definizione di funzione liscia tra varietà, $ \varphi $ è liscia in quanto lo è la composizione
	
	\begin{equation}
		\id_{\varphi(U)} \circ \varphi \circ \varphi^{-1} = \id_{\varphi(U)}
	\end{equation}

	Sappiamo inoltre che è invertibile in quanto omeomorfismo e la sua inversa è ancora liscia tra varietà perché lo è la composizione
	
	\begin{equation}
		\varphi \circ \varphi^{-1} \circ \id_{U} = \id_{U}
	\end{equation}

	Questo dimostra che $ \varphi $ è un diffeomorfismo, dunque $ U \simeq \varphi(U) $.
\end{proof}

\begin{definition}\label{prop:diffeo-map}
	Siano $ U \subset N $ un aperto di una varietà differenziabile $ N $ di dimensione $ n $ e $ F : U \to F(U) \subset \R^{n} $ (aperto) un diffeomorfismo, allora $ (U,F) $ è una carta di $ N $, i.e. appartiene alla struttura differenziabile $ \mathfrak{N} $ di $ N $.
\end{definition}

\begin{proof}
	È sufficiente verificare che $ (U,F) $ sia compatibile con ogni carta $ (U_{\alpha},\varphi_{\alpha}) $ dell'atlante massimale $ \mathfrak{N} $ di $ N $ che definisce la sua struttura differenziabile, cioè appartiene all'atlante stesso. Consideriamo dunque i cambi di carte
	
	\begin{gather}
		F \circ (\varphi_{\alpha})^{-1} : \varphi_{\alpha}(U_{\alpha} \cap U) \to F(U_{\alpha} \cap U) \\
		\varphi_{\alpha} \circ F^{-1} : F(U_{\alpha} \cap U) \to \varphi_{\alpha}(U_{\alpha} \cap U)
	\end{gather}

	in cui $ F $ è liscia in quanto diffeomorfismo e $ \varphi_{\alpha} $ e la sua inversa sono lisce per la dimostrazione precedente, perciò i cambi di carte sono anch'essi lisci in quanto composizione di applicazioni lisce, rendendo quindi $ (U,F) \in \mathfrak{N} $.
\end{proof}

\subsection{Proiezioni tra varietà}

\begin{definition}
	Siano $ N $ e $ M $ due varietà differenziabili rispettivamente di dimensione $ n $ e $ m $. Le seguenti proiezioni sono lisce
	
	\sbs{0.5}{%
				\map{\pi_{N}}
					{N \times M}{N}
					{(x,y)}{x}
				}
		{0.5}{%
				\map{\pi_{M}}
					{N \times M}{M}
					{(x,y)}{y}
				}
\end{definition}

\begin{proof}
	Se consideriamo $ \{(U_{\alpha},\varphi_{\alpha})\}_{\alpha \in A} $ un atlante differenziabile per $ N $ e $ \{(V_{\beta},\psi_{\beta})\}_{\beta \in B} $ un atlante differenziabile per $ M $, allora $ \{(U_{\alpha} \times V_{\beta},\varphi_{\alpha} \times \psi_{\beta})\} $ è un atlante differenziabile per $ N \times M $. \\
	Dalla teoria sulle funzioni tra varietà, un'applicazione $ F : N \to M $ è liscia tra varietà se è continua e se è liscia la seguente composizione
	
	\begin{equation}
		\psi \circ F \circ \varphi^{-1} : \varphi(F^{-1}(V) \cap U) \to \R^{n}
	\end{equation}
	
	Per mostrare dunque che $ \pi_{N} $ sia liscia\footnote{%
		La dimostrazione per $ \pi_{M} $ è analoga.%
	}, consideriamo innanzitutto il dominio della composizione relativa:
	
	\begin{align}
		\begin{split}
			(\varphi_{\alpha} \times \psi_{\beta})(\pi^{-1}(U_{\alpha}) \cap (U_{\alpha} \times V_{\beta})) &= (\varphi_{\alpha} \times \psi_{\beta})((U_{\alpha} \times M) \cap (U_{\alpha} \times V_{\beta})) \\
			&= (\varphi_{\alpha} \times \psi_{\beta})(U_{\alpha} \times V_{\beta}) \\
			&= \varphi_{\alpha}(U_{\alpha}) \times \psi_{\beta}(V_{\beta})
		\end{split}	
	\end{align}
	
	dove $ \varphi_{\alpha}(U_{\alpha}) \subset \R^{n} $ e $ \psi_{\beta}(V_{\beta}) \subset \R^{m} $. A questo punto, essendo le proiezioni continue, $ \pi_{N} $ è liscia in quanto lo è la composizione
	
	\map{\varphi_{\alpha} \circ \pi_{N} \circ (\varphi_{\alpha} \times \psi_{\beta})^{-1}}
	{\varphi_{\alpha}(U_{\alpha}) \times \psi_{\beta}(V_{\beta})}{\R^{n}}
	{(a^{1},\dots,a^{n},b^{1},\dots,b^{m})}{%
		(\varphi_{\alpha} \circ \pi_{N}) \left( (\varphi_{\alpha}^{-1} \times \psi_{\beta}^{-1}) (a^{1},\dots,a^{n},b^{1},\dots,b^{m}) \right) \\
		&\mapsto (\varphi_{\alpha} \circ \pi_{N}) \left( \varphi_{\alpha}^{-1}(a^{1},\dots,a^{n}),\psi_{\beta}^{-1}(b^{1},\dots,b^{m}) \right) \\
		&\mapsto \varphi_{\alpha} \left( \varphi_{\alpha}^{-1}(a^{1},\dots,a^{n}) \right) \\
		&\mapsto (a^{1},\dots,a^{n})
		}
\end{proof}

\begin{remark}
	Siano $ N $, $ M_{1} $ e $ M_{2} $ tre varietà differenziabili e l'applicazione
	
	\map{F}
		{N}{M_{1} \times M_{2}}
		{q}{(F_{1}(q),F_{2}(q))}
	
	dove $ F_{1} : N \to M_{1} $ e $ F_{1} : N \to M_{2} $, allora $ F $ è liscia se e solo se lo sono $ F_{1} $ e $ F_{2} $.
\end{remark}

\begin{proof}[Dimostrazione ($ \implies $)]
	Se $ F $ è liscia allora le sue componenti $ F_{i} = \pi_{i} \circ F $ (con $ \pi_{i} : M_{1} \times M_{2} \to M_{i} $) devono essere lisce: questo è verificato perché $ \pi_{i} \in C^{\infty}(M_{1} \times M_{2}) $ per $ i=1,2 $, dunque la composizione che risulta in $ F_{i} $ è liscia.
\end{proof}

\begin{proof}[Dimostrazione ($ \impliedby $)]
	Se $ F_{1} $ e $ F_{2} $ sono lisce allora $ \varphi_{1} \circ F_{1} \circ \varphi^{-1} $ e $ \varphi_{2} \circ F_{2} \circ \varphi^{-1} $ sono lisce, dove $ (U_{1},\varphi_{1}) $ è una carta in $ M_{1} $, $ (U_{2},\varphi_{2}) $ è una carta in $ M_{2} $ e $ (U,\varphi) $ è una carta in $ N $. Da questo otteniamo che la seguente composizione è liscia
	
	\begin{equation}
		(\varphi_{1} \circ F_{1} \circ \varphi^{-1}, \varphi_{2} \circ F_{2} \circ \varphi^{-1}) = (\varphi_{1} \times \varphi_{2}) \circ F \circ \varphi^{-1}
	\end{equation}
	
	Sappiamo inoltre che la topologia prodotto conserva la continuità delle funzioni, dunque $ F $ è continua in quanto $ F = F_{1} \times F_{2} $. \\
	Essendo $ F $ continua e $ (\varphi_{1} \times \varphi_{2}) \circ F \circ \varphi^{-1} $ liscia, otteniamo dunque che $ F $ è liscia tra varietà.
\end{proof}

\subsection{Gruppi di Lie}

Un \textit{gruppo di Lie} $ G $ è un gruppo algebrico che sia anche una varietà differenziabile nel quale le operazioni di gruppo sono lisce

\sbs{0.5}{%
			\map{\mu}
				{G \times G}{G}
				{(g,h)}{g \cdot h}
			}
	{0.5}{%
			\map{i}
				{G}{G}
				{g}{g^{-1}}
			}

\subsubsection{\textit{Esempi}}

\paragraph{1) $ \R $ e somma}

Il gruppo $ (\R,+) $ è anche una varietà differenziabile. Le operazioni

\sbs{0.5}{%
			\map{+}
				{\R \times \R}{\R}
				{(x,y)}{x + y}
			}
	{0.5}{%
			\map{i}
				{\R}{\R}
				{x}{- x}
			}

sono lisce dunque è un gruppo di Lie.

\paragraph{2) $ \R \setminus \{0\} $ e prodotto}

Il gruppo $ (\R \setminus \{0\},\cdot) $ è una varietà differenziabile e dunque gruppo di Lie in quanto le seguenti operazioni sono lisce

\sbs{0.5}{%
			\map{\cdot}
				{\R \setminus \{0\} \times \R \setminus \{0\}}{\R \setminus \{0\}}
				{(x,y)}{x \cdot y}
			}
	{0.5}{%
			\map{i}
				{\R \setminus \{0\}}{\R \setminus \{0\}}
				{x}{x^{-1}}
			}

\paragraph{3) Cerchio unitario}

Il cerchio unitario

\begin{equation}
	\S^{1} = \{ z \in \C \, \mid \, \norm{z} = 1 \} \subset \C = \R^{2}
\end{equation}

dove $ \C $ indica l'insieme dei numeri complessi, è un gruppo di Lie in quanto le seguenti operazioni sono lisce

\sbs{0.5}{%
			\map{\cdot}
				{\S^{1} \times \S^{1}}{\S^{1}}
				{(z,w)}{z \cdot w}
			}
	{0.5}{%
			\map{i}
				{\S^{1}}{\S^{1}}
				{z}{z^{-1}}
			}

\paragraph{4) Prodotto diretto}

Il prodotto diretto di gruppi di Lie è ancora un gruppo di Lie, con il prodotto tra varietà differenziabili e con l'operazione componente per componente\footnote{%
	Vedi Esercizio \ref{exer3-1}.%
}.

\paragraph{5) Gruppo lineare generale}

Il gruppo delle matrici invertibili

\begin{equation}
	GL_{n}(\R) = \{ A \in M_{n}(\R) \, \mid \, \det(A) \neq 0 \} \subset M_{n}(\R) = \R^{n^{2}}
\end{equation}

con la struttura differenziale ereditata da $ M_{n}(\R) = \R^{n^{2}} $ (in quanto $ GL_{n}(\R) $ è aperto in questo spazio) è un gruppo di Lie rispetto alla moltiplicazione. La carta $ (GL_{n}(\R),\id) $ corrisponde a un atlante per $ GL_{n}(\R) $. \\
Per dimostrare che sia un gruppo di Lie, dobbiamo verificare che le operazioni di moltiplicazione e di inversione siano lisce: consideriamo quindi il prodotto tra matrici

\map{\mu}
	{GL_{n}(\R) \times GL_{n}(\R)}{GL_{n}(\R)}
	{(A,B)}{A B}

Se $ A = (a_{ij}) $ e $ B = (b_{jk}) $ allora

\begin{equation}
	(A B)_{ik} = \sum_{j=1}^{n} a_{ij} b_{jk}
\end{equation}

dunque $ \mu \in C^{\infty}(GL_{n}(\R) \times GL_{n}(\R)) $ in quanto somma e prodotto di applicazioni lisce. \\
L'inversione è definita come

\map{i}
	{GL_{n}(\R)}{GL_{n}(\R)}
	{A}{A^{-1}}

Considerando la matrice dei cofattori $ \mathcal{C} $ definita mediante

\begin{equation}
	\mathcal{C}_{ij} \doteq (-1)^{i+j} m_{ij} = (-1)^{i+j} \det(A_{ij})
\end{equation}

dove le $ A_{ij} $ sono le sottomatrici di $ A $ (le matrici a cui sono state tolte la riga $ i $ e la colonna $ j $), si ha che l'inverso di una matrice $ A $ è dato da

\begin{equation}
	A^{-1} = \dfrac{1}{\det(A)} \, \mathcal{C}^{T}
\end{equation}

dunque l'inversione è liscia in quanto composizione di applicazioni lisce.

\subsection{Derivate parziali su varietà differenziabili}

Siano una varietà differenziabile $ N $ di dimensione $ n $ e una carta $ (U,\varphi) $ intorno a un punto $ p \in N $, dal quale consideriamo il diffeomorfismo $ \varphi : U \to \varphi(U) \subset \R^{n} $. Prendendo la proiezione

\map{r^{i}}
	{\R^{n}}{\R}
	{(x^{1},\dots,x^{n})}{x^{i}}

con $ i=1,\dots,n $, otteniamo che $ x^{i} = r^{i} \circ \varphi $, in quanto $ \varphi(p) = (x^{1}(p),\dots,x^{n}(p)) \in \R^{n} $. \\
Sia una funzione liscia $ f : N \to \R $, vogliamo definire la \textit{derivata parziale di} $ f $ \textit{rispetto alla coordinata} $ x^{i} $ \textit{nel punto} $ p $, la quale dipenderà dunque dalla carta scelta che contiene $ p $. L'idea alla base della derivata parziale di una funzione su una varietà è quella di riportare tutto a $ \R^{n} $ e svolgere la derivata usuale, dunque definiamo

\begin{equation}
	\pdv{f}{x^{i}} \, (p) \doteq \eval{ \pdv{x^{i}} }_{p} f
\end{equation}

\sbs{0.45}{%
			Per poter calcolare la derivata parziale, passiamo attraverso la composizione
			
			\begin{equation}
				f \circ \varphi^{-1} : \varphi(U) \to \R
			\end{equation}
		
			visualizzabile a lato.
			}
	{0.55}{%
			\img{0.55}{img22}
			}

Tramite questa, possiamo quindi esprimere la derivata parziale come:

\begin{equation}
	\eval{ \pdv{x^{i}} }_{p} f = \pdv{(f \circ \varphi^{-1})}{r^{i}} \, (\varphi(p)) %
	\doteq \eval{ \pdv{r^{i}} }_{\varphi(p)} (f \circ \varphi^{-1})
\end{equation}

La dipendenza dalle carte scelte per la varietà è analoga alla scelta delle coordinate per $ \R^{n} $ (e.g. standard vs. polari).

\begin{definition}
	Presa una funzione $ f $ liscia su una varietà, la sua derivata parziale è ancora liscia
	
	\begin{equation}
		\pdv{f}{x^{i}} \doteq \pdv{r^{i}} (f \circ \varphi^{-1}) : U \to \R %
		\qq{dove} \pdv{f}{x^{i}} \in C^{\infty}(U)
	\end{equation}
\end{definition}

\begin{proof}
	Possiamo scrivere
	
	\begin{equation}
		\pdv{f}{x^{i}} \, (p) = \pdv{f}{x^{i}} \, (\varphi^{-1}(\varphi(p))) %
		\doteq \pdv{(f \circ \varphi^{-1})}{r^{i}} \, (\varphi(p))
	\end{equation}

	da cui
	
	\begin{equation}
		\pdv{f}{x^{i}} \circ \varphi^{-1} = \pdv{(f \circ \varphi^{-1})}{r^{i}} \qcomma \forall i=1,\dots,n
	\end{equation}
	
	Per ipotesi, $ f $ è liscia e dunque, per definizione, lo è anche la composizione $ f \circ \varphi^{-1} $: l'equazione sopra è la derivata in uno spazio euclideo di una funzione liscia quindi è ancora liscia.
\end{proof}

\subsubsection{\textit{Esempio}}

Siano una varietà differenziabile $ N $ e una sua carta $ (U,\varphi) $ con $ \varphi = (x^{1},\dots,x^{n}) $. Allora

\begin{equation}
	\pdv{x^{i}}{x^{j}} = \delta^{ij}
\end{equation}

Infatti, usando la definizione

\begin{equation}
	\pdv{x^{i}}{x^{j}} \, (p) = \pdv{(x^{i} \circ \varphi^{-1})}{r^{j}} \, (\varphi(p))
\end{equation}

con $ p \in U $ e $ r^{j} $ le coordinate standard di $ \R^{n} $. A questo punto, siccome $ x^{i} = r^{i} \circ \varphi $, abbiamo che

\begin{equation}
	\pdv{x^{i}}{x^{j}} \, (p) = \pdv{r^{i}}{r^{j}} \, (\varphi(p)) = \delta^{ij}
\end{equation}

\subsection{Jacobiano per applicazioni tra varietà}

Sia un'applicazione liscia $ F : N \to M $ con $ N $ e $ M $ varietà differenziabili di dimensione $ n $ e $ m $ rispettivamente; fissiamo due carte $ (U,\varphi) \in N $ e $ (V,\psi) \in M $, dove $ F(U) \subset V $: è sempre possibile fare in modo che quest'ultima relazione sia valida e inoltre non restrittivo in quanto è possibile prendere l'aperto $ U $ arbitrariamente piccolo affinché si verifichi questa condizione\footnote{%
	Qualunque aperto contenuto in una carta definisce ancora una carta nella struttura differenziabile: se $ U \subset U_{\alpha} $ dove $ (U_{\alpha},\varphi_{\alpha}) $ è una carta della varietà differenziabile, allora $ \left( U, \eval{\varphi_{\alpha}}_{U} \right) $ è ancora una carta della stessa varietà.%
}. \\
Sia la componente $ i $-esima di $ F $ rispetto alle coordinate $ \psi = (y^{1},\dots,y^{m}) $

\begin{equation}
	F^{i} = y^{i} \circ F = r^{i} \circ \psi \circ F \qcomma i=1,\dots,m
\end{equation}

L'applicazione $ F^{i} : U \to \R $ appartiene a $ C^{\infty}(U) $, quindi possiamo definire le sue derivate parziali

\begin{equation}
	\pdv{F^{i}}{x^{j}} : U \to \R
\end{equation}

le quali possono essere organizzate nella \textit{matrice jacobiana} $ J $ \textit{di} $ F $ rispetto alle carte $ (U,\varphi) $ e $ (V,\psi) $ nel punto $ p \in U $

\begin{equation}
	J(F)(p) \doteq \left[ \pdv{F^{i}}{x^{j}} \, (p) \right] %
	= \bmqty{ % 
				\dpdv{F^{1}}{x^{1}} \, (p) & \cdots & \dpdv{F^{1}}{x^{n}} \, (p) \\ \\
				\vdots & \ddots & \vdots \\ \\
				\dpdv{F^{m}}{x^{1}} \, (p) & \cdots & \dpdv{F^{m}}{x^{n}} \, (p)
		 		} %
	\in M_{m,n}(\R)
\end{equation}

Per definizione di derivata parziale, possiamo scrivere gli elementi di $ J $ come

\begin{equation}
	J_{ij}(F)(p) = \pdv{F^{i}}{x^{j}} \, (p) = \pdv{(F^{i} \circ \varphi^{-1})}{r^{j}} \, (\varphi(p))
\end{equation}

Nel caso in cui $ N = \R^{n} $ e $ M = \R^{m} $

\begin{equation}
	\pdv{F^{i}}{x^{j}} = \pdv{F^{i}}{r^{j}}
\end{equation}

in quanto le carte hanno come applicazioni delle identità. \\
Se la dimensione della varietà di partenza è uguale alla dimensione della varietà di arrivo, i.e. $ n = m $, allora $ J \in M_{n}(\R) $ e il suo determinante $ \det(J) $ si chiama \textit{determinante jacobiano}.

\subsubsection{Matrice jacobiana del cambio di coordinate}

Siano $ N $ una varietà differenziabile, e $ (U,\varphi) $ e $ (V,\psi) $ due sue carte con $ \varphi = (x^{1},\dots,x^{n}) $ e $ \psi = (y^{1},\dots,y^{n}) $. Consideriamo il diffeomorfismo tra aperti di $ \R^{n} $

\begin{equation}
	\psi \circ \varphi^{-1} : \varphi(U \cap V) \to \psi(U \cap V)
\end{equation}

Possiamo calcolare la sua matrice jacobiana, sapendo che le $ y^{i} : U \cap V \to \R $ sono lisce

\begin{equation}
	J(\psi \circ \varphi^{-1})(\varphi(p)) = \left[ \pdv{(\psi \circ \varphi^{-1})^{i}}{r^{j}} \, (\varphi(p)) \right] %
	= \left[ \pdv{y^{i}}{x^{j}} \, (p) \right] \in M_{n}(\R)
\end{equation}

poiché, scrivendo esplicitamente le componenti, otteniamo

\begin{align}
	\begin{split}
		\pdv{(\psi \circ \varphi^{-1})^{i}}{r^{j}} \, (\varphi(p)) &= \pdv{(r^{i} \circ \psi \circ \varphi^{-1})}{r^{j}} \, (\varphi(p)) \\
		&= \pdv{(y^{i} \circ \varphi^{-1})}{r^{j}} \, (\varphi(p)) \\
		&= \left( \pdv{y^{i}}{r^{j}} \circ \varphi^{-1} \right) \, (\varphi(p)) \\
		&= \pdv{y^{i}}{x^{j}} \, (p)
	\end{split}
\end{align}

\subsection{Teorema della funzione inversa}

Sia $ F : N \to M $ un'applicazione liscia, diremo che $ F $ è un \textit{diffeomorfismo locale} in un punto $ p \in N $ se esiste un intorno $ U $ di $ p $ in $ N $ tale che $ F(U) \subset M $ sia aperto e $ F : U \to F(U) $ sia un diffeomorfismo.

\begin{theorem}[Teorema della funzione inversa (IFT) in analisi]
	Sia un'applicazione liscia $ F : W \to \R^{n} $ con $ W \subset \R^{n} $, allora $ F $ è un diffeomorfismo locale in $ p \in W $ se e solo se lo jacobiano associato è invertibile, i.e.
	
	\begin{equation}
		\det(J(F)(p)) = \det( \left[ \pdv{F^{i}}{x^{j}} \, (p) \right] ) \neq 0 %
		\implies %
		J(F)(p) \in GL_{n} (\R)
	\end{equation}
\end{theorem}

Possiamo quindi individuare un intorno $ U \subset W $ di un punto $ p \in W $ il quale sia diffeomorfo alla sua immagine tramite $ F $ controllando semplicemente un numero, il quale dipende esclusivamente dal punto e non dal suo intorno. \\
Questo teorema implica un altro teorema analogo in geometria differenziale:

\begin{theorem}[IFT in geometria differenziale]\label{thm:ift}
	Sia $ F : N \to M $ un'applicazione liscia dove $ \dim(N) = \dim(M) $ e sia un punto $ p \in N $, allora $ F $ è un diffeomorfismo locale in $ p $ se e solo se la matrice jacobiana per varietà è invertibile, i.e.
	
	\begin{equation}
		\det(J(F)(p)) = \det( \left[ \pdv{F^{i}}{x^{j}} \, (p) \right] ) \neq 0
	\end{equation}
	
	per carte $ (U,\varphi) \in N $ e $ (V,\psi) \in M $ tali che $ F(U) \subseteq V $, dove $ p \in U $ e $ F(p) \in V $.
\end{theorem}

Questa estensione alle varietà è naturale in quanto possiamo scrivere l'applicazione tra varietà localmente come da un aperto di $ \R^{n} $ a $ \R^{n} $ e quindi utilizzare il teorema della funzione inversa dell'analisi.

\begin{proof}
	Osserviamo che
	
	\begin{equation}
		J(\psi \circ F \circ \varphi^{-1})(\varphi(p)) = \left[ \pdv{F^{i}}{x^{j}} \, (p) \right]
	\end{equation}

	in quanto, esattamente come per il cambio di carte
	
	\begin{align}
		\begin{split}
			J(\psi \circ F \circ \varphi^{-1})(\varphi(p)) &= \left[ \pdv{(\psi \circ F \circ \varphi^{-1})^{i}}{r^{j}} \, (\varphi(p)) \right] \\
			&= \left[ \pdv{(r^{i} \circ \psi \circ F \circ \varphi^{-1})}{r^{j}} \, (\varphi(p)) \right] \\
			&= \left[ \pdv{(y^{i} \circ F \circ \varphi^{-1})}{r^{j}} \, (\varphi(p)) \right] \\
			&= \left[ \pdv{(F^{i} \circ \varphi^{-1})}{r^{j}} \, (\varphi(p)) \right] \\
			&= \left[ \pdv{F^{i}}{x^{j}} \, (p) \right]
		\end{split}
	\end{align}

	Quindi
	
	\begin{equation}
		\det( J(\psi \circ F \circ \varphi^{-1})(\varphi(p)) ) \neq 0 \, \iff \, \det( \left[ \pdv{F^{i}}{x^{j}} \, (p) \right] ) \neq 0
	\end{equation}

	ma dominio e codominio della funzione di cui calcoliamo la matrice jacobiana sono rispettivamente
	
	\begin{equation}
		\psi \circ F \circ \varphi^{-1} : \varphi(U) \to \R^{n}
	\end{equation}

	e il suo determinante jacobiano è diverso da zero, dunque $ \psi \circ F \circ \varphi^{-1} $ è un diffeomorfismo locale in $ \varphi(p) $ per IFT in analisi, il quale è equivalente a dire che $ F $ sia un diffeomorfismo locale in $ p $, in quanto sia $ \psi $ che $ \varphi $ sono diffeomorfismi. \\
	Il fatto che $ \psi \circ F \circ \varphi^{-1} $ sia un diffeomorfismo locale in $ \varphi(p) $ significa che esiste un aperto $ W \ni 
	\varphi(p) $ tale che
	
	\begin{equation}
		\psi \circ F \circ \varphi^{-1} : W \to \psi(F(\varphi^{-1}(W)))
	\end{equation}

	sia un diffeomorfismo, in cui sia $ W $ che $ \psi(F(\varphi^{-1}(W))) $ sono aperti di $ \R^{n} $. Se questo è un diffeomorfismo, allora lo è anche $ \psi^{-1} \circ \psi \circ F \circ \varphi^{-1} \circ \varphi = F $ in un intorno di $ p $, i.e. $ \varphi^{-1}(W) $.
\end{proof}

\begin{corollary}[IFT]\label{cor:ift}
	Siano $ N $ una varietà differenziabile di dimensione $ n $ e $ F^{1},\dots,F^{n} = F $ delle funzioni lisce definite in un aperto $ U \subset N $ facente parte di una carta $ (U,\varphi) \in N $ dove $ \varphi = (x^{1},\dots,x^{n}) $, allora le $ F^{1},\dots,F^{n} $ definiscono coordinate intorno a un punto $ p \in U $ se e solo se il loro jacobiano è invertibile, i.e.
	
	\begin{equation}
		\det( \left[ \pdv{F^{i}}{x^{j}} \, (p) \right] ) \neq 0
	\end{equation}
\end{corollary}

\begin{proof}[Dimostrazione ($ \implies $)]
	Supponiamo che le $ F^{1},\dots,F^{n} $ siano un sistema di coordinate intorno al punto $ p $: questo è equivalente a dire che $ (U,F) $ è una carta intorno a $ p $. Se abbiamo una carta, $ F $ è un diffeomorfismo il che implica che sia un diffeomorfismo locale in $ p $ e dunque, per IFT
	
	\begin{equation}
		\det( \left[ \pdv{F^{i}}{x^{j}} \, (p) \right] ) \neq 0
	\end{equation}
\end{proof}

\begin{proof}[Dimostrazione ($ \impliedby $)]
	Se il determinante dello jacobiano è diverso da zero, tramite IFT, abbiamo che $ F $ è un diffeomorfismo locale in $ p $, i.e. $ F : W \to F(W) \subset \R^{n} $ è un diffeomorfismo, dove $ p \in W \subset U $. \\
	La funzione $ F $ è compatibile con tutte le carte $ (U_{\alpha},\varphi_{\alpha}) $ della varietà differenziabile, in quanto i cambi di carte
				
	\begin{gather}
			F \circ \varphi_{\alpha}^{-1} : \varphi_{\alpha}(U_{\alpha} \cap W) \to F(U_{\alpha} \cap W) \\
			\varphi_{\alpha} \circ F^{-1} : F(U_{\alpha} \cap W) \to \varphi_{\alpha}(U_{\alpha} \cap W)
	\end{gather}
			
	sono lisci poiché $ F $ è un diffeomorfismo nei domini considerati, perciò $ (W,F) $ è ancora una carta per la varietà differenziabile, dunque le $ F^{1},\dots,F^{n} $ sono un sistema di coordinate attorno a $ p $.
\end{proof}

\section{Spazio tangente a una varietà differenziale in un suo punto}

Negli aperti di $ \R^{n} $ possiamo definire lo spazio tangente in un punto come l'insieme dei vettori di $ n $ componenti uscenti da quel punto. Considerando un aperto $ U \subset \R^{n} $, per lo spazio tangente vale

\begin{equation}
	T_{p}(U) = T_{p}(\R^{n}) \simeq \der_{p}(C_{p}^{\infty}(\R^{n}))
\end{equation}

dunque questo può essere pensato come lo spazio delle derivazioni puntuali dell'algebra dei germi delle funzioni lisce, dove $ C_{p}^{\infty}(U) \equiv C_{p}^{\infty}(\R^{n}) $ in quanto due funzioni sono equivalenti in un punto se coincidono in un aperto, arbitrariamente piccolo, che contiene il punto stesso. Siccome non è possibile avere una visualizzazione dello spazio tangente nelle varietà come quello in $ \R^{n} $, quest'ultimo approccio si presta maggiormente per definire lo spazio tangente su una varietà. \\
Siano $ N $ una varietà differenziabile e $ p \in N $ un suo punto, definiamo ora lo \textit{spazio tangente a una varietà in un suo punto} come

\begin{equation}
	T_{p}(N) \doteq \der_{p}(C_{p}^{\infty}(N))
\end{equation}

dove $ C_{p}^{\infty}(N) $ è l'insieme dei germi di funzioni lisce in un intorno di $ p $. Un elemento di $ C_{p}^{\infty}(N) $ è una classe di equivalenza di coppie $ [(f,U)] $, dove $ p \in U $ con $ U \subset N $ aperto, secondo la seguente relazione di equivalenza

\begin{gather}
	(f,U) \sim (g,V) \mid f \in C^{\infty}(U) \wedge g \in C^{\infty}(V) \nonumber \\
	\Updownarrow \\
	\E W \text{ intorno di } p \mid f(q) \equiv g(q) , \, \forall q \in W \nonumber
\end{gather}

i.e. due germi di funzioni appartengono alla stessa classe equivalenza se e solo se le funzioni coincidono in un intorno di $ W \ni p $. \\
Come nel caso euclideo, $ C_{p}^{\infty}(N) $ è ancora un'algebra\footnote{%
	Vedi Esercizio \ref{exer1-7}.%
} su $ \R $ con operazioni di somma e prodotto per scalare (per far sì che sia uno spazio vettoriale) e tra funzioni

\begin{gather}
	\lambda [(f,U)] + \mu [(g,V)] = [(\lambda f + \mu g , U \cap V)] \\
	[(f,U)] \cdot [(g,V)] = [(f g , U \cap V)]
\end{gather}

per qualsiasi $ [(f,U)], [(g,V)] \in C_{p}^{\infty}(N) $ e qualsiasi $ \lambda, \mu \in \R $. \\
Le derivazioni puntuali $ \der_{p}(C_{p}^{\infty}(N)) $, per definizione, sono l'insieme di applicazioni del tipo

\begin{equation}
	D : C_{p}^{\infty}(N) \to \R
\end{equation}

che siano $ \R $-lineari rispetto alla struttura di spazio vettoriale in $ C_{p}^{\infty}(N) $ e che rispettino la regola di Leibniz:

\begin{equation}
	D([(f,U)] [(g,V)]) = D([(f,U)]) \, g(p) + f(p) \, D([(g,V)])
\end{equation}

L'insieme $ \der_{p}(C_{p}^{\infty}(N)) $ è uno spazio vettoriale\footnote{%
	Vedi Esercizio \ref{exer1-8}.%
} su $ \R $ rispetto alle operazioni

\begin{align}
	\begin{split}
		(D_{u} + D_{v}) ([(f,U)]) &= D_{u}([(f,U)]) + D_{v}([(f,U)]) \\
		D (\lambda [(f,U)]) &= \lambda D ([(f,U)])
	\end{split}
\end{align}

per qualsiasi $ D,D_{u},D_{v} \in \der_{p}(C_{p}^{\infty}(\R^{n})) $ e qualsiasi $ \lambda \in \R $.

\subsection{Derivazioni e derivate parziali}

Siano una varietà differenziabile $ N $ di dimensione $ n $, una funzione $ f \in C^{\infty}(N) $, un punto $ p \in N $ e una carta $ (U, \varphi) \in N $ intorno a $ p $ con $ \varphi = (x^{1},\dots,x^{n}) $: la \textit{derivata parziale di} $ f $ \textit{rispetto alle componenti di} $ \varphi $ \textit{nel punto} $ p $ è definita come

\begin{equation}
	\pdv{f}{x^{i}} \, (p) \doteq \eval{ \pdv{x^{i}} }_{p} f = \pdv{(f \circ \varphi^{-1})}{r^{i}} \, (\varphi(p)) \qcomma i=1,\dots,n
\end{equation}

Possiamo pensare alla derivata parziale come elemento delle derivazioni puntuali, i.e.

\begin{equation}
	\eval{ \pdv{x^{i}} }_{p} \in \der_{p}(C_{p}^{\infty}(N))
\end{equation}

in quanto le applicazioni

\begin{equation}
	\eval{ \pdv{x^{i}} }_{p} : C_{p}^{\infty}(N) \to \R
\end{equation}

sono $ \R $-lineari e soddisfano la regola di Leibniz. \\
La definizione di questa derivazione, con dominio l'insieme dei germi di funzione, è la seguente

\begin{equation}
	\pdv{f}{x^{i}} \, (p) \doteq \eval{ \pdv{x^{i}} }_{p} ([(f,U)])
\end{equation}

la quale è ben definita perché

\begin{equation}
	f \sim g \iff f \circ \varphi^{-1} \sim g \circ \varphi^{-1}
\end{equation}

cioè

\begin{equation}
	 f \equiv g \text{ in un aperto } W \ni p \iff f \circ \varphi^{-1} = g \circ \varphi^{-1} \text{ in un aperto } \varphi(W) \ni \varphi(p)
\end{equation}

Verifichiamo ora che la derivata parziale sia $ \R $-lineare e soddisfi la regola di Leibniz, in modo tale da dimostrare che sia un elemento di $ \der_{p}(C_{p}^{\infty}(N)) $. \\
Siano due classi $ [(f,U)],[(g,V)] \in C_{p}^{\infty}(N) $, consideriamo la derivata parziale di una loro combinazione lineare

\begin{align}
	\begin{split}
		\eval{ \pdv{x^{i}} }_{p} (\lambda [(f,U)] + \mu [(g,V)]) &= \eval{ \pdv{x^{i}} }_{p} ([(\lambda f + \mu g, U \cap V)]) \\
		&= \eval{ \pdv{x^{i}} }_{p} (\lambda f + \mu g) \\
		&= \eval{ \pdv{r^{i}} }_{\varphi(p)} ((\lambda f + \mu g) \circ \varphi^{-1}) \\
		&= \eval{ \pdv{r^{i}} }_{\varphi(p)} (\lambda f \circ \varphi^{-1} + \mu g \circ \varphi^{-1}) \\
		&= \lambda \eval{ \pdv{r^{i}} }_{\varphi(p)} (f \circ \varphi^{-1}) + \mu \eval{ \pdv{r^{i}} }_{\varphi(p)} (g \circ \varphi^{-1}) \\
		&= \lambda \eval{ \pdv{x^{i}} }_{p} f + \mu \eval{ \pdv{x^{i}} }_{p} g \\
		&= \lambda \eval{ \pdv{x^{i}} }_{p} [(f,U)] + \mu \eval{ \pdv{x^{i}} }_{p} [(g,V)]
	\end{split}
\end{align}

per qualsiasi $ \lambda,\mu \in \R $, questo prova la $ \R $-linearità della derivata parziale. \\
Per la regola di Leibniz:

\begin{align}
	\begin{split}
		\eval{ \pdv{x^{i}} }_{p} ([(f,U)]& \cdot [(g,V)]) = \eval{ \pdv{x^{i}} }_{p} ([(fg,U \cap V)] \\
		&= \eval{ \pdv{x^{i}} }_{p} (f g) \\
		&= \eval{ \pdv{r^{i}} }_{\varphi(p)} ((f g) \circ \varphi^{-1}) \\
		&= \eval{ \pdv{r^{i}} }_{\varphi(p)} ((f \circ \varphi^{-1}) (g \circ \varphi^{-1})) \\
		&= \left( \eval{ \pdv{r^{i}} }_{\varphi(p)} (f \circ \varphi^{-1}) \right) (g \circ \varphi^{-1})(\varphi(p)) + (f \circ \varphi^{-1})(\varphi(p)) \left( \eval{ \pdv{r^{i}} }_{\varphi(p)} (g \circ \varphi^{-1}) \right) \\
		&= \left( \eval{ \pdv{x^{i}} }_{p} f \right) g(p) + f(p) \left( \eval{ \pdv{x^{i}} }_{p} g \right) \\
		&= \left( \eval{ \pdv{x^{i}} }_{p} [(f,U)] \right) ([(g,V)])(p) + ([(f,U)])(p) \left( \eval{ \pdv{x^{i}} }_{p} [(g,V)] \right)
	\end{split}
\end{align}

Nel caso in cui la dimensione della varietà sia unitaria, scriveremo

\begin{equation}
	\left( \eval{ \pdv{x^{i}} }_{p} \right)_{n=1} \doteq \eval{ \dv{t} }_{p}
\end{equation}

\subsection{Differenziale di un'applicazione liscia tra varietà}

Siano due varietà differenziabili $ N $ e $ M $ con dimensione $ n $ e $ m $ rispettivamente, e un'applicazione liscia $ F : N \to M $. Preso un punto $ p \in N $, definiamo l'applicazione lineare \textit{differenziale di} $ F $ \textit{nel punto} $ p $ come

\begin{equation}
	F_{*p} : T_{p}(N) \to T_{F(p)}(M)
\end{equation}

Sia un elemento delle derivazioni puntuali $ X_{p} \in T_{p}(N) = \der_{p}(C_{p}^{\infty}(N)) $, vogliamo definire l'azione del differenziale su $ X_{p} $

\begin{equation}
	F_{*p}(X_{p}) \in T_{F(p)}(M) = \der_{F(p)}(C_{F(p)}^{\infty}(M))
\end{equation}

Per fare questo, prendiamo un aperto $ V \subset M $ e un germe di funzioni lisce $ [(f,V)] \in C_{F(p)}^{\infty}(M) $, e definiamo:

\begin{equation}
	F_{*p}(X_{p})([(f,V)]) \doteq X_{p} \left( \left[ \left( f \circ \eval{F}_{F^{-1}(V)}, F^{-1}(V) \right) \right] \right) %
	\equiv X_{p} \left( f \circ \eval{F}_{F^{-1}(V)} \right) %
	\equiv X_{p}(f \circ F)
\end{equation}

dove

\begin{gather}
	F_{*p}(X_{p}) : C_{F(p)}^{\infty}(M) \to \R \\
	X_{p} : C_{p}^{\infty}(N) \to \R \\
	\left[ \left( f \circ \eval{F}_{F^{-1}(V)}, F^{-1}(V) \right) \right] \in C_{p}^{\infty}(N) \\
	f \circ F : N \to \R
\end{gather}

\sbs{0.5}{%
			\diagr{%
				T_{p}(N) \arrow[rr, "F_{*_{p}}"]             \&    \& T_{F(p)}(M)                                         \\
				N \arrow[rr, "F"] \arrow[rrdd, "f \circ F"'] \&    \& M \arrow[dd, "f"']                                  \\
				\&    \&                                                     \\
				\&    \& \R                                                  	
				}	
			}
	{0.5}{%
			\diagr{%
				C_{p}^{\infty}(N) \arrow[rd, "X_{p}"']       \&    \& C_{F(p)}^{\infty}(M) \arrow[ld, "F_{*_{p}}(X_{p})"] \\
				\& \R \&                                                    
				}
			}

Nella definizione, la prima identificazione è data dal fatto che questa definizione non dipende dal rappresentante di $ [(f \circ \eval{F}_{F^{-1}(V)}, F^{-1}(V))] $ scelto; per la seconda, sottintenderemo la restrizione a $ F^{-1}(V) $ per comodità di notazione. \\
In breve

\begin{equation}
	F_{*p}(X_{p})(f) \doteq X_{p}(f \circ F) \in \R
\end{equation}

L'applicazione $ F_{*p}(X_{p}) : C_{F(p)}^{\infty}(M) \to \R $ può essere vista come una derivazione puntuale dell'algebra $ C_{F(p)}^{\infty}(M) $ in quanto rispetta la $ \R $-linearità e la regola di Leibniz, poiché rispettate da $ X_{p} \in \der_{p}(C_{p}^{\infty}(N)) $:

\begin{align}
	\begin{split}
		F_{*p}(X_{p})(\lambda f + \mu g) &= X_{p}((\lambda f + \mu g) \circ F) \\
		&= X_{p}(\lambda f \circ F + \mu g \circ F) \\
		&= \lambda X_{p}(f \circ F) + \mu X_{p}(g \circ F) \\
		&= \lambda F_{*p}(X_{p})(f) + \mu F_{*p}(X_{p})(g)
	\end{split}
\end{align}

\begin{align}
	\begin{split}
		F_{*p}(X_{p})(f g) &= X_{p}((f g) \circ F) \\
		&= X_{p}((f \circ F) (g \circ F)) \\
		&= \left( X_{p}(f \circ F) \right) (g \circ F)(p) + (f \circ F)(p) \left( X_{p}(g \circ F) \right) \\
		&= \left( F_{*p}(X_{p})(f) \right) g(F(p)) + f(F(p)) \left( F_{*p}(X_{p})(g) \right)
	\end{split}
\end{align}

per qualsiasi $ f, g \in [(f,V)] \in C_{F(p)}^{\infty}(M) $ e qualsiasi $ \lambda, \mu \in \R $. \\
Queste proprietà mostrano dunque che $ F_{*p}(X_{p}) \in \der_{F(p)}(C_{F(p)}^{\infty}(M)) $.

\subsubsection{Proprietà del differenziale di un'applicazione}

\paragraph{1. Linearità}

Per mostrare che $ F_{*p} : T_{p}(N) \to T_{F(p)}(M) $ sia un'applicazione lineare, prendiamo due vettori $ X_{p},Y_{p} \in T_{p}(N) $, due numeri reali $ \alpha,\beta \in \R $, e una funzione $ f \in [(f,V)] \in C_{F(p)}^{\infty}(M) $ e facciamo agire il differenziale su una composizione lineare dei due vettori considerati:

\begin{align}
	\begin{split}
		F_{*p}(\alpha X_{p} + \beta Y_{p})(f) &= (\alpha X_{p} + \beta Y_{p})(f \circ F) \\
		&= \alpha X_{p}(f \circ F) + \beta Y_{p}(f \circ F) \\
		&= \alpha F_{*p}(X_{p})(f) + \beta F_{*p}(Y_{p})(f)
	\end{split}
\end{align}

da cui

\begin{equation}
	F_{*p}(\alpha X_{p} + \beta Y_{p}) = \alpha F_{*p}(X_{p}) + \beta F_{*p}(Y_{p})
\end{equation}

\paragraph{2. Regola della catena}

Siano $ N $, $ M $ e $ P $ tre varietà differenziabili, e $ F : N \to M $ e $ G : M \to P $ due applicazioni lisce, il differenziale della composizione delle due applicazioni in un punto $ p \in N $ vale

\begin{equation}
	(G \circ F)_{*p} = G_{*F(p)} \circ F_{*p}
\end{equation}

in quanto

\begin{gather}
	F_{*p} : T_{p}(N) \to T_{F(p)}(M) \\
	G_{*F(p)} : T_{F(p)}(M) \to T_{G(F(p))}(P)
\end{gather}

La dimostrazione di questa relazione si ottiene applicando la composizione a $ X_{p} \in T_{p}(N) $ a sua volta applicato a $ f \in [(f,W)] \in C_{G(F(p))}^{\infty}(P) $

\begin{align}
	\begin{split}
		(G \circ F)_{*p} (X_{p})(f) &= X_{p} (f \circ G \circ F) \\
		&= F_{*p} (X_{p}) (f \circ G) \\
		&= G_{*F(p)} (F_{*p} (X_{p})(f)) \\
		&= (G_{*F(p)} \circ F_{*p})(X_{p})(f)
	\end{split}
\end{align}

\paragraph{3. Proprietà funtoriali}\label{par:funt-prop}

Il differenziale dell'identità $ \id_{N} : N \to N $ in punto $ p \in N $ è l'identità dello spazio tangente, i.e.
	
\begin{equation}
	(\id_{N})_{*p} = \id_{T_{p}(N)}
\end{equation}

in quanto

\begin{equation}
		(\id_{N})_{*p} (X_{p})(f) = X_{p} (f \circ \id_{N}) %
		= X_{p}(f) %
		\qcomma \forall f \in [(f,U)] \in C^{\infty}_{p}(N)
\end{equation}

Sia un diffeomorfismo $ F : N \to M $, allora il differenziale $ F_{*p} : T_{p}(N) \to T_{F(p)}(M) $ è un isomorfismo tra spazi vettoriali: essendo $ F $ un diffeomorfismo, esiste una funzione $ G : M \to N $ per cui

\begin{equation}
	\begin{cases}
		G \circ F = \id_{N} \\
		F \circ G = \id_{M}
	\end{cases}
\end{equation}

dunque

\begin{gather}
	(G \circ F)_{*p} = G_{*F(p)} \circ F_{*p} %
	= (\id_{N})_{*p} %
	= \id_{T_{p}(N)} \\
	%
	(F \circ G)_{*F(p)} = F_{*p} \circ G_{*F(p)} %
	= (\id_{M})_{*F(p)} %
	= \id_{T_{F(p)}(M)}
\end{gather}

i.e.

\begin{equation}
	\begin{cases}
		G_{*F(p)} \circ F_{*p} = \id_{T_{p}(N)} \\
		F_{*p} \circ G_{*F(p)} = \id_{T_{F(p)}(M)}
	\end{cases}%
	\implies %
	(F_{*p})^{-1} = G_{*F(p)}
\end{equation}

dunque esiste un'applicazione $ \R $-lineare invertibile che lega i due spazi vettoriali, i.e. $ T_{p}(N) \simeq T_{F(p)}(M) $ attraverso il differenziale $ F_{*p} $.

\paragraph{4. Invarianza differenziabile della dimensione}

\begin{theorem}
	Sia $ F : U \to V $ un diffeomorfismo tra gli aperti $ U \subset \R^{n} $ e $ V \subset \R^{m} $, allora $ n = m $.
\end{theorem}

Conseguentemente a questo teorema, la definizione di varietà differenziabile è ben posta in quanto un diffeomorfismo non può legare due aperti di dimensione diversa.

\begin{proof}
	Essendo $ F : U \to V $ un diffeomorfismo, sappiamo che\footnote{%
		Si possono identificare gli spazi tangenti in $ U $ e $ V $ con i rispettivi in $ \R^{n} $ e $ \R^{m} $ in quanto l'algebra considerata non dipende dall'aperto scelto intorno al punto, e.g. $ C_{p}^{\infty}(U) = C_{p}^{\infty}(\R^{n}) $.%
	}
	
	\begin{equation}
		F_{*p} : T_{p}(U) = T_{p}(\R^{n}) \to T_{F(p)}(V) = T_{F(p)}(\R^{m})
	\end{equation}

	è un isomorfismo, ma un isomorfismo tra due spazi di dimensione finita può esistere solo se i due spazi hanno la stessa dimensione, il che implica
	
	\begin{equation}
		\dim(T_{p}(\R^{n})) = \dim(T_{F(p)}(\R^{m})) \implies n = m
	\end{equation}
\end{proof}

Nel caso topologico non si può utilizzare questo procedimento in quanto ha senso calcolare il differenziale solo per funzioni che siano lisce, condizione non necessaria per le funzioni in topologia.

\paragraph{5. Base per $ T_{p}(N) $}

Siano $ N $ una varietà differenziabile, $ (U,\varphi) $ una sua carta con $ \varphi = (x^{1},\dots,x^{n}) $ e $ p \in U $ un punto della varietà: l'insieme

\begin{equation}
	\B_{T_{p}(N)} = \left\{ \eval{ \pdv{x^{1}} }_{p}, \dots, \eval{ \pdv{x^{n}} }_{p} \right\}
\end{equation}

è una base per $ T_{p}(N) = \der_{p}(C_{p}^{\infty}(N)) $ in quanto

\begin{equation}
	\eval{ \pdv{x^{i}} }_{p} \in T_{p}(N) \qcomma \forall i=1,\dots,n
\end{equation}

conseguentemente

\begin{equation}
	\dim(T_{p}(N)) = \dim(N)
\end{equation}

\begin{proof}
	Sappiamo che l'applicazione $ \varphi : U \to \varphi(U) \subset \R^{n} $ appartenente alla carta $ (U,\varphi) $ non è solo un omeomorfismo ma anche un diffeomorfismo, dunque il suo differenziale
	
	\begin{equation}
		\varphi_{*p} : T_{p}(U) = T_{p}(N) \to T_{\varphi(p)}(\varphi(U)) = T_{\varphi(p)}(\R^{n})
	\end{equation}

	è un isomorfismo tra spazi vettoriali. \\
	Calcoliamo ora l'azione di $ \varphi_{*p} $ su un vettore generico dell'insieme $ \mathcal{B}_{T_{p}(N)} $ applicato a una qualunque funzione $ f \in [(f,U)] \in C_{p}^{\infty}(N) $:
	
	\begin{align}
		\begin{split}
			\varphi_{*p} \left( \eval{ \pdv{x^{i}} }_{p} \right) (f) &= \eval{ \pdv{x^{i}} }_{p} (f \circ \varphi) \\
			&= \eval{ \pdv{r^{i}} }_{\varphi(p)} (f \circ \varphi \circ \varphi^{-1}) \\
			&= \eval{ \pdv{r^{i}} }_{\varphi(p)} f
		\end{split}
	\end{align}

	perciò
	
	\begin{equation}
		\varphi_{*p} \left( \eval{ \pdv{x^{i}} }_{p} \right) = \eval{ \pdv{r^{i}} }_{\varphi(p)} \qcomma \forall i=1,\dots,n
	\end{equation}

	Essendo la derivata parziale in $ r^{i} $ un elemento di base per $ T_{\varphi(p)}(\R^{n}) $ e $ \varphi_{*p} $ un isomorfismo che, in quanto tale, porta basi in basi, le derivate parziali in $ x^{i} $ sono quindi una base per $ T_{p}(N) $, i.e.
	
	\begin{equation}
		\ev{ \eval{ \pdv{x^{i}} }_{p} } = T_{p}(N)
	\end{equation}
\end{proof}

\paragraph{6. Matrice dell'applicazione differenziale (aperti euclidei)}

Siano $ U \subset \R^{n} $ e $ V \subset \R^{m} $ due aperti euclidei, $ F : U \to V $ un'applicazione liscia e $ p \in U $ un punto, il differenziale dell'applicazione sarà

\begin{equation}
	F_{*p} : T_{p}(U) = T_{p}(\R^{n}) \to T_{F(p)}(V) = T_{F(p)}(\R^{m})
\end{equation}

Siano le basi rispettivamente per $ T_{p}(U) $ e $ T_{F(p)}(V) $:

\begin{gather}
	\B_{T_{p}(U)} = \left\{ \eval{ \pdv{r^{1}} }_{p}, \dots, \eval{ \pdv{r^{n}} }_{p} \right\} \\
	\B_{T_{F(p)}(V)} = \left\{ \eval{ \pdv{r^{1}} }_{F(p)}, \dots, \eval{ \pdv{r^{m}} }_{F(p)} \right\}
\end{gather}

Essendo il differenziale un'applicazione lineare, gli elementi della matrice relativa a $ F_{*p} $ sono dati dalle incognite $ a_{kj} $ della seguente equazione:

\begin{equation}
	F_{*p} \left( \eval{ \pdv{r^{j}} }_{p} \right) = \sum_{k=1}^{m} a_{kj} \left( \eval{ \pdv{r^{k}} }_{F(p)} \right) \qcomma j=1,\dots,n
\end{equation}

Per trovarle, applichiamo entrambi i membri dell'equazione allo stesso oggetto, i.e. la proiezione naturale in $ \R^{m} $

\map{r^{i}}
	{\R^{m}}{\R}
	{(x^{1},\dots,x^{m})}{x^{i}}

dunque prendiamo il secondo membro

\begin{align}
	\begin{split}
		\sum_{k=1}^{m} a_{kj} \left( \eval{ \pdv{r^{k}} }_{F(p)} r^{i} \right) &= \sum_{k=1}^{m} a_{kj} \pdv{r^{i}}{r^{k}} \, (F(p)) \\
		&= \sum_{k=1}^{m} a_{kj} \pdv{r^{i}}{r^{k}} \\
		&= \sum_{k=1}^{m} a_{kj} \delta_{ik} \\
		&= a_{ij} \in M_{m,n}(\R)
	\end{split}
\end{align}

Siccome

\map{F}
	{U}{V}
	{q}{(F^{1}(q),\dots,F^{m}(q))}

allora

\map{r^{i} \circ F}
	{U}{V}
	{q}{F^{i}(q)}

A questo punto possiamo calcolare il primo membro applicato a $ r^{i} $

\begin{equation}
	F_{*p} \left( \eval{ \pdv{r^{j}} }_{p} \right) (r^{i}) = \eval{ \pdv{r^{j}} }_{p} (r^{i} \circ F) %
	= \eval{ \pdv{r^{j}} }_{p} F^{i} %
	= \pdv{F^{i}}{r^{j}} \, (p)
\end{equation}

perciò

\begin{equation}
	a_{ij} = \pdv{F^{i}}{r^{j}} \, (p)
\end{equation}

dunque le incognite corrispondono alle entrate della matrice jacobiana della funzione $ F $, i.e. $ [a_{ij}] = J(F)(p) $. \\
In generale, sia un'applicazione lineare

\map{L}
	{\R^{n}}{\R^{m}}
	{v}{A v}

dove $ v = \bmqty{ r^{1} & \cdots & r^{n} }^{T} $ e $ A = [A_{ij}] \in M_{m,n}(\R) $, allora il suo jacobiano coincide proprio con la matrice della rappresentazione della funzione, i.e. $ J(L) \equiv A $. Dato questo, possiamo scrivere che se un'applicazione è lineare allora $ L_{*v} \equiv L $. \\
Per un esempio, vedi Esercizio \ref{exer2-8}.

\paragraph{7. Cambio di base}

Siano una varietà differenziabile $ N $, un suo punto $ p \in N $, e due sue carte $ (U,\varphi) $ e $ (U,\psi) $ con $ \varphi = (x^{1},\dots,x^{n}) $ e $ \psi = (y^{1},\dots,y^{n}) $. Le basi per $ T_{p}(N) $ date da queste due funzioni sono legate da

\begin{equation}
	\eval{ \pdv{x^{j}} }_{p} = \sum_{k=1}^{n} \pdv{y^{k}}{x^{j}} \, (p) \left( \eval{ \pdv{y^{k}} }_{p} \right)
\end{equation}

\begin{proof}
	La derivata in $ x^{j} $ appartiene allo spazio tangente $ T_{p}(N) $, dunque
	
	\begin{equation}
		\eval{ \pdv{x^{j}} }_{p} = \sum_{k=1}^{n} a_{kj} \left( \eval{ \pdv{y^{k}} }_{p} \right)
	\end{equation}

	Per trovare le $ a_{kj} $ applico entrambi i membri dell'equazione precedente alla funzione $ y^{i} : U \to \R $. Per il secondo membro
	
	\begin{equation}
		\sum_{k=1}^{n} a_{kj} \left( \eval{ \pdv{y^{k}} }_{p} y^{i} \right) = \sum_{k=1}^{n} a_{kj} \pdv{y^{i}}{y^{k}} \, (p) %
		= \sum_{k=1}^{n} a_{kj} \pdv{y^{i}}{y^{k}} %
		= \sum_{k=1}^{n} a_{kj} \delta_{ik} %
		= a_{ij}
	\end{equation}

	per il primo membro
	
	\begin{equation}
		\eval{ \pdv{x^{j}} }_{p} y^{j} = \pdv{y^{i}}{x^{j}} \, (p)
	\end{equation}

	perciò
	
	\begin{equation}
		a_{ij} = \pdv{y^{i}}{x^{j}} \, (p)
	\end{equation}
\end{proof}

Per un esempio, vedi Esercizio \ref{exer2-9}.

\paragraph{8. Matrice dell'applicazione differenziale (varietà differenziabili)}

Siano un'applicazione liscia $ F : N \to M $, un punto $ p \in N $, e due carte

\begin{equation}
	\begin{cases}
		(U,\varphi) = (U; x^{1},\dots,x^{n}) \in N \\
		(V,\psi) = (V; y^{1},\dots,y^{m}) \in M \\
		F(U) \subset V
	\end{cases}
\end{equation}

Siano inoltre le basi per gli spazi tangenti

\begin{gather}
	\B_{T_{p}(N)} = \left\{ \eval{ \pdv{x^{1}} }_{p}, \dots, \eval{ \pdv{x^{n}} }_{p} \right\} \\
	\B_{T_{F(p)}(M)} = \left\{ \eval{ \pdv{y^{1}} }_{F(p)}, \dots, \eval{ \pdv{y^{m}} }_{F(p)} \right\}
\end{gather}

La matrice associata al differenziale $ F_{*p} : T_{p}(N) \to T_{F(p)}(M) $ rispetto alla base $ \B_{T_{p}(N)} $ è la matrice jacobiana

\begin{equation}
	J(F)(p) = \left[ \pdv{F^{i}}{x^{j}} \, (p) \right] \in M_{m,n}(\R)
\end{equation}

perciò

\begin{equation}
	F_{*p} \left( \eval{ \pdv{x^{j}} }_{p} \right) = \sum_{k=1}^{n} \pdv{F^{k}}{x^{j}} \, (p) \left( \eval{ \pdv{y^{k}} }_{F(p)} \right)
\end{equation}

\begin{proof}
	In generale, essendo il differenziale una funzione lineare, possiamo scrivere
	
	\begin{equation}
		F_{*p} \left( \eval{ \pdv{x^{j}} }_{p} \right) = \sum_{k=1}^{n} a_{kj} \left( \eval{ \pdv{y^{k}} }_{F(p)} \right)
	\end{equation}
	
	Se applichiamo il secondo membro alla funzione $ y^{i} $
	
	\begin{equation}
		\sum_{k=1}^{n} a_{kj} \left( \eval{ \pdv{y^{k}} }_{F(p)} \right) (y^{i}) = \sum_{k=1}^{n} a_{kj} \pdv{y^{i}}{y^{k}} \, (F(p)) %
		= \sum_{k=1}^{n} a_{kj} \pdv{y^{i}}{y^{k}} %
		= \sum_{k=1}^{n} a_{kj} \delta_{ik} %
		= a_{ij}
	\end{equation}
	
	mentre per il primo membro
	
	\begin{equation}
		F_{*p} \left( \eval{ \pdv{x^{j}} }_{p} \right) (y^{i}) = \eval{ \pdv{x^{j}} }_{p} (y^{i} \circ F) %
		= \eval{ \pdv{x^{j}} }_{p} F^{i} %
		= \pdv{F^{i}}{x^{j}} \, (p)
	\end{equation}

	perciò
	
	\begin{equation}
	a_{ij} = \pdv{F^{i}}{x^{j}} \, (p)
	\end{equation}
\end{proof}

\paragraph{9. IFT tramite differenziale}

Possiamo enunciare il teorema della funzione inversa utilizzando il differenziale al posto dello jacobiano:

\begin{theorem}\label{thm:diffeo-loc-iso-ift}
	Sia $ F : N \to M $ un'applicazione liscia dove $ \dim(N) = \dim(M) $ e sia un punto $ p \in N $, allora $ F $ è un diffeomorfismo locale in $ p $ se e solo se il suo differenziale $ F_{*p} $ è un isomorfismo.
\end{theorem}

Questo perché lo jacobiano è invertibile se e solo se il differenziale è un isomorfismo: perché un'applicazione sia un isomorfismo è necessario che la matrice associata sia invertibile, i.e. che il determinante della matrice associata non sia nullo.

\subsection{Categorie}

Una \textit{categoria} $ \mathfrak{C} $ è definita come una collezione\footnote{%
	Utilizzeremo il termine "collezione" al posto di "insieme" per evitare contraddizioni logiche, ad esempio in situazioni dove potrebbe essere presente la locuzione "insiemi di insiemi".%
} $ \ob(\mathfrak{C}) $ di oggetti, e.g. insiemi, varietà differenziali o gruppi. Per ogni coppia di elementi $ A,B \in \ob(\mathfrak{C}) $ appartenenti alla categoria $ \mathfrak{C} $ esiste un insieme non vuoto $ \mor(A,B) $ dei \textit{morfismi}\footnote{%
	Un morfismo è una mappa tra due oggetti che conserva la struttura dell'oggetto sorgente nell'oggetto obbiettivo; ad esempio, può essere pensato come la generalizzazione nella teoria delle categorie delle funzioni in teoria degli insiemi o delle funzioni continue in topologia.%
} da $ A $ a $ B $ tale che siano soddisfatte le seguenti condizioni:

\begin{equation}
	\begin{cases}
		\E g \circ f \in \mor(A,C), & \forall f \in \mor(A,B), \, \forall g \in \mor(B,C) \\
		\E \bigone_{A} \in \mor(A,A) \mid f \circ \bigone_{A} = f, & \forall f \in \mor(A,B) \\
		\E \bigone_{B} \in \mor(B,B) \mid \bigone_{B} \circ f = f, & \forall f \in \mor(A,B) \\
		(f \circ g) \circ h = f \circ (g \circ h), & \forall f \in \mor(A,B), \forall g \in \mor(B,C), \forall h \in \mor(C,D)
	\end{cases}
\end{equation}

Il simbolo $ \circ $ indica la composizione ma, nella teoria delle categorie, non è necessariamente la composizione come si intende per le funzioni in teoria degli insiemi; nonostante ciò, vedremo solo esempi in cui queste due definizioni di composizione coincidono.

\subsubsection{\textit{Esempi}}

In tutti gli esempi che seguono tutte le proprietà delle categorie sono soddisfatte e la composizione $ \circ $ corrisponde a quella usuale.

\paragraph{0) Insiemi}

Sia $ \mathfrak{C} $ la categoria degli insiemi, allora la collezione di oggetti e l'insieme dei morfismi sono dati da:

\begin{gather}
	\ob(\mathfrak{C}) = \{ \text{collezione di tutti gli insiemi} \} \\
	\mor(A,B) = \{ \text{applicazioni } f : A \to B \} \qcomma \forall A,B \in \ob(\mathfrak{C})
\end{gather}

\paragraph{1) Spazi vettoriali su $ \R $}

Sia $ \mathfrak{C} $ la categoria degli spazi vettoriali su $ \R $, allora la collezione di oggetti e l'insieme dei morfismi sono dati da:

\begin{gather}
	\ob(\mathfrak{C}) = \{ \text{collezione di tutti gli spazi vettoriali su } \R \} \\
	\mor(V,W) \doteq Hom(V,W) = \{ \text{applicazioni $ \R $-lineari } f : V \to W \} \qcomma \forall V,W \in \ob(\mathfrak{C})
\end{gather}

\paragraph{2) Spazi topologici}

Sia $ \mathfrak{Top} $ la categoria degli spazi topologici, allora la collezione di oggetti e l'insieme dei morfismi sono dati da:

\begin{gather}
	\ob(\mathfrak{Top}) = \{ \text{collezione di tutti gli spazi topologici} \} \\
	\mor(X,Y) \doteq C^{0}(X,Y) = \{ \text{applicazioni continue } f : X \to Y \} \qcomma \forall X,Y \in \ob(\mathfrak{Top})
\end{gather}

\paragraph{3) Varietà differenziabili}

Sia $ \mathfrak{C} $ la categoria delle varietà differenziabili, allora la collezione di oggetti e l'insieme dei morfismi sono dati da:

\begin{gather}
	\ob(\mathfrak{C}) = \{ \text{collezione di tutte le varietà differenziabili} \} \\
	\mor(N,M) = \{ \text{applicazioni lisce } F : N \to M \} \qcomma \forall N,M \in \ob(\mathfrak{C})
\end{gather}

\paragraph{4) Varietà differenziabili puntate}

Sia $ \mathfrak{C} $ la categoria delle varietà differenziabili puntate, i.e. varietà in cui è fissato un punto\footnote{%
	Due varietà differenziali puntate sono diverse anche se si cambia solo il punto fissato, e.g. $ (M,p) \neq (M,q) $ per $ p \neq q $.%
}, allora la collezione di oggetti e l'insieme dei morfismi sono dati da:

\begin{gather}
	\ob(\mathfrak{C}) = \{ \text{collezione di tutte le coppie } (M,p) \, \mid M \text{ varietà differenziabili}, \, p \in M \} \\
		\mor((N,p),(M,q)) = \{ \text{applicazioni lisce } F : N \to M \, \mid \, F(p)=q \} \qcomma \forall (N,p),(M,q) \in \ob(\mathfrak{C})
\end{gather}

\subsubsection{Isomorfismi}

Data una categoria $ \mathfrak{C} $, la definizione di oggetti isomorfi è la seguente:

\begin{gather}
	A \simeq B \text{ isomorfi} \nonumber \\
	\Updownarrow \\ %
	\E f \in \mor(A,B) \wedge g \in \mor(B,A) \, \mid \, g \circ f = \bigone_{A} \wedge f \circ g = \bigone_{B} \nonumber
\end{gather}

dove $ A,B \in \ob(\mathfrak{C}) $, $ \bigone_{A} \in \mor(A,A) $ e $ \bigone_{B} \in \mor(B,B) $. \\
Per esempio, per la categoria degli insiemi, due oggetti sono isomorfi se esistono due applicazioni che associno questi due oggetti tra loro e che siano l'una inversa dell'altra.

\subsubsection{\textit{Esempi}}

\paragraph{0) Insiemi}

Due oggetti nella categoria degli insiemi sono isomorfi se hanno la stessa cardinalità, i.e. $ A \simeq B \iff \# A = \# B $, perciò esiste una bigezione tra i due oggetti.

\paragraph{1) Spazi vettoriali su $ \R $}

Due oggetti nella categoria degli spazi vettoriali sono isomorfi quando lo sono come spazi vettoriali, i.e.

\begin{equation}
	V \simeq W \iff V \stackrel{\text{sp.vett.}}{\simeq} W
\end{equation}

\paragraph{2) Spazi topologici}

Due oggetti nella categoria degli spazi topologico sono isomorfi quando sono omeomorfi, i.e.

\begin{equation}
	X \simeq Y \iff V \stackrel{\text{omeo.}}{\simeq} W
\end{equation}

\paragraph{3) Varietà differenziabili}

Due oggetti nella categoria delle varietà differenziabili sono isomorfi quando sono diffeomorfi, i.e.

\begin{equation}
	N \simeq M \iff N \stackrel{\text{diff.}}{\simeq} M
\end{equation}

\paragraph{4) Varietà differenziabili puntate}

Due oggetti nella categoria delle varietà differenziabili puntate sono isomorfi quando sono diffeomorfi e il diffeomorfismo associa il punto di una varietà al punto dell'altra, i.e.

\begin{equation}
	(N,p) \simeq (M,q) \iff \E F : N \to M \text{ diffeomorfismo} \, \mid \, N \stackrel{\text{diff.}}{\simeq} M \wedge F(p)=q
\end{equation}

\subsection{Funtori}

Siano due categorie $ \mathfrak{C} $ e $ \mathfrak{D} $, un \textit{funtore} $ \mathcal{F} : \mathfrak{C} \to \mathfrak{D} $ è una coppia di applicazioni che agisce nel seguente modo: una delle applicazioni mappa la collezione di oggetti $ \ob(\mathfrak{C}) $ in $ \ob(\mathfrak{D}) $; l'altra applicazione mappa i morfismi $ \mor(C_{1}, C_{2}) $ in $ \mor(D_{1} ,D_{2}) $, dove $ C_{1}, C_{2} \in \ob(\mathfrak{C}) $ e $ D_{1}, D_{2} \in \ob(\mathfrak{D}) $ sono due coppie qualunque di oggetti delle rispettive categorie. I funtori si dividono in \textit{covarianti} e \textit{controvarianti}.

\paragraph{Funtori covarianti}

Siano $ \mathfrak{C} $ e $ \mathfrak{D} $ due categorie, un \textit{funtore covariante} è una coppia di applicazioni identificate da $ \mathcal{F} $ tali che

\begin{equation}
	\begin{cases}
		\mathcal{F}(A) \in \ob(\mathfrak{D}), & \forall A \in \ob(\mathfrak{C}) \\
		\mathcal{F}(f) \in \mor(\mathcal{F}(A),\mathcal{F}(B)), & \forall f \in \mor(A,B)
	\end{cases}
\end{equation}

rispettando l'identità e la composizione, i.e.

\begin{equation}
	\begin{cases}
		\mathcal{F}(\bigone_{A}) = \bigone_{\mathcal{F}(A)} \in \mor(\mathcal{F}(A),\mathcal{F}(A)), & \forall \bigone_{A} \in \mor(A,A) \\
		\mathcal{F}(g \circ f) = \mathcal{F}(g) \circ \mathcal{F}(f), & \forall f \in \mor(A,B), \forall g \in \mor(B,C)
	\end{cases}
\end{equation}

Siano $ \mathfrak{C} $ la categoria delle varietà puntate e $ \mathfrak{D} $ la categoria degli spazi vettoriali, il \textit{funtore differenziale covariante} associa

\begin{equation}
	\begin{cases}
		\mathcal{F}((N,p)) = T_{p}(N), & \forall (N,p) \in \ob(\mathfrak{C}), \, T_{p}(N) \in \ob(\mathfrak{D}) \\
		\mathcal{F}(F) = F_{*p}, & \forall F \in \mor((N,p),(M,q)), \, F_{*p} \in \mor(T_{p}(N),T_{q=F(p)}(M))
	\end{cases}
\end{equation}

Per dimostrare che sia un funtore covariante, verifichiamo che rispetti l'identità

\begin{align}
	\mathcal{F}(\bigone_{N}) = (\bigone_{N})_{*p} %
	= \bigone_{T_{p}(N)} %
	= \bigone_{\mathcal{F}((N,p))}
\end{align}

e la composizione

\begin{align}
	\mathcal{F}(G \circ F) = (G \circ F)_{*p} %
	= G_{*F(p)} \circ F_{*p} %
	= \mathcal{F}(G) \circ \mathcal{F}(F)
\end{align}

per qualunque varietà $ N $, $ M $ e $ P $ con $ p \in N $ e qualunque coppia di applicazioni lisce tra varietà che rispettino le seguenti condizioni

\begin{gather}
	F : (N,p) \to (M,F(p)) \\
	G : (M,F(p)) \to (P,G(F(p)))
\end{gather}

\paragraph{Funtori controvarianti}

Siano $ \mathfrak{C} $ e $ \mathfrak{D} $ due categorie, un \textit{funtore controvariante} è una coppia di applicazioni identificate da $ \mathcal{F} $ tali che

\begin{equation}
	\begin{cases}
		\mathcal{F}(A) \in \ob(\mathfrak{D}), & \forall A \in \ob(\mathfrak{C}) \\
		\mathcal{F}(f) \in \mor(\mathcal{F}(B),\mathcal{F}(A)), & \forall f \in \mor(A,B)
	\end{cases}
\end{equation}

rispettando l'identità e la composizione, i.e.

\begin{equation}
	\begin{cases}
		\mathcal{F}(\bigone_{A}) = \bigone_{\mathcal{F}(A)} \in \mor(\mathcal{F}(A),\mathcal{F}(A)), & \forall \bigone_{A} \in \mor(A,A) \\
		\mathcal{F}(g \circ f) = \mathcal{F}(f) \circ \mathcal{F}(g) & \forall f \in \mor(A,B), \forall g \in \mor(B,C)
	\end{cases}
\end{equation}

\begin{remark}
	Sia $ \mathcal{F} : \mathfrak{C} \to \mathfrak{D} $ un funtore (covariante o controvariante), allora
	
	\begin{equation}
		A \simeq B \implies \mathcal{F}(A) \simeq \mathcal{F}(B) \qcomma \forall A,B \in \mathfrak{C}
	\end{equation}
\end{remark}

\begin{proof}
	Per funtori covarianti, la definizione di isomorfismo asserisce che
	
	\begin{gather}
		A \simeq B \nonumber \\
		\Updownarrow \\ %
		\E f \in \mor(A,B) \wedge g \in \mor(B,A) \, \mid \, g \circ f = \bigone_{A} \wedge f \circ g = \bigone_{B} \nonumber
	\end{gather}
	
	Applicando il funtore a entrambi i membri di entrambe le condizioni:
	
	\sbs{0.5}{%
				\begin{align}
					\begin{split}
						\mathcal{F}(g \circ f) &= \mathcal{F}(\bigone_{A}) \\
						\mathcal{F}(g) \circ \mathcal{F}(f) &= \bigone_{\mathcal{F}(A)}
					\end{split}
				\end{align}
				}
		{0.5}{%
				\begin{align}
					\begin{split}
						\mathcal{F}(f \circ g) &= \mathcal{F}(\bigone_{B}) \\
						\mathcal{F}(f) \circ \mathcal{F}(g) &= \bigone_{\mathcal{F}(B)}
					\end{split}
				\end{align}
				}
	
	le quali dimostrano che $ \mathcal{F}(A) \simeq \mathcal{F}(B) $ secondo la definizione di isomorfismo. \\
	La dimostrazione per funtori controvarianti è analoga.
\end{proof}

\subsubsection{\textit{Esempi}}

\paragraph{1) Spazio duale}

Sia $ \mathfrak{C} $ la categoria degli spazi vettoriali su $ \R $, consideriamo il funtore controvariante $ \mathcal{F} : \mathfrak{C} \to \mathfrak{C} $ tale che

\begin{equation}
	\begin{cases}
		\mathcal{F}(V) = V^{*}, & \forall V \in \ob(\mathfrak{C}) \\
		\mathcal{F}(f) = f^{*}, & \forall f \in \mor(V,W) \doteq Hom(V,W)
	\end{cases}
\end{equation}

dove $ V^{*} $ è lo spazio duale\footnote{%
	Lo spazio duale $ V^{*} $ corrisponde all'insieme degli omeomorfismi da $ V $ a $ \R $, i.e. $ V^{*} = Hom(V,\R) $.%
} di $ V $ i cui oggetti sono applicazioni da $ V $ in $ \R $ e

\map{f^{*}}
	{W^{*} = Hom(W,\R)}{V^{*} = Hom(\mathcal{F}(W),\mathcal{F}(V)) = Hom(W^{*},V^{*})}
	{\alpha}{f^{*}(\alpha) = \alpha \circ f}

Per dimostrare che sia un funtore controvariante, verifichiamo che rispetti l'identità

\begin{align}
	\mathcal{F}(\bigone_{V}) = \bigone_{V}^{*} %
	= \bigone_{V^{*}} %
	= \bigone_{\mathcal{F}(V)}
\end{align}

in quanto

\begin{equation}
	\begin{cases}
		\bigone_{V} : V \to V \\
		\bigone_{V}^{*} : V^{*} \to V^{*}
	\end{cases}
\end{equation}

e la composizione

\begin{align}
	\mathcal{F}(g \circ f) = (g \circ f)^{*} %
	= f^{*} \circ g^{*} %
	= \mathcal{F}(f) \circ \mathcal{F}(g) %
	\qcomma f \in \mor(V,W), \, g \in \mor(W,Z)
\end{align}

in quanto

\begin{equation}
	(g \circ f)^{*}(\alpha) = \alpha \circ g \circ f %
	= f^{*} (\alpha \circ g) %
	= f^{*} (g^{*}(\alpha)) %
	= f^{*} \circ g^{*} \circ \alpha %
	\qcomma \forall \alpha \in W^{*}
\end{equation}

\paragraph{2) Varietà puntate e isomorfismi}

Siano $ \mathfrak{C} $ la categoria delle varietà puntate e $ \mathfrak{D} $ la categoria degli spazi vettoriali. Il funtore differenziale covariante associa

\begin{equation}
	\begin{cases}
		\mathcal{F}((N,p)) = T_{p}(N), & \forall (N,p) \in \ob(\mathfrak{C}), \, T_{p}(N) \in \ob(\mathfrak{D}) \\
		\mathcal{F}(F) = F_{*p} & \forall F \in \mor((N,p),(M,q)), \, F_{*p} \in \mor(T_{p}(N),T_{q=F(p)}(M))
	\end{cases}
\end{equation}

A questo punto, se due varietà puntate sono diffeomorfe $ (N,p) \simeq (M,q) $ allora i loro due spazi tangenti saranno isomorfi $ T_{p}(N) \simeq T_{q=F(p)}(M) $.

\subsection{Curve su una varietà differenziabile}

\sbs{0.5}{%
			Una \textit{curva} liscia $ c $ su una varietà differenziabile $ N $ è un'applicazione liscia tale che
			
			\begin{equation}
				c : (a,b) \to N \qcomma (a,b) \subset \R
			\end{equation}
		
			Se $ 0 \in (a,b) $ diremo che $ c $ \textit{inizia} in $ p = c(0) $.
			}
	{0.5}{%
			\img{0.9}{img24}
			}

Definiamo il \textit{vettore tangente} o \textit{vettore velocità} $ c'(t_{0}) $ alla curva $ c $ nel punto $ c(t_{0}) \in c((a,b)) \subset N $ come

\begin{equation}
	c'(t_{0}) \doteq c_{*t_{0}} \left( \eval{ \dv{t} }_{t_{0}} \right) \in T_{c(t_{0})}(N)
\end{equation}

dove il differenziale è dato da

\begin{equation}
	c_{*t_{0}} : T_{t_{0}}((a,b)) = T_{t_{0}}(\R) \to T_{c(t_{0})}(N)
\end{equation}

Il vettore tangente $ c'(t_{0}) $ è dunque l'immagine del vettore $ \eval{ \dv*{t} }_{t_{0}} $ tramite l'applicazione lineare del differenziale $ c_{*t} $; il vettore $ \eval{ \dv*{t} }_{t_{0}} $ rappresenta la base canonica di $ T_{t_{0}}(\R) $.

\begin{remark}
	Sia $ c : (a,b) \to \R $ una curva sui reali. Possiamo considerare sia la derivata di $ c $ nel punto $ t_{0} \in (a,b) $ sia il vettore tangente alla curva tramite le notazioni seguenti:
	
	\begin{equation}
		\begin{cases}
			\dot{c}(t_{0}) \doteq \eval{ \ddv{t} }_{t_{0}} c(t) \in \R & \text{derivata} \\ \\
			%
			c'(t_{0}) \in T_{c(t_{0})}(\R) & \text{vettore tangente}
		\end{cases}
	\end{equation}
\end{remark}

Siccome $ c'(t_{0}) \in T_{c(t_{0})}(\R) $, possiamo scrivere il vettore tangente come

\begin{equation}
	c'(t_{0}) = a \eval{ \dv{r} }_{c(t_{0})}
\end{equation}

dove $ a \in \R $ e $ \ev{ \eval{ \dv*{r} }_{c(t_{0})} } = T_{c(t_{0})}(\R) $. \\
Per determinare $ a $, applichiamo entrambi i membri dell'equazione alla funzione identità

\map{r}
	{\R}{\R}
	{x}{x}

Per il secondo membro

\begin{equation}
	\left( a \eval{ \dv{r} }_{c(t_{0})} \right) r = a \eval{ \dv{r}{r} }_{c(t_{0})} %
	= a \, \dv{r}{r} %
	= a
\end{equation}

e per il primo membro

\begin{equation}
	c'(t_{0})(r) = c_{*t_{0}} \left( \eval{ \ddv{t} }_{t_{0}} \right)(r) %
	= \eval{ \ddv{t} }_{t_{0}} (r \circ c) %
	= \eval{ \ddv{t} }_{t_{0}} c %
	= \dot{c}(t_{0})
\end{equation}

perciò

\begin{equation}
	c'(t_{0}) = \dot{c}(t_{0}) \eval{ \dv{r} }_{c(t_{0})}
\end{equation}

nel caso in cui il codominio sia $ \R $.

\begin{definition}[Espressione locale del vettore tangente a una curva liscia]\label{prop:loc-exp-tan-cur}
	Siano $ c : (a,b) \to N $ una curva liscia e $ (U,\varphi) \in N $ una carta con $ \varphi = (x^{1},\dots,x^{n}) $ tale che $ c((a,b)) \subset U $.
	
	\img{0.5}{img25}
	
	Siano $ c^{i} $ le componenti di $ c $ rispetto alla carta fissata
	
	\begin{equation}
		c^{i} = x^{i} \circ c : (a,b) \to \R \qcomma i=1,\dots,n
	\end{equation}
	
	Il vettore tangente alla curva si scrive come
	
	\begin{equation}
		c'(t_{0}) = \sum_{i=1}^{n} \dot{c}_{i}(t_{0}) \eval{ \pdv{x^{i}} }_{c(t_{0})}
	\end{equation}

	In particolare, se $ N = \R^{n} $ allora
	
	\begin{equation}
		c'(t_{0}) = \sum_{i=1}^{n} \dot{c}_{i}(t_{0}) \eval{ \pdv{r^{i}} }_{c(t_{0})}
	\end{equation}

	dove $ \dot{c}_{i}(t_{0}) $ può essere identificato con il vettore colonna $ \bmqty{ \dot{c}_{1}(t_{0}) & \cdots & \dot{c}_{n}(t_{0}) }^{T} $.
\end{definition}

Un esempio potrebbe essere la curva

\map{c}
	{\R}{\R^{2}}
	{t}{(t,t^{3})}

da cui $ c'(t_{0}) = (1, 3 t_{0}^{2}) $, nella quale espressione si identifica il vettore tangente con il punto di $ \R^{2} $ al secondo membro. \\
Per un altro esempio, vedi Esercizio \ref{exer2-10}.

\begin{proof}
	Per definizione $ c'(t_{0}) \in T_{c(t_{0})}(N) $, dunque
	
	\begin{equation}
		c'(t_{0}) = \sum_{i=1}^{n} a_{i} \eval{ \pdv{x^{i}} }_{c(t_{0})}
	\end{equation}

	con $ a^{i} \in \R $ per $ i=1,\dots,n $, in quanto $ \ev{ \eval{ \pdv*{x^{i}} }_{c(t_{0})} } = T_{c(t_{0})}(N) $. \\
	Valutiamo ora entrambi i membri applicati alla funzione coordinata
	
	\map{x^{j}}
		{U}{\R}
		{(x^{1},\dots,x^{n})}{x^{j}}
	
	per $ j=1,\dots,n $. \\
	Per il secondo membro
	
	\begin{equation}
		\left( \sum_{i=1}^{n} a_{i} \eval{ \pdv{x^{i}} }_{c(t_{0})} \right) (x^{j}) = \sum_{i=1}^{n} a_{i} \eval{ \pdv{x^{j}}{x^{i}} }_{c(t_{0})} %
		= \sum_{i=1}^{n} a_{i} \pdv{x^{j}}{x^{i}} %
		= \sum_{i=1}^{n} a_{i} \delta_{ij} %
		= a_{j}
	\end{equation}

	mentre per il primo
	
	\begin{equation}
		c'(t_{0})(x^{j}) = c_{*t_{0}} \left( \eval{ \dv{t} }_{t_{0}} \right)(x^{j}) %
		= \eval{ \dv{t} }_{t_{0}} (x^{j} \circ c) %
		= \eval{ \dv{t} }_{t_{0}} c^{j} %
		= \dot{c}_{j}(t_{0})
	\end{equation}

	dunque
	%
	\begin{equation}
		a_{j} = \dot{c}_{j}(t_{0})
	\end{equation}
\end{proof}

\subsubsection{Spazio tangente e curve}

Sia una curva liscia $ c : (a,b) \to N $ che inizi in $ p = c(0) \in N $, allora $ c'(0) \in T_{p}(N) $. Di seguito dimostriamo come ogni vettore in $ T_{p}(N) $ sia il vettore velocità di una curva liscia in $ N $ che inizia in $ p $.

\begin{definition}
	Siano una varietà differenziabile $ N $, un suo punto $ p \in N $, un vettore $ X_{p} \in T_{p}(N) $ e $ \varepsilon \in \R^{+} $, allora
	
	\begin{equation}
		\E c : (-\varepsilon,\varepsilon) \to N \text{ curva liscia}, \, c(0) = p \mid c'(0) = X_{p}
	\end{equation}

	Questo mostra che
	
	\begin{equation}
		T_{p}(N) = \{ c'(0) \mid c : (-\varepsilon,\varepsilon) \to N \text{ curva liscia}, \, c(0) = p \}
	\end{equation}
\end{definition}

Questo insieme, nonostante equivalente, è più intuitivo dell'insieme delle derivazioni puntali dell'algebra dei germi delle funzioni lisce in un punto $ p $, i.e. $ T_{p}(N) = \der_{p}(C_{p}^{\infty}(N)) $.

\begin{proof}
	Consideriamo il seguente schema:
	
	\img{0.8}{img26}

	Fissiamo una carta $ (U,\varphi) \in N $ con $ \varphi = (x^{1},\dots,x^{n}) $, $ p \in U $ e $ \varphi(p) = 0 $, la quale fisserà a sua volta una base per lo spazio tangente $ \ev{ \eval{ \pdv*{x^{i}} }_{p} } = T_{p}(N) $. A questo punto possiamo scrivere un vettore dello spazio tangente come
	
	\begin{equation}
		X_{p} = \sum_{i=1}^{n} a^{i} \eval{ \pdv{x^{i}} }_{p} \qcomma i=1,\dots,n
	\end{equation}

	in quanto abbiamo dimostrato che\footnote{%
		L'applicazione $ \varphi : U \to \varphi(U) $ è un diffeomorfismo quindi il suo differenziale è un isomorfismo che dunque porta basi in basi.%
	}
	
	\begin{equation}
		\varphi_{*p} \left( \eval{ \pdv{x^{i}} }_{p} \right) = \eval{ \pdv{r^{i}} }_{\varphi(p)}
	\end{equation}
	
	Prendiamo ora $ \varepsilon \in \R^{+} $ da cui l'intervallo $ (-\varepsilon,\varepsilon) \ni 0 $. Scegliamo $ \varepsilon $ in modo tale che l'applicazione liscia
	
	\map{\alpha}
		{(-\varepsilon,\varepsilon)}{\R^{n}}
		{t}{(a^{1} t, \dots, a^{n} t)}

	la quale associa all'intervallo sorgente una retta, risulti in $ \alpha((-\varepsilon,\varepsilon)) \subset \varphi(U) $. \\
	La curva $ c : (-\varepsilon,\varepsilon) \to N $ corrisponderà dunque a $ c = \varphi^{-1} \circ \alpha $ la quale è liscia in quanto composizione di funzioni lisce. \\
	Il vettore tangente alla curva in 0 è dunque uguale a
	
	\begin{align}
		\begin{split}
			c'(0) &= c_{*0} \left( \eval{ \dv{t} }_{0} \right) \\
			&= (\varphi^{-1} \circ \alpha)_{*0} \left( \eval{ \dv{t} }_{0} \right) \\
			&= (\varphi^{-1}_{*\alpha(0)} \circ \alpha_{*0}) \left( \eval{ \dv{t} }_{0} \right) \\
			&= (\varphi^{-1}_{*0} \circ \alpha_{*0}) \left( \eval{ \dv{t} }_{0} \right) \\
			&= \varphi^{-1}_{*0} \left( \alpha_{*0} \left( \eval{ \dv{t} }_{0} \right) \right) \\
			&= \varphi^{-1}_{*0} \left( \sum_{i=1}^{n} \dot{\alpha}_{i}(0) \eval{ \pdv{r^{i}} }_{0} \right) \\
			&= \varphi^{-1}_{*0} \left( \sum_{i=1}^{n} a^{i} \eval{ \pdv{r^{i}} }_{0} \right) \\
			&= \sum_{i=1}^{n} a^{i} \varphi^{-1}_{*0} \left( \eval{ \pdv{r^{i}} }_{0} \right) \\
			&= \sum_{i=1}^{n} a^{i} \eval{ \pdv{x^{i}} }_{p} \\
			&= X_{p}
		\end{split}
	\end{align}

	dove $ \varphi(p) = 0 $, $ \alpha_{*0} $ con $ 0 \in \R $ mentre $ \varphi^{-1}_{*0} $ con $ \alpha(0) = 0 \in \R^{n} $, inoltre
	
	\begin{equation}
		\alpha_{*0} \left( \eval{ \dv{t} }_{0} \right) = \alpha'(0) %
		= \sum_{i=1}^{n} a^{i} \eval{ \pdv{r^{i}} }_{0}
	\end{equation}

	per la Proposizione \ref{prop:loc-exp-tan-cur} in quanto $ \alpha $ è anch'essa una curva liscia e possiamo portare $ \varphi_{*0} $ dentro la sommatoria in quanto applicazione lineare.
\end{proof}

\subsection{Differenziale tramite curve}

Siano un vettore $ X_{p} \in T_{p}(N) $ e una funzione liscia $ f \in C^{\infty}(N) $, la notazione $ X_{p}(f) \in \R $ indica la derivata (direzionale) di $ f $ rispetto al vettore $ X_{p} $. \\
Siccome lo spazio tangente può essere pensato come costituito da vettori velocità, esisterà una curva $ c : (-\varepsilon,\varepsilon) \to N $ con $ p = c(0) $ tale che $ X_{p} = c'(0) $: da queste premesse, possiamo scrivere la derivata direzionale in funzione della curva considerata

\begin{equation}
	X_{p} f = c'(0) (f) %
	= c_{*0} \left( \eval{ \dv{t} }_{0} \right) (f) %
	= \eval{ \dv{t} }_{0} (f \circ c) %
	= \dot{(f \circ c)}(0)
\end{equation}

dove $ f \circ c : (-\varepsilon,\varepsilon) \to \R $, perciò

\begin{equation}
	X_{p} f = \dot{(f \circ c)}(0)
\end{equation}

\begin{definition}[Differenziale tramite curve]
	Siano un'applicazione liscia $ F : N \to M $, un punto $ p \in N $ e un vettore $ X_{p} \in T_{p}(N) $, allora il differenziale $ F_{*p} : T_{p}(N) \to T_{F(p)}(M) $ si può scrivere come
	
	\begin{equation}
		F_{*p}(X_{p}) = (F \circ c)' (0)
	\end{equation}

	dove $ c : (-\varepsilon,\varepsilon) \to N $ con $ p = c(0) $ è una curva per il quale vale $ X_{p} = c'(0) $; la curva $ F \circ c : (-\varepsilon,\varepsilon) \to M $ è ancora una curva che passa per $ F(p) = F(c(0)) $, il cui vettore tangente in $ F(p) $ equivale al differenziale $ F_{*p}(X_{p}) $.
	
	\img{0.9}{img27}
\end{definition}

\begin{proof}
	\begin{align}
		\begin{split}
			F_{*p}(X_{p}) &= F_{*p}(c'(0)) \\
			&= F_{*p} \left( c_{*0} \left( \eval{ \dv{t} }_{0} \right) \right) \\
			&= (F_{*c(0)} \circ c_{*0}) \left( \eval{ \dv{t} }_{0} \right) \\
			&= (F \circ c)_{*0} \left( \eval{ \dv{t} }_{0} \right) \\
			&= (F \circ c)' (0)
		\end{split}
	\end{align}

	in quanto $ p = c(0) $.
\end{proof}

\begin{corollary}
	Consideriamo il caso in cui lo spazio ambiente siano i numeri reali. Siano un'applicazione liscia $ f : N \to \R $ e un vettore $ X_{p} \in T_{p}(N) $, allora
	
	\begin{equation}
		f_{*p}(X_{p}) = (X_{p} f) \eval{ \dv{r} }_{f(p)}
	\end{equation}

	dove $ f_{*p} : T_{p}(N) \to T_{f(p)}(\R) $ e $ \ev{ \eval{ \dv*{r} }_{f(p)} } = T_{f(p)}(\R) $.
\end{corollary}

\begin{proof}
	Sia $ c : (-\varepsilon,\varepsilon) \to N $ con $ p = c(0) $ una curva per il quale vale $ X_{p} = c'(0) $. Per le funzioni da $ \R $ in $ \R $ la composizione può essere scritta in funzione della derivata e, nel caso specifico di $ f \circ c : (-\varepsilon,\varepsilon) \to \R $, abbiamo che
	
	\begin{equation}
		(f \circ c)'(0) = \dot{(f \circ c)}(0) \eval{ \dv{r} }_{f(p)}
	\end{equation}

	a questo punto possiamo scrivere il differenziale come
	
	\begin{equation}
		f_{*p}(X_{p}) = (f \circ c)' (0) %
		= \dot{(f \circ c)}(0) \eval{ \dv{r} }_{f(p)} %
		= (X_{p} f) \eval{ \dv{r} }_{f(p)}
	\end{equation}
\end{proof}

\subsubsection{\textit{Esempi}}

\paragraph{1) Differenziale della moltiplicazione per una matrice}\label{example:trasl-diff}

Consideriamo l'insieme delle matrici invertibili

\begin{equation}
	GL_{n}(\R) = \left\{ A \in M_{n}(\R) \st \det A \neq 0 \right\} \subset M_{n}(\R)
\end{equation}

il quale è un aperto di $ M_{n}(\R) = \R^{n^{2}} $ e una varietà differenziabile di dimensione $ n^{2} $. \\
Fissata una matrice $ g \in GL_{n}(\R) $, definiamo l'applicazione liscia

\map{L_{g}}
	{GL_{n}(\R)}{GL_{n}(\R)}
	{h}{g h}

chiamata \textit{traslazione a sinistra tramite} $ g $. \\
L'obbiettivo è calcolare il differenziale di questa applicazione sulla matrice identità\footnote{%
	Considereremo il differenziale applicato alla matrice identità in quanto il differenziale è un'applicazione lineare, dunque da questo esempio è possibile derivare il risultato del differenziale applicato ad altre matrici.%
} $ I \in GL_{n}(\R) $:

\map{L_{g_{*I}}}
	{T_{I}(GL_{n}(\R))}{T_{g}(GL_{n}(\R))}
	{X_{I}}{(L_{g})_{*I}(X_{I})}

Innanzitutto, il dominio del differenziale è dato da

\begin{equation}
	T_{I}(GL_{n}(\R)) = \left\{ \sum_{i,j=1}^{n} X_{ij} \eval{ \pdv{r_{ij}} }_{I} \st X=[X_{ij}] \in M_{n}(\R) \right\}
\end{equation}

dove $ \ev{ \eval{ \pdv*{r_{ij}} }_{I} } = T_{I}(GL_{n}(\R)) $; a ogni vettore $ X_{I} \in T_{I}(GL_{n}(\R)) $ si può associare il vettore $ X \in M_{n}(\R) $ prendendo semplicemente $ X=[X_{ij}] $ dalla definizione dello spazio $ T_{I}(GL_{n}(\R)) $. Per quanto riguarda il codominio

\begin{equation}
	T_{g}(GL_{n}(\R)) = \left\{ \sum_{i,j=1}^{n} X_{ij} \eval{ \pdv{r_{ij}} }_{g} \st X=[X_{ij}] \in M_{n}(\R) \right\}
\end{equation}

esattamente come per lo spazio di partenza, anche questo si identifica con $ M_{n}(\R) $. \\
Per calcolare il differenziale di $ L_{g} $, consideriamo una curva di matrici in $ GL_{n}(\R) $ il cui vettore tangente sia $ X_{I} $, i.e.

\begin{equation}
	\begin{cases}
		c : (-\varepsilon,\varepsilon) \to GL_{n}(\R) \\
		c(0) = I \\
		c'(0) = X_{I} = \displaystyle\sum_{i,j=1}^{n} X_{ij} \eval{ \pdv{r_{ij}} }_{I}
	\end{cases}
\end{equation}

perciò

\begin{equation}
	(L_{g})_{*I}(X_{I}) = (L_{g} \circ c)'(0) %
	= (g \, c(t))'(0) %
	= \sum_{i,j=1}^{n} \dot{(g \, c(t))}_{ij}(0) \eval{ \pdv{r_{ij}} }_{g}
\end{equation}

dove $ g \, c(t) $ presuppone la moltiplicazione tra matrici e ricordando che un vettore tangente può essere scritto come

\begin{equation}
	c'(t_{0}) = \sum_{i=1}^{n} \dot{c}_{i}(t_{0}) \eval{ \pdv{x^{i}} }_{t_{0}}
\end{equation}

considerando $ c : (a,b) \to N $ e $ (U,\varphi) \in N $ con $ \varphi=(x^{1},\dots,x^{n}) $ da cui $ c^{i} = x^{i} \circ c $. \\
Sappiamo che vale la regola di Leibniz per le matrici, i.e.

\begin{equation}
	\dot{A(t) \, B(t)} = \dot{A}(t) \, B(t) + A(t) \, \dot{B}(t) \qcomma \forall A(t),B(t) \in M_{n}(\R)
\end{equation}

e che $ g $ non dipende da $ t $, dunque

\begin{equation}
	(L_{g})_{*I}(X_{I}) = \sum_{i,j=1}^{n} \dot{(g \, c(t))}_{ij}(0) \eval{ \pdv{r_{ij}} }_{g} %
	= \sum_{i,j=1}^{n} (g \, \dot{c}(0))_{ij} \eval{ \pdv{r_{ij}} }_{g}
\end{equation}

Siccome

\begin{align}
	\begin{split}
		c'(0) &= X_{I} \\
		\sum_{i,j=1}^{n} \dot{c}(0)_{ij} \eval{ \pdv{r_{ij}} }_{I} &= \sum_{i,j=1}^{n} X_{ij} \eval{ \pdv{r_{ij}} }_{I} \\
		\dot{c}(0)_{ij} &= X_{ij}
	\end{split}
\end{align}

abbiamo che $ \dot{c}(0) = X = [X_{ij}] $ perciò

\begin{equation}
	(L_{g})_{*I}(X_{I}) = \sum_{i,j=1}^{n} (g \, \dot{c}(0))_{ij} \eval{ \pdv{r_{ij}} }_{g} %
	= \sum_{i,j=1}^{n} (g \, X)_{ij} \eval{ \pdv{r_{ij}} }_{g}
\end{equation}

che può essere identificato naturalmente con la matrice $ g \, X $ quindi, tramite le identificazioni $ X_{I} \equiv X $ e $ (L_{g})_{*I}(X_{I}) \equiv g \, X $ derivate dall'identificazione tra gli spazi $ T_{I}(GL_{n}(\R)) = T_{g}(GL_{n}(\R)) = M_{n}(\R) $, possiamo scrivere

\begin{equation}
	(L_{g})_{*I}(X) = g \, X
\end{equation}

Il differenziale di $ L_{g} $ valutato nella matrice identità $ I $ corrisponde dunque alla moltiplicazione a sinistra per $ g $, i.e. $ (L_{g})_{*I} \equiv L_{g} $ in quanto la moltiplicazione a sinistra per $ g $ è un'applicazione lineare\footnote{%
	$ (L_{g})_{*I} $ e $ L_{g} $ hanno però dei domini differenti: la prima agisce sui vettori tangenti di $ T_{I}(GL_{n}(\R)) $ mentre la seconda sulle matrici invertibili $ GL_{n}(\R) $.%
}.

\paragraph{2) Differenziale della moltiplicazione e dell'inversione in un gruppo di Lie}\label{example:differ-prod-inv-lie}

Siano $ G $ un gruppo di Lie\footnote{%
	Se $ G $ è un gruppo di Lie, allora è una varietà differenziale e un gruppo algebrico con operazioni lisce
	
	\sbs{0.45}{%
				\maps{\mu}
					{G \times G}{G}
					{(g,h)}{g h}
				}
		{0.45}{%
				\maps{i}
					{G}{G}
					{g}{g^{-1}}
				}%
} ed $ e \in G $ l'identità. L'obbiettivo è calcolare i differenziali:

\begin{gather}
	\mu_{*(e,e)} : T_{(e,e)}(G \times G) \to T_{e}(G) \\
	i_{*e} : T_{e}(G) \to T_{e}(G)
\end{gather}

Sappiamo che, prese $ N $ e $ M $ due varietà differenziali con $ p \in N $ e $ q \in M $, si ha che lo spazio tangente al prodotto delle varietà differenziali è isomorfo al prodotto degli spazi tangenti alle singole varietà\footnote{%
	Vedi Esercizio \ref{exer2-11}.%
}, i.e.

\begin{equation}
	T_{(p,q)}(N \times M) \simeq T_{p}(N) \times T_{q}(M)
\end{equation}

Per calcolare $ \mu_{*(e,e)} $ prendiamo $ X_{e},Y_{e} \in T_{e}(G) $ e usiamo la curva liscia

\begin{equation}
	\begin{cases}
		\gamma : (-\varepsilon,\varepsilon) \to G \times G \\
		\gamma(0) = (e,e) \\
		\gamma'(0) = (X_{e},Y_{e})
	\end{cases}
\end{equation}

Inoltre sappiamo che il differenziale $ \mu_{*(e,e)} $ è lineare, dunque

\begin{equation}
	\mu_{*(e,e)}(X_{e},Y_{e}) = \mu_{*(e,e)}(X_{e},0) + \mu_{*(e,e)}(0,Y_{e}) \qcomma 0 \in T_{e}(G)
\end{equation}

A questo punto, possiamo considerare i due addendi separatamente: prendiamo la curva liscia

\begin{equation}
	\begin{cases}
		c : (-\varepsilon,\varepsilon) \to G \\
		c(0) = e \\
		c'(0) = X_{e}
	\end{cases}
\end{equation}

da cui per il primo addendo

\begin{equation}
	\mu_{*(e,e)}(X_{e},0) = \mu'(c(t),e)(0)
\end{equation}

in quanto $ \gamma \doteq (c(t),e) $ è una curva per il quale valgono

\begin{gather}
	\gamma(0) = (e,e) \\
	\gamma'(0) = (c'(0),0) = (X_{e},0)
\end{gather}

Siccome  $ \mu $ è la moltiplicazione abbiamo che

\begin{equation}
	\mu_{*(e,e)}(X_{e},0) = \mu'(c(t),e)(0) %
	= (c(t) \, e)'(0) %
	= c'(0) %
	= X_{e}
\end{equation}

Il ragionamento è analogo per il secondo addendo, perciò

\begin{equation}
	\mu_{*(e,e)}(X_{e},Y_{e}) = X_{e} + Y_{e}
\end{equation}

Per quanto riguarda il differenziale dell'inversione $ i_{*e}(X_{e}) $, prendiamo ancora una curva liscia

\begin{equation}
	\begin{cases}
		c : (-\varepsilon,\varepsilon) \to G \\
		c(0) = e \\
		c'(0) = X_{e}
	\end{cases}
\end{equation}

da cui

\begin{equation}
	i_{*e}(X_{e}) = (i \circ c)'(0)
\end{equation}

Se consideriamo la moltiplicazione tra la curva e la sua inversa, otteniamo

\begin{equation}
	\mu(c(t),i(c(t))) = e
\end{equation}

Siccome valgono le relazioni

\begin{equation}
	\begin{cases}
		(G \circ F)_{*p} = G_{*F(p)} \circ F_{*p} \\
		(\id_{G})_{*e} = \id_{T_{e}(G)} \\
		c(0) = e \\
		c'(0) = X_{e}
	\end{cases}
\end{equation}

possiamo scrivere l'azione del differenziale sulla moltiplicazione considerata sopra:

\begin{align}
	\begin{split}
		\mu(c(t),i(c(t))) &= (\mu \circ (\id_{G} \times i) \circ c) (t) \\
		&\stackrel{\mathcal{F}}{\to} ( \mu \circ (\id_{G} \times i) \circ c )_{*0} \left( \eval{ \dv{t} }_{0} \right) \\
		&= ( \mu_{*(\id_{G}(c(0)), i(c(0)))} \circ (\id_{G} \times i)_{*c(0)} \circ c_{*0} ) \left( \eval{ \dv{t} }_{0} \right) \\
		&= ( \mu_{*(\id_{G}(e), i(e))} \circ (\id_{G} \times i)_{*e} ) \left( c_{*0} \left( \eval{ \dv{t} }_{0} \right) \right) \\
		&= ( \mu_{*(e,e)} \circ ((\id_{G})_{*e} \times i_{*e}) ) (c'(0)) \\
		&= ( \mu_{*(e,e)} \circ (\id_{T_{e}(G)} \times i_{*e}) ) (X_{e}) \\
		&= \mu_{*(e,e)} (\id_{T_{e}(G)} (X_{e}), i_{*e} (X_{e})) \\
		&= \mu_{*(e,e)} (X_{e}, i_{*e} (X_{e})) \\
		&= X_{e} + i_{*e} (X_{e})
	\end{split}
\end{align}

dove $ \mathcal{F} $ indica il funtore differenziale. \\
Applicando il differenziale all'elemento $ e $ otteniamo $ 0 \in G $, dunque

\begin{gather}
		\mu(c(t),i(c(t))) = e \\
		\downarrow \mathcal{F} \nonumber \\
		X_{e} + i_{*e} (X_{e}) = 0 \\
		\Downarrow \nonumber \\
		i_{*e}(X_{e}) = - X_{e}
\end{gather}

\section{Immersioni, sommersioni e sottovarietà}

\subsection{Immersioni e sommersioni}

Siano un'applicazione liscia tra varietà e il suo differenziale

\begin{gather}
	F : N \to M \\
	\downarrow \mathcal{F} \nonumber \\
	F_{*p} : T_{p}(N) \to T_{F(p)}(M)
\end{gather}

L'applicazione $ F $ è detta:

\begin{itemize}
	\item \textit{immersione} nel punto $ p \in N $ se il suo differenziale è iniettivo\footnote{%
		Equivalentemente $ \ker(F_{*p}) = \{0 \in T_{p}(N)\} $, in quanto il differenziale è lineare.%
	}
	
	\item \textit{sommersione} nel punto $ p \in N $ se il suo differenziale è suriettivo
\end{itemize}

Un'applicazione liscia è un'immersione (risp. una sommersione) se lo è in ogni punto del suo dominio.

\begin{definition}
	Per il teorema della dimensione\footnote{%
		Data un'applicazione lineare $ T : V \to W $, abbiamo che
		
		\begin{equation*}
			\dim(T(V)) + \dim(\ker(T)) = \dim(V)
		\end{equation*}
		
		dove $ \dim(T(V)) \doteq \rank(T) $ viene anche chiamato rango di $ T $.%
	}, se $ F : N \to M $ è un'immersione allora $ \dim(N) \leqslant \dim(M) $, mentre se è una sommersione allora $ \dim(N) \geqslant \dim(M) $.
\end{definition}

\begin{proof}
	Per l'immersione, l'iniettività del differenziale implica che il suo kernel ha dimensione nulla, i.e. $ \dim(\ker(F_{*p})) = 0 $, in quanto il differenziale è un'applicazione lineare. \\
	Siccome la dimensione dello spazio tangente a una varietà coincide con la dimensione della varietà, i.e.
	
	\begin{equation}
		\dim(T_{p}(N)) = \dim(\R^{n}) = \dim(N)
	\end{equation}

	e dato che, in generale, l'immagine di una funzione è contenuta nel codominio, i.e.
	
	\begin{equation}
		\Im(F) \subseteq M %
		\implies %
		\dim(\Im(F)) \leqslant \dim(M)
	\end{equation}
	
	tramite il teorema della dimensione otteniamo
	
	\begin{align}
		\begin{split}
			\dim(\Im(F_{*p})) + \cancelto{0}{ \dim(\ker(F_{*p})) } &= \dim(T_{p}(N)) \\
			\dim(\Im(F_{*p})) &= \dim(N) \\
			\dim(T_{p}(M)) &\geqslant \dim(N) \\
			\dim(M) &\geqslant \dim(N)
		\end{split}
	\end{align}

	Per la sommersione, la suriettività del differenziale implica che la sua immagine coincide con il codominio, i.e.
	
	\begin{equation}
		\Im(F_{*p}) = T_{F(p)}(M) %
		\implies %
		\dim(\Im(F_{*p})) = \dim(T_{F(p)}(M))
	\end{equation}

	Siccome, in generale, la dimensione del kernel di una funzione suriettiva non è nulla, i.e. $ \dim(\ker(F_{*p})) \geqslant 0 $, tramite il teorema della dimensione otteniamo
	
	\begin{align}
		\begin{split}
			\dim(\Im(F_{*p})) + \dim(\ker(F_{*p})) &= \dim(T_{p}(N)) \\
			\dim(T_{F(p)}(M)) + \dim(\ker(F_{*p})) &= \dim(N) \\
			\dim(T_{F(p)}(M)) &\leqslant \dim(N) \\
			\dim(M) &\leqslant \dim(N)
		\end{split}
	\end{align}
\end{proof}

\begin{remark}
	Se $ F : N \to M $ è un'immersione e una sommersione in $ p $, i.e. $ \dim(N) = \dim(M) $, allora, per il teorema della funzione inversa\footnote{%
		Vedi Teorema \ref{thm:ift}.%
	}, avremo che il differenziale è invertibile, dunque $ F $ è un diffeomorfismo locale intorno a $ p $.
\end{remark}

\begin{remark}
	In generale, un'immersione non è detto sia iniettiva e una sommersione non è detto sia suriettiva (lo sono necessariamente rispettivamente solo i loro differenziali).
\end{remark}

\begin{remark}
	La composizione e il prodotto cartesiano di immersioni sono immersioni\footnote{%
		Vedi Esercizio \ref{exer2-21}.%
	}.
\end{remark}

\subsubsection{\textit{Esempi}}

\paragraph{1) Immersione canonica}

Sia l'applicazione lineare (inclusione)

\map{i}
	{\R^{n}}{\R^{m}}
	{(r^{1},\dots,r^{n})}{(r^{1},\dots,r^{n},0,\dots,0)}

con $ n \leqslant m $ e con $ m-n $ zeri aggiunti nei punti dell'immagine. \\
Essendo lineare, il differenziale coincide con l'applicazione stessa, i.e. $ i_{*p} = i $, la quale è iniettiva, dunque $ i $ è un'immersione. \\
Questa applicazione definisce la forma canonica delle immersioni, cioè ogni immersione è localmente analoga all'inclusione sopra considerata\footnote{%
	Vedi Teorema \ref{thm:loc-imm}.%
}.

\paragraph{2) Sommersione canonica}

Sia l'applicazione lineare (proiezione)

\map{\pi}
	{\R^{n}}{\R^{m}}
	{(r^{1},\dots,r^{n})}{(r^{1},\dots,r^{m})}

con $ n \geqslant m $. \\
Essendo lineare, il differenziale coincide con l'applicazione stessa, i.e. $ \pi_{*p} = \pi $, la quale è suriettiva, dunque $ \pi $ è una sommersione. \\
Questa applicazione definisce la forma canonica delle sommersioni, cioè ogni sommersione è localmente analoga alla proiezione sopra considerata\footnote{%
	Vedi Teorema \ref{thm:loc-sub}.%
}.

\paragraph{3)}

Sia l'inclusione

\map{i}
	{U}{N}
	{p}{p}

dove $ U \subset N $ aperto. Questa applicazione non è suriettiva se $ U \subsetneq N $ (è incluso ma non coincidente) ma il suo differenziale lo è, in quanto $ i_{*p} : T_{p}(U) = T_{p}(N) \to T_{p}(N) $ corrisponde all'identità (è anche iniettivo). \\
Dunque l'inclusione considerata non è suriettiva ma è un'immersione e una sommersione.

\subsubsection{Rango di un'applicazione liscia}

Siano un'applicazione liscia $ F : N \to M $, un punto $ p \in N $ e due carte

\begin{equation}
	\begin{cases}
		(U,\varphi) = (U; x^{1},\dots,x^{n}) \in N, & \varphi(p) = 0 \\
		(V,\psi) = (V; y^{1},\dots,y^{m}) \in M, & \psi(F(p)) = 0 \\
		F(U) \subseteq V
	\end{cases}
\end{equation}

Definiamo il \textit{rango} di $ F $ nel punto $ p $ come

\begin{equation}
	\rank(F(p)) \doteq \rank\left( \left[ \pdv{F^{i}}{x^{j}} \, (p) \right] \right) %
	\equiv \rank(F_{*p})
\end{equation}

dove $ F^{i} = y^{i} \circ F = r^{i} \circ \psi \circ F $; la seconda equazione è data dal fatto che la matrice di $ F_{*p} $ rispetto alle basi

\begin{gather}
	\mathcal{B}_{T_{p}(N)} = \left\{ \eval{ \pdv{x^{1}} }_{p}, \dots, \eval{ \pdv{x^{n}} }_{p} \right\} \\
	\mathcal{B}_{T_{F(p)}(M)} = \left\{ \eval{ \pdv{y^{1}} }_{F(p)}, \dots, \eval{ \pdv{y^{m}} }_{F(p)} \right\}
\end{gather}

è proprio lo jacobiano di $ F $ nel punto $ p $ associato alle carte definite sopra, i.e.

\begin{equation}
	J(F)(p) = \left[ \pdv{F^{i}}{x^{j}} \, (p) \right] \in M_{m,n}(\R)
\end{equation}

Il rango di un'applicazione liscia definito in questo modo è ben definito in quanto lo è quello di un'applicazione lineare: il rango della matrice associata non dipende dalle basi scelte per calcolare la matrice stessa perché due matrici ottenute a partire da due basi diverse differiscono tra loro per la moltiplicazione di una matrice invertibile.

\begin{remark}
	Se $ F $ è un'immersione in $ p $, allora $ \rank(F(p)) = n \leqslant m $ mentre se è una sommersione $ \rank(F(p)) = m \leqslant n $, in entrambi casi il rango è massimo.
\end{remark}

\subsection{Punti e valori critici e regolari}

Siano un'applicazione liscia tra varietà e il suo differenziale

\begin{gather}
	F : N \to M \\
	\downarrow \mathcal{F} \nonumber \\
	F_{*p} : T_{p}(N) \to T_{F(p)}(M)
\end{gather}

Un punto $ p \in N $ è detto:

\begin{itemize}
	\item \textit{punto critico per} $ F $ se il differenziale in quel punto non è suriettivo
	
	\item \textit{punto regolare per} $ F $ se il differenziale in quel punto è suriettivo ($ F $ è una sommersione in $ p $)
\end{itemize}

I punti critici e regolari appartengono al dominio della funzione. Come notazione

\begin{gather}
	\PC_{F} = \{ p \in N \mid p \text{ punto critico per } F \} \subset N \\
	\PR_{F} = \{ p \in N \mid p \text{ punto regolare per } F \} \subset N
\end{gather}

dove valgono le relazioni

\begin{equation}
	\begin{cases}
		N = \PC_{F} \cup \PR_{F} \\
		\PC_{F} \cap \PR_{F} = \emptyset
	\end{cases}
\end{equation}

\begin{remark}
	L'insieme dei punti regolari di $ F $ è un aperto del dominio di $ F $.\footnote{%
		Vedi Esercizio \ref{exer2-17}.%
	}
\end{remark}

Consideriamo l'insieme dei punti di $ N $ che hanno come immagine un punto $ q \in M $ attraverso $ F $, i.e.

\begin{equation}
	F^{-1}(q) = \{ p \in N \mid F(p) = q \}
\end{equation}

Il punto $ q \in M $ è detto:

\begin{itemize}
	\item \textit{valore regolare per} $ F $ se ogni punto $ p \in F^{-1}(q) $ è un punto regolare
	
	\item \textit{valore critico per} $ F $ se non è un valore regolare o, equivalentemente, se esiste un punto $ p \in F^{-1}(q) $ tale che $ p $ sia un punto critico
\end{itemize}

I valori critici e regolari appartengono al codominio della funzione. Come notazione

\begin{gather}
	\VR_{F} = \{ q \in M \mid q \text{ valore regolare per } F \} \subset M \\
	\VC_{F} = \{ q \in M \mid q \text{ valore critico per } F \} \subset M
\end{gather}

Possiamo scrivere le definizioni usando la seguente notazione

\begin{gather}
	q \in \VR_{F} \iff p \in \PR_{F}, \, \forall p \in F^{-1}(q) \\
	q \in \VC_{F} \iff \E p \in F^{-1}(q) \mid p \in \PC_{F}
\end{gather}

Analogamente ai punti regolari e critici, valgono le seguenti relazioni

\begin{equation}
	\begin{cases}
		M = \VC_{F} \cup \VR_{F} \\
		\VC_{F} \cap \VR_{F} = \emptyset
	\end{cases}
\end{equation}

\begin{remark}
	Se $ q \in M $ ma $ q \notin F(N) $ allora $ q $ è un valore regolare, i.e.
	
	\begin{equation}
		M \setminus F(N) \subset \VR_{F}
	\end{equation}
\end{remark}

\begin{remark}
	Vale la relazione
	
	\begin{equation}
		F(\PC_{F}) = \VC_{F}
	\end{equation}

	ma in generale
	
	\begin{equation}
		\begin{cases}
			F(\PR_{F}) \not\subset \VR_{F} \\
			F(\PR_{F}) \not\supset \VR_{F}
		\end{cases}
	\end{equation}

	La prima relazione è relativa al fatto che possono esserci dei punti regolari che, tramite $ F $, vengono mappati in valori critici; la seconda viene dai punti del codominio che non sono immagine del dominio tramite $ F $, i.e. per qualsiasi $ q \in M \setminus F(N) $ abbiamo che $ q \in \mathcal{VR}_{F} $ per definizione ma $ q \notin F(\mathcal{PR}_{F}) $ in quanto $ q \notin F(N) $.
\end{remark}

\begin{remark}
	L'insieme dei valori regolari di $ F $ è un aperto del codominio di $ F $.\footnote{%
		Vedi Esercizio \ref{exer2-18}.%
	}
\end{remark}

\subsubsection{Codominio $ \R $}

Siano l'applicazione liscia $ f : N \to \R $ e un punto $ p \in N $, se $ p \in \PC_{f} $ allora il differenziale $ f_{*p} : T_{p}(N) \to T_{f(p)}(\R) $ non è suriettivo. In generale, a seconda della dimensione dell'immagine dell'applicazione, si presentano due casi:

\begin{gather}
	\dim(\Im(f_{*p})) \leqslant \dim(T_{f(p)}(\R)) = \dim(\R) = 1 \\
	\Downarrow \nonumber \\
	\begin{cases}
		\dim(\Im(f_{*p})) = 1 \\
		\dim(\ker(f_{*p})) = 0
	\end{cases} %
	\lor \quad
	\begin{cases}
		\dim(\Im(f_{*p})) = 0 \\
		\dim(\ker(f_{*p})) = 1
	\end{cases}
\end{gather}

Il primo caso non può verificarsi in quanto imporrebbe la suriettività del differenziale, contraddicendo l'ipotesi, dunque abbiamo che $ f_{*p} = 0 $ o, equivalentemente, che l'immagine di $ f_{*p} $ sia l'insieme nullo, i.e.

\begin{equation}
	\Im(f_{*p}) = f_{*p}(T_{p}(N)) = \{0 \in T_{f(p)}(\R)\}
\end{equation}

perciò

\begin{equation}
	f_{*p}(X_{p}) = 0 \in T_{f(p)}(\R) \qcomma \forall X_{p} \in T_{p}(N)
\end{equation}

Sappiamo che

\begin{equation}
	f_{*p}(X_{p}) = (X_{p} f) \eval{ \dv{r} }_{f(p)}
\end{equation}

dove $ X_{p} f \in \R $ e $ \ev{ \eval{ \dv*{r} }_{f(p)} } = T_{f(p)}(\R) $. Sia dunque la carta $ (U,\varphi) \in N $ con $ \varphi = (x^{1},\dots,x^{n}) $ intorno a $ p $, allora

\begin{equation}
	X_{p} = \sum_{i=1}^{n} a^{i} \eval{ \pdv{x^{i}} }_{p}
\end{equation}

perciò

\begin{equation}
	X_{p} f = \sum_{i=1}^{n} a^{i} \pdv{f}{x^{i}} \, (p)
\end{equation}

Affinché $ f_{*p}(X_{p}) = 0 $, è necessario che $ X_{p} f = 0 $, il che implica

\begin{equation}
	\pdv{f}{x^{i}} \, (p) = 0 \qcomma \forall i=1,\dots,n
\end{equation}

in quanto abbiamo considerato un vettore $ X_{p} \in T_{p}(N) $ arbitrario. \\
A questo punto possiamo scrivere la condizione per cui un punto $ p \in N $ sia un punto critico per un'applicazione $ f : N \to \R $

\begin{equation}
	p \in \PC_{f} %
	\iff %
	\dpdv{f}{x^{i}} \, (p) = 0 \qcomma %
	\begin{cases}
		\forall i=1,\dots,n \\
		\forall (U,\varphi) = (U; x^{1},\dots,x^{n}) \in N \, \wedge \, p \in (U,\varphi)
	\end{cases}
\end{equation}

In particolare, se $ N = (a,b) \subset \R $, la condizione per cui un punto $ t_{0} \in (a,b) $ sia un punto critico diventa

\begin{equation}
	t_{0} \in \PC_{f} \iff \dot{f}(t_{0}) = 0
\end{equation}

\subsubsection{\textit{Esempio}}

\sbs{0.4}{%
			Sia la funzione
			
			\map{f}
				{\R}{\R}
				{x}{x^{4} - 2 x^{2}}
			}
	{0.6}{%
			\img{1}{img28}
			}

Abbiamo che

\begin{equation}
	\begin{cases}
		\PC_{f} = \{ x \in \R \mid \dot{f}(x) = 4 x^{3} - 4 x = 0 \} = \{ -1, 0, 1 \} \\
		\PR_{f} = \R \setminus \PC_{f} = \R \setminus \{ -1, 0, 1 \} \\
		\VC_{f} = f(\PC_{f}) = \{ -1, 0 \} \\
		\VR_{f} = \R \setminus \VC_{f} = \R \setminus \{ -1, 0 \}
	\end{cases}
\end{equation}

Sappiamo inoltre che $ f(\PR_{f}) \not\subset \VR_{f} $ in quanto, ad esempio, $ \sqrt{2} \in \PR_{f} $ ma $ f(\sqrt{2}) = 0 \notin \VR_{f} $.

\subsubsection{Massimi e minimi relativi}

In analisi, per una funzione $ f : \R \to \R $ il punto $ x_{0} \in \R $ è un massimo (risp. minimo) locale per $ f $ se esiste un intorno $ I $ contenente $ x_{0} $ tale che $ f(x) \leqslant f(x_{0}) $ (risp. $ f(x) \geqslant f(x_{0}) $) per qualsiasi $ x \in I $. Se $ x_{0} $ è un punto di massimo o minimo relativo allora la derivata della funzione nel punto è nulla\footnote{%
	La derivata della funzione $ f $ nel punto $ x_{0} $ è definita come
	
	\begin{equation*}
		\dot{f}(x_{0}) = \lim_{x \to x_{0}} \dfrac{f(x) - f(x_{0})}{x - x_{0}}
	\end{equation*}

	Per un massimo locale vale $ f(x) \leqslant f(x_{0}) $ per qualsiasi $ x \in I $ dunque $ f(x) - f(x_{0}) \leqslant 0 $; considerando i limiti da destra e da sinistra e sfruttando la permanenza del segno
	
	\begin{align*}
		\begin{split}
			\dot{f}(x_{0})^{+} = \lim_{x \to x_{0}^{+}} \dfrac{f(x) - f(x_{0})}{x - x_{0}} \leqslant 0 \\
			\dot{f}(x_{0})^{-} = \lim_{x \to x_{0}^{-}} \dfrac{f(x) - f(x_{0})}{x - x_{0}} \geqslant 0
		\end{split}
	\end{align*}

	perciò $ \dot{f}(x_{0}) = 0 $.%
}, i.e. $ \dot{f}(x_{0}) = 0 $, dunque $ x_{0} \in \PC_{f} $. \\
Siano una varietà differenziabile $ N $ e una funzione $ f \in C^{\infty}(N) $, un punto $ p \in N $ è un \textit{massimo locale} (risp. \textit{minimo locale}) per $ f $ se esiste un intorno $ U \subset N $ di $ p $ tale che $ f(q) \leqslant f(p) $ (risp. $ f(q) \geqslant f(p) $ ) per qualsiasi $ q \in U $.

\begin{definition}
	Se $ p $ è un punto di massimo locale (risp. minimo locale) per $ f : N \to \R $ allora $ p \in \PC_{f} $.
\end{definition}

\begin{proof}
	Ricordiamo che
	
	\begin{equation}
		p \in \PC_{f} %
		\iff %
		\begin{cases}
			\dpdv{f}{x^{i}} \, (p) = 0 \\ \\
			\forall i=1,\dots,n
		\end{cases} %
		\iff %
		f_{*p}(X_{p}) = 0, \, \forall X_{p} \in T_{p}(N)
	\end{equation}

	con
	
	\begin{equation}
		f_{*p}(X_{p}) = (X_{p} f) \eval{ \dv{r} }_{f(p)}
	\end{equation}
	
	La derivata direzionale si può calcolare usando le curve su varietà, dunque presa la curva liscia
				
	\begin{equation}
		\begin{cases}
			c : (-\varepsilon,\varepsilon) \to N \\
			c(0) = p \\
			c'(0) = X_{p}
		\end{cases}
	\end{equation}

	abbiamo che
	
	\begin{equation}
		X_{p} f = \dot{(f \circ c)}(0)
	\end{equation}
	
	quindi
	
	\begin{equation}
		f_{*p}(X_{p}) = \dot{(f \circ c)}(0) \eval{ \dv{r} }_{f(p)}
	\end{equation}
	
	Sappiamo che, preso $ U $ intorno di $ p $, per un massimo locale in $ p $ abbiamo che $ f(q) \leqslant f(p) $ per qualsiasi $ q \in U $, quindi prendiamo $ \varepsilon $ in modo tale che $ c((-\varepsilon,\varepsilon)) \subset U $. Se $ p $ è un massimo locale per $ f $ allora 0 è un massimo locale per $ f \circ c $, i.e.
	
	\begin{equation}
		(f \circ c)(t) \leqslant (f \circ c)(0) = f(p) \qcomma \forall t \in (-\varepsilon,\varepsilon)
	\end{equation}

	Dall'analisi, affinché $ p $ sia un massimo, è necessario che $ \dot{(f \circ c)}(0) = 0 $, il che implica che $ f_{*p}(X_{p}) = 0 $ e perciò $ p \in \PC_{f} $. \\	
	La dimostrazione per il minimo locale è analoga.
\end{proof}

\subsubsection{Matrice hessiana}

Siano una varietà compatta $ N $ con $ \dim(N) = 2 $ e una funzione $ f : N \to \R $. Un punto $ p \in N $  è detto punto critico \textit{non degenere} per $ f $ se $ p \in \PC_{f} $ e se, presa una carta $ (U\,\varphi) \in N $ con $ \varphi = (x^{1},x^{2}) $, il determinante della \textit{matrice hessiana} associata a $ f $ in $ p $ è diverso da zero, i.e.

\begin{equation}
	\det(\mathcal{H}(f)(p)) \doteq \det( \bmqty{ \dpdv[2]{f}{(x^{1})} & \dpdv[2]{f}{x^{1}}{x^{2}} \\ \\ \dpdv[2]{f}{x^{2}}{x^{1}} & \dpdv[2]{f}{(x^{2})} } ) %
	\neq 0
\end{equation}

La maggior parte delle funzioni possiedono punti critici non degeneri. \\
Analogamente all'analisi, si possono classificare i punti critici non degeneri in base alla loro matrice hessiana:

\begin{itemize}
	\item $ \mathcal{H}(f)(p) > 0 \text{ definita positiva\footnotemark} \implies p \text{ massimo locale} $
	
	\item $ \mathcal{H}(f)(p) < 0 \text{ definita negativa} \implies p \text{ minimo locale} $
	
	\item $ \mathcal{H}(f)(p) \gtreqqless 0 \text{ semidefinita} \implies p \text{ punto di sella} $
\end{itemize}
\footnotetext{%
	Siano una matrice $ A \in M_{n,n}(\R) $ e un vettore riga $ v \in M_{1,n}(\R) $, la matrice $ A $ è \textit{definita positiva} se il prodotto $ v A v^{T} \in \R $ è positivo per qualunque vettore $ v $ diverso dal vettore nullo i.e.
	
	\begin{equation*}
		v A v^{T} > 0 \qcomma \forall v \in M_{1,n}(\R) \, \wedge \, v \neq 0
	\end{equation*}

	Analogamente, per una matrice \textit{definita negativa} deve verificarsi $ v A v^{T} < 0 $, mentre per una \textit{semidefinita} (positiva o negativa) abbiamo che $ v A v^{T} \geqslant 0 $ oppure $ v A v^{T} \leqslant 0 $. Infine, la matrice si dirà \textit{indefinita} se esistono due vettori non nulli $ v_{1} $ e $ v_{2} $ tali che
	
	\begin{equation*}
		v_{1} A v_{1}^{T} > 0 \, \wedge \, v_{2} A v_{2}^{T} < 0
	\end{equation*}

	Per determinare la definitezza di una matrice si possono usare, ad esempio, il criterio di Sylvester o la regola di Cartesio.%
}

Ad esempio, il paraboloide ellittico $ z = - x^{2} - y^{2} $ ha come massimo locale il punto $ (0,0) $, per la funzione

\map{f}
	{\R^{2}}{\R}
	{(x,y)}{- x^{2} - y^{2}}

dove

\begin{equation}
	\mathcal{H}(f)(p) = \bmqty{ -2 & 0 \\ 0 & -2 } < 0
\end{equation}

Il paraboloide ellittico $ z = x^{2} + y^{2} $ (riflessione tramite il piano $ xy $ del paraboloide precedente) ha $ (0,0) $ come minimo locale mentre il paraboloide iperbolico $ z = x^{2} - y^{2} $ ha $ (0,0) $ come punto di sella.

\subsubsection{Teoria di Morse}

La teoria di Morse lega lo studio dei punti critici di una varietà alla sua topologia.

\begin{theorem}[Morse per superfici]
	Siano $ N $ una varietà compatta e connessa con $ \dim(N)=2 $ ed $ f $ una funzione liscia che ha tutti i punti critici non degeneri. Siano
	
	\begin{itemize}
		\item $ M = \# \text{ punti di massimo per } f $
		
		\item $ m = \# \text{ punti di minimo per } f $
		
		\item $ s = \# \text{ punti di sella per } f $
	\end{itemize}

	i quali sono finiti perché la varietà è compatta. \\
	Vale la relazione
	
	\begin{equation}
		M - s + m = \chi(N)
	\end{equation}

	dove $ \chi(N) $ è la \textit{caratteristica di Eulero} della superficie $ N $. La caratteristica di Eulero è data da
	
	\begin{equation}
		\chi(N) = F - L + V
	\end{equation}

	dove $ F $ sono le facce, $ L $ sono i lati e $ V $ sono i vertici che compaiono nella suddivisione di una varietà compatta in triangoli (curvilinei).
\end{theorem}

\sbs{0.6}{%
			Al fine di generalizzare questo teorema per $ n $ dimensioni, consideriamo il toro $ \T^{2} $ e la funzione "altezza" $ f : \T^{2} \to \R $ la quale ha come immagine del toro l'altezza dei punti dello stesso. La caratteristica di Eulero del toro calcolata mediante la funzione altezza corrisponde dunque a
			
			\begin{equation}
				\chi(\T^{2}) = M - s + m = 1 - 2 + 1 = 0
			\end{equation}
			}
	{0.4}{%
			\img{0.7}{img30}
			}

Se consideriamo la sfera $ \S^{2} $ tramite la funzione altezza, otteniamo invece

\begin{equation}
	\chi(\S^{2}) = 1 - 0 + 1 = 2
\end{equation}

\sbs{0.4}{%
			In generale, per una varietà compatta di dimensione due, la quale è diffeomorfa alla superficie con $ g $ buchi $ \Sigma_{g} $, otteniamo che la sua caratteristica di Eulero corrisponde a
			
			\begin{equation}
				\chi(\Sigma_{g}) = 1 - 2 g + 1 = 2 (1 - g)
			\end{equation}
			}
	{0.6}{%
			\img{0.9}{img31}
			}

\subsection{Sottovarietà}

Siano $ N $ una varietà differenziabile con $ \dim(N) = n $ e $ S \subset N $ sottospazio topologico di $ N $ (con la topologia indotta da $ N $), allora $ S $ è una \textit{sottovarietà di} $ N $ di dimensione $ \dim(S) = k \leqslant n $ se per ogni punto $ p \in S $ esiste una carta $ (U,\varphi) \in N $ con $ \varphi = (x^{1},\dots,x^{n}) $ intorno a $ p $ tale che l'aperto $ U \cap S $ sia costituito dai punti di $ U $ che abbiano le ultime $ n-k $ funzioni coordinate nulle\footnote{%
	Due carte che hanno le stesse coordinate ma invertite sono identiche, i.e. $ (U;x^{1},x^{2}) \equiv (U;x^{2},x^{1}) $.%
}, i.e.

\begin{gather}
	S \text{ sottovarietà} \nonumber \\
	\Downarrow \\
	\forall p \in S, \, \E (U,\varphi) \in N \mid (U,\varphi) \ni p \wedge U \cap S = \{ q \in U \mid x^{k+1}(q) = \dots = x^{n}(q) = 0 \} \nonumber
\end{gather}

Ad esempio, se consideriamo $ N = \R^{2} $ con carta $ (\R^{2};x,y) $ e $ S = \R \times \{0\} $ allora

\begin{equation}
	\R^{2} \cap S = S = \{ (x,y) \in \R^{2} \mid y=0 \}
\end{equation}

mentre per $ N = \R^{3} $ con carta $ (\R^{3};x,y,z) $ e $ S = \R \times \{(0,0)\} $ allora

\begin{equation}
	\R^{3} \cap S = S = \{ (x,y,z) \in \R^{3} \mid y = 0 \wedge z = 0 \}
\end{equation}

\begin{remark}\label{rem:subvar-open}
	Sia $ S \subset N $ un aperto di una varietà $ N $ con $ \dim(N)=n $, allora $ S $ è una sottovarietà di $ N $ con $ \dim(S)=n $. \\
	Questo è vero in quanto, presa una carta $ (U,\varphi) \in N $, la condizione
	
	\begin{equation}
		U \cap S = \{ q \in U \mid x^{k+1}(q) = \dots = x^{n}(q) = 0 \}
	\end{equation}

	si verifica per $ k=n $, i.e. zero coordinate si annullano in $ U \cap S $. \\
	Ad esempio, non potrebbe essere che solo una coordinata si annulla
	
	\begin{equation}
		U \cap S = \{ q \in U \mid x^{i}(q) = 0 \}
	\end{equation}

	in quanto $ U \simeq \R^{n} $ tramite il diffeomorfismo $ \varphi $, $ U \cap S \simeq \R^{n-1} $ e $ U \cap S \simeq U $ ma i diffeomorfismi conservano la dimensione.
\end{remark}

Siano $ N $ una varietà differenziabile con $ \dim(N)=n $, $ S \subset N $ una sottovarietà con $ \dim(S)=k $ e $ p \in S $ un punto della sottovarietà, allora una carta $ (U,\varphi) \in N $ con $ \varphi = (x^{1},\dots,x^{n}) $ tale che $ (U,\varphi) \ni p $ è chiamata \textit{carta adattata di} $ N $ \textit{intorno a} $ p $ \textit{relativamente a} $ S $ se vale la condizione per la sottovarietà, i.e.

\begin{equation}
	U \cap S = \{ q \in U \mid x^{k+1}(q) = \dots = x^{n}(q) = 0 \}
\end{equation}

\subsubsection{\textit{Esempi}}

\paragraph{1)}

Consideriamo $ N = \R^{2} $ con carta $ U = (-1,1) \times (-1,1) $ (quadrato di lato 2) e $ S = (-1,1) \times \{0\} $. La carta $ U $ è adattata relativamente a $ S $ in quanto

\begin{equation}
	U \cap S = \{ q \in U \mid y(q) = 0 \} = S
\end{equation}

\paragraph{2)}

Consideriamo $ N = \R^{2} $ con carta $ V = (0,2) \times (-1,1) $ (quadrato di lato 2 decentrato verso destra) e $ S = (-1,1) \times \{0\} $. La carta $ V $ non è adattata relativamente a $ S $ in quanto

\begin{gather}
	V \cap S = (0,1) \times \{0\}\nonumber \\
	\neq \\
	\{ q \in V \mid y(q) = 0 \} = (0,2) \times \{0\}\nonumber
\end{gather}

\paragraph{3)}

Consideriamo $ N = \R^{n} $ con carta $ (U,\varphi) = (\R^{n}; r^{1},\dots,r^{n}) $ e $ S = \R^{k} \times \{0\} \subset N $. La carta $ (U,\varphi) $ è una carta adattata intorno ad ogni punto $ p \in S $ relativamente a $ S $ in quanto

\begin{equation}
	U \cap S = S = \{ q \in \R^{n} \mid r^{k+1}(q) = \dots = r^{n}(q) = 0 \}
\end{equation}

\subsubsection{Sottovarietà come varietà differenziabile}

Possiamo definire l'applicazione

\map{\varphi_{S}}
	{U \cap S}{\R^{k}}
	{q}{(x^{1}(q),\dots,x^{k}(q))}

dove $ (U,\varphi) $ è una carta adattata di $ N $ intorno a $ p $ relativamente a $ S $, come restrizione a $ S $ dell'applicazione

\map{\eval{ \varphi }_{U \cap S}}
	{U \cap S}{\R^{n}}
	{q}{(x^{1}(q),\dots,x^{k}(q),0,\dots,0)}

Siccome $ \varphi $ è un omeomorfismo, allora lo è anche $ \varphi_{S} $ in quanto restrizione di un omeomorfismo. \\
Deduciamo dunque che $ S $ è una varietà topologica con $ \dim(S) = k $:

\begin{itemize}
	\item $ S $ è $ N_{2}+T_{2} $ in quanto $ S \subset N $ è un sottospazio topologico
	
	\item per un qualsiasi $ p \in S $ si può definire una carta $ (U \cap S,\varphi_{S}) $ intorno a $ p $ dal quale si ottiene un atlante topologico per $ S $
\end{itemize}

\begin{definition}
	Sia $ S $ una sottovarietà di una varietà differenziabile $ N $ rispettivamente con $ \dim(S) = k $ e $ \dim(N) = n $, allora $ S $ è una varietà differenziabile con $ \dim(S) = k $.
\end{definition}

Chiameremo la struttura differenziabile costruita nella proposizione \textit{struttura differenziale indotta da} $ N $.

\begin{proof}
	Per dimostrare che $ S $ sia una varietà differenziabile, dobbiamo far vedere che per qualunque punto $ p \in S $ e qualunque coppia di carte $ (U \cap S,\varphi_{S}) $ e $ (V \cap S,\psi_{S}) $ indotte da carte adattate $ (U,\varphi) $ e $ (V,\psi) $ si abbia che $ (U \cap S,\varphi_{S}) $ sia $ C^{\infty} $-compatibile con $ (V \cap S,\psi_{S}) $. \\
	Essendo $ (U,\varphi) $ e $ (V,\psi) $ carte di $ N $, queste sono $ C^{\infty} $-compatibili, i.e. sono lisci i seguenti cambi di carta
	
	\begin{gather}
		\psi \circ \varphi^{-1} : \varphi(U \cap V) \to \psi(U \cap V) \\
		\varphi \circ \psi^{-1} : \psi(U \cap V) \to \varphi(U \cap V)
	\end{gather}

	Scrivendo queste applicazioni su coordinate euclidee
	
	\begin{align}
		\begin{split}
			(\psi \circ \varphi^{-1}) \left( r^{1}(\varphi(q)),\dots,r^{n}(\varphi(q)) \right) &= \left( r^{1}(\psi(q)),\dots,r^{n}(\psi(q)) \right) \\
			(\psi \circ \varphi^{-1}) \left( x^{1}(q),\dots,x^{n}(q) \right) &= \left( y^{1}(q),\dots,y^{n}(q) \right)
		\end{split}
	\end{align}

	\begin{align}
		\begin{split}
			(\varphi \circ \psi^{-1}) \left( r^{1}(\psi(q)),\dots,r^{n}(\psi(q)) \right) &= \left( r^{1}(\varphi(q)),\dots,r^{n}(\varphi(q)) \right) \\
			(\varphi \circ \psi^{-1}) \left( y^{1}(q),\dots,y^{n}(q) \right) &= \left( x^{1}(q),\dots,x^{n}(q) \right)
		\end{split}
	\end{align}

	otteniamo che le funzioni $ x^{i} $ e $ y^{i} $ per qualsiasi $ i=1,\dots,n $ sono lisce rispetto a $ (y^{1},\dots,y^{n}) $ e $ (x^{1},\dots,x^{n}) $ rispettivamente\footnote{%
		Le funzioni $ x^{i} $ e $ y^{i} $ dipendono da $ (y^{1},\dots,y^{n}) $ e $ (x^{1},\dots,x^{n}) $ rispettivamente tramite i cambi di carta ($ f(x) = y $ dunque $ y $ dipende da $ x $ tramite $ f $).%
	}, in quanto immagini dei cambi di carte, i quali sono lisci per definizione di varietà differenziabile. \\
	Siano le carte adattate intorno a $ p $
	
	\begin{gather}
		\begin{cases}
			(U,\varphi) = (U; x^{1},\dots,x^{n}) \in N \\
			U \cap S = \{ q \in U \mid x^{k+1}(q) = \dots = x^{n}(q) = 0 \}
		\end{cases} \\
		\begin{cases}
			(V,\psi) = (V; y^{1},\dots,y^{n}) \in N \\
			V \cap S = \{ q \in V \mid y^{k+1}(q) = \dots = y^{n}(q) = 0 \}
		\end{cases}
	\end{gather}
	
	Prendiamo ora le applicazioni
	
	\map{\varphi_{S}}
		{U \cap S}{\R^{k}}
		{q}{(x^{1}(q),\dots,x^{k}(q))}

	\map{\psi_{S}}
		{V \cap S}{\R^{k}}
		{q}{(y^{1}(q),\dots,y^{k}(q))}

	i quali cambi di carte
	
	\map{\psi_{S} \circ \varphi_{S}^{-1}}
		{\varphi_{S}(U \cap V \cap S)}{\psi_{S}(U \cap V \cap S)}
		{\left( r^{1}(\varphi_{S}(q)),\dots,r^{k}(\varphi_{S}(q)) \right)}{%
			\left( r^{1}(\psi_{S}(q)),\dots,r^{k}(\psi_{S}(q)) \right) \\
			(x^{1}(q),\dots,x^{k}(q)) &\mapsto (y^{1}(q),\dots,y^{k}(q))}
	
	\map{\varphi_{S} \circ \psi_{S}^{-1}}
		{\psi_{S}(U \cap V \cap S)}{\varphi_{S}(U \cap V \cap S)}
		{\left( r^{1}(\psi_{S}(q)),\dots,r^{k}(\psi_{S}(q)) \right)}{%
			\left( r^{1}(\varphi_{S}(q)),\dots,r^{k}(\varphi_{S}(q)) \right) \\
			(y^{1}(q),\dots,y^{k}(q)) &\mapsto (x^{1}(q),\dots,x^{k}(q))}

	sono ancora lisci in quanto le applicazioni $ x^{i} $ e $ y^{i} $, essendo lisce per $ (y^{1},\dots,y^{n}) $ e $ (x^{1},\dots,x^{n}) $ rispettivamente, saranno ancora lisce per la restrizione a $ k < n $ coordinate. \\
	Dimostrando che i cambi di carta sono lisci, otteniamo che $ S $ è una varietà differenziabile.
\end{proof}

\begin{definition}\label{prop:subman-incl-immersion}
	Sia $ N $ una varietà differenziabile con $ \dim(N) = n $, $ S \subset N $ è una sottovarietà di $ N $ con $ \dim(S) = k < n $ se e solo se l'inclusione naturale
	
	\map{i}
		{S}{N}
		{q}{q}

	è un'immersione. \\
	La sottovarietà $ S $ è dotata della struttura differenziabile di dimensione $ k $ descritta dalla proposizione precedente.
\end{definition}

\begin{proof}
	Innanzitutto, dobbiamo mostrare che $ i $ sia liscia: per fare ciò, scegliamo due carte arbitrarie di dominio e codominio rispettivamente e dimostriamo che la composizione delle applicazioni di queste carte con l'inclusione sia ancora un'applicazione liscia. \\
	Siano un punto $ p \in S $, $ (U,\varphi) \in N $ con $ \varphi = (x^{1},\dots,x^{n}) $ una carta adattata di $ N $ intorno a $ p $ relativamente a $ S $ e la corrispondente applicazione su $ S $
	
	\map{\varphi_{S}}
		{U \cap S}{\R^{k}}
		{q}{(x^{1}(q),\dots,x^{k}(q))}
	
	Sappiamo che
	
	\begin{equation}
		U \cap S = \{ q \in U \mid x^{k+1}(q) = \dots = x^{n}(q) = 0 \}
	\end{equation}

	La composizione
	
	\map{\varphi \circ i \circ \varphi_{S}^{-1}}
		{\varphi_{S}(U \cap S)}{\R^{n}}
		{\varphi_{S}(q) = (\, r^{1}(\varphi(q)),\dots,r^{k}(\varphi(q)) \,)}{(\, r^{1}(\varphi(q)),\dots,r^{k}(\varphi(q)),0,\dots,0 \,)}
	
	aggiunge $ n-k $ zeri all'immagine, dunque
	
	\begin{equation}
		(\varphi \circ i \circ \varphi_{S}^{-1}) (r^{1},\dots,r^{k}) = (r^{1},\dots,r^{k},0,\dots,0)
	\end{equation}

	è un'applicazione liscia. \\	
	Essendo $ p \in S $ e le carte scelte arbitrariamente, la dimostrazione vale per ogni punto in $ S $ e quindi $ i $ è un'applicazione liscia. \\
	Deduciamo anche che $ \varphi \circ i \circ \varphi_{S}^{-1} $ è un'immersione in quanto immersione canonica localmente (e quindi globalmente). \\
	Il differenziale della composizione, tramite la regola della catena, si può scrivere come
	
	\begin{equation}
		(\varphi \circ i \circ \varphi_{S}^{-1})_{*\varphi_{S}(p)} = \varphi_{*p} \circ i_{*p} \circ (\varphi_{S}^{-1})_{*\varphi_{S}(p)}
	\end{equation}
	
	A questo punto, siccome $ (\varphi \circ i \circ \varphi_{S}^{-1})_{*\varphi_{S}(p)} $ è iniettiva in quanto $ \varphi \circ i \circ \varphi_{S}^{-1} $ è un'immersione, e $ \varphi_{*p} $ e $ (\varphi_{S}^{-1})_{*\varphi_{S}(p)} $ sono isomorfismi in quanto differenziali di diffeomorfismi (in quanto sono le applicazioni delle carte), allora $ i_{*p} $ è iniettiva in $ p $. Essendo $ p $ arbitrario, l'inclusione $ i $ è un'immersione.
\end{proof}

\begin{remark}
	Se $ F : N \to M $ è un'immersione e $ S \subset N $ è una sottovarietà di $ N $ allora $ \eval{F}_{S} : S \to M $ è un'immersione\footnote{%
		Vedi Esercizio \ref{exer2-22}.%
	}.
\end{remark}

\subsubsection{\textit{Esempi}}

\paragraph{1) Seno del topologo}

Siano un'applicazione $ f $ e il suo grafico $ \Gamma(f) $ definiti come:

\sbs{0.5}{%
			\map{f}
			{(0,1)}{\R}
			{x}{\sin(\dfrac{1}{x})}
			
			\begin{equation}
				\Gamma(f) = \{ (x,f(x)) \in \R^{2} \mid x \in (0,1) \}
			\end{equation}
			}
	{0.5}{%
			\img{0.8}{img32}
			}

Sia inoltre l'insieme $ S = \Gamma(f) \cup I $ dove

\begin{equation}
	I = \{ (x,y) \in \R^{2} \mid x=0 \wedge y \in (-1,1) \}
\end{equation}

Consideriamo dunque $ S \subset \R^{2} $ con la topologia indotta da $ \R^{2} $ e verifichiamo se $ S $ sia una sottovarietà di $ \R^{2} $. \\
L'insieme $ S $ con la struttura topologica indotta da $ \R^{2} $ non è una sottovarietà di $ \R^{2} $ in quanto non è nemmeno una varietà differenziabile. Preso un punto $ p \in I $ e un intorno circolare $ U \ni p $, otteniamo che l'insieme $ U \cap S $ non è connesso in quanto costituito da infiniti segmenti disgiunti: a questo punto, $ S $ non può essere localmente euclideo poiché $ \R^{n} $ è connesso, quindi $ S $ non è una varietà differenziabile. \\
Naturalmente, non è nemmeno possibile trovare una carta adattata che renda $ S $ una sottovarietà.

\sbs{0.5}{%
			\paragraph{2) Cuspide cubica}
	
			L'insieme definito come	
	
			\begin{equation}
				S = \{ (x,y) \in \R^{2} \mid y^{2}=x^{3} \}
			\end{equation}

			con la struttura topologica indotta da $ \R^{2} $ non è una sottovarietà di $ \R^{2} $ ma è una varietà di dimensione unitaria (curva) rispetto alla struttura topologica indotta da $ \R^{2} $.
			}
	{0.5}{%
			\img{0.75}{img33}
			}

Possiamo definire l'applicazione che proietta i punti di $ S $ lungo l'asse verticale, i.e.

\map{\psi}
	{S}{\R}
	{(x,y)}{y}

la quale individua un'unica carta $ (S,\psi) $ per $ S $ che definisce una struttura differenziabile (è compatibile con sé stessa). \\
Mostriamo ora che $ S $ non è una sottovarietà di $ \R^{2} $ attraverso il fatto che l'inclusione $ i : S \to \R^{2} $ non è un'immersione nel punto 0 e dunque non può esserlo per qualsiasi $ p \in S $. \\

\sbs{0.4}{%
			Supponiamo per assurdo che questa inclusione sia un'immersione e sia $ (U,\varphi) $ una carta di $ S $ intorno a 0 con $ \varphi : U \to (a,b) \subset \R $ tale che $ \varphi^{-1}(0) = (0,0) $ e $ \varphi $ sia un diffeomorfismo.
			}
	{0.6}{%
			\img{0.7}{img34}
			}

Definiamo l'applicazione liscia

\begin{equation}
	h = i \circ \varphi^{-1} : (a,b) \to \R^{2}
\end{equation}

la quale è un'immersione in quanto composizione di un diffeomorfismo e di un'immersione\footnote{%
	La composizione di immersioni è ancora un'immersione: questo deriva dal fatto che la composizione di funzioni iniettive (in questo caso differenziali) è ancora una funzione iniettiva. In questo caso, il differenziale $ (i \circ \varphi^{-1})_{*t} $ è iniettivo in quanto composizione del differenziale dell'immersione $ i $ (iniettivo) e dal differenziale del diffeomorfismo $ \varphi $ (il quale è un isomorfismo), i.e. $ (i \circ \varphi^{-1})_{*t} = i_{*\varphi^{-1}(t)} \circ (\varphi^{-1})_{*t} $.%
}, tale che $ h(0)=(0,0) $. Scrivendo $ h(t) = (h_{1}(t),h_{2}(t)) $, sappiamo che

\begin{equation}
	\begin{cases}
		h_{1}(0) = 0 \\
		h_{2}(0) = 0 \\
		h_{1}(t) \geqslant 0 \\
		h_{1}(t) = (h_{2}(t))^{\sfrac{2}{3}}
	\end{cases}
\end{equation}

Deriviamo l'ultima equazione rispetto a $ t $

\begin{equation}
	\dot{h}_{1}(t) = \dfrac{2}{3} \, (h_{2}(t))^{-\sfrac{1}{3}} \, \dot{h}_{2}(t)
\end{equation}

per $ t \to 0 $ abbiamo che

\begin{equation}
		\dot{h}_{1}(0) = \dfrac{2}{3} \, (h_{2}(0))^{-\sfrac{1}{3}} \, \dot{h}_{2}(0) %
		= \dfrac{2 \, \dot{h}_{2}(0)}{3 \, (h_{2}(0))^{\sfrac{1}{3}}}
\end{equation}

siccome $ h_{2}(0)=0 $, otteniamo che $ \dot{h}_{2}(0) $ deve tendere a 0 per far sì che $ \dot{h}_{1}(0) $ sia finito. \\
Sappiamo inoltre che $ \dot{h}_{1}(0) \neq 0 $ perché se consideriamo la funzione $ h $ come una curva

\map{h}
	{(a,b)}{\R^{2}}
	{t}{(h_{1}(t),h_{2}(t))}

e la differenziamo in $ 0 $

\begin{equation}
	h'(0) = h_{*0} \left( \eval{ \dv{t} }_{0} \right) %
	= \dot{h}_{1}(0) \eval{ \pdv{x} }_{(0,0)} + \dot{h}_{2}(0) \eval{ \pdv{y} }_{(0,0)}
\end{equation}

e se $ \dot{h}_{1}(0) = \dot{h}_{2}(0) = 0 $ allora $ h'(0) = 0 $ e quindi $ h_{*0} ( \eval{ \dv*{t} }_{0} ) = 0 $, i.e.

\map{h_{*0}}
	{T_{0}((a,b))}{T_{(0,0)}(\R^{2})}
	{\eval{ \dv{t} }_{0}}{0}

ma questo non è possibile in quanto $ h $ è un'immersione quindi il suo differenziale deve essere iniettivo. \\
Da questo ragionamento, sappiamo ora che $ \dot{h}_{2}(0)=0 $ e $ \dot{h}_{1}(0) \neq 0 $:

\begin{itemize}
	\item se $ \dot{h}_{1}(0) > 0 $ allora esiste un intorno di $ 0 \in (a,b) $ dove $ h_{1} $ è strettamente crescente, dunque esiste un punto $ t_{0}<0 $ tale che $ h_{1}(t_{0}) < h_{1}(0) $ il quale è assurdo perché $ h_{1}(0) = 0 $ e $ h_{1}(t) \geqslant 0 $, i.e. non esiste un $ t_{0} $ tale che $ h_{1}(t_{0}) < 0 $;
	
	\item se $ \dot{h}_{1}(0) < 0 $ allora esiste un intorno di $ 0 \in (a,b) $ dove $ h_{1} $ è strettamente decrescente, dunque esiste un punto $ t_{0}>0 $ tale che $ h_{1}(0) > h_{1}(t_{0}) $ il quale è assurdo perché $ h_{1}(0) = 0 $ e $ h_{1}(t) \geqslant 0 $, i.e. non esiste un $ t_{0} $ tale che $ h_{1}(t_{0}) < 0 $.
\end{itemize}

Questo assurdo invalida la supposizione che l'inclusione $ i : S \to \R^{2} $ sia un'immersione e dunque, per la proposizione, che $ S $ sia una sottovarietà di $ \R^{2} $. \\
La dimensione di $ S $ non può che essere unitaria in quanto, essendo un sottoinsieme di $ \R^{2} $ ci potrebbero essere solo altri due casi

\begin{itemize}
	\item $ \dim(S)=0 $ : l'insieme $ S $ è un insieme discreto di punti, fatto che non si verifica per la cuspide cubica, la quale è connessa e inoltre è immagine di un insieme continuo
	
	\item $ \dim(S)=2 $ : esiste un intorno dell'origine omeomorfo a $ \R^{2} $, ma questo non è possibile perché, togliendo l'origine dall'intorno\footnote{%
		Un esempio di intorno nell'origine di $ S $ è l'insieme $ U $ definito sopra per la carta.%
	} questo diventa non connesso e quindi non omeomorfo a $ \R^{2} \setminus \{0\} $ il quale è ancora un insieme connesso.
\end{itemize}

\paragraph{3) Sfera (metodo generale per la determinazione di sottovarietà)}

Sappiamo che la sfera unitaria $ \S^{2} $ è una varietà differenziabile di dimensione 2, dimostriamo quindi che è anche una sottovarietà di $ \R^{3} $. \\
La sfera $ \S^{2} $ è definita come

\begin{equation}
	\S^{2} = \{ (x,y,z) \in \R^{3} \mid x^{2} + y^{2} + z^{2} = 1 \}
\end{equation}

L'idea di base è quella di sfruttare la seguente applicazione liscia

\map{f}
	{\R^{3}}{\R}
	{(x,y,z)}{x^{2} + y^{2} + z^{2} - 1}

in modo tale da avere la sfera come controimmagine del punto 0, i.e.

\begin{equation}
\S^{2} = f^{-1}(0)
\end{equation}

Osserviamo che $ 0 \in \VR_{f} $, cioè la controimmagine di 0, i.e. la sfera, è costituita da tutti e soli punti regolari tramite $ f $, cioè punti in cui il differenziale di $ f $ è suriettivo. In simboli

\begin{gather}
	p \in \PR_{f}, \, \forall p \in f^{-1}(0) = \S^{2} \nonumber \\
	\Updownarrow \\
	f_{*p} : T_{p}(\R^{3}) \to T_{0}(\R) \text{ suriettivo} \nonumber
\end{gather}

Sappiamo inoltre che queste condizioni sono equivalenti al fatto che le derivate parziali di $ f $ rispetto a $ (x,y,z) $ non si annullano contemporaneamente nei punti della sfera, i.e.

\begin{equation}
	\begin{cases}
		\dpdv{f}{x} \, (p) = 2x(p) \\ \\
		%
		\dpdv{f}{y} \, (p) = 2y(p) \\ \\
		%
		\dpdv{f}{z} \, (p) = 2z(p)
	\end{cases}
\end{equation}

si annullano solo per $ p = (0,0,0) $, il quale non appartiene alla sfera, i.e. $ (0,0,0) \notin \S^{2} $, quindi $ 0 \in \VR_{f} $. \\
Sia $ p \in \S^{2} $ e supponiamo che $ x(p) \neq 0 $ (i.e. la prima coordinata di $ p $ non è nulla) e sia l'applicazione liscia

\map{g}
	{\R^{3}}{\R^{3}}
	{(x,y,z)}{(f(x,y,z),y,z)}

Essendo questa applicazione definita tra due spazi della stessa dimensione, possiamo calcolarne lo jacobiano, il quale vale

\begin{equation}
	J(g)(p) = \bmqty{ \dpdv{f}{x} \, (p) & \dpdv{f}{y} \, (p) & \dpdv{f}{z} \, (p) \\ \\ 0 & 1 & 0 \\ \\ 0 & 0 & 1 } %
	= \bmqty{ 2x(p) & 2y(p) & 2z(p) \\ \\ 0 & 1 & 0 \\ \\ 0 & 0 & 1 }
\end{equation}

Il determinante dello jacobiano non si annulla in quanto la coordinata $ x(p) $ non si annulla per ipotesi:

\begin{equation}
	\det(J(g)(p)) = \pdv{f}{x} \, (p) = 2x(p) \neq 0
\end{equation}

Per il corollario del teorema della funzione inversa\footnote{%
	Vedi Corollario \ref{cor:ift}.%
}, esiste un intorno aperto $ U \subset \R^{3} $ di $ p $ tale che $ \eval{g}_{U} : U \to g(U) $ sia un diffeomorfismo e dunque $ \left( U,\eval{g}_{U} \right) $ è una carta di $ \R^{3} $. In questo caso, $ \left( U,\eval{g}_{U} \right) $ con $ \eval{g}_{U} = (f(x,y,z),y,z) $ è una carta adatta di $ \R^{3} $ intorno a $ p $ relativamente a $ \S^{2} $. Per verificare quest'ultima proposizione, è necessario che $ U \cap \S^{2} $ sia un'insieme dato dall'annullarsi di alcune coordinate:

\begin{equation}
	U \cap \S^{2} = \{ q \in U \mid g^{1}(q) = f(q) = 0 \} \equiv \S^{2}
\end{equation}

A questo punto, intorno a tutti i punti $ p $ che hanno $ x(p) \neq 0 $ esiste una carta adattata. Il ragionamento è analogo per le altre due coordinate. \\
In conclusione, per definizione, $ \S^{2}$ è una sottovarietà di $\R^{3} $ con $ \dim(\S^{2}) = 2 $, in quanto solo una coordinata delle carte adattate si annulla.

\subsubsection{Preimmagini e sottovarietà}

\begin{theorem}[Preimmagine di un valore regolare (caso $ M = \R $)]
	Siano $ N $ una varietà differenziabile con $ \dim(N) = n $, $ f : N \to \R $ un'applicazione liscia e $ c \in \VR_{f} \cap f(N) $ un valore regolare che risiede nell'immagine della funzione\footnote{%
		Siccome valgono le implicazioni
		
		\begin{equation*}
			q \notin f(N) \implies q \in \VR_{f} \implies f(\PR_{f}) \not\supset \VR_{f}
		\end{equation*}
	
		è necessario includere l'appartenenza all'immagine della funzione.%
	}, i.e. $ f^{-1}(c) \neq \emptyset $, allora $ f^{-1}(c) $ è una sottovarietà di $ N $ con $ \dim(f^{-1}(c)) = n-1 $.
\end{theorem}

\begin{proof}
	Possiamo sempre supporre che\footnote{%
		Se consideriamo l'applicazione
		%		
		\maps{h}
			{N}{\R}
			{x}{f(x) - c}
		
		abbiamo che $ h^{-1}(0) = f^{-1}(c) $ e $ c \in \VR_{f} \iff 0 \in \VR_{h} $, perché $ \pdv*{f}{x} = \pdv*{h}{x} $; questo ci permette di considerare l'applicazione $ h $ al posto di $ f $ ottenendo gli stessi risultati.%
	} $ c = 0 \in \R $. \\
	Sia dunque un punto $ p \in f^{-1}(0) $, vogliamo trovare una carta adattata di $ N $ intorno a $ p $ relativamente a $ f^{-1}(0) $. Il fatto che $ 0 \in \VR_{f} $ implica che almeno una delle derivate parziali di $ f $ non si annulli se consideriamo una qualunque carta $ (U,\varphi) $ con $ \varphi = (x^{1},\dots,x^{n}) $ intorno a $ p $. \\
	Supponiamo ora che non si annulli la prima derivata\footnote{%
		Nel caso in cui la prima derivata fosse nulla, possiamo permutare le coordinate della carta ottenendo una carta identica per cui la prima derivata è diversa da zero.%
	} e consideriamo l'applicazione liscia

	\map{g}
		{U}{\R^{n}}
		{q}{(f(q),x^{2}(q),\dots,x^{n}(q))}

	Il determinante dello jacobiano di $ g $ nel punto $ p $ vale

	\begin{equation}
		\det(J(g)(p)) = \det( \bmqty{ %
			\dpdv{f}{x^{1}} \, (p) & \cdots & \cdots & \dpdv{f}{x^{n}} \, (p) \\ \\ %
			0 & 1 & \cdots & 0 \\ \\ %
			\vdots & \vdots & \ddots & \vdots \\ \\ %
			0 & 0 & \cdots & 1
			} ) %
		= \dpdv{f}{x^{1}} \, (p) \neq 0
	\end{equation}

	dunque, dal corollario del teorema della funzione inversa, esiste un aperto $ U_{p} \subset U \subset N $ tale che $ \left( U_{p},\eval{g}_{U_{p}} \right) $ sia una carta di $ N $ intorno a $ p $. Prendendo l'intersezione
	
	\begin{equation}
		U_{p} \cap f^{-1}(0) = \{ q \in U_{p} \mid f(q)=0 \} \equiv f^{-1}(0)
	\end{equation}

	otteniamo che $ \left( U_{p},\eval{g}_{U_{p}} \right) $ è una carta adattata e quindi, siccome il punto $ p \in f^{-1}(0) $ è arbitrario, ogni punto di $ f^{-1}(0) $ ammette una carta adattata: questo dimostra che $ f^{-1}(0) $ è una sottovarietà di $ N $ con $ \dim(f^{-1}(0)) = n-1 $, in quanto solo una delle coordinate in $ U_{p} \cap f^{-1}(0) $ si annulla.
\end{proof}

\begin{definition}
	Siano $ N $ una varietà con $ \dim(N)=n $ e $ S $ una sottovarietà di $ N $ con $ \dim(S)=k \leqslant n $. La \textit{codimensione} di $ S $ in $ N $ è definita come
	
	\begin{equation}
		\operatorname{cod}_{N}(S) \doteq n-k
	\end{equation}

	e rappresenta il numero di variabili che si annullano in una carta adattata di $ N $ relativamente a $ S $.
\end{definition}

Possiamo dimostrare ora il teorema della preimmagine di un valore regolare nel caso di un'applicazione tra varietà.

\begin{theorem}[Preimmagine di un valore regolare]\label{thm:preimg}
	Siano $ N $ e $ M $ due varietà differenziabili rispettivamente con $ \dim(N)=n $ e $ \dim(M)=m $, $ F : N \to M $ un'applicazione liscia tra varietà e $ c \in \VR_{f} \cap F(N) $ un valore regolare che risiede nell'immagine della funzione, i.e. $ F^{-1}(c) \neq \emptyset $, allora $ F^{-1}(c) $ è una sottovarietà di $ N $ con $ \dim(F^{-1}(c)) = n-m $ o equivalentemente $ \operatorname{cod}_{N}(F^{-1}(c)) = m $. \\
	Inoltre
	
	\begin{equation}
		T_{p}(F^{-1}(c)) = \ker(F_{*p}) \qcomma \forall p \in F^{-1}(c)
	\end{equation} 

	il che significa che lo spazio tangente alla sottovarietà coincide con l'insieme dei vettori di $ T_{p}(N) $ che si annullano tramite il differenziale.
\end{theorem}

\begin{proof}
	Fissiamo un punto $ p \in F^{-1}(c) $ e scegliamo due carte
	
	\begin{equation}
		\begin{cases}
			(U,\varphi) = (U; x^{1},\dots,x^{n}) \in N, & \varphi(p) = 0 \in \R^{n} \\
			(V,\psi) = (V; y^{1},\dots,y^{n}) \in M, & \psi(F(p)) = \psi(c) = 0 \in \R^{m} \\
			F(U) \subseteq V, & F \text{ continua}
		\end{cases}
	\end{equation}
	
	\img{0.8}{img35}

	Localmente possiamo scrivere l'applicazione $ F $ come
	
	\begin{equation}
		\psi \circ F \circ \varphi^{-1} : \varphi(U) \to \psi(V)
	\end{equation}

	dove $ \varphi(U) \subset \R^{n} $ e $ \psi(V) \subset \R^{m} $. \\
	Siccome $ c \in \VR_{f} $, il differenziale $ F_{*p} : T_{p}(N) \to T_{F(p)}(M) $ è suriettivo o equivalentemente $ p \in \PR_{F} $ o ancora $ F $ è una sommersione in $ p $. Da questo sappiamo anche che $ \rank(F(p)) = m \leqslant n $ il quale, localmente, è anche il rango dello jacobiano di $ F $ calcolato in $ p $
	
	\begin{equation}
		\rank(J(F)(p)) = \rank \left( \left[ \pdv{F^{i}}{x^{j}} \, (p) \right] \right) = m
	\end{equation}

	dove $ F^{i} = y^{i} \circ F = r^{i} \circ \psi \circ F $, e quindi esiste un minore $ \tilde{J} $ di ordine $ m $ il cui determinante è diverso da zero: possiamo sempre supporre che il minore in questione sia quello formato dalle prime $ m $ righe ed $ m $ colonne, in quanto la permutazione di coordinate all'interno delle carte le lascia invariate, i.e.
	
	\begin{equation}
		\det( \left[ \pdv{F^{i}}{x^{j}} \, (p) \right] ) = \det(\tilde{J}(F)(p)) \neq 0 %
		\qcomma i,j=1,\dots,m
	\end{equation}

	Definiamo ora l'applicazione liscia
	
	\map{G}
		{U}{\R^{n}}
		{q}{(F^{1}(q),\dots,F^{m}(q),x^{m+1}(q),\dots,x^{n}(q))}

	dove abbiamo sostituito le prime $ m $ coordinate con l'immagine di $ q $ tramite $ (F^{1},\dots,F^{m}) $. Lo jacobiano dell'applicazione $ G $ è dato da
	
	\begin{equation}
		J(G)(p) = %
		\bmqty{ %
			\dpdv{F^{1}}{x^{1}} \, (p) & \cdots & \dpdv{F^{1}}{x^{m}} \, (p) & \dpdv{F^{1}}{x^{m+1}} \, (p) & \cdots & \dpdv{F^{1}}{x^{n}} \, (p) \\ \\ %
			\vdots & \ddots & \vdots & \vdots & \ddots & \vdots \\ \\ %
			\dpdv{F^{m}}{x^{1}} \, (p) & \cdots & \dpdv{F^{m}}{x^{m}} \, (p) & \dpdv{F^{m}}{x^{m+1}} \, (p) & \cdots & \dpdv{F^{m}}{x^{n}} \, (p) \\ \\ %
			0 & \cdots & 0 & 1 & \cdots & 0 \\ \\ %
			\vdots & \ddots & \vdots & \vdots & \ddots & \vdots \\ \\ %
			0 & \cdots & 0 & 0 & \cdots & 1 %
		 } %
	 	= \bmqty{ %
	 		\tilde{J}(F)(p) & \left[ \dpdv{F^{i}}{x^{k}} \, (p) \right] \\ \\ %
	 		0_{n-m,m} & \bigone_{n-m,n-m} %
	 		}
	\end{equation}

	dove $ i=1,\dots,m $ e $ k=m+1,\dots,n $. \\
	Essendo una matrice a blocchi, il determinante di questa matrice è ottenuto dai determinanti dei blocchi, i.e.\footnote{%
		Le dimensioni delle matrici nulla e unitaria sono state rimosse in quanto ininfluenti nel calcolo dei loro determinanti.%
	}
	
	\begin{equation}
		\det(J(G)(p)) = %
		\det ( \bmqty{ %
			\det( \tilde{J}(F)(p) ) & \det( \left[ \dpdv{F^{i}}{x^{k}} \, (p) \right] ) \\ \\ %
			\det( 0 ) & \det( \bigone ) %
			} ) %
		= \det( \tilde{J}(F)(p) ) \neq 0
	\end{equation}

	Per il Corollario \ref{cor:ift} del teorema della funzione inversa, esiste un aperto $ U_{p} \in N $ tale che $ \left( U_{p},\eval{G}_{U_{p}} \right) $ sia una carta di $ N $ intorno a $ p $. Considerando la seguente intersezione
	
	\begin{equation}
		U_{p} \cap F^{-1}(c) = \{ q \in U_{p} \mid F^{1}(q) = \dots = F^{m}(q) = 0 \} \equiv F^{-1}(c)
	\end{equation}

	abbiamo che $ \left( U_{p},\eval{G}_{U_{p}} \right) $ è una carta adattata di $ N $ intorno a $ p $ relativamente a $ F^{-1}(c) $, ma $ p $ è un punto arbitrario, quindi esiste una carta adattata intorno ad ogni punto. Siccome si annullano $ m $ coordinate, $ F^{-1}(c) $ è una sottovarietà di $ N $ con $ \dim(F^{-1}(c)) = n-m $ o equivalentemente $ \operatorname{cod}_{N}(F^{-1}(c)) = m $. \\
	Dobbiamo ora dimostrare che
	
	\begin{equation}
		T_{p}(F^{-1}(c)) = \ker(F_{*p}) \qcomma \forall p \in F^{-1}(c)
	\end{equation} 
	
	Questa uguaglianza equivale a dimostrare le due inclusioni seguenti:
	
	\begin{equation}
		\begin{cases}
			T_{p}(F^{-1}(c)) \subset \ker(F_{*p}) \\
			\ker(F_{*p}) \subset T_{p}(F^{-1}(c))
		\end{cases}
	\end{equation}
	
	Per la prima, ricordiamo che il vettore tangente a una varietà può essere descritto come il vettore velocità di una curva liscia che passa per il punto della varietà considerato, i.e.
	
	\begin{equation}
		\begin{cases}
			\gamma : (-\varepsilon,\varepsilon) \to F^{-1}(c) \\
			\gamma(0) = p \\
			\gamma'(0) = X_{p} \in T_{p}(F^{-1}(c))
		\end{cases}
	\end{equation}

	Per definizione di controimmagine e dalla definizione della curva $ \gamma $, abbiamo che
	
	\begin{equation}
		F(\gamma(t)) = (F \circ \gamma)(t) = c \qcomma \forall t \in (-\varepsilon,\varepsilon)
	\end{equation}
	
	dunque il differenziale di $ F $ nel punto $ p $ è dato da
	
	\begin{equation}
		F_{*p}(X_{p}) = (F \circ \gamma)'(0) %
		= c'(0) %
		= 0 \in T_{F(p)}(M)
	\end{equation}

	dove $ c $ è intesa come l'applicazione costante, il cui vettore tangente è il vettore nullo, perciò $ X_{p} \in \ker(F_{*p}) $ e la prima inclusione è dimostrata. \\
	Per la seconda inclusione è sufficiente verificare che
	
	\begin{equation}
		\dim(T_{p}(F^{-1}(c))) = \dim(\ker(F_{*p}))
	\end{equation}

	in quanto, se due spazi vettoriali (uno incluso nell'altro) hanno la stessa dimensione, allora vale l'uguaglianza tra i due. \\
	Sappiamo che
	
	\begin{equation}
		\dim(T_{p}(F^{-1}(c))) = \dim(F^{-1}(c)) = n-m
	\end{equation}

	e il differenziale $ F_{*p} : T_{p}(N) \to T_{F(p)}(M) $ è suriettivo in quanto $ p \in \PR_{F} $ dunque, tramite il teorema della dimensione in algebra lineare
	\begin{align}
		\begin{split}
			\dim(\Im(F_{*p})) + \dim(\ker(F_{*p})) &= \dim(T_{p}(N)) \\
			\dim(T_{F(p)}(M)) + \dim(\ker(F_{*p})) &= \dim(N) \\
			\dim(M) + \dim(\ker(F_{*p})) &= n \\
			m + \dim(\ker(F_{*p})) &= n \\
			\dim(\ker(F_{*p})) &= n-m
		\end{split}		
	\end{align}

	dove $ \dim(\Im(F_{*p})) = \dim(T_{F(p)}(M)) $ proprio perché il differenziale $ F_{*p} $ è suriettivo, perciò $ T_{p}(F^{-1}(c)) = \ker(F_{*p}) $. \\
	A questo punto
	
	\begin{equation}
		\begin{cases}
			T_{p}(F^{-1}(c)) \subset \ker(F_{*p}) \\
			\ker(F_{*p}) \subset T_{p}(F^{-1}(c))
		\end{cases}%
		\implies %
		T_{p}(F^{-1}(c)) = \ker(F_{*p}) \qcomma \forall p \in F^{-1}(c)
	\end{equation}
\end{proof}

\begin{remark}
	Il teorema della preimmagine di un valore regolare dà una condizione sufficiente affinché si possa individuare una sottovarietà, i.e. $ F^{-1}(c) $ come sottovarietà con $ c \in \VR_{F} $, ma è possibile che anche la preimmagine di un valore critico sia una sottovarietà.
\end{remark}

\subsubsection{\textit{Esempi}}

\paragraph{1) Preimmagine di un valore critico come sottovarietà}

Sia l'applicazione

\map{f}
	{\R^{2}}{\R}
	{(x,y)}{y^{2}}

Il punto $ (0,0) \in \R^{2} $ annulla entrambe le derivate parziali

\begin{equation}
	\pdv{f}{x} \, (0,0) = \pdv{f}{y} \, (0,0) = 0
\end{equation}

dunque $ (0,0) \in \PC_{f} $ e perciò $ f(0,0) = 0 \in \VC_{f} $, nonostante ciò la sua controimmagine è una sottovarietà di $ \R^{2} $, i.e. l'asse delle ascisse $ f^{-1}(0) = \{(x,0)\} $.

\paragraph{2) Sfera}

La sfera unitaria è definita come

\begin{equation}
	\S^{n} = \{ (x^{1},\dots,x^{n+1}) \in \R^{n+1} \mid (x^{1})^{2} + \cdots + (x^{n+1})^{2} = 1 \} \subset \R^{n+1}
\end{equation}

Sia l'applicazione liscia

\map{f}
	{\R^{n+1}}{\R}
	{(x^{1},\dots,x^{n+1})}{(x^{1})^{2} + \cdots + (x^{n+1})^{2} - 1}

in modo tale che $ f^{-1}(0) = \S^{n} $, dove $ 0 \in \VR_{f} $, in quanto le derivate parziali di $ f $ non si annullano tutte contemporaneamente se non nell'origine (la quale non è inclusa nella sfera) dunque $ \S^{n} $ è una sottovarietà di $ \R^{n+1} $ con $ \dim(\S^{n}) = n+1-1 = n $. \\
Lo spazio tangente di $ \S^{n} $, dal teorema della preimmagine, è dato dal $ \ker $ del differenziale di $ f $, i.e. 

\begin{equation}
	T_{p}(\S^{n}) = \ker(f_{*p})
\end{equation}

Per calcolarlo consideriamo una curva liscia definita come

\begin{equation}
	\begin{cases}
		\gamma : (-\varepsilon,\varepsilon) \to \S^{n} \\
		\gamma(0) = p \\
		\gamma'(0) = X_{p}
	\end{cases}
\end{equation}

e il differenziale della funzione $ f $ nel punto $ p $

\begin{equation}
	f_{*p} : T_{p}(\R^{n+1}) \to T_{f(p)}(\R)
\end{equation}

siccome $ \gamma(t) = (x^{1}(t),\dots,x^{n+1}(t)) $, abbiamo che

\begin{align}
	\begin{split}
		f_{*p}(X_{p}) &= (f \circ \gamma)'(0) \\
		&= ((x^{1}(t))^{2} + \cdots + (x^{n+1}(t))^{2} - 1)'(0) \\
		&= \dot{((x^{1}(t))^{2} + \cdots + (x^{n+1}(t))^{2} - 1)}(0) \eval{ \dv{r} }_{f(p)} \\
		&= (2 x^{1}(t) \, \dot{x}^{1}(t) + \cdots + 2 x^{n+1}(t) \, \dot{x}^{n+1}(t))(0) \eval{ \dv{r} }_{f(p)} \\
		&= (2 x^{1}(0) \, \dot{x}^{1}(0) + \cdots + 2 x^{n+1}(0) \, \dot{x}^{n+1}(0)) \eval{ \dv{r} }_{f(p)} \\
		&= \left( 2 (x^{1}(0), \dots, x^{n+1}(0)) \cdot (\dot{x}^{1}(0), \dots, \dot{x}^{n+1}(0)) \right) \eval{ \dv{r} }_{f(p)} \\
		&= 2 (\gamma(0) \cdot X_{p}) \eval{ \dv{r} }_{f(p)} \\
		&= 2 (p \cdot X_{p}) \eval{ \dv{r} }_{f(p)}
	\end{split}
\end{align}

dove nel terzo passaggio è presente una derivata (indicata dal puntino sopra l'intera parentesi), il vettore $ X_{p} $ può essere rappresentato come un punto in $ \R^{n+1} $, i.e. $ X_{p} = (\dot{x}^{1}(0),\dots,\dot{x}^{n+1}(0)) $, e il punto in $ p \cdot X_{p} $ indica il prodotto scalare in $ \R^{n+1} $. \\
Abbiamo perciò derivato la forma del differenziale:

\map{f_{*p}}
	{T_{p}(\R^{n+1})}{T_{f(p)}(\R)}
	{X_{p}}{(2 \, p \cdot X_{p}) \eval{ \dv{r} }_{f(p)}}
	
A questo punto, il suo $ \ker $, e dunque lo spazio tangente alla sfera $ \S^{n} $, è costituito da tutti i vettori $ X_{p} $ di $ \R^{n+1} $ che sono ortogonali al punto $ p $, i.e. $ p \cdot X_{p} = 0 $. In altre parole, presa la sfera e un piano tangente a essa in un punto, tutti i vettori complanari al piano tangente sono tangenti alla sfera in quel punto.

\paragraph{3)}\label{example:subvar-cond}

Sia l'insieme

\begin{equation}
	S = \{ (x,y,z) \in \R^{3} \mid x^{3} + y^{3} + z^{3} = 1 \wedge x+y+z=0 \} \subset \R^{3}
\end{equation}

Vogliamo dimostrare che $ S $ sia una sottovarietà di $ \R^{3} $ di dimensione unitaria e lo facciamo cercando un'applicazione la cui controimmagine del punto\footnote{%
	La dimensione del punto $ (0,\dots,0) \in \Im(f) $ dipende dal numero di condizioni che vengono poste all'interno della definizione dello spazio $ S $; in questo caso ci sono due condizioni, dunque $ (0,0) \in \Im(f) \subset \R^{2} $.%
} $ (0,\dots,0) $ sia $ S $, tale che questo punto sia un valore regolare per l'applicazione. \\
Sia dunque l'applicazione

\map{F}
	{\R^{3}}{\R^{2}}
	{(x,y,z)}{(x^{3}+y^{3}+z^{3}-1, x+y+z)}

dove $ F^{-1}(0,0) = S $. Verificare che $ (0,0) \in \VR_{F} $ è equivalente a verificare che il differenziale $ F_{*p} $, tramite il suo jacobiano, abbia rango massimo pari a 2 (questo rende il differenziale suriettivo in tutti i punti di $ F^{-1}(0,0) $): per verificare ciò, calcoliamo lo jacobiano di $ F $

\begin{equation}
	J(F)(x,y,x) = \bmqty{ 3x^{2} & 3y^{2} & 3z^{2} \\ \\ 1 & 1 & 1 }\in M_{2,3}(\R)
\end{equation}

I punti che rendono unitario (e quindi minore di 2) il rango\footnote{%
	Ricordiamo che, per il calcolo del rango, si considerano i determinanti dei minori quadrati della matrice.%
} di $ F $ sono i punti critici di $ F $:

\begin{equation}
	\PC_{F} = \{ (x,y,z) \in \R^{3} \mid x^{2}-y^{2}=0 \lor y^{2}-z^{2}=0 \}
\end{equation}

A questo punto

\begin{equation}
	(0,0) \in \VR_{F} \iff S \cap \PC_{F} = \emptyset
\end{equation}

Per verificare questa condizione, costruiamo un sistema la cui soluzione indica l'intersezione tra lo spazio $ S $ e i punti critici di $ F $:

\begin{equation}
	\begin{cases}
		x^{3} + y^{3} + z^{3} = 1 \\
		x+y+z=0 \\
		x = \pm y \\
		y = \pm z
	\end{cases}
\end{equation}

Questo sistema non ha soluzione\footnote{%
	Perché abbia soluzione, per la seconda equazione si deve avere che una delle tre coordinate sia nulla e le altre due siano una l'opposta dell'altra, ma questa situazione non è permessa dalla prima equazione.%
} o, equivalentemente, $ S \cap \mathcal{PC}_{F} = \emptyset $ e dunque $ S $ è una sottovarietà di $ \R^{3} $. \\
Per il calcolo dello spazio tangente alla varietà $ S $, vedi Esercizio \ref{BONUS2-3}.

\paragraph{4) Grafico di funzione}

Siano un'applicazione liscia e il suo grafico

\begin{gather}
	F : \R^{n} \to \R \\
	\nonumber \\
	\Gamma(F) = \{ (x^{1},\dots,x^{n},y) \in \R^{n+1} \mid y = F(x^{1},\dots,x^{n}) \} \subset \R^{n+1}
\end{gather}

Vogliamo dimostrare che $ \Gamma(F) $ sia una sottovarietà di $ \R^{n+1} $ di dimensione $ n $ e per fare ciò costruiamo una carta adattata attorno a ogni suo punto. \\
Sia l'applicazione liscia

\map{\varphi}
	{\R^{n+1}}{\R^{n+1}}
	{(x^{1},\dots,x^{n},y)}{(x^{1},\dots,x^{n},y - F(x^{1},\dots,x^{n}))}

la cui inversa è liscia

\map{\varphi^{-1}}
	{\R^{n+1}}{\R^{n+1}}
	{(a^{1},\dots,a^{n},b)}{(a^{1},\dots,a^{n},b + F(a^{1},\dots,a^{n}))}

dunque $ \varphi $ è un diffeomorfismo: questo significa che $ (\R^{n+1},\varphi) $ è una carta di $ \R^{n+1} $ con la struttura euclidea standard. Questa carta può anche essere pensata come carta adattata di $ \R^{n+1} $ relativamente a $ \Gamma(F) $ per qualunque punto $ p \in \Gamma(F) $: questo è vero in quanto l'insieme

\begin{equation}
	\R^{n+1} \cap \Gamma(F) = \{ q \in \R^{n+1} \mid y - F(x^{1},\dots,x^{n}) = 0 \} = \Gamma(F)
\end{equation}

rispetta la condizione per le sottovarietà; siccome una sola coordinata si annulla nelle carte adattate, la dimensione del grafico di $ F $ è pari a

\begin{equation}
	\dim(\Gamma(F)) = n+1-1 = n
\end{equation}

\paragraph{5) Sottoinsieme del proiettivo reale}

Sia l'insieme $ S \subset \rp{2} $ definito come

\begin{equation}
	S = \{ [x_{0},x_{1},x_{2}] \in \rp{2} \mid x_{0}^{2} - x_{1}^{2} + x_{2}^{2} = 0 \}
\end{equation}

questo insieme è ben definito poiché un altro rappresentante della classe $ (y_{0},y_{1},y_{2}) \in [x_{0},x_{1},x_{2}] $ che sia proporzionale a quello della definizione per mezzo di $ \lambda \in \R \setminus \{0\} $, i.e. $ (y_{0},y_{1},y_{2}) = \lambda (x_{0},x_{1},x_{2}) $, rispetta ancora la condizione dell'insieme $ S $. \\
Prendendo le tre carte $ \{(U_{i},\varphi_{1})\}_{i=0,1,2} $ di $ \rp{2} $, queste sono carte adattate relativamente a $ S $ e dunque lo rendono una sottovarietà del proiettivo reale. \\
Di seguito i tre casi per le tre carte:

\begin{itemize}
	\item Prendiamo la carta $ (U_{0},\varphi_{0}) $ dove
	
	\begin{equation}
		U_{0} = \{ [(x_{0},x_{1},x_{2})] \in \rp{2} \mid x_{0} \neq 0 \}
	\end{equation}
	%	
	\map{\varphi_{0}}
		{U_{0}}{\R^{2}}
		{[(x_{0},x_{1},x_{2})]}{(x,y) = \left( \dfrac{x_{1}}{x_{0}}, \dfrac{x_{2}}{x_{0}} \right)}
	
	da cui otteniamo che la condizione diventa $ 1 - x^{2} + y^{2} = 0 $ e dunque
	
	\begin{equation}
		\varphi_{0}(S) = \{ (x,y) \in \R^{2} \mid x^{2} - y^{2} = 1 \}
	\end{equation}

	che rappresenta i punti di un'iperbole equilatera, la quale è una sottovarietà di $ \R^{2} $ in quanto controimmagine del valore regolare\footnote{%
		Le derivate parziali $ 2x $ e $ 2y $ si annullano solamente in 0 ma $ f^{-1}(0) \neq 0 $.%
	} $ 0 \in \VR_{f} $ tramite l'applicazione
	
	\map{f}
		{\R^{2}}{\R}
		{(x,y)}{x^{2} - y^{2} - 1}
	
	A questo punto, $ S $ è una sottovarietà di $ \rp{2} $ in quanto controimmagine del valore regolare $ 0 \in \R $ tramite la composizione dell'applicazione $ f $	e il diffeomorfismo $ \varphi_{0} $, i.e.
	
	\begin{equation}
		S = \varphi_{0}^{-1} (f^{-1}(0)) = (f \circ \varphi_{0})^{-1} (0)
	\end{equation}
	
	\item Prendiamo la carta $ (U_{1},\varphi_{1}) $ dove
	
	\begin{equation}
		U_{1} = \{ [(x_{0},x_{1},x_{2})] \in \rp{2} \mid x_{1} \neq 0 \}
	\end{equation}
	%	
	\map{\varphi_{1}}
		{U_{1}}{\R^{2}}
		{[(x_{0},x_{1},x_{2})]}{(x,y) = \left( \dfrac{x_{0}}{x_{1}}, \dfrac{x_{2}}{x_{1}} \right)}
	
	da cui otteniamo che la condizione diventa $ x^{2} - 1 + y^{2} = 0 $ e dunque
	
	\begin{equation}
		\varphi_{1}(S) = \{ (x,y) \in \R^{2} \mid x^{2} + y^{2} = 1 \}
	\end{equation}
	
	che rappresenta i punti di un cerchio\footnote{%
		Dal punto di vista della geometria proiettiva non esiste distinzione tra parabola, iperbole ed ellisse.%
	}, il quale è una sottovarietà di $ \R^{2} $ in quanto controimmagine del valore regolare $ 0 \in \VR_{f} $ tramite l'applicazione
	
	\map{f}
	{\R^{2}}{\R}
	{(x,y)}{x^{2} + y^{2} - 1}
	
	A questo punto, $ S $ è una sottovarietà di $ \rp{2} $ in quanto controimmagine del valore regolare $ 0 \in \R $ tramite la composizione dell'applicazione $ f $	e il diffeomorfismo $ \varphi_{1} $, i.e.
	
	\begin{equation}
		S = \varphi_{1}^{-1} (f^{-1}(0)) = (f \circ \varphi_{1})^{-1} (0)
	\end{equation}
	
	\item Prendiamo la carta $ (U_{2},\varphi_{2}) $ dove
	
	\begin{equation}
		U_{2} = \{ [(x_{0},x_{1},x_{2})] \in \rp{2} \mid x_{2} \neq 0 \}
	\end{equation}
	%	
	\map{\varphi_{2}}
		{U_{2}}{\R^{2}}
		{[(x_{0},x_{1},x_{2})]}{(x,y) = \left( \dfrac{x_{0}}{x_{2}}, \dfrac{x_{1}}{x_{2}} \right)}
	
	da cui otteniamo che la condizione diventa $ x^{2} - y^{2} + 1 = 0 $ e dunque
	
	\begin{equation}
		\varphi_{2}(S) = \{ (x,y) \in \R^{2} \mid y^{2} - x^{2} = 1 \}
	\end{equation}
	
	che rappresenta i punti di un'iperbole equilatera (ruotata di $ \sfrac{\pi}{2} $ rispetto alla prima), la quale è una sottovarietà di $ \R^{2} $ in quanto controimmagine del valore regolare $ 0 \in \VR_{f} $ tramite l'applicazione
	
	\map{f}
		{\R^{2}}{\R}
		{(x,y)}{y^{2} - x^{2} - 1}
	
	A questo punto, $ S $ è una sottovarietà di $ \rp{2} $ in quanto controimmagine del valore regolare $ 0 \in \R $ tramite la composizione dell'applicazione $ f $	e il diffeomorfismo $ \varphi_{2} $, i.e.
	
	\begin{equation}
		S = \varphi_{2}^{-1} (f^{-1}(0)) = (f \circ \varphi_{2})^{-1} (0)
	\end{equation}
\end{itemize}

\paragraph{6) Gruppo delle matrici con determinante unitario}\label{example:sl-subman}

Sia l'insieme

\begin{equation}
	SL_{n}(\R) = \{ A \in GL_{n}(\R) \mid \det(A) = 1 \} \subset GL_{n}(\R) \subset M_{n}(\R) = \R^{n^{2}}
\end{equation}

il quale è una varietà differenziabile di dimensione $ n^{2} - 1 $. \\
Vogliamo dimostrare che $ SL_{n}(\R) $ sia una sottovarietà di $ GL_{n}(\R) $ di dimensione $ n^{2}-1 $ o equivalentemente $ \operatorname{cod}_{GL_{n}(\R)}(SL_{n}(\R)) = 1 $ e, per dimostrarlo, usiamo il teorema della preimmagine di un valore regolare. \\
Sia l'applicazione liscia

\map{f}
	{GL_{n}(\R)}{\R}
	{A}{\det(A)}

Siccome $ f^{-1}(1) = SL_{n}(\R) $, dobbiamo mostrare che $ 1 \in \VR_{f} $ perché $ SL_{n}(\R) $ sia una sottovarietà di $ GL_{n}(\R) $: siccome il codominio di $ f $ è $ \R $, un punto è critico per $ f $ se e solo se le derivate parziali di $ f $ si annullano in quel punto. Usiamo come coordinate le componenti della matrice, i.e. $ A = [a_{ij}] $ con $ i,j=1,\dots,n $, e per scrivere il determinante lo sviluppo di Laplace:

\begin{equation}
	\det(A) = \sum_{i=1}^{n} (-1)^{i+j} \, a_{ij} \, m_{ij}
\end{equation}

dove gli $ m_{ij} = \det(A_{ij}) $ sono i minori di $ A $ e le $ A_{ij} $ sono le sottomatrici ricavate da $ A $ rimuovendo la $ i $-esima riga e la $ j $-esima colonna. \\
A questo punto, calcoliamo la derivata di $ f $ rispetto alle coordinate $ a_{ij} $

\begin{equation}
	\pdv{f}{a_{ij}} = (-1)^{i+j} \, m_{ij}
\end{equation}

La condizione che individua i punti critici è la seguente

\begin{align}
	\pdv{f}{a_{ij}} = 0 \iff m_{ij} = 0 \qcomma \forall i,j=1,\dots,n
\end{align}

ma
%
\begin{equation}
	m_{ij} = 0, \, \forall i,j=1,\dots,n \implies \det(A) = 0
\end{equation}

Siccome quindi $ 1 \notin \VC_{f} $ allora $ 1 \in \VR_{f} $, in quanto $ \VC_{f} \cap \VR_{f} = \emptyset $: questo prova che $ f^{-1}(1) = SL_{n}(\R) $ è una sottovarietà di $ GL_{n}(\R) $ di dimensione $ n^{2}-1 $. \\
Dimostreremo che\footnote{%
	Vedi Sottosezione \ref{s-sec:slng-sublie}.}

\begin{equation}
	T_{I}(SL_{n}(\R)) = \{ A \in M_{n}(\R) \mid \tr(A)=0 \} \subset T_{I}(GL_{n}(\R)) = M_{n}(\R)
\end{equation}

Per il caso complesso, vedi Esercizio \ref{exer2-16}.

\subsubsection{Teorema del rango costante}

\begin{theorem}[Rango costante in analisi]
	Sia un'applicazione $ F : W \to \R^{m} $ con $ W \subset \R^{n} $ aperto e supponiamo che esista un intorno $ I_{1} $ di un punto $ p \in W $ tale che il rango dello jacobiano dell'applicazione in $ I_{1} $ sia pari a un valore costante $ k $, i.e.
	
	\begin{equation}
		\rank(J(F)(q)) = \rank \left( \left[ \pdv{F^{i}}{x^{j}} \, (q) \right] \right) = k \leqslant \min\{n,m\} \qcomma \forall q \in I_{1}
	\end{equation}

	Allora esistono due diffeomorfismi
	
	\begin{gather}
		g : I_{1} \to \R^{n} \, \mid \, g(p) = 0 \\
		h : I_{2} \to \R^{m} \, \mid \, h(F(p)) = 0
	\end{gather}
	
	dove $ I_{2} $ è un intorno di $ F(p) $ in $ \R^{m} $ tale che $ F(I_{1}) \subset I_{2} $, la cui composizione con $ F $
	
	\map{h \circ F \circ g^{-1}}
		{\R^{n}}{\R^{m}}
		{(r^{1},\dots,r^{n})}{(r^{1},\dots,r^{k},0,\dots,0)}

	è chiamata \textit{forma canonica} di $ F $, scritta solitamente nella forma
	
	\begin{equation}
		(h \circ F \circ g^{-1})(r^{1},\dots,r^{n}) = (r^{1},\dots,r^{k},0,\dots,0)
	\end{equation}
	
	\img{0.7}{img36}
\end{theorem}

\begin{theorem}[Rango costante in geometria differenziale]\label{thm:const-rank}
	Siano $ F : N \to M $ un'applicazione liscia tra varietà $ N $ e $ M $ di dimensione rispettivamente $ n $ e $ m $, e $ I $ un intorno di un punto $ p \in N $ tale che
	
	\begin{equation}
		\rank(F(q)) = k \leqslant \min\{n,m\} \qcomma \forall q \in I
	\end{equation}

	allora esistono due carte con condizioni
	
	\begin{equation}
		\begin{cases}
			(U,\varphi) \in N, & \varphi(p) = 0 \\
			(V,\psi) \in M, & \psi(F(p)) = 0 \\
			F(U) \subseteq V
		\end{cases}
	\end{equation}

	tali che la composizione tra le applicazioni delle carte ed $ F $ sia
	
	\map{\psi \circ F \circ \varphi^{-1}}
		{\varphi(U)}{\psi(V)}
		{(r^{1},\dots,r^{n})}{(r^{1},\dots,r^{k},0,\dots,0)}

	dove $ \varphi(U) \subset \R^{n} $ e $ \psi(V) \subset \R^{m} $, alternativamente scritta come
	
	\begin{equation}
		(\psi \circ F \circ \varphi^{-1})(r^{1},\dots,r^{n}) = (r^{1},\dots,r^{k},0,\dots,0)
	\end{equation}
\end{theorem}

Questi teoremi derivano dal teorema della funzione inversa come caso particolare di quest'ultimo.

\begin{proof}
	La dimostrazione si basa sull'idea di restringere gli aperti considerati in modo tale che si possa applicare il teorema del rango costante in analisi. \\
	Consideriamo il seguente schema:
	
	\img{1}{img37}

	Nell'intorno $ W \subset N $ di $ p $ il rango di $ F $ è costante, i.e.
	
	\begin{equation}
		\rank(F(q)) = k \qcomma \forall q \in W
	\end{equation}

	Consideriamo due carte arbitrarie
	
	\begin{equation}
		\begin{cases}
			(\tilde{U},\tilde{\varphi}) \in N, & \tilde{\varphi}(p) = 0 \\
			(\tilde{V},\tilde{\psi}) \in M, & \tilde{\psi}(F(p)) = 0 \\
			F(\tilde{U}) \subseteq \tilde{V}
		\end{cases}
	\end{equation}

	 possiamo dunque scrivere la composizione
	
	\begin{equation}
		f \doteq \tilde{\psi} \circ F \circ \tilde{\varphi}^{-1} : \tilde{\varphi}(\tilde{U}) \to \tilde{\psi}(\tilde{V})
	\end{equation}

	con $ \tilde{\varphi}(\tilde{U}) \subset \R^{n} $ e $ \tilde{\psi}(\tilde{V}) \subset \R^{m} $; questa applicazione porta $ 0 \in \tilde{\varphi}(\tilde{U}) $ in $ 0 \in \tilde{\psi}(\tilde{V}) $, in quanto entrambe le carte sono centrare rispettivamente in $ p $ e $ F(p) $. \\
	Essendo dominio e codominio di quest'applicazione due aperti di spazi euclidei, possiamo applicare il teorema del rango costante in analisi: sappiamo che
	
	\begin{equation}
		\rank(f(\tilde{\varphi}(q))) = \rank(\tilde{\psi} \circ F \circ \tilde{\varphi}^{-1})(\tilde{\varphi}(q)) = k \qcomma \forall q \in \tilde{U}
	\end{equation}

	perciò esistono due aperti $ I_{1} \subset \tilde{\varphi}(\tilde{U}) $ e $ I_{2} \subset \tilde{\psi}(\tilde{V}) $ entrambi centrati nella rispettiva origine e due diffeomorfismi
	
	\begin{gather}
		g : I_{1} \to g(I_{1}) \subset \R^{n} \\
		h : I_{2} \to h(I_{2}) \subset \R^{m}
	\end{gather}

	tali che $ g(0) = 0 $ e $ h(0) = 0 $, per i quali l'applicazione $ f $ può essere scritta nella forma canonica
	
	\map{h \circ f \circ g^{-1}}
		{g(I_{1})}{h(I_{2})}
		{(r^{1},\dots,r^{n})}{(r^{1},\dots,r^{k},0,\dots,0)}

	A questo punto, possiamo considerare le carte
	
	\begin{equation}
		\begin{cases}
			(U,\varphi) \doteq (\tilde{\varphi}^{-1}(I_{1}), g \circ \tilde{\varphi}) \in N, & \varphi(p) = 0 \\
			(V,\psi) \doteq (\tilde{\psi}^{-1}(I_{2}), h \circ \tilde{\psi}) \in M, & \psi(F(p)) = 0
		\end{cases}
	\end{equation}
	
	Esplicitando la composizione
	
	\begin{equation}
		h \circ f \circ g^{-1} = h \circ \tilde{\psi} \circ F \circ \tilde{\varphi}^{-1} \circ g^{-1} = \psi \circ F \circ \varphi^{-1}
	\end{equation}

	otteniamo dunque
	
	\map{\psi \circ F \circ \varphi^{-1}}
		{\varphi(U)}{\psi(V)}
		{(r^{1},\dots,r^{n})}{(r^{1},\dots,r^{k},0,\dots,0)}
\end{proof}

\begin{theorem}[Preimmagine di un'applicazione di rango costante]\label{thm:preimg-rank}
	Siano un'applicazione liscia $ F : N \to M $ tra varietà differenziabili $ N $ e $ M $ di dimensione rispettivamente $ n $ ed $ m $, e un punto $ c \in M $ tale che $ F^{-1}(c) \neq \emptyset $. Supponiamo esista un aperto $ W \subset N $ tale che $ F^{-1}(c) \subset W $ nel quale il rango di $ F $ sia costante, i.e.
	
	\begin{equation}
		\rank(F(q)) = k \qcomma \forall q \in W
	\end{equation}

	Allora $ F^{-1}(c) $ è una sottovarietà di $ N $ con dimensione e codimensione
	
	\begin{gather}
		\dim(F^{-1}(c)) = n - k \\
		\operatorname{cod}_{N}(F^{-1}(c)) = k
	\end{gather}
\end{theorem}

\begin{proof}
	Sia un punto $ p \in F^{-1}(c) $, costruiamo dunque una carta di $ N $ intorno a $ p $ relativamente a $ F^{-1}(c) $. Siano le carte

	\begin{equation}
		\begin{cases}
			(U,\varphi) = (U; x^{1},\dots,x^{n}) \in N \qcomma \varphi(p) = 0 \in \R^{n} \\
			(V,\psi) = (V; y^{1},\dots,y^{m}) \in M \qcomma \psi(F(p)) = 0  \in \R^{m} \\
			F(U) \subseteq V
		\end{cases}
	\end{equation}

	e la composizione ricavata dal Teorema \ref{thm:const-rank}
	
	\map{\psi \circ F \circ \varphi^{-1}}
		{\varphi(U)}{\psi(V)}
		{(r^{1},\dots,r^{n})}{(r^{1},\dots,r^{k},0,\dots,0)}

	dove $ \varphi(U) \subset \R^{n} $ e $ \psi(V) \subset \R^{m} $. Queste carte esistono perché possiamo sempre supporre che $ U \subset W $, dove $ W $ è un intorno di $ p $ in cui il rango di $ F $ è costante con valore $ k $. \\
	Osserviamo che
	
	\begin{align}
		\begin{split}
			(\psi \circ F \circ \varphi^{-1})^{-1}(0) &= \{ \varphi(q) \in \varphi(U) \mid r^{1}(\varphi(q)) = \cdots = r^{k}(\varphi(q)) = 0 \} \\
			&= \{ \varphi(q) \in \varphi(U) \mid x^{1}(q) = \cdots = x^{k}(q) = 0 \}
		\end{split}
	\end{align}

	siccome $ F \equiv \eval{F}_{U} $
	
	\begin{align}
		\begin{split}
			(\psi \circ F \circ \varphi^{-1})^{-1}(0) &= \left( \psi \circ \eval{F}_{U} \circ \varphi^{-1} \right)^{-1}(0) \\
			&= \left( \varphi \circ \left( \eval{F}_{U} \right)^{-1} \circ \psi^{-1} \right)(0) \\
			&= \varphi \left( \left( \eval{F}_{U} \right)^{-1}(\psi^{-1}(0)) \right) \\
			&= \varphi \left( \left( \eval{F}_{U} \right)^{-1}(F(p)) \right) \\
			&= \varphi \left( \left( \eval{F}_{U} \right)^{-1}(c) \right) \\
			&= \varphi \left( U \cap F^{-1}(c) \right)
		\end{split}
	\end{align}

	dunque
	
	\begin{equation}
		\varphi( U \cap F^{-1}(c) ) = \{ \varphi(q) \in \varphi(U) \mid x^{1}(q) = \cdots = x^{k}(q) = 0 \}
	\end{equation}

	facendo la controimmagine di questo insieme tramite $ \varphi $ otteniamo
	
	\begin{equation}
		U \cap F^{-1}(c) = \{ q \in U \mid x^{1}(q) = \cdots = x^{k}(q) = 0 \}
	\end{equation}

	la quale è la condizione per cui $ (U,\varphi) $ è una carta adattata di $ N $ intorno a $ p $ (arbitrario) relativamente a $ F^{-1}(c) $, rendendo perciò $ F^{-1}(c) $ una sottovarietà di $ N $ di dimensione $ n-k $.
\end{proof}

\subsubsection{\textit{Esempio}}

Dimostriamo che l'insieme delle matrici ortogonali

\begin{equation}
	O(n) = \{ A \in GL_{n}(\R) \mid A^{T} A = I_{n} \} \subset GL_{n}(\R)
\end{equation}

è una sottovarietà di $ GL_{n}(\R) $. \\
Sia l'applicazione

\map{F}
	{GL_{n}(\R)}{GL_{n}(\R)}
	{A}{A^{T} A}

dunque $ O(n) = F^{-1}(I) $, dove identifichiamo $ I \equiv I_{n} $. \\
Introduciamo le applicazioni traslazione a sinistra e a destra:

\map{L_{C}}
	{GL_{n}(\R)}{GL_{n}(\R)}
	{A}{C A}
%
\map{R_{C}}
	{GL_{n}(\R)}{GL_{n}(\R)}
	{A}{A C}

le quali sono diffeomorfismi con inverse $ L_{C}^{-1} = L_{C^{-1}} $ e $ R_{C}^{-1} = R_{C^{-1}} $. \\
Vale la seguente equazione

\begin{equation}
	L_{C^{T}} \circ R_{C} \circ F = F \circ R_{C}
\end{equation}

in quanto

\begin{align}
	\begin{split}
		(F \circ R_{C})(A) &= F(AC) \\
		&= (AC)^{T} (AC) \\
		&= C^{T} A^{T} A C \\
		&= C^{T} F(A) C \\
		&= (L_{C^{T}} \circ R_{C} \circ F)(A)
	\end{split}
\end{align}

Considerando $ A \in GL_{n}(\R) $, calcoliamo il differenziale dell'equazione

\begin{align}
	\begin{split}
		(L_{C^{T}} \circ R_{C} \circ F)_{*A} &= (F \circ R_{C})_{*A} \\
		(L_{C^{T}})_{*A^{T} A C} \circ (R_{C})_{*A^{T} A} \circ F_{*A} &= F_{*A C} \circ (R_{C})_{*A}
	\end{split}
\end{align}

Siccome le traslazioni a sinistra e a destra sono diffeomorfismi, i loro differenziali sono isomorfismi, i quali non modificano il rango di applicazioni lineari, perciò

\begin{equation}
	\rank(F_{*A}) = \rank(F_{*A C})
\end{equation}

da cui, essendo $ C $ arbitrario, possiamo scrivere $ A C = B $ e

\begin{equation}
	\rank(F(A)) = \rank(F(B)) \qcomma \forall A,B \in GL_{n}(\R)
\end{equation}

dunque il rango di $ F $ non dipende dalla matrice a cui si applica, i.e. il rango di $ F $ è costante in tutto $ GL_{n}(\R) $. \\
Applicando dunque il Teorema \ref{thm:preimg-rank}, $ O(n) $ è una sottovarietà di $ GL_{n}(\R) $.

\subsection{Teoremi di immersione e sommersione locale}

Un'applicazione liscia $ F : N \to M $ tra varietà differenziabili di dimensione rispettivamente $ n $ ed $ m $ ha \textit{rango massimale} in un punto $ p \in N $ se

\begin{equation}
	\rank(F(p)) = k = \min\{n,m\}
\end{equation}

Se $ k = m \leqslant n $ allora $ F $ è una sommersione in $ p $ ($ F_{*p} $ suriettivo) invece se $ k = n \leqslant m $ allora $ F $ è un'immersione in $ p $ ($ F_{*p} $ iniettivo).

\begin{lemma}
	Sia un'applicazione liscia $ F : N \to M $ e sia un punto $ p \in N $ tale che il rango di $ F $ in $ p $ sia massimale, allora esiste un intorno di $ p $ dove il rango di $ F $ è massimale. \\
	Equivalentemente, la condizione di \textit{rango massimale} è una \textit{condizione aperta}.
\end{lemma}

\begin{proof}
	Siano le carte
	
	\begin{equation}
		\begin{cases}
			(U,\varphi) = (U; x^{1},\dots,x^{n}) \in N, & \varphi(p) = 0 \in \R^{n} \\
			(V,\psi) = (V; y^{1},\dots,y^{m}) \in M, & \psi(F(p)) = 0  \in \R^{m} \\
			F(U) \subseteq V
		\end{cases}
	\end{equation}

	Definiamo l'insieme
	
	\begin{equation}
		W \doteq \left\{ q \in U \st \rank(J(F)(q)) = \rank \left( \left[ \pdv{F^{i}}{x^{j}} \, (q) \right] \right) = k \right\} \subset U \subset N
	\end{equation}

	Per ipotesi $ p \in W $ e dunque $ W \neq \emptyset $. \\
	Per mostrare che $ W $ sia aperto in $ N $, mostriamo prima che sia aperto in $ U $: siccome $ U $ è aperto in $ N $, allora anche $ W $ sarà aperto in $ N $. \\
	Possiamo riscrivere la condizione che definisce l'insieme $ W $ inserendo una disuguaglianza per il rango dello jacobiano:
	
	\begin{equation}
		\rank(J(F)(q)) = k %
		\implies %
		\rank(J(F)(q)) \geqslant k
	\end{equation}

	Siccome il rango è massimale e non può essere maggiore di $ k $, le condizioni sono equivalenti al fine della definizione di $ W $, i.e. possiamo riscrivere $ W $ nel seguente modo
	
	\begin{equation}
		W = \{ q \in U \mid \rank(J(F)(q)) \geqslant k \}
	\end{equation}

	Il complementare di $ W $ è
	
	\begin{align}
		\begin{split}
			U \setminus W &= \{ q \in U \mid \rank(J(F)(q)) < k \} \\
			&= \{ q \in U \mid m_{1}(q) = \cdots = m_{t}(q) = 0 \}
		\end{split}
	\end{align}

	dove i $ m_{i}(q) $ sono i minori\footnote{%
		Un minore di ordine $ k $ di una matrice di dimensione $ n \times m $ è il determinante di una sottomatrice quadrata $ k \times k $ ricavata dalla prima.%
	} di ordine $ k $ di $ J(F)(q) $ e
	
	\begin{equation}
		t = %
		\begin{cases}
			\displaystyle \binom{m}{k}, & k = n \\ \\
			\displaystyle \binom{n}{k}, & k = m
		\end{cases}
	\end{equation}

	Essendo i minori delle funzioni continue e $ U \setminus W $ la controimmagine di 0 (chiuso) del sistema finito di $ t $ funzioni continue
	
	\begin{equation}
		\begin{cases}
			m_{1}(q) = 0 \\
			\vdots \\
			m_{t}(q) = 0
		\end{cases}
	\end{equation}

	o equivalentemente la controimmagine di 0 della funzione continua
	
	\map{f}
		{U}{\R^{t}}
		{q}{(m_{1}(q),\dots,m_{t}(q))}

	allora l'insieme $ U \setminus W = f^{-1}(0) $ è chiuso e dunque il suo complementare $ W $ è aperto.
\end{proof}

\begin{remark}
	Preso un intorno $ I $ di un punto $ p $, valgono le seguenti implicazioni
	
	\begin{gather}
		\rank(F(p)) = k \text{ massimale} \implies \rank(F(q)) = k \text{ costante}, \; \forall q \in I \\
		\nonumber \\
		\rank(F(p)) = k \text{ costante} \notimplies \rank(F(p)) = k \text{ massimale}, \; \forall q \in I
	\end{gather}
\end{remark}

\begin{corollary}
	Sia $ F : N \to M $ un'immersione in $ p \in N $ ($ F_{*p} $ iniettivo), allora $ F $ è un'immersione in un intorno di $ p $. Analogamente, se $ F $ è una sommersione in $ p $ ($ F_{*p} $ suriettivo), allora $ F $ è una sommersione in un intorno di $ p $. \\
	Equivalentemente, la condizione per un'applicazione di essere un'immersione o una sommersione è una condizione aperta.
\end{corollary}

\begin{proof}
	Essendo il rango dell'applicazione $ F $ massimale in $ p $ per un'immersione (risp. sommersione), esiste un intorno di $ p $ in cui il rango è costantemente uguale al massimo, dunque l'applicazione continua a essere un'immersione (risp. sommersione) anche in questo intorno.
\end{proof}

Combinando questo corollario con il teorema del rango costante si ottengono i teoremi di immersione e sommersione locale.

\begin{theorem}[Immersione locale]\label{thm:loc-imm}
	Sia $ F : N \to M $ un'immersione in $ p \in N $, le cui dimensioni delle varietà rispettano $ n \leqslant m $, allora esistono due carte
	
	\begin{equation}
		\begin{cases}
			(U,\varphi) = (U; x^{1},\dots,x^{n}) \in N, & \varphi(p) = 0 \in \R^{n} \\
			(V,\psi) = (V; y^{1},\dots,y^{m}) \in M, & \psi(F(p)) = 0  \in \R^{m} \\
			F(U) \subseteq V
		\end{cases}
	\end{equation}
	
	tali che l'applicazione possa essere scritta come \textit{immersione canonica} (inclusione)
	
	\map{\psi \circ F \circ \varphi^{-1}}
		{\varphi(U)}{\psi(V)}
		{(r^{1},\dots,r^{n})}{(r^{1},\dots,r^{n},0,\dots,0)}

	dove nell'immagine ci sono $ m-n $ zeri.
\end{theorem}

\begin{theorem}[Sommersione locale]\label{thm:loc-sub}
	Sia $ F : N \to M $ una sommersione in $ p \in N $, le cui dimensioni delle varietà rispettano $ n \geqslant m $, allora esistono due carte
	
	\begin{equation}
		\begin{cases}
			(U,\varphi) = (U; x^{1},\dots,x^{n}) \in N, & \varphi(p) = 0 \in \R^{n} \\
			(V,\psi) = (V; y^{1},\dots,y^{m}) \in M, & \psi(F(p)) = 0 \in \R^{m} \\
			F(U) \subseteq V
		\end{cases}
	\end{equation}
	
	tali che l'applicazione possa essere scritta come \textit{sommersione canonica} (proiezione)
	
	\map{\psi \circ F \circ \varphi^{-1}}
		{\varphi(U)}{\psi(V)}
		{(r^{1},\dots,r^{n})}{(r^{1},\dots,r^{m})}
	
	dove nell'immagine ci sono le prime $ m $ coordinate del punto del dominio.
\end{theorem}

\begin{corollary}\label{cor:sub-open}
	Sia $ F : N \to M $ una sommersione, allora $ F $ è un'applicazione aperta.
\end{corollary}

\begin{proof}
	La sommersione è localmente aperta in quanto la sommersione canonica è un'applicazione aperta, ma un'applicazione localmente aperta è anche aperta. Infatti, se $ U \subset N $ è aperto ed $ F $ è localmente aperta, allora
	
	\begin{equation}
		\forall p \in U, \, \E U_{p} \text{ intorno di } p \mid U_{p} \subset U \, \wedge \, \eval{F}_{U_{p}} \text{ aperta}
	\end{equation}

	ma $ U $ si può scrivere come
	
	\begin{equation}
		U = \bigcup_{p \in U} U_{p}
	\end{equation}

	e applicando $ F $ si ottiene
	
	\begin{align}
		F(U) = F \left( \bigcup_{p \in U} U_{p} \right) %
		= \bigcup_{p \in U} F(U_{p})
	\end{align}

	che è aperto in quanto unione di aperti.
\end{proof}

\begin{corollary}\label{cor:sub-comp-conn-surj}
	Sia $ F : N \to M $ una sommersione tra varietà $ N $ compatta e $ M $ compatta e connessa, allora $ F $ è suriettiva.
\end{corollary}

\begin{proof}
	Per il Corollario \ref{cor:sub-open}, $ F $ è aperta in quanto sommersione mentre, per il lemma dell'applicazione chiusa\footnote{%
		Vedi Lemma \ref{lemma:clos-app}.%
	}, $ F $ è anche chiusa. A questo punto, l'immagine di $ N $ attraverso $ F $ è sia aperta che chiusa ma è sottoinsieme di $ M $ che è compatto ($ F(N) \subseteq M $): l'unico sottoinsieme di un compatto che è sia aperto che chiuso è il compatto stesso, dunque $ F(N) = M $ e perciò $ F $ è suriettiva.
\end{proof}

\begin{corollary}
	Siano $ N $ una varietà differenziabile compatta e $ F : N \to \R^{m} $ un'applicazione liscia, allora deve esistere almeno un punto critico per l'applicazione, i.e. $ \PC_{F} \neq \emptyset $.
\end{corollary}

\begin{proof}
	Se per assurdo $ \PC_{F} = \emptyset $ allora $ N = \PR_{F} $ e quindi $ F $ è una sommersione: per il Corollario \ref{cor:sub-comp-conn-surj}, $ F(N) = \R^{m} $ ($ F $ suriettiva) ma questo è assurdo in quanto $ N $ è compatta e dunque la sua immagine è compatta mentre $ \R^{m} $ non lo è (un'applicazione suriettiva porta compatti in compatti).
\end{proof}

\begin{corollary}\label{cor:imm-sph}
	Non esiste un'immersione dalla sfera $ \S^{n} $ a $ \R^{n} $.
\end{corollary}

\begin{proof}
	Se per assurdo esistesse un'immersione $ F : \S^{n} \to \R^{n} $ allora, siccome le dimensioni di dominio e codominio sono uguali, questa sarebbe anche una sommersione, in contrasto con il Corollario \ref{cor:sub-comp-conn-surj}.
\end{proof}

\begin{remark}
	Il Teorema \ref{thm:preimg}

	\begin{equation}
		\begin{cases}
			F : N \to M \text{ liscia} \\
			c \in \VR_{F} \cap F(N)
		\end{cases} %
		\implies %
		F^{-1}(c) \text{ sottovarietà di } N
	\end{equation}

	segue dal Teorema \ref{thm:preimg-rank}
	
	\begin{equation}
		\begin{cases}
			F : N \to M \text{ liscia} \\
			\rank(F(q)) = k, & \forall q \in W \supset F^{-1}(c)
		\end{cases} %
		\implies %
		F^{-1}(c) \text{ sottovarietà di } N
	\end{equation}
\end{remark}

\begin{proof}
	Consideriamo le varietà differenziabili $ N $ ed $ M $ rispettivamente di dimensione $ n $ ed $ m $. Siccome $ c \in \VR_{F} \cap F(N) $, un punto $ p $ della controimmagine di $ c $ apparterrà a $ \PR_{F} \cap F^{-1}(c) $, i.e.
	
	\begin{equation}
		\begin{cases}
			c \in \VR_{F} \cap F(N) \\
			p \in F^{-1}(c)
		\end{cases} %
		\implies %
		p \in \PR_{F} \cap F^{-1}(c)
	\end{equation}
	
	Siccome $ p $ è un punto regolare, $ F $ è una sommersione in $ p $ ($ n \geqslant m $): essendo la condizione di sommersione in un punto una condizione aperta, esiste un intorno $ U_{p} \subset N $ di $ p $ tale che
	
	\begin{equation}
		\rank(F(q)) = m \qcomma \forall q \in U_{p}
	\end{equation}
	
	Il rango della funzione $ F $ è massimale in quanto F è una sommersione ed è costante nell'aperto $ U_{p} $ in quanto massimale in $ p $ (la condizione di rango massimale è aperta). \\
	A questo punto, al variare di $ p $, abbiamo diversi aperti di cui consideriamo l'unione
	
	\begin{equation}
		W = \bigcup_{p \in \PR_{F} \cap F^{-1}(c)} U_{p}
	\end{equation}
	
	la quale contiene $ F^{-1}(c) $, i.e. $ W \supset F^{-1}(c) $. \\
	Avendo considerato l'unione di aperti in cui il rango è costante (e massimale), questo sarà ancora costante (e massimale) in tutto $ W $, i.e.
	
	\begin{equation}
		\rank(F(q)) = m \qcomma \forall q \in W
	\end{equation}

	il che implica
	
	\begin{equation}
		\rank(F(q)) = m \qcomma \forall q \in F^{-1}(c)
	\end{equation}
	
	in quanto $ F^{-1}(c) \subset W $. Per il teorema della preimmagine di un'applicazione di rango costante, otteniamo dunque che $ F^{-1}(c) $ è una sottovarietà di $ N $. \\	
	Questo dimostra l'implicazione poiché abbiamo mostrato che il teorema della preimmagine di un valore regolare segue dal teorema della preimmagine di un'applicazione di rango costante, in quanto abbiamo usato il secondo teorema per dimostrare il primo.
\end{proof}

\subsection{Immagini di applicazioni lisce}\label{s-sec:img-smooth-app}

Sia un'applicazione liscia tra varietà $ F : N \to M $: ci chiediamo sotto quali ipotesi l'immagine $ F(N) $ della varietà differenziabile $ N $ attraverso $ F $ sia una sottovarietà di $ M $. \\
Definiamo innanzitutto l'\textit{embedding topologico} come un'applicazione $ f : N \to M $ che individua un omeomorfismo tra $ N $ e $ f(N) $, dove quest'ultimo ha la topologia ereditata da $ M $. \\
Supponiamo che $ N \subset M $ sia una sottovarietà di $ M $, allora l'inclusione canonica

\map{i}
	{N}{M}
	{q}{q}

è un'immersione iniettiva e un embedding topologico\footnote{%
	Un omeomorfismo è un'applicazione tra spazi topologici che sia una bigezione continua, invertibile e con inversa continua, questo implica che l'inclusione $ i : N \to M $ è iniettiva.%
}. \\
Da questa inclusione, possiamo definire l'\textit{embedding differenziale}\footnote{%
	L'\textit{embedding differenziale} verrà indicato semplicemente come \textit{embedding} o \textit{embedding liscio}, altrimenti verrà esplicitato \textit{embedding topologico} nel caso in cui saremo interessati solo agli aspetti topologici.%
} come un'applicazione liscia $ F : N \to M $ che soddisfa le seguenti due condizioni:

\begin{itemize}
	\item $ F $ è un'immersione
	
	\item $ F $ è un embedding topologico
\end{itemize}

il che implica che un embedding è iniettivo. Ad esempio, come visto sopra, l'inclusione canonica è un embedding.

\begin{remark}
	Come l'embedding topologico induce un omeomorfismo tra il dominio e l'immagine del dominio tramite l'applicazione, l'embedding differenziale induce un diffeomorfismo tra il dominio e l'immagine del dominio tramite l'applicazione.
\end{remark}

\begin{theorem}\label{thm:emb-subvar}
	Sia $ F : N \to M $ un embedding differenziabile, allora $ F(N) $ è una sottovarietà di $ M $ con $ \dim(F(N)) = \dim(N) $, dove $ F(N) \stackrel{diff.}{\simeq} N $.
\end{theorem}

\begin{proof}
	Sfruttiamo il teorema di immersione locale\footnote{%
		Vedi Teorema \ref{thm:loc-imm}.%
	}: essendo $ F : N \to M $ un'immersione ($ n \leqslant m $), dato $ p \in N $, esistono due carte
	
	\begin{equation}
		\begin{cases}
			(U,\varphi) = (U; x^{1},\dots,x^{n}) \in N, & \varphi(p) = 0 \\
			(V,\psi) = (V; y^{1},\dots,y^{m}) \in M, & \psi(F(p)) = 0 \\
			F(U) \subseteq V
		\end{cases}
	\end{equation}
	
	tali che
	
	\map{\psi \circ F \circ \varphi^{-1}}
		{\varphi(U)}{\psi(V)}
		{(r^{1},\dots,r^{n})}{(r^{1},\dots,r^{n},0,\dots,0)}

	dove $ \varphi(U) \subset \R^{n} $, $ \psi(V) \subset \R^{m} $ e nell'immagine ci sono $ m-n $ zeri. \\
	Applicando la funzione a $ \varphi(U) $ otteniamo
	
	\begin{align}
		\begin{split}
			(\psi \circ F \circ \varphi^{-1})(\varphi(U)) &= \{ \psi(q) \in \psi(V) \mid r^{n+1}(\psi(q)) = \cdots = r^{m}(\psi(q)) = 0 \} \\
			\psi(F(U)) &= \{ \psi(q) \in \psi(V) \mid y^{n+1}(q) = \cdots = y^{m}(q) = 0 \} \\
			F(U) &= \{ q \in V \mid y^{n+1}(q) = \cdots = y^{m}(q) = 0 \}
		\end{split}
	\end{align}

	Nel caso in cui $ F(N) \cap V = F(U) $, avremmo che $ (V,\psi) $ è una carta adattata di $ M $ intorno a $ F(p) $ relativamente a $ F(N) $: in generale però questa uguaglianza è falsa in quanto $ F(U) \subsetneq F(N) \cap V $ e questa condizione non è sufficiente perché $ V $ sia una carta adattata. \\
	Ad esempio, considerando lo schema per un'applicazione $ F : N \to M $
	
	\img{0.9}{img38}
	
	vediamo che, siccome un'estremità di $ F(N) $ si avvicina indefinitamente a $ F(p) $ senza toccarlo, $ F(U) \subset F(N) \cap V $ ma non è possibile trovare un aperto $ V \in \R^{2} $ tale che $ F(U) = F(N) \cap V $. \\
	Sfruttando ora il fatto che $ F $ sia anche un'embedding topologico, sappiamo che $ F(U) $ è un aperto di $ F(N) \subset M $, cioè esiste un aperto $ V' \subset M $ tale che $ F(U) = F(N) \cap V' $ (questo perché la topologia di $ F(N) $ è indotta da $ M $). Intersecando quest'ultima uguaglianza con l'aperto $ V $ della carta e chiamando $ W \doteq V \cap V' $, otteniamo
	
	\begin{equation}
		F(U) \cap V = F(N) \cap V' \cap V = F(N) \cap W
	\end{equation}

	siccome $ F(U) \subseteq V $ abbiamo che $ F(U) \cap V = F(U) $, perciò
	
	\begin{equation}
		F(N) \cap W = \{ q \in V \mid y^{n+1}(q) = \cdots = y^{m}(q) = 0 \}
	\end{equation}

	dalla definizione di $ F(U) $ sopra e dunque $ \left( W,\eval{\psi}_{W} \right) $ è una carta adattata di $ M $ intorno a $ F(p) $ relativamente a $ F(N) $ (restringendo l'aperto da $ V $ e $ W $, si ottiene la definizione per le sottovarietà). A questo punto, dato che il punto $ p $ è arbitrario, $ F(N) $ è una sottovarietà di $ M $ di dimensione $ n $.
\end{proof}

\begin{remark}
	Se $ F : N \to M $ è un'immersione iniettiva, questo non implica che $ F(N) $ sia una sottovarietà di $ M $. Chiameremo dunque $ F(N) \subset M $ \textit{sottovarietà immersa}\footnote{%
		Una sottovarietà immersa non è necessariamente una sottovarietà.%
	}, la quale è diffeomorfa a $ N $ ma non ha la struttura differenziale ereditata da $ M $.
\end{remark}

\begin{remark}
	Se $ F : N \to M $ è un'immersione iniettiva ed $ N $ è compatta allora $ F $ è un embedding. Per il lemma dell'applicazione chiusa\footnote{%
		Vedi Lemma \ref{lemma:clos-app}.%
	}, essendo $ F $ continua, $ N $ compatta ed $ M $ di Hausdorff (in quanto varietà differenziabile), segue che $ F : N \to F(N) $ è un omeomorfismo e quindi un embedding topologico.
\end{remark}

\begin{theorem}[Whitney]
	Se $ N $ è una varietà differenziabile di dimensione $ n $ allora esiste $ F : N \to \R^{2n} $ embedding differenziabile. \\
	Questo significa che $ F(N) $, identificabile con $ N $ tramite il diffeomorfismo $ F $, è una sottovarietà di $ \R^{2n} $.
\end{theorem}

\begin{theorem}[Whitney (forma debole)]
	Se $ N $ è una varietà differenziabile di dimensione $ n>1 $ allora esiste $ F : N \to \R^{2n-1} $ immersione iniettiva.
\end{theorem}

Nel caso in cui $ n=1 $, non esiste un'immersione iniettiva tra $ N $ e $ \R $, e.g.\footnote{%
	Vedi Corollario \ref{cor:imm-sph}.%
} tra $ \S^{1} $ e $ \R $.

\subsubsection{\textit{Esempi}}

Per altri esempi, vedi Esercizi \ref{exer2-19}, \ref{exer2-20}, \ref{exer2-23}, \ref{exer2-24}.

\paragraph{1)}

Nell'esempio della dimostrazione del Teorema \ref{thm:emb-subvar}, $ F(N) $ non può essere una sottovarietà di $ \R^{2} $ perché non è localmente euclideo. Questo perché possiede la topologia indotta da $ \R^{2} $ in quanto $ F $ è un embedding topologico: un qualunque intorno di $ F(p) $ non è omeomorfo a $ \R^{n} $ in quanto il primo, togliendo il punto $ F(p) $, è formato da tre componenti connesse mentre il secondo può essere o formato da due componenti connesse (per $ \R $) oppure connesso (per $ n > 1 $).

\paragraph{2) Cuspide cubica}

Sia l'applicazione della cuspide cubica $ y^{2} = x^{3} $

\map{F}
	{\R}{\R^{2}}
	{t}{(t^{2},t^{3})}

Questa è iniettiva perché per due punti dell'immagine uguali otteniamo che

\begin{equation}
	(t_{1}^{2},t_{1}^{3}) = (t_{2}^{2},t_{2}^{3}) %
	\implies %
	\begin{cases}
		(t_{1})^{2} = (t_{2})^{2} \\
		(t_{1})^{3} = (t_{2})^{3}
	\end{cases} %
	\implies %
	\begin{cases}
		t_{1} = \pm t_{2} \\
		t_{1} = t_{2}
	\end{cases} %
	\implies %
	t_{1} = t_{2}
\end{equation}

Per controllare se sia un'immersione, dobbiamo controllare che il differenziale sia iniettivo e dunque che lo jacobiano abbia rango massimale (unitario in questo caso) ovunque nel dominio:

\begin{equation}
	J(F)(t) = \bmqty{ 2t \\ 3t^{2} }
\end{equation}

dunque non è un'immersione perché nell'origine ($ t = 0 $) il differenziale non è iniettivo. \\
Non è nemmeno un embedding in quanto la cuspide cubica non è una sottovarietà\footnote{%
	Con un abuso di nomenclatura, si potrebbe dire che la cuspide cubica è una "sottovarietà topologica" di $ \R^{2} $ in quanto totalmente omeomorfa (anche nell'origine) a $ \R $ e questo rende $ F $ un embedding topologico.%
} di $ \R^{2} $.

\sbs{0.5}{%
			\paragraph{3) Cubica nodale}
			
			Sia l'applicazione della cubica nodale $ y^{2} = x^{3} + x^{2} $
			
			\map{F}
				{\R}{\R^{2}}
				{t}{(t^{2}-1,t^{3}-t)}
			}
	{0.5}{%
			\img{1}{img39}
			}

Questa non è iniettiva perché per esempio $ F(\pm 1) = (0,0) $, dunque non è nemmeno un embedding topologico. \\
Per controllare se sia un'immersione, cerchiamo un punto che annulla contemporaneamente le derivate di $ F $ rispetto a $ t $:

\begin{equation}
	\dot{F}(t) = (2t, 3 t^{2} - 1) \neq (0,0) \qcomma \forall t \in \R
\end{equation}

dunque $ F $ è un'immersione. \\
Anche la cuspide cubica non è localmente euclidea nell'origine, dunque non può essere una sottovarietà di $ \R^{2} $.

\paragraph{4) Lemniscata di Bernoulli}
				
Sia l'applicazione della lemniscata di Bernoulli $ x^{2} = 4 y^{2} (1-y^{2}) $ (quartica)
				
\map{F}
	{(-\pi,\pi)}{\R^{2}}
	{t}{(\sin(2t),\sin(t))}

\img{0.7}{img40}

Questa è iniettiva perché per due punti dell'immagine uguali otteniamo che

\begin{equation}
	(\sin(2t_{1}), \sin(t_{1})) = (\sin(2t_{2}), \sin(t_{2})) %
	\implies %
	\begin{cases}
		\sin(t_{1}) = \sin(t_{2}) = 0 \\
		\sin(t_{1}) = \sin(t_{2}) \neq 0
	\end{cases}
\end{equation}

Nel primo caso

\begin{equation}
	\begin{cases}
		\sin(t_{1}) = \sin(t_{2}) = 0 \\
		t \in (-\pi,\pi)
	\end{cases} %
	\implies %
	t_{1} = t_{2} = 0
\end{equation}

mentre nel secondo

\begin{align}
	\begin{split}
		\sin(2t_{1}) &= \sin(2t_{2}) \\
		2 \sin(t_{1}) \cos(t_{1}) &= 2 \sin(t_{2}) \cos(t_{2}) \\
		\cos(t_{1}) &= \cos(t_{2})
	\end{split}		
\end{align}

perciò

\begin{equation}
	\begin{cases}
		\sin(t_{1}) = \sin(t_{2}) \\
		\cos(t_{1}) = \cos(t_{2})
	\end{cases} %
	\implies %
	\begin{cases}
		t_{1} = t_{2} + 2 k \pi, & k \in \Z \\
		t \in (-\pi,\pi)
	\end{cases} %
	\implies %
	t_{1} = t_{2}
\end{equation}

in quanto qualunque valore di $ k $ diverso da 0 esce dall'intervallo del dominio. \\
Questa applicazione è un'immersione in quanto

\begin{equation}
	\dot{F}(t) = (2 \cos(2t), \cos(t)) \neq (0,0) \qcomma \forall t \in \R
\end{equation}

e dunque non sono presenti punti critici, i.e. $ \PC_{F} = \emptyset $. \\
Nonostante sia un'immersione iniettiva, non è un embedding (né topologico né differenziabile) in quanto $ F((-\pi,\pi)) $ non è localmente euclidea nell'origine.

\paragraph{5) Toro}\label{example:embed-torus}

Sia l'applicazione

\map{F}
	{\R}{\T^{2}}
	{t}{(e^{2 \pi i t}, e^{2 \pi i \alpha t})}

dove

\begin{equation}
	\T^{2} = \S^{1} \times \S^{1} \subset \C^{2} = \R^{4}
\end{equation}

e $ \alpha \in \R \setminus \Q $ (i.e. $ \alpha $ irrazionale). \\
Siccome il toro $ \T^{2} $ è incluso in $ \R^{4} $, possiamo riscrivere la funzione come

\map{F}
	{\R}{\R^{4}}
	{t}{(\cos(2 \pi t), \sin(2 \pi t), \cos(2 \pi \alpha t), \sin(2 \pi \alpha t))}

Verifichiamo se $ F $ sia iniettiva tramite la dimostrazione dell'implicazione:

\begin{equation}
	(e^{2 \pi i t_{1}}, e^{2 \pi i \alpha t_{1}}) = (e^{2 \pi i t_{2}}, e^{2 \pi i \alpha t_{2}}) %
	\implies %
	t_{1} = t_{2}
\end{equation}

Possiamo scrivere

\begin{equation}
	\begin{cases}
		e^{2 \pi i t_{1}} = e^{2 \pi i t_{2}} \\
		e^{2 \pi i \alpha t_{1}} = e^{2 \pi i \alpha t_{2}}
	\end{cases} %
	\implies %
	\begin{cases}
		t_{1} - t_{2} = n, & n \in \Z \\
		t_{1} - t_{2} = \alpha m, & m \in \Z
	\end{cases} %
	\implies %
	\alpha m = n %
	\implies %
	\alpha \in \Q
\end{equation}

ma questo non è possibile in quanto abbiamo posto $ \alpha \in \R \setminus \Q $, dunque

\begin{equation}
	n = m = 0 \implies t_{1} = t_{2}
\end{equation}

perciò l'applicazione è iniettiva. \\
Per verificare se sia un'immersione, consideriamo la derivata di $ F $ in $ \R^{4} $

\begin{equation}
	\dot{F}(t) = 2 \pi (- \sin(2 \pi t), \cos(2 \pi t), - \alpha \sin(2 \pi \alpha t), \alpha \cos(2 \pi \alpha t))
\end{equation}

la quale non si annulla per alcun $ t \in \R $, dunque $ F $ è un'immersione. \\
Nonostante $ F $ sia un'immersione iniettiva, questa non è un embedding in quanto la sua immagine non è una sottovarietà di $ \R^{4} $. \\
Dimostriamo questo fatto mediante il seguente lemma:

\begin{lemma}
	Chiamando $ \mathcal{D}(A) $ l'insieme dei punti di accumulazione\footnote{%
		Per un punto di accumulazione $ x_{0} $ di un insieme $ U $, ogni intorno del punto $ x_{0} $ contiene almeno un punto dell'insieme $ U $ diverso da sé stesso.
		
		\begin{equation}
			x_{0} \text{ punto di accumulazione per } U %
			\implies %
			\forall I \ni x_{0}, \, \E x \in U \mid x \in I \subset U \, \wedge \, x \neq x_{0}
		\end{equation} %
	} di un insieme $ A $, possiamo affermare che, per l'applicazione $ F $ considerata in precedenza, vale la seguente relazione:
	
	\begin{equation}
		F(0) = (1,1) \in \mathcal{D}(F(\Z))
	\end{equation}

	Cioè qualunque intorno di $ (1,1) \in \T^{2} $ include almeno un punto di $ F(\Z) $ o equivalentemente
	
	\begin{equation}
		\forall \varepsilon > 0, \, \E k \in \Z \setminus \{0\} \mid \left| (e^{2 \pi i k}, e^{2 \pi i \alpha k}) - (1,1) \right| < \varepsilon
	\end{equation}
\end{lemma}

\begin{proof}
	Possiamo riscrivere la condizione del lemma nel seguente modo:
	
	\begin{align}
		\begin{split}
			\left| (e^{2 \pi i k}, e^{2 \pi i \alpha k}) - (1,1) \right| &< \varepsilon \\
			\left| (1, e^{2 \pi i \alpha k}) - (1,1) \right| &< \varepsilon \\
			\left| e^{2 \pi i \alpha k} - 1 \right| &< \varepsilon
		\end{split}
	\end{align}

	Sappiamo che $ \S^{1} $ è compatto quindi la successione di punti del cerchio
	
	\begin{equation}
		z_{n} = e^{2 \pi i \alpha k} \qcomma n \in \Z
	\end{equation}

	ammette una sottosuccessione convergente: possiamo supporre che (a meno di cambiare semplicemente gli indici della successione) la successione stessa converga a un punto $ z_{n} \to z_{0} \in \S^{1} $ e ciò significa che esistono due numeri (diversi) $ n_{1},n_{2} \in \Z $ tali che si possa scrivere
	
	\begin{equation}
		\left| e^{2 \pi i \alpha n_{1}} - 1 \right| < \dfrac{\varepsilon}{2} %
		\quad \wedge \quad %
		\left| e^{2 \pi i \alpha n_{2}} - 1 \right| < \dfrac{\varepsilon}{2}
	\end{equation}

	ponendo dunque $ k = n_{1} - n_{2} \neq 0 $ e sfruttando la disuguaglianza triangolare
	
	\begin{align}
		\begin{split}
			\left| e^{2 \pi i \alpha k} - 1 \right| &= \left| e^{2 \pi i \alpha (n_{1} - n_{2})} - 1 \right| \\
			&= \left| e^{ -2 \pi i \alpha n_{2}} \left( e^{2 \pi i \alpha n_{1}} - e^{2 \pi i \alpha n_{2}} \right) \right| \\
			&= \cancelto{1}{ \left| e^{ -2 \pi i \alpha n_{2}} \right| } \left| e^{2 \pi i \alpha n_{1}} - e^{2 \pi i \alpha n_{2}} \right| \\
			&= \left| e^{2 \pi i \alpha n_{1}} - e^{2 \pi i \alpha n_{2}} \right| \\
			&= \left| e^{2 \pi i \alpha n_{1}} - e^{2 \pi i \alpha n_{2}} +1 -1 \right| \\
			&\leqslant \left| e^{2 \pi i \alpha n_{1}} - 1 \right| + \left| e^{2 \pi i \alpha n_{2}} - 1 \right| \\
			&< \dfrac{\varepsilon}{2} + \dfrac{\varepsilon}{2} \\
			&= \varepsilon
		\end{split}
	\end{align}

	dunque $ \left| e^{2 \pi i \alpha k} - 1 \right| < \varepsilon $.
\end{proof}

Supponiamo per assurdo che $ F $ sia un embedding: l'applicazione $ f : \R \to F(\R) $, dove $ F(\R) $ ha la topologia indotta dal toro $ \T^{2} $, è un diffeomorfismo e dunque anche un omeomorfismo. Per le proprietà degli omeomorfismi, esiste una bigezione tra i punti di accumulazione del dominio e quelli del codominio, i.e. $ \mathcal{D}(\Z) \to \mathcal{D}(F(\Z)) $, però $ \mathcal{D}(\Z) = \emptyset $ in quanto i numeri interi sono isolati in $ \R $ mentre $ \mathcal{D}(F(\Z)) \ni (1,1) $. Questa contraddizione porta a dire che $ F $ non è un embedding e quindi $ F(\R) $ non è una sottovarietà di $ \T^{2} $.

\begin{remark}
	Si dimostra che $ F(\R) $ è denso in $ \T^{2} $, cioè il toro è completamente ricoperto dall'immagine della funzione o equivalentemente ogni intorno di ogni punto del toro ha al suo interno un punto dell'immagine (per dimostrare questo, viene utilizzata la teoria dei numeri).
\end{remark}

\subsubsection{Immagini di applicazioni lisce contenute in varietà}

\begin{theorem}\label{thm:smooth-restriction-subman}
	Siano $ F : N \to M $ un'applicazione liscia tra varietà e $ S \subset M $ una sottovarietà di $ M $, e supponiamo che $ F(N) \subseteq S $, allora la restrizione del codominio della funzione al sottoinsieme $ S $, i.e. $ \tilde{F} : N \to S $, è ancora un'applicazione liscia. \\
	Possiamo considerare questa restrizione del codominio grazie alla condizione $ F(N) \subseteq S $, la quale rende le due applicazioni identiche dal punto di vista dell'immagine risultante, i.e.
	
	\begin{equation}
		\tilde{F}(p) = F(p) \qcomma \forall p \in N
	\end{equation}
\end{theorem}

\begin{proof}
	Sappiamo che $ \tilde{F} $ è continua in quanto $ S $ è un sottospazio topologico di $ M $ in quanto sottovarietà. \\
	Siano un punto $ p \in N $ e due carte
	
	\begin{equation}
		\begin{cases}
			(U,\varphi) = (U; x^{1},\dots,x^{n}) \in N, & \varphi(p) = 0 \\
			(V,\psi) = (V; y^{1},\dots,y^{m}) \in M, & \psi(F(p)) = 0 \\
			F(U) \subseteq V
		\end{cases}
	\end{equation}
	
	tale che $ (V,\psi) $ sia una carta adattata intorno a $ F(p) $ relativamente a $ S $. \\
	Poniamo $ \dim(S) = s $, possiamo dunque scrivere la condizione per le sottovarietà
	
	\begin{equation}
		V \cap S = \{ q \in V \mid y^{s+1}(q) = \cdots = y^{m}(q) = 0 \}
	\end{equation}
	
	e la restrizione dell'applicazione della carta adattata
	
	\map{\eval{\psi}_{V \cap S} \doteq \psi_{S}}
		{V \cap S}{\psi(V \cap S) \subset \R^{s}}
		{q}{(y^{1}(q),\dots,y^{s}(q))}

	Osserviamo che, dalle carte, $ F(U) \subseteq V $ e che, dalle ipotesi, $ F(N) \subseteq S $ da cui $ F(U) \subseteq S $ perciò $ F(U) \subseteq V \cap S $. \\
	Consideriamo ora la composizione ristretta a $ U $
	
	\map{\psi \circ F}
		{U}{\psi(V) \subset \R^{m}}
		{p}{\left( y^{1}(F(p)), \dots, y^{s}(F(p)), 0,\dots,0 \right)}

	in cui sono presenti $ m - s $ zeri in quanto $ F(U) \subseteq V \cap S $, e poi la restrizione di questa a $ S $
	
	\map{\psi_{S} \circ \tilde{F}}
		{U}{\psi_{S}(V \cap S) \subset \R^{s}}
		{p}{\left( y^{1}(F(p)), \dots, y^{s}(F(p)) \right)}

	Siccome le $ y^{i} $, $ F $ e $ \psi_{S} $ sono tutte applicazioni lisce rispetto a $ p $, anche la composizione $ \psi_{S} \circ \tilde{F} $ sarà liscia, il che implica che la restrizione $ \tilde{F} $ è liscia.
\end{proof}

\subsubsection{\textit{Esempi}}

\paragraph{1) Lemniscata di Bernoulli}

Consideriamo le seguenti due varianti della lemniscata di Bernoulli tramite le applicazioni lisce

\sbs{0.5}{%
			\map{F}
				{(-\pi,\pi)}{\R^{2}}
				{t}{(\sin(2t), \sin(t))}
			}
	{0.5}{%
			\map{G}
				{(-\pi,\pi)}{\R^{2}}
				{t}{(\sin(2t), -\sin(t))}
			}

e chiamiamo le loro immagini

\begin{gather}
	F((-\pi,\pi)) \doteq \mathfrak{8}_{F} \\
	G((-\pi,\pi)) \doteq \mathfrak{8}_{G}
\end{gather}

le quali sono varietà differenziabili la cui struttura è indotta rispettivamente da $ F $ e da $ G $ (bigezioni), ma non sono sottovarietà di $ \R^{2} $. Le funzioni che hanno lo stesso dominio e come codominio le immagini tramite $ F $ e $ G $ del dominio sono diffeomorfismi, i.e.

\sbs{0.5}{%
			\map{\tilde{F}}
				{(-\pi,\pi)}{\mathfrak{8}_{F}}
				{t}{(\sin(2t), \sin(t))}
			}
	{0.5}{%
			\map{\tilde{G}}
				{(-\pi,\pi)}{\mathfrak{8}_{G}}
				{t}{(\sin(2t), -\sin(t))}
			}
		
		\img{0.8}{img41}

Consideriamo ora l'applicazione $ H $ che mappa i punti come $ G $ ma nell'immagine di $ F $, i.e.

\map{\tilde{H}}
	{(-\pi,\pi)}{\mathfrak{8}_{F}}
	{t}{(\sin(2t), -\sin(t))}

il cui codominio possiede la struttura differenziabile indotta da $ G $. Nonostante la funzione $ G $ sia liscia e l'immagine di $ H $ sia contenuta in $ \R^{2} $, $ \mathfrak{8}_{F} $ non è una sottovarietà di $ \R^{2} $ e $ H $ non è nemmeno continua (perciò non liscia): per dimostrarlo possiamo considerare un aperto nel codominio e dimostrare che la sua controimmagine non è aperta nel dominio. Consideriamo dunque il segmento aperto $ U = (a,b) \in \mathfrak{8}_{F} $: la sua controimmagine tramite $ F $ è aperta in $ (-\pi,\pi) $, mentre non è aperta se si prende la sua controimmagine tramite $ H $, i.e.

\begin{equation}
	U = (a,b) \subset \mathfrak{8}_{F} %
	\rightarrow %
	\begin{cases}
		F^{-1}(U) = (F^{-1}(a),F^{-1}(b)), & \text{aperta} \\
		H^{-1}(U) = (-\pi,H^{-1}(a)) \cup \{0\} \cup (H^{-1}(b),\pi), & \text{non aperta}
	\end{cases}
\end{equation}

Il fatto che immagine e preimmagine non siano entrambe aperte rende $ H $ non continua.

\img{0.8}{img42}

Questo esempio mostra come, presa una funzione liscia tra varietà differenziabili, la restrizione del codominio da una varietà a un sottoinsieme (varietà) di questa varietà non è ancora una funzione liscia, a meno che il sottoinsieme della varietà non sia una sottovarietà di questa (quindi con la relativa struttura differenziale ereditata): in questo caso, restringere il codominio $ \R^{2} $ della funzione $ G $ alla varietà $ \mathfrak{8}_{F} \subset \R^{2} $ (la quale non è una sottovarietà di $ \R^{2} $) non rende la funzione $ H $ nemmeno continua. Quest'ultimo risultato può essere derivato anche dal fatto che la lemniscata di Bernoulli non sia un sottospazio topologico di $ \R^{2} $.

\paragraph{2) Gruppo lineare speciale}\label{example:slnr-smooth}

Sia l'insieme

\begin{equation}
	SL_{n}(\R) = \{ A \in GL_{n}(\R) \mid \det(A) = 1 \} \subset GL_{n}(\R)
\end{equation}

il quale è una sottovarietà di $ GL_{n}(\R) $. \\
Consideriamo l'applicazione del prodotto tra matrici di $ SL_{n}(\R) $

\map{\mu}
	{SL_{n}(\R) \times SL_{n}(\R)}{SL_{n}(\R)}
	{(A,B)}{A B}

e verifichiamo che sia liscia. \\
La moltiplicazione tra matrici è liscia in $ GL_{n}(\R) $, la quale ha una carta globale con l'omeomorfismo verso $ \R^{n^{2}} $, rendendo dunque il prodotto tra matrici una semplice concatenazione di somme e prodotti tra numeri reali. Questo però non implica automaticamente che la moltiplicazione tra matrici in $ SL_{n}(\R) $ sia liscia, a meno che non si utilizzi la proprietà di quest'ultima di essere una sottovarietà di $ GL_{n}(\R) $. \\
Sia l'applicazione liscia del prodotto tra matrici di $ GL_{n}(\R) $

\map{F}
	{GL_{n}(\R) \times GL_{n}(\R)}{GL_{n}(\R)}
	{(A,B)}{A B}

se restringiamo il dominio a $ SL_{n}(\R) \times SL_{n}(\R) $, otteniamo l'applicazione

\map{G}
	{SL_{n}(\R) \times SL_{n}(\R)}{GL_{n}(\R)}
	{(A,B)}{A B}

definita come $ G \doteq F \circ (i \times i) $ dove $ i : SL_{n}(\R) \to GL_{n}(\R) $ indica l'inclusione naturale: essendo $ F $ e $ i $ lisce (e dunque\footnote{%
	Vedi Esercizio \ref{exer2-12}.%
} anche $ i \times i $), allora anche $ G $ è liscia. \\
Siccome la moltiplicazione di due matrici con determinante unitario è ancora una matrice con determinante unitario, abbiamo che $ G(SL_{n}(\R) \times SL_{n}(\R)) \subset SL_{n}(\R) $ e quindi possiamo restringere il codominio, ottenendo la funzione

\map{\mu}
	{SL_{n}(\R) \times SL_{n}(\R)}{SL_{n}(\R)}
	{(A,B)}{A B}

la quale è ancora liscia per il Teorema \ref{thm:smooth-restriction-subman}.

\section{Fibrato tangente e i campi di vettori}

\subsection{Fibrato tangente}

Sia $ M $ una varietà differenziabile di dimensione $ n $. Il \textit{fibrato tangente a} $ M $ è definito come l'unione disgiunta di tutti gli spazi tangenti ai punti di $ M $, i.e.

\begin{equation}
	T(M) \doteq \bigsqcup_{p \in M} T_{p}(M)
\end{equation}

La proiezione naturale è definita come

\map{\pi}
	{T(M)}{M}
	{v}{p}

dove notazioni alternative per $ v \in T_{p}(M) $ sono $ (p,v) $ (per evidenziare il collegamento con il punto $ p \in M $) e $ X_{p} $. \\
Costruiamo ora una struttura differenziabile sul fibrato tangente: questa ci permetterà di dimostrare il fibrato tangente sia una varietà differenziabile e che la proiezione $ \pi : T(M) \to M $ sia liscia.

\subsubsection{Struttura topologica sul fibrato tangente}

Sia $ T(U) $ il fibrato tangente a un aperto coordinato $ U \subset M $, i.e. l'aperto di una carta definita come

\begin{equation}
	(U,\varphi) = (U; x^{1},\dots,x^{n}) \in M \qcomma \varphi(p) = 0
\end{equation}

dunque

\begin{equation}
	T(U) = \bigsqcup_{p \in U} T_{p}(U) = \bigsqcup_{p \in U} T_{p}(M)
\end{equation}

questo perché possiamo scrivere

\begin{equation}
	T_{p}(U) = \der_{p}(C_{p}^{\infty}(U)) = \der_{p}(C_{p}^{\infty}(M)) = T_{p}(M)
\end{equation}

in quanto il concetto di germe di funzione è locale. \\
Esiste una bigezione naturale $ \tilde{\varphi} $ tra $ T(U) $ e $ \R^{2n} $: prendendo $ (p,v) \in T_{p}(M) $ e sapendo che le derivate parziali rispetto a $ x^{i} $ in $ p $ generano lo spazio tangente, i.e.

\begin{equation}
	\ev{ \eval{ \pdv{x^{i}} }_{p} } = T_{p}(M)
\end{equation}

possiamo scrivere

\begin{equation}
	v = \sum_{i=1}^{n} c^{i}(v) \eval{ \pdv{x^{i}} }_{p}
\end{equation}

dove $ c(v) = (c^{1}(v),\dots,c^{n}(v)) \in \R^{n} $. \\
A questo punto, la bigezione può essere scritta come

\map{\tilde{\varphi}}
	{T(U)}{\varphi(U) \times \R^{n} = \R^{2n}}
	{(p,v)}{(\varphi(p),c(v)) \\
		&\mapsto (x^{1}(p),\dots,x^{n}(p),c^{1}(v),\dots,c^{n}(v))
		}

dove $ \varphi(U) \subset \R^{n} $. Per poter scrivere l'immagine in funzione di solo $ v $, consideriamo il differenziale

\map{\varphi_{*} = \varphi_{*p}}
	{T_{p}(U)}{T_{\varphi(p)}(\varphi(U)) =  T_{\varphi(p)}(\R^{n})}
	{\eval{ \pdv{x^{i}} }_{p}}{\eval{ \pdv{r^{i}} }_{\varphi(p)}}

che applicato a $ v $

\begin{equation}
	\varphi_{*}(v) = \varphi_{*} \left( \sum_{i=1}^{n} c^{i}(v) \eval{ \pdv{x^{i}} }_{p} \right) %
	= \sum_{i=1}^{n} c^{i}(v) \, \varphi_{*} \left( \eval{ \pdv{x^{i}} }_{p} \right) %
	= \sum_{i=1}^{n} c^{i}(v) \eval{ \pdv{r^{i}} }_{\varphi(p)}
\end{equation}

il quale risultato può essere identificato con $ c(v) = (c^{1}(v),\dots,c^{n}(v)) \in \R^{n} $, i.e. $ \varphi_{*}(v) = c(v) $. \\
Considerando la proiezione $ \pi(v) = p $, si può riscrivere $ \tilde{\varphi} $ come

\begin{equation}
	\tilde{\varphi} = (\varphi \circ \pi) \times \varphi_{*}
\end{equation}

perciò

\map{\tilde{\varphi}}
	{T(U)}{\varphi(U) \times \R^{n} = \R^{2n}}
	{v}{(\varphi \circ \pi,\varphi_{*})(v) \\
		&\mapsto (\varphi(\pi(v)),\varphi_{*}(v)) \\
		&\mapsto (\varphi(p), c(v))
		}

Questa applicazione è invertibile (essendo una bigezione) e la sua inversa è

\map{\tilde{\varphi}^{-1}}
	{\varphi(U) \times \R^{n}}{T(U)}
	{(\varphi(p),c(v))}{v = \sum_{i=1}^{n} c^{i}(v) \eval{ \pdv{x^{i}} }_{p}}

Definiamo dunque la topologia su $ T(U) $ (indotta dall'applicazione $ \tilde{\varphi} $) come

\begin{equation}
	A \subset T(U) \text{ aperto} \iff \tilde{\varphi}(A) \text{ aperto in } \varphi(U) \times \R^{n}
\end{equation}

Da queste considerazioni, possiamo dunque trarre le seguenti conseguenze:

\begin{itemize}
	\item $ \tilde{\varphi} : T(U) \to \varphi(U) \times \R^{n} $ è un omeomorfismo;
	
	\item Preso $ V \subset U $, la topologia indotta da $ T(U) $ su $ T(V) $, dove $ T(V) \subset T(U) $, è la stessa della topologia indotta su $ T(V) $ dall'applicazione $ \eval{\tilde{\varphi}}_{T(V)} : T(V) \to \varphi(V) \times \R^{n} $.
\end{itemize}

Per costruire una topologia su $ T(M) $ consideriamo dunque l'insieme

\begin{equation}
	\B_{T(M)} = \bigcup_{\alpha} \{ A \subset T(U_{\alpha}) \text{ aperti} \}
\end{equation}

cioè l'unione di tutti gli aperti $ A $ contenuti nei rispettivi $ T(U_{\alpha}) $ al variare di $ \alpha $, dove gli $ U_{\alpha} $ sono aperti coordinati dell'atlante massimale $ \{ (U_{\alpha},\varphi_{\alpha}) \} \in M $.

\begin{definition}
	L'insieme $ \B_{T(M)} $ è una base per una topologia su $ T(M) $.
\end{definition}

\begin{proof}
	Per la dimostrazione è necessario che
	
	\begin{itemize}
		\item $ \B_{T(M)} $ sia un ricoprimento di $ T(M) $;
		
		\item l'intersezione di due elementi di $ \B_{T(M)} $ si possa scrivere come unione di elementi di $ \B_{T(M)} $.
	\end{itemize}

	Per definizione $ T(U_{\alpha}) \in \B_{T(M)} $ in quanto aperto; inoltre vale la seguente implicazione
	
	\begin{equation}
		\bigcup_{\alpha} U_{\alpha} = M %
		\implies %
		\bigcup_{\alpha} T(U_{\alpha}) = T(M)
	\end{equation}

	perciò $ \B_{T(M)} $ è un ricoprimento di $ T(M) $. \\
	Siano due aperti $ A \subset T(U) $ e $ B \subset T(V) $, i.e. $ A,B \in \B_{T(M)} $, con $ U $ e $ V $ due aperti coordinati di $ M $. Dimostriamo che
	
	\begin{equation}
		T(U) \cap T(V) = T(U \cap V)
	\end{equation}

	tramite
	
	\begin{align}
		\begin{split}
			T(U) \cap T(V) &= \left( \bigsqcup_{p \in U} T_{p}(U) \right) \cap \left( \bigsqcup_{q \in V} T_{q}(V) \right) \\
			&= \left( \bigsqcup_{p \in U} T_{p}(M) \right) \cap \left( \bigsqcup_{q \in V} T_{q}(M) \right) \\
			&= \bigsqcup_{r \in U \cap V} T_{r}(M) \\
			&= \bigsqcup_{r \in U \cap V} T_{r}(U \cap V) \\
			&= T(U \cap V)
		\end{split}
	\end{align}
	
	da cui ricaviamo che
	
	\begin{equation}
		A \cap B \subset T(U) \cap T(V) %
		\implies %
		A \cap B \subset T(U \cap V)
	\end{equation}

	dove $ U \cap V $ è ancora un aperto coordinato della carta $ \left( U \cap V,\eval{\varphi}_{U \cap V} \right) \in M $. Dobbiamo verificare ora che $ A \cap B \subset T(U \cap V) $ sia aperto in $ T(U \cap V) $ (i.e. che l'intersezione di due elementi della base si possa scrivere come unione di elementi della base) perché $ \B_{T(M)} $ sia effettivamente una base: osserviamo che
	
	\begin{equation}
		A \cap B = (A \cap B) \cap T(U \cap V) %
		= (A \cap T(U \cap V)) \cap (B \cap T(U \cap V))
	\end{equation}

	perciò, perché $ A \cap B $ sia aperto in $ T(U \cap V) $, è sufficiente che $ A \cap T(U \cap V) $ sia aperto in $ T(U \cap V) $ (risp. per $ B $). Questo è vero perché $ T(U \cap V) \subset T(U) $ e ha la topologia indotta da $ T(U) $ (perché $ U \cap V \subset U $) e i suoi aperti sono gli aperti di $ T(U) $ (e.g. $ A $) intersecati con $ T(U \cap V) $ stesso, i.e. $ A \cap T(U \cap V) $; il ragionamento è analogo per $ B $. \\
	Questo dimostra che l'intersezione di due aperti qualsiasi $ A \cap B $ della base può essere scritta come unione di aperti della base stessa, in quanto $ A \cap T(U \cap V) $ e $ B \cap T(U \cap V) $ sono aperti di $ T(U \cap V) $, dunque la loro intersezione può essere scritta come un aperto per la topologia di $ T(U \cap V) $ ereditata da $ T(U) $.
\end{proof}

A questo punto $ T(M) $ è uno spazio topologico con la topologia che ha $ \B_{T(M)} $ come base.\footnote{%
	Questa topologia per $ T(M) $ è l'unica possibile.%
} \\
Dobbiamo far vedere ora che $ T(M) $ è una varietà topologica e per farlo dobbiamo mostrare che sia $ N_{2} $, $ T_{2} $ e localmente euclideo.

\paragraph{Localmente euclideo}

\begin{definition}
	Sia $ \{(U_{\alpha},\varphi_{\alpha})\} $ un atlante topologico per $ M $, allora $ \{(T(U_{\alpha}),\tilde{\varphi}_{\alpha})\} $ è un atlante topologico per $ T(M) $.
\end{definition}

\begin{proof}
	L'insieme $ \{(T(U_{\alpha}),\tilde{\varphi}_{\alpha})\} $ è un atlante topologico per $ T(M) $ in quanto i $ T(U_{\alpha}) $ sono un ricoprimento di $ T(M) $, i.e.
	
	\begin{equation}
		\bigcup_{\alpha} T(U_{\alpha}) = T(M)
	\end{equation}
	
	e le applicazioni
	
	\begin{equation}
		\tilde{\varphi}_{\alpha} : T(U_{\alpha}) \to \varphi_{\alpha}(U_{\alpha}) \times \R^{n}
	\end{equation}
	
	sono omeomorfismi verso aperti di $ \R^{2n} $, dunque
	
	\begin{equation}
		\dim(T(M)) = 2 \dim(M) = 2 n
	\end{equation}
\end{proof}

\paragraph{$ T_{2} $ o di Hausdorff}

Per definizione, un insieme è di Hausdorff se, dati due punti qualunque dell'insieme, esistono sempre due intorni disgiunti che contengono questi due punti. \\
Siano due punti distinti di $ T(M) $

\begin{equation}
	\begin{cases}
		(p,v) \in T_{p}(M) \\
		(q,w) \in T_{q}(M)
	\end{cases}
\end{equation}

Presi $ v \neq w $, abbiamo due casi

\begin{itemize}
	\item se $ p = q $, allora $ T_{p}(M) = T_{q}(M) \subset T(U) $ dove $ U $ è un aperto coordinato di $ (U,\varphi) \ni p=q $, ma $ T(U) \simeq \varphi(U) \times \R^{n} $ dunque $ T(U) $ eredita la proprietà $ T_{2} $ da $ \varphi(U) \times \R^{n} $ e a sua volta anche $ T_{p}(M) = T_{q}(M) $ la ereditano da $ T(U) $, i.e.
	
	\begin{equation}
		\E A,B \subset T(U) = T(M) \text{ aperti}, \, A \ni (p,v) \wedge B \ni (q,w) \mid A \cap B = \emptyset
	\end{equation}
	
	\item se $ p \neq q $, siccome $ M $ è $ T_{2} $ esisteranno due aperti disgiunti coordinati che contengono $ p $ e $ q $, i.e.
	
	\begin{equation}
		\E (U,\varphi),(V,\psi) \in M, \, (U,\varphi) \ni p \wedge (V,\psi) \ni q \mid U \cap V = \emptyset
	\end{equation}

	per cui, siccome $ T(U) \ni (p,v) $ e $ T(V) \ni (q,w) $ e
	
	\begin{equation}
		T(U) \cap T(V) = T(U \cap V) = \emptyset \quad \because \quad U \cap V = \emptyset
	\end{equation}

	abbiamo che $ T(M) $ è di Hausdorff.
\end{itemize}

\paragraph{$ N_{2} $ o a base numerabile}

Al fine di dimostrare il secondo criterio di numerabilità $ N_{2} $ per $ T(M) $, utilizziamo il seguente lemma:

\begin{lemma}
	Ogni varietà differenziabile ha una base numerabile di insiemi coordinati o equivalentemente esiste un atlante differenziabile numerabile.
\end{lemma}

\begin{proof}
	Siano $ \{(U_{\alpha},\varphi_{\alpha})\} $ l'atlante massimale di $ M $ che definisce la struttura differenziabile di $ M $ e $ \mathfrak{U} = \{U_{i}\}_{i \in I} $ una base numerabile per la topologia di $ M $ (questa esiste in quanto $ M $ è $ N_{2} $). \\
	Presi un punto $ p \in M $ e un aperto $ U_{\alpha} $ di una carta dell'atlante, esiste sempre un aperto $ B_{p,\alpha} \in \mathfrak{U} $ tale che $ p \in B_{p,\alpha} \subset U_{\alpha} $, in quanto $ \mathfrak{U} $ è una base. Consideriamo ora la famiglia di aperti $ \B = \{B_{p,\alpha}\} $ senza duplicati, la quale sarà una sottofamiglia di $ \mathfrak{U} $ e, in quanto tale, sarà numerabile perché $ \mathfrak{U} $ è numerabile. Inoltre $ \left( B_{p,\alpha}, \eval{\varphi}_{B_{p,\alpha}} \right) $ è una carta di $ M $ e dunque $ \left\{ \left( B_{p,\alpha}, \eval{\varphi}_{B_{p,\alpha}} \right) \right\} $ è un atlante differenziabile numerabile per $ M $. \\
	Resta da dimostrare che $ \B $ sia una base (già numerabile) per $ M $: siano $ p \in M $ e $ U \ni p $ aperto di $ M $, siccome esiste sempre un aperto di una carta che sia contenuto in un aperto della varietà, possiamo scrivere $ p \in U_{\alpha} \subset U $, dove la carta considerata dell'atlante è $ \left( U \cap U_{\alpha}, \eval{\varphi}_{U \cap U_{\alpha}} \right) $; sappiamo anche che esiste $ B_{p,\alpha} $ tale che $ p \in B_{p,\alpha} \subset U_{\alpha} $, dunque $ p \in B_{p,\alpha} \subset U $. Questo significa che esiste sempre un aperto $ B_{p,\alpha} \in \B $ che si interponga tra ogni punto della varietà e ogni aperto che contiene il punto stesso, i.e. $ \B $ è una base per $ M $.
\end{proof}

Sia $ \{(U_{i},\varphi_{i})\}_{i \in I} $ un atlante differenziabile numerabile di $ M $ (la sua esistenza segue dal lemma dimostrato sopra), allora gli omeomorfismi

\begin{equation}
	\tilde{\varphi}_{i} : T(U_{i}) \to \varphi_{i}(U_{i}) \times \R^{n}
\end{equation}

permettono di dire che $ T(U_{i}) $ sia $ N_{2} $, in quanto $ T(U_{i}) \simeq \varphi(U_{i}) \times \R^{n} $ e $ \varphi(U_{i}) \times \R^{n} $ è $ N_{2} $. \\
Fissato $ i \in I $, sia $ \{B_{ij}\}_{j \in J} $ una base numerabile per $ T(U_{i}) $ (con $ I $ e $ J $ numerabili), allora l'unione

\begin{equation}
	\B_{T(M)} = \bigcup_{\substack{ i \in I \\ j \in J }} \{B_{ij}\}
\end{equation}

è una base numerabile per $ T(M) $ e dunque $ T(M) $ è $ N_{2} $.

\subsubsection{Struttura differenziabile sul fibrato tangente}

Dopo aver dimostrato che $ T(M) $ è una varietà topologica, dimostriamo che è anche una varietà differenziabile con atlante differenziabile $ \{(T(U_{\alpha}),\tilde{\varphi}_{\alpha})\}_{\alpha \in A} $ dove $ \{(U_{\alpha},\varphi_{\alpha})\}_{\alpha \in A} $ è un atlante differenziabile per $ M $. L'unico fatto da verificare è che le carte siano $ C^{\infty} $-compatibili, i.e. i cambi di carte siano lisci. \\
Definiamo

\begin{equation}
	U_{\alpha \beta} \doteq U_{\alpha} \cap U_{\beta}
\end{equation}

da cui

\begin{equation}
	T(U_{\alpha \beta}) \doteq T(U_{\alpha} \cap U_{\beta}) = T(U_{\alpha}) \cap T(U_{\beta})
\end{equation}

Possiamo quindi scrivere i cambi di carta come

\begin{gather}
	\tilde{\varphi}_{\beta} \circ \tilde{\varphi}_{\alpha}^{-1} : \tilde{\varphi}_{\alpha}(T(U_{\alpha \beta})) \to \tilde{\varphi}_{\beta}(T(U_{\alpha \beta})) \\
	\tilde{\varphi}_{\alpha} \circ \tilde{\varphi}_{\beta}^{-1} : \tilde{\varphi}_{\beta}(T(U_{\alpha \beta})) \to \tilde{\varphi}_{\alpha}(T(U_{\alpha \beta}))
\end{gather}

i quali devono essere lisci per qualsiasi $ \alpha,\beta \in A $. \\
Scriviamo le singole applicazioni come

\map{\tilde{\varphi}_{\alpha}}
	{T(U_{\alpha})}{\varphi_{\alpha}(U_{\alpha}) \times \R^{n}}
	{v}{(\varphi_{\alpha} \circ \pi, \varphi_{*p})(v)}

dove\footnote{%
	Qui il simbolo $ \longleftrightarrow $ indica l'identificazione.%
}

\begin{equation}
	\begin{cases}
		(U_{\alpha},\varphi_{\alpha}) = (U_{\alpha}; x^{1},\dots,x^{n}) \in M \\ \\
		v = \displaystyle \sum_{i=1}^{n} a^{i}(v) \eval{ \pdv{x^{i}} }_{p} \\ \\
		\pi(v) = p \\ \\
		\varphi_{*}(v) = \displaystyle \sum_{i=1}^{n} a^{i}(v) \eval{ \pdv{r^{i}} }_{\varphi(p)} \longleftrightarrow a(v) = (a^{1}(v),\dots,a^{n}(v)) \in \R^{n}
	\end{cases}	
\end{equation}

A questo punto possiamo riscrivere l'applicazione come

\map{\tilde{\varphi}_{\alpha}}
	{T(U_{\alpha})}{\varphi_{\alpha}(U_{\alpha}) \times \R^{n}}
	{v}{(\varphi_{\alpha} \circ \pi, \varphi_{*p})(v) \\
		&\mapsto \left( x^{1}(\pi(v)),\dots,x^{n}(\pi(v)), a^{1}(v),\dots,a^{n}(v) \right)}

e la sua inversa

\map{\tilde{\varphi}_{\alpha}^{-1}}
	{\varphi_{\alpha}(U_{\alpha}) \times \R^{n}}{T(U_{\alpha})}
	{(\varphi_{\alpha}(p), a^{1},\dots,a^{n})}{\sum_{i=1}^{n} a^{i}(v) \eval{ \pdv{x^{i}} }_{p}}

Per la funzione $ \tilde{\varphi}_{\beta} $ abbiamo invece

\begin{equation}
	\begin{cases}
		(U_{\beta},\varphi_{\beta}) = (U_{\beta}; y^{1},\dots,y^{n}) \in M \\ \\
		v = \displaystyle \sum_{i=1}^{n} b^{i}(v) \eval{ \pdv{y^{i}} }_{p} \\ \\
		\pi(v) = p \\ \\
		\varphi_{*}(v) = \displaystyle \sum_{i=1}^{n} b^{i}(v) \eval{ \pdv{r^{i}} }_{\varphi(p)} \longleftrightarrow b(v) = (b^{1}(v),\dots,b^{n}(v)) \in \R^{n}
	\end{cases}	
\end{equation}

da cui

\map{\tilde{\varphi}_{\beta}}
	{T(U_{\beta})}{\varphi_{\beta}(U_{\beta}) \times \R^{n}}
	{v}{(\varphi_{\beta} \circ \pi, \varphi_{*p})(v) \\
		&\mapsto \left( y^{1}(\pi(v)),\dots,y^{n}(\pi(v)), b^{1}(v),\dots,b^{n}(v) \right)}

con inversa

\map{\tilde{\varphi}_{\beta}^{-1}}
	{\varphi_{\beta}(U_{\beta}) \times \R^{n}}{T(U_{\beta})}
	{(\varphi_{\beta}(p), b^{1},\dots,b^{n})}{\sum_{i=1}^{n} b^{i}(v) \eval{ \pdv{y^{i}} }_{p}}

In $ T(U_{\alpha \beta}) $, il vettore $ v $ può essere scritto nei seguenti due modi

\begin{equation}
	v = \sum_{j=1}^{n} a^{j}(v) \eval{ \pdv{x^{j}} }_{p} = %
	\sum_{k=1}^{n} b^{k}(v) \eval{ \pdv{y^{k}} }_{p}
\end{equation}

da cui otteniamo

\sbs{0.5}{%
			\begin{align}
				\begin{split}
					v(y^{i}) &= \sum_{j=1}^{n} a^{j}(v) \, \pdv{y^{i}}{x^{j}} \, (p) \\
					&= \sum_{k=1}^{n} b^{k}(v) \, \pdv{y^{i}}{y^{k}} \, (p) \\
					&= \sum_{k=1}^{n} b^{k}(v) \, \delta^{ik} \\
					&= b^{i}(v)
				\end{split}
			\end{align}
			}
	{0.5}{%
			\begin{align}
				\begin{split}
					v(x^{i}) &= \sum_{k=1}^{n} b^{k}(v) \, \pdv{x^{i}}{y^{k}} \, (p) \\
					&= \sum_{j=1}^{n} a^{j}(v) \, \pdv{x^{i}}{x^{j}} \, (p) \\
					&= \sum_{k=1}^{n} a^{j}(v) \, \delta^{ij} \\
					&= a^{i}(v)
				\end{split}
			\end{align}
			}

perciò

\begin{gather}
		b^{i}(v) = \sum_{k=1}^{n} a^{k}(v) \, \pdv{y^{i}}{x^{k}} \, (p) \\
		a^{i}(v) = \sum_{k=1}^{n} b^{k}(v) \, \pdv{x^{i}}{y^{k}} \, (p)
\end{gather}

Possiamo dunque scrivere l'azione dei cambi di carte come

\begin{align}
	\begin{split}
		(\tilde{\varphi}_{\beta} \circ \tilde{\varphi}_{\alpha}^{-1}) (\varphi_{\alpha}(p), a^{1},\dots,a^{n}) &= \tilde{\varphi}_{\beta} \left( \sum_{i=1}^{n} a^{i}(v) \eval{ \pdv{x^{i}} }_{p} \right) \\
		&= \tilde{\varphi}_{\beta}(v) \\
		&= (\varphi_{\beta} \circ \pi, \varphi_{*p})(v) \\
		&= ( \varphi_{\beta}(p), b^{1}(v),\dots,b^{n}(v) ) \\
		&= \left( (\varphi_{\beta} \circ \varphi_{\alpha}^{-1})(\varphi_{\alpha}(p)), \sum_{i=1}^{n} a^{k}(v) \, \pdv{y^{1}}{x^{k}} \, (p) ,\dots, \sum_{i=1}^{n} a^{k}(v) \, \pdv{y^{n}}{x^{k}} \, (p) \right)
	\end{split}
\end{align}

ricordando che le entrate della matrice jacobiana sono equivalenti a

\begin{equation}
	\pdv{y^{j}}{x^{k}} \, (p) = \pdv{(\varphi_{\beta} \circ \varphi_{\alpha}^{-1})^{j}}{r^{k}} \, (\varphi_{\alpha}(p))
\end{equation}

otteniamo, applicando un ragionamento analogo per il cambio di carte inverso, le seguenti uguaglianze

\begin{multline}
	(\tilde{\varphi}_{\beta} \circ \tilde{\varphi}_{\alpha}^{-1}) (\varphi_{\alpha}(p), a^{1},\dots,a^{n}) = \\
	= \left( (\varphi_{\beta} \circ \varphi_{\alpha}^{-1})(\varphi_{\alpha}(p)), \sum_{i=1}^{n} a^{i}(v) \, \pdv{(\varphi_{\beta} \circ \varphi_{\alpha}^{-1})^{1}}{r^{i}} \, (\varphi_{\alpha}(p)) ,\dots, \sum_{i=1}^{n} a^{i}(v) \, \pdv{(\varphi_{\beta} \circ \varphi_{\alpha}^{-1})^{n}}{r^{i}} \, (\varphi_{\alpha}(p)) \right)
\end{multline}
%
\begin{multline}
	(\tilde{\varphi}_{\alpha} \circ \tilde{\varphi}_{\beta}^{-1}) (\varphi_{\beta}(p), b^{1},\dots,b^{n}) = \\
	= \left( (\varphi_{\alpha} \circ \varphi_{\beta}^{-1})(\varphi_{\beta}(p)), \sum_{i=1}^{n} b^{i}(v) \, \pdv{(\varphi_{\alpha} \circ \varphi_{\beta}^{-1})^{1}}{r^{i}} \, (\varphi_{\beta}(p)) ,\dots, \sum_{i=1}^{n} b^{i}(v) \, \pdv{(\varphi_{\alpha} \circ \varphi_{\beta}^{-1})^{n}}{r^{i}} \, (\varphi_{\beta}(p)) \right)
\end{multline}

Questi cambi di carte sono lisci poiché:

\begin{itemize}
	\item per le prime $ n $ componenti identificate da $ (\varphi_{\alpha} \circ \varphi_{\beta}^{-1})(\varphi_{\beta}(p)) $: $ \varphi_{\beta} \circ \varphi_{\alpha}^{-1} $ e $ \varphi_{\alpha} \circ \varphi_{\beta}^{-1} $ sono i cambi di carte in $ M $ tra $ (U_{\alpha},\varphi_{\alpha}) $ e $ (U_{\beta},\varphi_{\beta}) $;
	
	\item per le seconde $ n $ componenti: le derivate parziali dei cambi di carte applicate nei punti $ \varphi_{\alpha}(p) $ e $ \varphi_{\beta}(p) $ moltiplicate per numeri reali $ a^{i}(v) $ e $ b^{i}(v) $ e sommate tra loro sono ancora lisce.
\end{itemize}

Questo mostra che $ T(M) $ è una varietà differenziabile di dimensione $ 2n $.

\subsubsection{Proiezione dal fibrato tangente}

Dimostriamo che sia liscia la proiezione dal fibrato tangente di una varietà alla varietà stessa

\map{\pi}
	{T(M)}{M}
	{(p,v)}{p}

Per mostrarlo, prendiamo due carte arbitrarie e facciamo vedere che l'espressione locale della proiezione è liscia. \\
Preso un vettore $ (p,v) \in T_{p}(M) $, siano le carte

\begin{equation}
	\begin{cases}
		(U,\varphi) = (U; x^{1},\dots,x^{n}) \in M, & \varphi(p) = 0 \in \R^{n} \\
		(T(U),\tilde{\varphi}) \in T(M), & \tilde{\varphi}(p,v) = 0 \in \phi(U) \times \R^{n}
	\end{cases}
\end{equation}

dove

\begin{gather}
	v = \displaystyle \sum_{i=1}^{n} c^{i}(v) \eval{ \pdv{x^{i}} }_{p} \\
	\nonumber \\
	c(v) = (c^{1}(v),\dots,c^{n}(v))
\end{gather}

Ricordando l'applicazione $ \tilde{\varphi} $

\begin{equation}
	\tilde{\varphi}(v) = (\varphi(p),c(v)) = (x^{1}(p),\dots,x^{n}(p),c^{1}(v),\dots,c^{n}(v))
\end{equation}

abbiamo che

\begin{equation}
	(\varphi \circ \pi \circ \tilde{\varphi}^{-1}) (\varphi(p),c^{1}(v),\dots,c^{n}(v)) = (\varphi \circ \pi)(v) = (\varphi \circ \pi) \left( \sum_{i=1}^{n} c^{i}(v) \eval{ \pdv{x^{i}} }_{p} \right) = \varphi(p)
\end{equation}

dunque $ \pi $ è localmente liscia (e anche lineare) in quanto proiezione sulle prime $ n $ coordinate e perciò liscia dappertutto.

\subsection{Funzioni a campana}

Siano $ M $ una varietà differenziabile e $ f : M \to \R $ un'applicazione su varietà (non necessariamente liscia). Definiamo il \textit{supporto di} $ f $, indicato con $ \supp(f) $, come il sottoinsieme di $ M $ costituito dalla chiusura dell'insieme dei punti dove $ f $ non si annulla, i.e. preso l'insieme dei punti dove la funzione si annulla

\begin{equation}
	Z(f) = \{ p \in M \mid f(p) = 0 \}
\end{equation}

e il complementare di questo

\begin{equation}
	Z(f)^{C} = M \setminus Z(f)
\end{equation}

definiamo

\begin{equation}
	\supp(f) \doteq \overline{Z(f)^{C}}
\end{equation}

\sbs{0.6}{%
			Ad esempio, per la funzione
			
			\img{1}{img43}
			}
	{0.4}{%
			\begin{gather}
				Z(f) = (-\infty,-1] \cup [1,+\infty) \\
				Z(f)^{C} = (-1,-1) \\
				\supp(f) = \overline{Z(f)^{C}} = [-1,1]
			\end{gather}
			}

\sbs{0.5}{%
			O ancora per la funzione
			
			\img{0.8}{img44}
			}
	{0.5}{%
			\map{f}
				{(-1,1)}{\R}
				{x}{\tan(\dfrac{\pi x}{2})}
			
			\begin{gather}
				Z(f) = \{0\} \\
				Z(f)^{C} = (-1,0) \cup (0,1) \\
				\supp(f) = \overline{Z(f)^{C}} = (-1,1)
			\end{gather}
		
			In quanto $ (-1,1) = M $ è chiuso in $ M $.
			}

Siano $ M $ una varietà differenziabile, $ U \subset M $ un aperto della varietà e $ q \in U $ un suo punto, l'applicazione $ f : M \to \R $ è una \textit{funzione a campana} in $ q $ con supporto in $ U $ se

\begin{itemize}
	\item $ f $ è continua
	
	\item $ f \equiv 1 $ in un intorno di $ q $
	
	\item $ \supp(f) \subset U $
\end{itemize}

Il primo esempio descritto sopra è una funzione a campana in un qualunque punto appartenente a $ (-1/2,1/2) $ con supporto in qualunque aperto $ U \supset [-1,1] $.

\paragraph{Funzione a campana liscia da $ \R^{n} $ a $ \R $}

Partendo da una funzione liscia da $ \R $ in $ \R $, costruiremo una funzione a campana liscia da $ \R^{n} $ a $ \R $. Consideriamo quindi la funzione liscia

\sbs{0.4}{%
			\map{f}
				{\R}{\R}
				{x}{%
					\begin{cases}
						e^{\sfrac{-1}{x}}, & x > 0 \\
						0, & x \leqslant 0
					\end{cases}
					}
			}
	{0.6}{%
			\img{0.9}{img45}
			}

Il primo passo è definire una nuova funzione $ g $ a partire da $ f $ come

\begin{equation}
	g(x) \doteq \dfrac{f(x)}{f(x) + f(1-x)}
\end{equation}

questa sarà ancora liscia se il denominatore non si annulla, in quanto

\begin{equation}
	\begin{cases}
		x \leqslant 0 \implies f(x) + f(1-x) = f(1-x) \neq 0 \quad \because \quad 1-x \geqslant 0 \\
		x > 0 \implies f(x) + f(1-x) \geqslant f(x) > 0
	\end{cases}
\end{equation}

Se consideriamo il caso

\begin{equation}
	x \geqslant 1 %
	\implies %
	1-x \leqslant 0 %
	\implies %
	f(1-x) = 0 %
	\implies %
	g(x) = \dfrac{f(x)}{f(x)} \equiv 1
\end{equation}

perciò il grafico di $ g(x) $ avrà la forma

\img{0.5}{img46}

In modo tale da poter spostare i punti di contatto, introduciamo la seguente modifica

\begin{equation}
	h(x) \doteq g \left( \dfrac{x - a^{2}}{b^{2} - a^{2}} \right) %
	\qcomma %
	\begin{cases}
		a,b \in \R \\
		0 < a < b
	\end{cases}
\end{equation}

perciò

\begin{equation}
	\begin{cases}
		h(x) = 1, & x \geqslant b^{2} \\
		h(x) = 0, & x \leqslant a^{2}
	\end{cases}
\end{equation}

da cui la forma "spostata"

\img{0.5}{img47}

Il prossimo passo è rendere la funzione simmetrica

\begin{equation}
	k(x) \doteq h(x^{2})
\end{equation}

perciò

\begin{equation}
	\begin{cases}
		k(x) = 1, & x \leqslant - b \wedge x \geqslant b \\
		k(x) = 0, & -a \leqslant x \leqslant a
	\end{cases}
\end{equation}

da cui la forma a "scodella"

\img{0.5}{img48}

Considerando ora l'immagine speculare rispetto all'asse $ x $ e spostata di un'unità verso l'alto

\begin{equation}
	\sigma(x) \doteq 1 - k(x)
\end{equation}

\img{0.5}{img49}

A questo punto, generalizziamo al dominio in $ \R^{n} $ considerando $ x = (x^{1},\dots,x^{n}) $ e poi la norma di $ x $

\map{\rho}
	{\R^{n}}{\R}
	{x}{\sigma(\norm{x})}

da cui la forma in $ n+1 $ dimensioni:

\img{0.5}{img50}

Considerando il disco centrato in $ q $ di raggio $ r $

\begin{equation}
	D_{r}(q) = \{ x \in \R^{n} \mid \norm{x - q} < r \}
\end{equation}

abbiamo che $ \rho $ è una funzione a campana liscia in qualunque punto in $ D_{a}(0) $ con supporto $ U $, dove $ U $ è un qualunque aperto tale che contenga il supporto di $ \rho $, i.e.

\begin{equation}
	U \supset \supp(\rho) = \overline{D_{b}(0)} = \{ x \in \R^{n} \mid \norm{x} \leqslant b \}
\end{equation}

I valori principali assunti dalla funzione sono

\begin{equation}
	\begin{cases}
		\rho(x) = 1, & x \in \overline{D_{a}(0)} \\
		\rho(x) = 0, & x \in \R^{n} \setminus D_{b}(0)
	\end{cases}
\end{equation}

Si può spostare il centro della funzione in un punto $ q \in \R^{n} $ tramite la sostituzione

\begin{equation}
	\rho(x) \to \rho(x-q)
\end{equation}

Per la generalizzazione di questo procedimento a funzioni a campana su varietà, consideriamo il seguente teorema:

\begin{theorem}\label{thm:bump-fun}
	Siano $ M $ una varietà differenziabile di dimensione $ n $, $ U \subset M $ un aperto della varietà e $ q \in U $ un suo punto, allora esiste una funzione a campana liscia $ \rho : M \to \R $ in $ q $ con supporto contenuto in $ U $.
\end{theorem}

\begin{proof}
	L'idea è di spostare il problema al costruire la funzione a campana in $ \R^{n} $. \\
	Consideriamo una carta $ (V_{r},\varphi) \in M $ tale che
	
	\begin{equation}
		\begin{cases}
			q \in V_{r} \subset U \\
			\varphi(q) = 0 \\
			\varphi(V_{r}) = D_{r}(0) \in \R^{n}
		\end{cases}
	\end{equation}

	ricordando che $ \varphi $ è un diffeomorfismo e che è sempre possibile restringere l'aperto $ V_{r} $ in modo tale da avere come immagine il disco considerato. \\
	Consideriamo ora due numeri reali $ a < b < r $ e i dischi in $ \R^{n} $ centrati in 0 che hanno questi come raggio, i.e. $ D_{a}(0) \subset D_{b}(0) \subset D_{r}(0) $ dove
	
	\begin{equation}
		\begin{cases}
			\varphi^{-1}(D_{a}(0)) \doteq V_{a} \\
			\varphi^{-1}(D_{b}(0)) \doteq V_{b} \\
			\varphi^{-1}(D_{r}(0)) = V_{r}
		\end{cases} %
		\implies %
		V_{a} \subset V_{b} \subset V_{r}
	\end{equation}
	
	\img{0.8}{img51}
	
	Presa la funzione a campana su $ \R^{n} $
	
	\begin{equation}
		\begin{cases}
			\sigma : \R^{n} \to \R \\
			\sigma(x) = 1, & x \in \overline{D_{a}(0)} \\
			\sigma(x) = 0, & x \in \R^{n} \setminus D_{b}(0) \\
			\supp(\sigma) = \overline{D_{b}(0)}
		\end{cases}
	\end{equation}

	possiamo definire la funzione su $ M $ per $ p \in M $ come
	
	\map{\rho}
		{M}{\R}
		{p}{%
			\begin{cases}
				(\sigma \circ \varphi)(p), & p \in V_{r} \\
				0, & p \in M \setminus \overline{V_{b}}
			\end{cases}
			}

	A questo punto dimostriamo che $ \rho $ sia una funzione a campana liscia in $ q $ con supporto contenuto in $ U $: è necessario che l'applicazione sia liscia, che esista un intorno in cui questa è pari a 1 e che il suo supporto sia contenuto in $ U $.
	
	\begin{itemize}
		\item Perché questa applicazione sia liscia è necessario che sia liscia per i due casi della definizione (definite in aperti) e che i valori dei due casi coincidano nell'intersezione degli aperti in cui sono definiti (la quale è ancora un aperto), i.e.
		
		\begin{equation}
			\begin{cases}
				\sigma \circ \varphi \in C^{\infty}(V_{r}) \\
				0 \in C^{\infty} \left( M \setminus \overline{V_{b}} \right) \\
				(\sigma \circ \varphi)(p) = 0, & \forall p \in V_{r} \cap (M \setminus \overline{V_{b}})
			\end{cases}
		\end{equation}
		
		La prima richiesta è soddisfatta in quanto composizione di funzioni lisce mentre la seconda richiesta è banale; per la terza, possiamo scrivere
		
		\begin{equation}
			V_{r} \cap \left( M \setminus \overline{V_{b}} \right) = V_{r} \setminus \overline{V_{b}}
		\end{equation}
		
		essendo $ \sigma(x) = 0 $ per qualsiasi $ x \in \R^{n} \setminus D_{b}(0) $, possiamo dunque scrivere
		
		\begin{equation}
			(\sigma \circ \varphi)(p) = 0 \qcomma \forall p \in V_{r} \setminus \overline{V_{b}}
		\end{equation}
		
		%
		
		\item Abbiamo che $ V_{a} \ni q $ e che
		
		\begin{equation}
			\rho(p) = 1 \qcomma \forall p \in V_{a}
		\end{equation}
	
		perciò esiste un intorno di $ q $ (in questo caso $ V_{a} $) in cui la funzione è pari a 1
		
		%
		
		\item Abbiamo che
		
		\begin{equation}
			\supp(\sigma) = \overline{D_{b}(0)}
		\end{equation}
	
		e che $ \varphi $ è un diffeomorfismo, dunque il supporto di $ \rho $ sarà
		
		\begin{equation}
			\supp(\rho) = \varphi^{-1} \left( \overline{D_{b}(0)} \right) %
			= \overline{V_{b}} \subset V_{r} \subset U
		\end{equation}
	\end{itemize}
	
	Questo prova dunque che esiste una funzione a campana liscia per ogni punto di una varietà con supporto contenuto in un intorno del punto.
\end{proof}

\begin{corollary}[Estensione di funzioni lisce]\label{cor:ext-smooth}
	Siano una varietà differenziabile $ M $, un punto $ p \in M $, un aperto $ U \subset M $ che sia intorno di $ p $ e una funzione $ f : M \to \R $ liscia solo su $ U $, i.e. $ f \in C^{\infty}(U) $, allora esistono un aperto $ V \subseteq U $ intorno di $ q $ e una funzione $ \tilde{f} : M \to \R $ liscia in tutta la varietà, i.e. $ \tilde{f} \in C^{\infty}(M) $, tale che $ \tilde{f} \equiv f $ in $ V $.
\end{corollary}

Attraverso questo corollario, in pratica, $ \tilde{f} $ è un'estensione liscia di $ \eval{f}_{V} $. \\
In generale non si può estendere una funzione oltre al dominio in cui essa è liscia però, tramite questo corollario, è possibile considerare la restrizione a un aperto di questa funzione ed estendere questa restrizione in modo liscio oltre il dominio della prima funzione. Un esempio è la funzione $ f(x) = \tan(x/2) $ la quale ha due asintoti verticali in $ \pm \pi $: prendendo un aperto $ V \subset [-\pi,\pi] $ e la restrizione di $ f $ ad esso, è possibile "incollare" in modo liscio alla funzione due estensioni di questa oltre il suo dominio.

\begin{proof}
	Sia $ \rho : M \to \R $ una funzione liscia a campana in $ q $ con supporto contenuto in $ U $ (esiste per il teorema precedente) che sia uguale a 1 nell'aperto $ V $. \\
	Definiamo
	
	\map{\tilde{f}}
		{M}{\R}
		{p}{%
			\begin{cases}
				\rho(p) \cdot f(p), & p \in U \\
				0, & p \in M \setminus \supp(\rho)
			\end{cases}
			}
	
	dove $ \cdot $ indica il prodotto tra numeri reali. \\
	Perché sia liscia è necessario che lo sia nei due casi della definizione e che questi coincidano nella loro intersezione:
	
	\begin{itemize}
		\item La prima condizione richiede che $ \tilde{f} \equiv f $ in $ V $. Questa è verificata in quanto
		
		\begin{equation}
			\rho(p) \equiv 1 \qcomma \forall p \in V \implies \tilde{f}(p) = (\rho \circ f)(p) \equiv f(p) \qcomma \forall p \in V
		\end{equation}
		
		\item Per la seconda richiesta, possiamo scrivere l'intersezione come
		
		\begin{equation}
			U \cap (M \setminus \supp(\rho)) = U \setminus \supp(\rho)
		\end{equation}
		
		la quale è aperta perché il supporto è un chiuso, dunque resta verificare che i casi coincidano. \\
		Ricordando che
		
		\begin{equation}
			\rho(p) = 0 \qcomma \forall p \in M \setminus \supp(\rho)
		\end{equation}
		
		perciò
		
		\begin{equation}
			\tilde{f}(p) = (\rho \circ f)(p) = 0 \qcomma \forall p \in U \setminus \supp(\rho)
		\end{equation}
		
		dunque la funzione è liscia dappertutto.
	\end{itemize}
\end{proof}

\subsection{Campi di vettori su varietà}

Sia $ M $ una varietà differenziabile di dimensione $ n $. Ricordando la definizione del fibrato tangente

\begin{equation}
	T(U) = \bigsqcup_{p \in M} T_{p}(M)
\end{equation}

e, preso un vettore $ (p,v) \equiv X_{p} \in T_{p}(M) $, la definizione della proiezione sulla varietà

\map{\pi}
	{T(M)}{M}
	{X_{p}}{p}

un \textit{campo di vettori} è un'applicazione

\map{X}
	{M}{T(M)}
	{p}{X_{p}}

e dunque l'inverso della proiezione, i.e.

\begin{equation}
	\begin{cases}
		\pi \circ X = \id_{M} \\
		X \circ \pi = \id_{T(M)}
	\end{cases}
\end{equation}

Un campo di vettori $ X $ su una varietà $ M $ è \textit{liscio} se per qualunque punto $ p \in M $ esiste una carta

\begin{equation}
	\begin{cases}
		(U,\varphi) = (U; x^{1},\dots,x^{n}) \in M, & \varphi(p) = 0 \\
		U \ni p
	\end{cases}
\end{equation}

tale che, esplicitando nel seguente modo il vettore $ X_{p} $ tangente alla varietà $ M $ nel punto $ p \in U $

\begin{equation}
	X_{p} = \sum_{j=1}^{n} a^{j}(p) \eval{ \pdv{x^{j}} }_{p} \in T_{p}(M)
\end{equation}

le funzioni $ a^{j} : U \to \R $ siano lisce in $ U $. \\
L'espressione locale dei vettori del campo è la seguente

\begin{equation}
	\eval{X}_{U} = \sum_{j=1}^{n} a^{j} \, \pdv{x^{j}}
\end{equation}

La definizione di campo di vettori liscio non dipende dalla carta scelta: presa un'altra carta

\begin{equation}
	\begin{cases}
		(V,\psi) = (V; y^{1},\dots,y^{n}) \in M, & \psi(p) = 0 \\
		V \ni p
	\end{cases}
\end{equation}

il campo di vettori si scriverà localmente come

\begin{equation}
	\eval{X}_{V} = \sum_{j=1}^{n} b^{j} \, \pdv{y^{j}}
\end{equation}

Nell'intersezione degli aperti delle carte possiamo dunque scrivere

\begin{equation}
	\eval{X}_{U \cap V} = \sum_{j=1}^{n} a^{j} \, \pdv{x^{j}} = \sum_{j=1}^{n} b^{j} \, \pdv{y^{j}}
\end{equation}

Applicando questo alle applicazioni $ x^{j} $ e $ y^{j} $

\begin{equation}
	\begin{cases}
		\displaystyle \eval{X}_{U \cap V} (x^{j}) = \sum_{j=1}^{n} a^{k} \, \pdv{x^{j}}{x^{k}} = a^{j} = \sum_{j=1}^{n} b^{k} \, \pdv{x^{j}}{y^{k}} \\ \\
		\displaystyle \eval{X}_{U \cap V} (y^{j}) = \sum_{j=1}^{n} a^{k} \, \pdv{y^{j}}{x^{k}} = \sum_{j=1}^{n} b^{k} \, \pdv{y^{j}}{y^{k}} = b^{j}
	\end{cases} %
	\qquad \forall j=1,\dots,n
\end{equation}

otteniamo

\begin{equation}
	\begin{cases}
		\displaystyle a^{j} = \sum_{j=1}^{n} b^{k} \, \pdv{x^{j}}{y^{k}} \\ \\
		\displaystyle b^{j} = \sum_{j=1}^{n} a^{k} \, \pdv{y^{j}}{x^{k}}
	\end{cases} %
	\qquad \forall j=1,\dots,n
\end{equation}

perciò

\begin{equation}
	a^{j} \in C^{\infty}(U \cap V) \iff b^{j} \in C^{\infty}(U \cap V) \qcomma \forall j=1,\dots,n
\end{equation}

poiché le entrate delle matrici jacobiane sono lisce.

\begin{definition}
	Sia $ X $ un campo di vettori su $ M $, allora $ X : M \to T(M) $ è un'applicazione liscia se e solo se il campo di vettori è liscio, i.e. se scrivendo
	
	\begin{equation}
		\eval{X}_{U} = \sum_{j=1}^{n} a^{j} \, \pdv{x^{j}}
	\end{equation}

	si ha che
	
	\begin{equation}
		a^{j} \in C^{\infty}(U) \qcomma \forall j=1,\dots,n
	\end{equation}
\end{definition}

\begin{proof}[Dimostrazione ($ \implies $)]
	Siano un'applicazione liscia $ X : M \to T(M) $ e una carta
	
	\begin{equation}
		(U,\varphi) = (U; x^{1},\dots,x^{n}) \in M \qcomma \varphi(p) = 0
	\end{equation}
	
	Possiamo scrivere il campo di vettori come
	
	\begin{equation}
		\eval{X}_{U} = \sum_{j=1}^{n} a^{j} \, \pdv{x^{j}}
	\end{equation}

	con $ a^{j} : U \to \R $. \\
	Sia una carta $ (T(U),\tilde{\varphi}) \in T(M) $ con il diffeomorfismo
	
	\map{\tilde{\varphi}}
		{T(U)}{\varphi(U) \times \R^{n}}
		{%
			X_{p} = \sum_{j=1}^{n} c^{j}(X_{p}) \eval{ \pdv{x^{j}} }_{p}
			}
		{%
			(\varphi(p),c(X_{p})) \\
			&\mapsto (x^{1}(p),\dots,x^{n}(p),c^{1}(X_{p}),\dots,c^{n}(X_{p}))
			}

	dove $ c^{j} \in C^{\infty}(T(U)) $ per $ j=1,\dots,n $, da cui possiamo considerare la restrizione
	
	\begin{equation}
		\eval{X}_{U} = \sum_{j=1}^{n} c^{j} \, \pdv{x^{j}}
	\end{equation}

	e dunque $ c^{j} = a^{j} \circ \pi $, il che implica che
	
	\begin{equation}
		a^{j} \in C^{\infty}(U) \qcomma \forall j=1,\dots,n
	\end{equation}
	
	in quanto $ \pi \in C^{\infty}(T(U)) $, e quindi il campo di vettori è liscio.
\end{proof}

\begin{proof}[Dimostrazione ($ \impliedby $)]
	Supponiamo di poter scrivere
	
	\begin{equation}
		\eval{X}_{U} = \sum_{j=1}^{n} a^{j} \, \pdv{x^{j}}
	\end{equation}
	
	con
	
	\begin{equation}
		a^{j} \in C^{\infty}(U) \qcomma \forall j=1,\dots,n
	\end{equation}

	Consideriamo la composizione
	
	\map{\eval{\tilde{\varphi} \circ X}_{U}}
		{U}{\varphi(U) \times \R^{n}}
		{p}{(\varphi(p),c(X_{p}))}

	da cui
	
	\begin{equation}
		\left( \tilde{\varphi} \circ \eval{X}_{U} \right)(p) = \tilde{\varphi}(X_{p}) %
		= (\varphi(p),c(X_{p})) %
		= (\varphi(p),a(p))
	\end{equation}

	Da questo risultato, siccome $ \varphi $  un diffeomorfismo e le $ a^{j} $ sono lisce, abbiamo che la composizione $ \tilde{\varphi} \circ \eval{X}_{U} $ è liscia: essendo anche $ \tilde{\varphi} $ un diffeomorfismo, il campo di vettori $ \eval{X}_{U} $ è liscio localmente e dunque lo è anche globalmente, i.e. $ X \in C^{\infty}(M) $.
\end{proof}

L'insieme dei campi di vettori lisci verrà indicato come

\begin{equation}
	\chi(M) = \{ \text{campi di vettori lisci } X \text{ su } M \}
\end{equation}

\subsubsection{$ \chi(M) $ come $ C^{\infty}(M) $-modulo}

\begin{definition}
	L'insieme $ \chi(M) $ è un $ C^{\infty}(M) $-modulo.
\end{definition}

\begin{proof}
	La dimostrazione segue dal Teorema \ref{thm:chi-mod} tramite la generalizzazione di questo alle varietà differenziabili.
\end{proof}

Perché $ \chi(M) $ sia un $ C^{\infty}(M) $-modulo è necessario che sia uno spazio vettoriale sul campo dei reali, i.e.

\begin{equation}
	\begin{cases}
		\lambda X + \mu Y \in \chi(M), & \forall X,Y \in \chi(M), \, \forall \lambda,\mu \in \R \\
		(\lambda X + \mu Y)_{p} = \lambda X_{p} + \mu Y_{p}
	\end{cases}
\end{equation}

Inoltre, si deve poter moltiplicare il campo per una funzione liscia: usiamo per questo la seguente applicazione

\map{K}
	{C^{\infty}(M) \times \chi(M)}{\chi(M)}
	{(f,X)}{f X}

con la condizione che

\begin{equation}
	(f X)(p) = f(p) \, X_{p}
\end{equation}

da cui

\begin{equation}
	\begin{cases}
		f X \in \chi(M) \\ \\
		\displaystyle \eval{X}_{U} = \sum_{j=1}^{n} a^{j} \, \pdv{x^{j}}
	\end{cases} %
	\implies %
	\eval{f X}_{U} = \sum_{j=1}^{n} f a^{j} \, \pdv{x^{j}}
\end{equation}

dove le funzioni $ f a^{j} $ sono lisce per qualsiasi $ j=1,\dots,n $. Essendo quindi $ \chi(M) $ un $ C^{\infty}(M) $-modulo, rimangono campi di vettori lisci anche tutte le combinazioni lineari tra campi e funzioni, i.e.

\begin{equation}
	(f+g)(X) = f X + g X
\end{equation}

\subsubsection{Derivata di una funzione rispetto a un campo di vettori}

Siano un campo di vettori liscio $ X \in \chi(M) $ e una funzione liscia $ f \in C^{\infty}(M) $, allora la derivata della funzione $ f $ rispetto al campo di vettori $ X $ è definita come

\map{X f}
	{M}{\R}
	{p}{(X f)(p) = X_{p} f}

La derivata può anche essere definita su germi delle funzioni in quanto tutte le funzioni di uno stesso germe coincidono in un intorno del punto, i.e.

\begin{equation}
	X_{p} f = X_{p} [f] \qcomma f \in [f] \in C_{p}^{\infty}(M)
\end{equation}

\begin{definition}
	Fissato un campo di vettori liscio $ X \in \chi(M) $, l'immagine della funzione
	
	\map{\varphi(X)}
		{C^{\infty}(M)}{C^{\infty}(M)}
		{f}{X f}

	è una funzione liscia, i.e. $ X f \in C^{\infty}(M) $.
\end{definition}

\begin{proof}
	È sufficiente dimostrare che la funzione
	
	\map{X f}
		{M}{\R}
		{p}{X_{p} f}

	sia liscia. \\
	Siano un punto della varietà $ p \in M $ e una carta
	
	\begin{equation}
		(U,\varphi) = (U; x^{1},\dots,x^{n}) \in M \qcomma \varphi(p) = 0
	\end{equation}
	
	allora
	
	\begin{equation}
		\eval{X}_{U} = \sum_{j=1}^{n} a^{j} \, \pdv{x^{j}}
	\end{equation}

	perciò
	
	\begin{equation}
		\eval{(X f)}_{U} = \sum_{j=1}^{n} a^{j} \, \pdv{f}{x^{j}}
	\end{equation}

	in quanto
	
	\begin{equation}
		X_{p} f = \sum_{j=1}^{n} a^{j}(p) \, \pdv{f}{x^{j}} \, (p)
	\end{equation}

	i quali sono lisci perché le $ a^{j} $ e le derivate parziali $ \pdv*{f}{x^{j}} $ sono lisce. \\
	Avendo fatto vedere che $ X f $ è localmente liscia, allora lo è anche globalmente.
\end{proof}

\begin{definition}
	Un campo di vettori $ X $ è liscio se e solo se la derivata $ X f $ è liscia per qualunque funzione liscia $ f $, i.e.
	
	\begin{equation}
		X \in \chi(M) \iff X f \in C^{\infty}(M), \, \forall f \in C^{\infty}(M)
	\end{equation}
\end{definition}

\begin{proof}[Dimostrazione ($ \implies $)]
	Vedi dimostrazione della proposizione precedente.
\end{proof}

\begin{proof}[Dimostrazione ($ \impliedby $)]
	Supponiamo che
	
	\begin{equation}
		X f \in C^{\infty}(M) \qcomma \forall f \in C^{\infty}(M)
	\end{equation}

	Sia una carta
	
	\begin{equation}
		(U,\varphi) = (U; x^{1},\dots,x^{n}) \in M \qcomma \varphi(p) = 0
	\end{equation}
	
	allora
	
	\begin{equation}
		\eval{X}_{U} = \sum_{j=1}^{n} a^{j} \, \pdv{x^{j}}
	\end{equation}

	con $ a^{j} : U \to \R $ per $ j=1,\dots,n $. Perché $ X \in \chi(M) $ dobbiamo dimostrare che $ a^{j} \in C^{\infty}(U) $. \\
	Per il corollario di estensione di funzioni\footnote{%
		Vedi Corollario \ref{cor:ext-smooth}.%
	}, esistono un aperto $ V \subset U $ tale che $ V \ni p $ e delle funzioni $ \tilde{x}^{k} \in C^{\infty}(M) $ tali che $ \eval{\tilde{x}^{k}}_{V} \equiv x^{k} $ per $ k=1,\dots,n $ (dalla carta, $ x^{k} \in C^{\infty}(U) $ per $ k=1,\dots,n $). Applichiamo ora la derivata a queste estensioni
	
	\begin{align}
		\begin{split}
			\eval{(X \tilde{x}^{k})}_{V} &= \left( \sum_{j=1}^{n} a^{j} \, \pdv{x^{j}} \right) \eval{(\tilde{x}^{k})}_{V} \\
			&= \left( \sum_{j=1}^{n} a^{j} \, \pdv{x^{j}} \right) (x^{k}) \\
			&= \sum_{j=1}^{n} a^{j} \, \pdv{x^{k}}{x^{j}} \\
			&= a^{j} \delta^{kj} \\
			&= a^{k}
		\end{split}
	\end{align}

	Essendo $ \eval{(X \tilde{x}^{k})}_{V} $ liscio per ipotesi, abbiamo che le $ a^{k} \in C^{\infty}(V) $ per $ k=1,\dots,n $ da cui si ottiene che queste sono lisce in tutto $ M $, in quanto il punto $ p $ considerato è arbitrario, e dunque $ X \in \chi(M) $.
\end{proof}

\subsubsection{Proprietà}

Riassumendo, sono presenti tre metodi equivalenti per determinare se un campo $ X $ su $ M $ sia liscio, i.e. $ X \in \chi(M) $:

\begin{enumerate}
	\item Presa una carta
	
	\begin{equation}
		(U,\varphi) = (U; x^{1},\dots,x^{n}) \in M \qcomma \varphi(p) = 0
	\end{equation}
	
	è possibile scrivere il campo localmente come
	
	\begin{equation}
		\eval{X}_{U} = \sum_{j=1}^{n} a^{j} \, \pdv{x^{j}}
	\end{equation}

	dove le $ a^{j} \in C^{\infty}(U) $ per $ j=1,\dots,n $;
	
	%
	
	\item Si può vedere il campo come un'applicazione liscia
	
	\map{X}
		{M}{T(M)}
		{p}{X_{p} \in T_{p}(M)}
	
	tale che
	
	\begin{equation}
		\begin{cases}
			\pi \circ X = \id_{M} \\
			X \circ \pi = \id_{T(M)}
		\end{cases}
	\end{equation}	
	
	dove $ \pi(X_{p}) = p $;
	
	%
	
	\item La derivata direzionale di una funzione è liscia per ogni funzione liscia, i.e. l'immagine dell'applicazione
	
	\map{\varphi(X)}
		{C^{\infty}(M)}{C^{\infty}(M)}
		{f}{X f}

	è liscia.
\end{enumerate}

Ricordiamo che la derivata $ X f $ si scrive come

\begin{equation}
	(X f)(p) = X_{p} f \equiv X_{p} [f]
\end{equation}

in quanto $ f \in [f] \in C_{p}^{\infty}(M) $ e la derivata non dipende dal rappresentante scelto dal germe.

\paragraph{Uguaglianza tra campi di vettori}

\begin{theorem}[Uguaglianza tra campi di vettori]
	Siano due campi di vettori lisci $ X,Y \in \chi(M) $, allora
	
	\begin{equation}
		X = Y \iff X f = Y f \qcomma \forall f \in C^{\infty}(M)
	\end{equation}
\end{theorem}

\begin{proof}[Dimostrazione ($ \implies $)]
	La prima implicazione è banale in quanto, se due campi di vettori $ X,Y \in \chi(M) $ sono uguali allora anche lo è anche la loro azione su una funzione liscia $ f $, i.e.
	
	\begin{equation}
		X = Y \implies X f = Y f \qcomma \forall f \in C^{\infty}(M)
	\end{equation}
\end{proof}

\begin{proof}[Dimostrazione ($ \impliedby $)]
	Supponiamo che
	
	\begin{equation}
		X f = Y f  \qcomma \forall f \in C^{\infty}(M)
	\end{equation}
	
	Andando a ritroso, abbiamo la seguente serie di implicazioni:
	
	\begin{equation}
		X = Y \iff X_{p} = Y_{p} \qcomma \forall p \in M
	\end{equation}

	considerando l'azione sui germi
	
	\begin{equation}
		\begin{aligned}
			X_{p} = Y_{p} \\
			\forall p \in M
		\end{aligned} %
		 \iff %
		 \begin{aligned}
			 X_{p} [g] &= Y_{p} [g] \\
			 \forall p \in M,& \quad \forall [g] \in C_{p}^{\infty}(M)
		 \end{aligned}
	\end{equation}

	prendendo un rappresentante
	
	\begin{equation}
		\begin{aligned}
			X_{p} [g] &= Y_{p} [g] \\
			\forall p \in M,& \quad \forall [g] \in C_{p}^{\infty}(M)
		\end{aligned}%
		\iff %
		\begin{aligned}
			X_{p} g &= Y_{p} g \\
			\forall p \in M,& \quad \forall g \in C^{\infty}(U)
		\end{aligned}
	\end{equation}
	
	dove $ g \in [g] $ e $ p \in U \subset M $. \\
	L'ipotesi però comprende funzioni qualsiasi $ f \in C^{\infty}(M) $ dunque, tramite il corollario di estensione di funzioni lisce\footnote{%
		Vedi Corollario \ref{cor:ext-smooth}.%
	}, consideriamo un aperto $ V \subset U $ che contenga $ p $ e una funzione $ \tilde{g} \in C^{\infty}(M) $ tale che $ \eval{\tilde{g}}_{V} \equiv g $. \\
	Applichiamo l'ipotesi a questa funzione ottenendo
	
	\begin{equation}
		X \tilde{g} = Y \tilde{g} \iff (X \tilde{g})(p) = (Y \tilde{g})(p) \qcomma \forall p \in M
	\end{equation}

	e dunque
	
	\begin{equation}
		X_{p} \tilde{g} = Y_{p} \tilde{g} \iff X_{p} g = Y_{p} g
	\end{equation}

	poiché $ \eval{\tilde{g}}_{V} \equiv g $ (equivalentemente $ \tilde{g} \sim g $). \\
	Seguendo le implicazioni al contrario, arriviamo dunque all'uguaglianza $ X = Y $.
\end{proof}

\paragraph{$ \R $-linearità}

\begin{definition}
	Sia un campo di vettori liscio $ X \in \chi(M) $, l'applicazione
	
	\map{\varphi(X)}
		{C^{\infty}(M)}{C^{\infty}(M)}
		{f}{X f}
	
	è $ \R $-lineare, i.e.
	
	\begin{equation}
		\varphi(X)(\lambda f + \mu g) = \lambda \varphi(X)(f) + \mu \varphi(X)(g) \qcomma \forall \lambda,\mu \in \R, \, \forall f,g \in C^{\infty}(M)
	\end{equation}
\end{definition}

\begin{proof}
	Preso un punto arbitrario $ p \in M $
	
	\begin{align}
		\begin{split}
			\varphi(X)(\lambda f + \mu g)(p) &= X(\lambda f + \mu g)(p) \\
			&= X_{p}(\lambda f + \mu g) \\
			&= \lambda X_{p} f + \mu X_{p} g \\
			&= \lambda (X f)(p) + \mu (X g)(p) \\
			&= (\lambda X f + \mu X g)(p) \\
			&= (\lambda \varphi(X)(f) + \mu \varphi(X)(g))(p)
		\end{split}
	\end{align}

	in quanto $ X_{p} \in T_{p}(M) = \der_{p}(C_{p}^{\infty}(M)) $ e quindi $ \R $-lineare.
\end{proof}

\paragraph{Regola di Leibniz}

I campi di vettori lisci rispettano la regola di Leibniz:

\begin{equation}
	X(f g) = (X f) g + f (X g) \qcomma \forall f,g \in C^{\infty}(M)
\end{equation}

\begin{proof}
	Preso un punto arbitrario $ p \in M $
	
	\begin{align}
		\begin{split}
			X(f g)(p) &= X_{p}(f g) \\
			&= (X_{p} f) \, g(p) + f(p) \, (X_{p} g) \\
			&= (X f)(p) \, g(p) + f(p) \, (X g)(p) \\
			&= ((X f) g + f (X g))(p)
		\end{split}
	\end{align}
	
	in quanto $ X_{p} \in T_{p}(M) = \der_{p}(C_{p}^{\infty}(M)) $ (i.e. si possono omettere le classi e considerare direttamente i rappresentanti di esse) e quindi rispetta la regola di Leibniz.
\end{proof}

\subsubsection{Derivazioni dell'algebra}

A questo punto, possiamo scrivere l'insieme delle derivazioni dell'algebra $ C^{\infty}(M) $

\begin{equation}
	\der(C^{\infty}(M)) = \{ D : C^{\infty}(M) \to C^{\infty}(M) \mid D \text{ è } \R-\text{lineare e soddisfa Leibniz} \}
\end{equation}

il quale è uno spazio vettoriale su $ \R $ con operazioni

\begin{equation}
	\begin{cases}
		(D_{1} + D_{2})(f) = D_{1}(f) + D_{2}(f), & \forall D_{1},D_{2} \in \der(C^{\infty}(M)) \\
		(\lambda D)(f) = \lambda (D f), & \forall \lambda \in \R, \, \forall D \in \der(C^{\infty}(M))
	\end{cases}
\end{equation}

La dimostrazione\footnote{%
	Vedi Esercizio \ref{exer1-8}.%
} è la stessa fatta per aperti di $ \R^{n} $. \\
Le derivazioni $ \der(C^{\infty}(M)) $ sono anche un $ C^{\infty}(M) $-modulo: il primo requisito è che siano uno spazio vettoriale e il secondo è che si possa moltiplicare una derivazione per una funzione liscia, come dalla seguente applicazione

\map{K}
	{C^{\infty}(M) \times \der(C^{\infty}(M))}{\der(C^{\infty}(M))}
	{(f,D)}{f D}

definendo l'azione dell'immagine di questa funzione come

\begin{equation}
	(f D)(g) \doteq f (D(g)) \qcomma \forall g \in C^{\infty}(M)
\end{equation}

in modo tale che $ K $ rispetti le seguenti proprietà

\begin{equation}
	\begin{cases}
		1_{C^{\infty}(M)} D = D & \text{ 1. elemento neutro (somma) } \\
		f (g D) = (f g) X & \text{ 2. compatibilità (moltiplicazione) } \\
		f (D_{1} + D_{2}) = f D_{1} + f D_{2} & \text{ 3. distributività (somma vettoriale) } \\
		(f + g) D = f D + g D & \text{ 4. distributività (somma scalare) }
	\end{cases}
\end{equation}

per qualsiasi $ f,g \in C^{\infty}(M) $ e qualsiasi $ D,D_{1},D_{2} \in \der(C^{\infty}(M)) $. \\
La dimostrazione\footnote{%
	Essendo $ \chi(M) $ un $ C^{\infty} $-modulo e tramite l'isomorfismo $ \chi(M) \simeq \der(C^{\infty}(M)) $ dimostrato nella sezione successiva, è sufficiente vedere la dimostrazione che $ \chi(U) $ sia un $ C^{\infty} $-modulo nell'Esercizio \ref{exer1-9}.%
} è la stessa fatta per aperti di $ \R^{n} $.

\subsubsection{Isomorfismo $ \chi(M) \simeq \der(C^{\infty}(M)) $}\label{ss-sec:iso-chi-der-man}

\begin{theorem}
	Sia l'applicazione
	
	\map{\varphi}
		{\chi(M)}{\der(C^{\infty}(M))}
		{X}{\varphi(X)}

	dove l'azione dell'immagine è quella dell'applicazione definita in precedenza per i campi di vettori, i.e.
	
	\begin{equation}
		\varphi(X)(f) = X f
	\end{equation}

	L'applicazione $ \varphi $ è un isomorfismo di $ C^{\infty}(M) $-moduli.
\end{theorem}

Per la dimostrazione, è necessario il seguente lemma:

\begin{lemma}
	Siano una derivazione $ D \in \der(C^{\infty}(M)) $ e un'applicazione liscia $ \tilde{f} \in C^{\infty}(M) $ tale che $ \eval{\tilde{f}}_{U} = 0 $ dove $ U \subset M $ è un aperto della varietà, allora
	
	\begin{equation}
		\eval{(D \tilde{f})}_{U} = 0 \in C^{\infty}(M)
	\end{equation}
\end{lemma}

\begin{proof}[Dimostrazione (lemma)]
	Siano $ p \in U $ un punto e $ \rho \in C^{\infty}(M) $ una funzione a campana liscia in $ p $ con supporto in $ U $, i.e. $ \supp(\rho) \subset U $, (esiste per il Teorema \ref{thm:bump-fun}). Consideriamo la funzione $ f = \rho \tilde{f} \in C^{\infty}(M) $ (le funzioni sono moltiplicate tra loro) la quale è nulla dappertutto perché in $ U $ abbiamo che $ \eval{\tilde{f}}_{U} = 0 $ mentre in $ M \setminus U $ abbiamo che $ \eval{\rho}_{M \setminus U} = 0 $. Applichiamo ora $ D $ a $ f $, il cui risultato sarà nullo in quanto $ f \equiv 0 $ e $ D $ è lineare
	
	\begin{equation}
		0 = D f %
		= D (\rho \tilde{f}) %
		= (D \rho) \tilde{f} + \rho (D \tilde{f})
	\end{equation}

	in quanto $ D $ soddisfa anche Leibniz. \\
	Preso un punto $ p \in U $, abbiamo che $ 0(p) = 0 \in \R $ e
	
	\begin{align}
		\begin{split}
			0(p) &= (D \rho)(p) \, \tilde{f}(p) + \rho(p) \, (D \tilde{f})(p) \\
			0 &= (D \rho)(p) \, 0 + 1 \, (D \tilde{f})(p) \\
			&= (D \tilde{f})(p)
		\end{split}
	\end{align}

	perciò
	
	\begin{equation}
		(D \tilde{f})(p) = 0 \qcomma \forall p \in U %
		\implies %
		\eval{(D \tilde{f})}_{U} = 0
	\end{equation}
\end{proof}

\begin{proof}[Dimostrazione (teorema $ \varphi $ isomorfismo)]
	Perché $ \varphi $ sia un isomorfismo, dobbiamo dimostrare che l'applicazione sia iniettiva, suriettiva e che
	
	\begin{equation}
		\varphi(g X + h Y) = g \varphi(X) + h \varphi(Y) \qcomma \forall g,h \in C^{\infty}(M), \, \forall X,Y \in \chi(M)
	\end{equation}

\begin{itemize}
	\item Perché sia iniettiva, siccome è $ \R $-lineare, è necessario e sufficiente che il nucleo dell'applicazione
	
	\begin{equation}
		\ker(\varphi) = \{ X \in \chi(M) \mid \varphi(X) = 0 \in \der(C^{\infty}(M)) \}
	\end{equation}

	dove
	
	\map{0}
		{C^{\infty}(M)}{C^{\infty}(M)}
		{f}{0}

	contenga solo il campo di vettori nullo. Per gli elementi del nucleo vale
	
	\begin{equation}
		X \in \ker(\varphi) \iff \varphi(X)(f) = 0 \in C^{\infty}(M) \qcomma \forall f \in C^{\infty}(M)
	\end{equation}

	il quale accade per la condizione
	
	\begin{equation}
		\begin{aligned}
			\varphi(X)(f) = 0 \\
			\forall f \in C^{\infty}(M)
		\end{aligned} %
		\iff %
		\begin{aligned}
			& X f = 0 \\
			\forall f &\in C^{\infty}(M)
		\end{aligned}
	\end{equation}

	da cui, per il teorema di uguaglianza, $ X = 0 \in \chi(M) $ e dunque $ \varphi $ è iniettiva poiché
	
	\begin{equation}
		\ker(\varphi) = \{ X = 0 \in \chi(M) \}
	\end{equation}

	%
	
	\item Perché sia suriettiva, è necessario che per qualunque derivazione $ D $ esista un campo di vettori $ X $ tale che l'immagine del campo tramite $ \varphi $ sia la derivazione $ D $, i.e.
	
	\begin{equation}
		\forall D \in \der(C^{\infty}(M)), \, \E X \in \chi(M) \, \mid \, \varphi(X)(f) = D f \qcomma \forall f \in C^{\infty}(M)
	\end{equation}
	
	Definiamo l'applicazione
	
	\map{X}
		{M}{T(M)}
		{p}{X_{p}}

	per cui
	
	\begin{equation}
		X_{p} [f] \doteq (D \tilde{f})(p)
	\end{equation}

	dove $ [f] \in C_{p}^{\infty}(M) $ di cui prendiamo un rappresentante $ f \in [f] $ il quale è liscio in un aperto $ U \subset M $, i.e. $ f \in C^{\infty}(U) $, di cui a sua volta prendiamo l'estensione $ \tilde{f} \in C^{\infty}(M) $ tale che $ \eval{\tilde{f}}_{V} \equiv f $ con $ p \in V \subset U $. Dobbiamo verificare che $ X_{p} $ sia ben definito, che $ X $ sia un campo di vettori liscio e che $ \varphi(X) = D $:
	
	\begin{itemize}
		\item Perché $ X_{p} $ sia ben definito, prendiamo $ g \in [f] $ (equivalentemente $ g \sim f $) tale che $ g \in C^{\infty}(U') $ con $ U' \ni p $. Prendiamone ora l'estensione $ \tilde{g} \in C^{\infty}(M) $ tale che $ \eval{\tilde{g}}_{V'} \equiv g $ con $ p \in V' \subset U' $ e verifichiamo che $ (D \tilde{g})(p) = (D \tilde{f})(p) $, il quale prova che la scelta del rappresentante per $ X_{p} [f] $ sia irrilevante. Siccome $ \tilde{g} - \tilde{f} \equiv 0 $ in un intorno di $ p $, abbiamo che (per il lemma dimostrato sopra e per la linearità di $ D $)
		
		\begin{align}
			\begin{split}
				(D (\tilde{g} - \tilde{f}))(p) &= 0 \\
				(D \tilde{g})(p) &= (D \tilde{f})(p)
			\end{split}
		\end{align}
		
		e dunque $ X_{p} $ è ben definito.
		
		\item Verifichiamo prima che $ X $ sia un campo di vettori, i.e. $ X_{p} \in \der_{p}(C_{p}^{\infty}(M)) = T_{p}(M) $ per qualsiasi $ p \in M $, e poi che $ X \in \chi(M) $. Perché sia un campo di vettori, è necessario che sia $ \R $-lineare: presa la classe $ [\lambda f + \mu g] $ ne consideriamo un'estensione da un aperto in cui entrambe le funzioni sono definite, da cui otteniamo
		
		\begin{align}
			\begin{split}
				X_{p} [\lambda f + \mu g] &= X (\lambda \tilde{f} + \mu \tilde{g})(p) \\
				&= (\lambda X \tilde{f} + \mu X \tilde{g})(p) \\
				&= \lambda (X \tilde{f})(p) + \mu (X \tilde{g})(p) \\
				&= \lambda X_{p} [f] + \mu X_{p} [g]
			\end{split}
		\end{align}
	
		Dobbiamo poi verificare che soddisfi la regola di Leibniz:
		
		\begin{align}
			\begin{split}
				X_{p} ([f] [g]) &= X_{p} ([f g]) \\
				&= X (\tilde{f g})(p) \\
				&= X (\tilde{f} \tilde{g})(p) \\
				&= X (\tilde{f})(p) \, \tilde{g}(p) + \tilde{f}(p) \, X (\tilde{g})(p) \\
				&= (X_{p} [f]) \, g(p) + f(p) \, (X_{p} [g])
			\end{split}
		\end{align}
	
		dunque $ X_{p} \in T_{p}(M) $; per mostrare che $ X \in \chi(M) $ mostriamo che $ X f \in C^{\infty}(M) $ per qualsiasi $ f \in C^{\infty}(M) $
		
		\begin{equation}
			(X f)(p) = X_{p} f %
			= X_{p} [f] %
			\doteq (D f)(p) \in C^{\infty}(M) %
			\qcomma \forall p \in M
		\end{equation}
	
		da cui otteniamo inoltre che $ X f = D f $.
		
		\item A questo punto, possiamo scrivere
		
		\begin{equation}
			\varphi(X)(f) = X f = D f \qcomma \forall f \in C^{\infty}(M) \implies \varphi(X) = D
		\end{equation}
	\end{itemize}

	%
	
	\item Per l'ultima condizione, sia $ f \in C^{\infty}(M) $
	
	\begin{align}
		\begin{split}
			\varphi(g X + h Y)(f) &= (g X + h Y)(f) \\
			&= (g X)(f) + (h Y)(f) \\
			&= g \, (X f) + h \, (Y f) \\
			&= g \, \varphi(X)(f) + h \, \varphi(Y)(f) \\
			&= (g \, \varphi(X) + h \, \varphi(Y))(f)
		\end{split}
	\end{align}
\end{itemize}

Abbiamo dunque che $ \varphi $ è un isomorfismo.
\end{proof}

\begin{lemma}\label{lemma:chi-subman-restr}
	Siano $ N \subset M $ una sottovarietà di $ M $ e $ X \in \chi(M) $ un campo di vettori liscio tale che $ X_{p} \in T_{p}(N) $ per qualsiasi $ p \in N $, allora $ \eval{X}_{N} \in \chi(N) $.
\end{lemma}

\begin{proof}
	Per ipotesi
	
	\map{\eval{X}_{N}}
		{N}{T(N)}
		{p}{X_{p} \in T_{p}(N)}
	
	dunque $ \eval{X}_{N} $ è un campo di vettori in $ N $, dobbiamo ora dimostrare che questo sia liscio in $ N $. \\
	Per verificarlo, consideriamo il fatto che $ N $ sia una sottovarietà di $ M $, la quale implica che $ T(N) $ sia una sottovarietà\footnote{%
		Vedi Esercizio \ref{exer2-25}.%
	} di $ T(M) $ con $ T(N) \subset T(M) $. La composizione $ F $ tra l'inclusione $ i : N \to M $ (la quale è un'immersione) e il campo di vettori $ X $

	\begin{equation}
		F \doteq X \circ i : N \to T(M)
	\end{equation}

	è un'applicazione liscia perché composizione di applicazioni lisce e $ F(N) \subset T(N) $, in quanto ogni punto di $ N $ viene prima mandato nello stesso punto in $ M $ tramite l'inclusione e poi, tramite $ X $, nel vettore $ X_{p} \in T_{p}(N) $. \\
	Sapendo che la restrizione del codominio di una funzione liscia a una sottovarietà del codominio originale produce ancora una funzione liscia\footnote{%
		Vedi Teorema \ref{thm:smooth-restriction-subman}.%
	}, consideriamo la restrizione $ \tilde{F} : N \to T(N) $ con $ \tilde{F}(p) = F(p) $, per la quale vale $ \tilde{F} \equiv \eval{X}_{N} $, rendendo dunque liscia la restrizione del campo. \\
	Questo prova che $ \eval{X}_{N} \in \chi(N) $.
\end{proof}

\subsubsection{\textit{Esempi}}

Tramite il lemma dimostrato sopra, è possibile trovare campi di vettori lisci sulla sfera $ \S^{n} $ semplicemente prendendo campi di vettori lisci in $ \R^{n+1} $ e verificando che questi siano ancora campi di vettori nella sfera, la quale è una sottovarietà di $ \R^{n+1} $.

\paragraph{1) Sfera di dimensione dispari}

Sia la sfera di dimensione dispari $ \S^{2n-1} \subset \R^{2n} $ con $ n > 0 $. Avendo le coordinate $ (x^{1},\dots,x^{n},y^{1},\dots,y^{n}) $ su $ \R^{2n} $, abbiamo che

\begin{equation}
	\S^{2n-1} = \{ (x^{1},\dots,x^{n},y^{1},\dots,y^{n}) \in \R^{2n} \mid (x^{1})^{2} + \cdots + (x^{n})^{2} + (y^{1})^{2} + \cdots + (y^{n})^{2} = 1 \}
\end{equation}

Siano il campo di vettori $ T(\R^{2n}) $ su $ \R^{2n} $ e la sua forma dopo l'identificazione con $ \R^{2n} $:

\begin{equation}
	X = \sum_{j=1}^{n} - y^{j} \pdv{x^{j}} + x^{j} \pdv{y^{j}} %
	\equiv (-y^{1},\dots,-y^{n},x^{1},\dots,x^{n})
\end{equation}

Per questo campo vale $ X \in \chi(\R^{2n}) $ e inoltre è possibile restringerlo alla sfera $ \S^{2n-1} $

\map{\eval{X}_{\S^{2n-1}}}
	{\S^{2n-1}}{T(\S^{2n-1})}
	{p}{X_{p} \in T_{p}(\S^{2n-1})}

dove $ p = (x^{1},\dots,x^{n},y^{1},\dots,y^{n}) \in \S^{2n-1} $ e gli $ X_{p} $ sono i vettori ortogonali al vettore normale all'ipersuperficie della sfera, i.e.

\begin{equation}
	X_{p} \cdot p = (-y^{1},\dots,-y^{n},x^{1},\dots,x^{n}) \cdot (x^{1},\dots,x^{n},y^{1},\dots,y^{n}) = 0 %
	\qcomma \forall p \in \S^{2n-1}
\end{equation}

è dunque vero che $ X_{p} \in T_{p}(\S^{2n-1}) $. \\
Per il lemma, essendo $ \S^{2n-1} \subset \R^{2n} $ una sottovarietà di $ \R^{2n} $, per il campo di vettori

\begin{equation}
	X : \S^{2n-1} \to T(\S^{2n-1})
\end{equation}

vale dunque $ X \in \chi(\S^{2n-1}) $.

\paragraph{2) Sfera di dimensione pari}

Non è possibile costruire un campo di vettori lisci su una sfera di dimensione pari in quanto non è possibile costruirne nemmeno uno continuo. \\
Presa la sfera di dimensione pari $ \S^{2n} $, non esiste dunque un campo di vettori $ X \in \chi(\S^{2n}) $ tale che $ X_{p} \neq 0 $ per qualsiasi $ p \in \S^{2n} $ o, equivalentemente, la sfera di dimensione pari non è \textit{pettinabile}.

\subsubsection{Parallelizzabilità}

Una varietà $ M $ di dimensione $ n $ è \textit{parallelizzabile} se esistono $ n $ campi di vettori lisci $ X_{1},\dots,X_{n} \in \chi(M) $ tali che

\begin{equation}
	\B_{T_{p}(M)} = \{ (X_{1})_{p},\dots,(X_{n})_{p} \}
\end{equation}

sia una base per $ T_{p}(M) $ per qualsiasi $ p \in M $, i.e.

\begin{equation}
	T_{p}(M) = \ev{ (X_{i})_{p} }
\end{equation}

Siccome è necessario che tutti i vettori di $ \B_{T_{p}(M)} $ siano linearmente indipendenti (in particolare nessuno di questi deve essere nullo), le sfere di dimensione pari non sono parallelizzabili. \\
Le sfere di dimensione dispari che sono parallelizzabili sono solo tre\footnote{%
	Per la sfera $ \S^{1} $, vedi Esercizio \ref{BONUS2-4}.%
}: $ \S^{1} $, $ \S^{3} $ ed $ \S^{7} $. Le altre sfere di dimensione dispari hanno un numero di campi di vettori linearmente indipendenti diverso dalla dimensione della sfera considerata.

\paragraph{Sfera $ \S^{3} $}

Dimostriamo la parallelizzabilità della sfera $ \S^{3} \subset \R^{4} $. \\
Consideriamo i seguenti campi di vettori di $ \S^{3} $ e la loro identificazione in $ \R^{4} $ con coordinate $ (x^{1},x^{2},x^{3},x^{4}) $:

\begin{gather}
	X = - x^{2} \pdv{x^{1}} + x^{1} \pdv{x^{2}} + x^{4} \pdv{x^{3}} - x^{3} \pdv{x^{4}} %
	= (-x^{2}, +x^{1}, +x^{4}, -x^{3}) \\
	%
	Y = - x^{3} \pdv{x^{1}} - x^{4} \pdv{x^{2}} + x^{1} \pdv{x^{3}} + x^{2} \pdv{x^{4}} = (-x^{3}, -x^{4}, +x^{1}, +x^{2}) \\
	%
	Z = - x^{4} \pdv{x^{1}} + x^{3} \pdv{x^{2}} - x^{2} \pdv{x^{3}} + x^{1} \pdv{x^{4}} = (-x^{4}, +x^{3}, -x^{2}, +x^{1})
\end{gather}

dove $ X,Y,Z \in \chi(\S^{3}) $ per il lemma. \\
Questi campi sono perpendicolari alla normale alla sfera e linearmente indipendenti:

\begin{equation}
	\begin{cases}
		X \cdot p = Y \cdot p = Z \cdot p = 0 \in \R, & \forall p \in \S^{3} \\
		X \cdot Y = X \cdot Z = Y \cdot Z = 0 \in \chi(\S^{3})
	\end{cases}
\end{equation}

Costituiscono dunque una base per lo spazio tangente alla sfera, i.e.

\begin{equation}
	\B_{T_{p}(\S^{3})} = \{ X_{p}, Y_{p}, Z_{p} \} \qcomma \forall p \in \S^{3}
\end{equation}

\subsection{Curve integrali e flussi di campi di vettori}

Siano $ M $ una varietà differenziabile di dimensione $ n $, $ p \in M $ un suo punto e $ X \in \chi(M) $ un campo di vettori liscio sulla varietà, una curva liscia $ c : (a,b) \to M $ con $ 0 \in (a,b) \subset \R $ è una \textit{curva integrale per il campo} $ X $ \textit{che inizia in} $ p $ se

\begin{equation}
	\begin{cases}
		c'(t) = X_{c(t)}, & \forall t \in (a,b) \\
		c(0) = p
	\end{cases}
\end{equation}

cioè il vettore tangente $ c'(t) $ alla curva $ c $ in ogni suo punto $ c(t) $ è uguale al vettore del campo di vettori $ X $ calcolato nello stesso punto $ X_{c(t)} $. \\
Siano una carta

\begin{equation}
	(U,\varphi) = (U; x^{1},\dots,x^{n}) \in M \qcomma \varphi(p) = 0
\end{equation}

e l'espressione locale del campo di vettori

\begin{equation}
	\eval{X}_{U} = \sum_{i=1}^{n} a^{i} \, \pdv{x^{i}} \qcomma a^{i} \in C^{\infty}(U)
\end{equation}

possiamo dunque scrivere l'equazione del sistema in forma locale

\begin{align}
	\begin{split}
		c'(t) &= X_{c(t)} \\ \\
		\sum_{i=1}^{n} \dot{c}^{i}(t) \eval{ \pdv{x^{i}} }_{c(t)} &= \sum_{i=1}^{n} a^{i}(c(t)) \eval{ \pdv{x^{i}} }_{c(t)} \\ \\
		\dot{c}^{i}(t) &= a^{i}(c(t))
	\end{split}
\end{align}

da cui il sistema di equazioni differenziali ordinarie (ODE) per $ i=1,\dots,n $

\begin{equation}
	\begin{cases}
		\dot{c}^{i}(t) = a^{i}(c(t)), & \forall t \in (a,b) \\
		c^{i}(0) = p^{i} = x^{i} (p)
	\end{cases}
\end{equation}

L'esistenza e l'unicità di una curva integrale per il campo $ X $ deriva dalla soluzione $ c(t) = (c^{1}(t),\dots,c^{n}(t)) $ a questo sistema, dimostrata tramite il seguente teorema:

\begin{theorem}[Esistenza e unicità della soluzione di un sistema di ODE (analisi)]
	Siano un aperto $ V \subset \R^{n} $, un punto $ p \in V $ e un'applicazione liscia $ f : V \to \R^{n} $, allora il sistema di ODE
	
	\begin{equation}
		\begin{cases}
			\dot{y}(t) = f(c(t)) \\
			y(0) = p
		\end{cases}
	\end{equation}
	
	dove $ y(t) = (y^{1}(t),\dots,y^{n}(t)) $, ha una soluzione massimale liscia unica
	
	\begin{equation}
		y : (a(p),b(p)) \to V
	\end{equation}

	i.e. presa un'altra soluzione del sistema $ z : (d,e) \to V $ che soddisfa
	
	\begin{equation}
		\begin{cases}
			\dot{z}(t) = f(z(t)) \\
			z(0) = p
		\end{cases}
	\end{equation}

	allora
	
	\begin{equation}
		\begin{cases}
			(d,e) \subseteq (a(p),b(p)) \\
			z = \eval{y}_{(d,e)}
		\end{cases}
	\end{equation}

	 La condizione di soluzione "massimale" indica che qualunque soluzione è definita all'interno dell'intervallo $ (a(p),b(p)) $.
\end{theorem}

Applichiamo ora questo teorema al sistema seguente

\begin{equation}
	\begin{cases}
		\dot{c}^{i}(t) = a^{i}(c(t)), & \forall t \in (a,b) \\
		c^{i}(0) = p^{i}
	\end{cases}
\end{equation}

per $ i=1,\dots,n $: grazie al teorema, sappiamo che esiste ed è unica la soluzione locale, i.e. una curva integrale $ c(t) $ per il campo di vettori $ X $. \\
Prendendo diverse carte della varietà, possiamo trovare una curva integrale per ogni aperto di esse: essendo la curva integrale unica per ogni carta, l'intersezione di aperti delle carte avrà come curva integrale la stessa curva, perciò esiste ed è unica la curva integrale massimale per un campo di vettori.

\img{0.6}{img52}

\begin{theorem}[Esistenza e unicità di curve integrali]
	Siano un campo di vettori liscio $ X \in \chi(M) $ e un punto $ p \in M $, allora esiste ed è unica la curva integrale massimale $ c : (a(p),b(p)) \to M $ per il campo $ X $, i.e.
	
	\begin{equation}
		\begin{cases}
			c'(t) = X_{c(t)} & \forall t \in (a,b) \\
			c(0) = p
		\end{cases}
	\end{equation}
\end{theorem}

La curva integrale per un campo dipende dal punto $ p $ della varietà scelto: il seguente teorema asserisce che questa dipendenza è liscia

\begin{theorem}
	Siano un aperto $ V \in \R^{n} $ e una funzione liscia $ f : V \to \R^{n} $, allora per qualunque punto $ p \in V $ esistono un intorno aperto $ W \subset V $ di $ p $, $ \varepsilon > 0 $ e una funzione liscia
	
	\begin{equation}
		F : (-\varepsilon,\varepsilon) \times W \to V
	\end{equation}

	tali che
	
	\begin{equation}
		\begin{cases}
			\dot{F}(t,q) = f(F(t,q)), & \forall (t,q) \in (-\varepsilon,\varepsilon) \times W \\
			F(0,q) = q
		\end{cases}
	\end{equation}

	dove il punto indica la derivata rispetto a $ t $.
\end{theorem}

Questo teorema può essere trasposto sulle varietà:

\begin{theorem}\label{thm:flux-var}
	Siano un campo di vettori liscio $ X \in \chi(M) $ e un punto $ p \in M $, allora esistono un aperto $ W \subset M $ di $ p $, $ \varepsilon > 0 $ e una funzione liscia
	
	\begin{equation}
		F : (-\varepsilon,\varepsilon) \times W \to V \subset M
	\end{equation}

	tali che
	
	\begin{equation}
		\begin{cases}
			F'(t,q) = X_{F(t,q)}, & \forall (t,q) \in (-\varepsilon,\varepsilon) \times W \\
			F(0,q) = q
		\end{cases}
	\end{equation}
\end{theorem}

L'applicazione $ F(t,q) $ è chiamata \textit{flusso locale del campo di vettori} $ X $; utilizzeremo la notazione $ F_{t}(q) \doteq F(t,q) $. \\
La \textit{linea di flusso in} $ q $ è definita come $ F_{t}(q) $ al variare di $ t \in (-\varepsilon,\varepsilon) $ con $ q $ fissato: presa la curva integrale di $ X $ che inizia in $ q $ indicata come

\begin{equation}
	F(\cdot,q) : (-\varepsilon,\varepsilon) \to V
\end{equation}

la sua immagine è la linea di flusso. Questa curva integrale non è massimale per $ X $: il dominio non dipende dal punto ma è scelto perché la curva soddisfi la condizione per qualsiasi $ q \in W $. \\
Se il flusso di $ X $ è \textit{globale}, i.e. definito in $ \R \times M $

\begin{equation}
	F : \R \times M \to M
\end{equation}

il campo di vettori $ X $ si dice \textit{completo}.

\begin{definition}
	Siano $ X \in \chi(M) $ un campo di vettori liscio e $ F_{t}(q) $ il suo flusso locale, e supponiamo che siano definiti $ F_{t} $, $ F_{s} $ e $ F_{t+s} $, allora
	
	\begin{equation}
		F_{t}(F_{s}(q)) = (F_{t} \circ F_{s})(q) = F_{t+s}(q)
	\end{equation}
\end{definition}

\begin{proof}
	Fissando il parametro $ s $ e il punto $ q $ e facendo variare il parametro $ t $, per la definizione di flusso possiamo scrivere
	
	\begin{equation}
		\begin{cases}
			F'_{t}(F_{s}(q)) = X_{F_{t}(F_{s}(q))}, & \forall t \in (-\varepsilon,\varepsilon) \\
			F_{0}(F_{s}(q)) = F(0,F_{s}(q)) = F_{s}(q)
		\end{cases}
	\end{equation}

	d'altra parte, ponendo $ u = t+s $
	
	\begin{equation}
		F'_{t+s}(q) = \dv{u} F_{u}(q) = F'_{u}(q)
	\end{equation}

	dunque
	
	\begin{equation}
		\begin{cases}
			F'_{t+s}(q) = F'_{u}(q) = X_{F_{u}(q)} = X_{F_{t+s}(q)}, & \forall t \in (-\varepsilon,\varepsilon) \\
			F_{0+s}(q) = F_{s}(q)
		\end{cases}
	\end{equation}

	siccome sia $ F_{t}(F_{s}(q)) $ che $ F_{t+s}(q) $ sono curve integrali per $ X $ e iniziano nello stesso punto $ F_{s}(q) $, per il teorema di unicità delle curve integrali
	
	\begin{equation}
		(F_{t} \circ F_{s})(q) = F_{t+s}(q)
	\end{equation}
\end{proof}

Nel caso in cui un campo di vettori $ X $ ammetta un flusso globale $ F : \R \times M \to M $, abbiamo che

\begin{equation}
	F_{t} \circ F_{s} = F_{t+s} \qcomma \forall t,s \in \R
\end{equation}

In particolare, se $ s = -t $ abbiamo che

\begin{equation}
	F_{t} \circ F_{-t} = F_{0} = \id_{M}
\end{equation}

i.e. $ F_{t} $ è invertibile con inversa $ (F_{t})^{-1} = F_{-t} $ e dunque $ F_{t} : M \to M $ è un diffeomorfismo per qualsiasi $ t \in \R $: da questo è possibile ottenere un \textit{gruppo di diffeomorfismi a un parametro} tramite l'applicazione

\map{G}
	{\R}{\operatorname{Diff}(M)}
	{t}{F_{t}}

dove $ \operatorname{Diff}(M) $ indica l'insieme dei diffeomorfismi su $ M $.

\subsubsection{\textit{Esempi}}

\paragraph{1) Campo di vettori completo}

Sia il campo di vettori liscio $ X \in \chi(\R^{2}) $ con la sua identificazione in $ \R^{2} $ definito come

\begin{equation}
	X = - y \, \pdv{x} + x \, \pdv{y} \equiv (-y,x)
\end{equation}

L'obiettivo è trovare la curva integrale passante per un generico punto $ p = (p^{1},p^{2}) \in \R^{2} $ e il flusso di $ X $. \\
Sia $ c : (a,b) \to \R^{2} $ una curva integrale per $ X $ con

\begin{equation}
	\begin{cases}
		c(t) = (x(t),y(t)) \\
		c(0) = p \\
		c'(t) = \dot{x}(t) \eval{ \dpdv{x} }_{c(t)} + \dot{y}(t) \eval{ \dpdv{y} }_{c(t)} \equiv (\dot{x}(t),\dot{y}(t))		
	\end{cases}
\end{equation}

per trovare la forma della curva, eguagliamo il vettore tangente alla curva con il campo di vettori

\begin{align}
	\begin{split}
		c'(t) &= X_{c(t)} \\
		\dot{x}(t) \eval{ \dpdv{x} }_{c(t)} + \dot{y}(t) \eval{ \dpdv{y} }_{c(t)} &= - y(t) \eval{ \dpdv{x} }_{c(t)} + x(t) \eval{ \dpdv{y} }_{c(t)} \\
		(\dot{x}(t), \dot{y}(t)) &= (- y(t), x(t))
	\end{split}
\end{align}

da cui otteniamo il sistema di ODE

\begin{equation}
	\begin{cases}
		\dot{x}(t) = - y(t) \\
		\dot{y}(t) = x(t) \\
		x(0) = p^{1} \\
		y(0) = p^{2}
	\end{cases}
\end{equation}

derivando la prima equazione rispetto a $ t $ otteniamo

\begin{equation}
	\ddot{x}(t) = - \dot{y}(t) = - x(t) %
	\implies %
	\begin{cases}
		\ddot{x}(t) + x(t) = 0 \\
		x(0) = p^{1}
	\end{cases}
\end{equation}

la cui soluzione è

\begin{equation}
	\begin{cases}
		x(t) = A \cos(t) + B \sin(t), & A,B \in \R \\
		x(0) = p^{1} = A
	\end{cases} %
	\implies %
	x(t) = p^{1} \cos(t) + B \sin(t)
\end{equation}

per trovare l'altra soluzione

\begin{equation}
	\begin{cases}
		y(t) = - \dot{x}(t) = p^{1} \sin(t) - B \cos(t) \\
		y(0) = p^{2} = - B
	\end{cases} %
	\implies %
	\begin{cases}
		x(t) = p^{1} \cos(t) - p^{2} \sin(t) \\
		y(t) = p^{1} \sin(t) + p^{2} \cos(t)
	\end{cases}
\end{equation}

dunque

\begin{equation}
	c(t) = (p^{1} \cos(t) - p^{2} \sin(t), p^{1} \sin(t) + p^{2} \cos(t))
\end{equation}

o alternativamente possiamo scrivere $ c(t) $ come vettore colonna

\begin{equation}
	c(t) = \bmqty{ x(t) \\ \\ y(t) } %
	= \bmqty{
		\cos(t) & - \sin(t) \\ \\
		\sin(t) & \cos(t) %
		} %
		\bmqty{ p^{1} \\ \\ p^{2} }
\end{equation}

Questo significa che le curve integrali, al variare del punto $ p $, sono cerchi concentrici di raggio $ \sqrt{(p^{1})^{2} + (p^{2})^{2}} $, i.e. la distanza del punto $ p $ dall'origine. \\
Il flusso del campo di vettori $ X $ è globale

\map{F}
	{\R \times \R^{2}}{\R^{2}}
	{(t,q) = (t,q^{1},q^{2})}{ %
								F_{t}(q) = \bmqty{
													\cos(t) & - \sin(t) \\ \\
													\sin(t) & \cos(t) %
													} %
											\bmqty{ p^{1} \\ \\ p^{2} }
								}

dunque il campo $ X $ è completo. \\
Verifichiamo ora che valga la legge di composizione dei flussi:

\begin{align}
	\begin{split}
		F_{t+s} &= F_{t} \circ F_{s} \\ \\
		%
		\bmqty{
				\cos(t+s) & - \sin(t+s) \\ \\
				\sin(t+s) & \cos(t+s) %
				} %
		&= %
		\bmqty{
				\cos(t) & - \sin(t) \\ \\
				\sin(t) & \cos(t) %
				} %
		\bmqty{
				\cos(s) & - \sin(s) \\ \\
				\sin(s) & \cos(s) %
				}
	\end{split}
\end{align}

questo è vero anche perché le due matrici rappresentano rotazioni rispettivamente di angoli $ t $ ed $ s $ e queste rispettano la stessa legge di composizione dei flussi.

\paragraph{2) Campo di vettori non completo}

Sia il campo di vettori liscio $ X \in \chi(\R \setminus \{0\}) $

\begin{equation}
	X = x^{2} \, \pdv{x}
\end{equation}

L'obiettivo è trovare la curva integrale passante per un generico punto $ p \in \R \setminus \{0\} $ e il flusso di $ X $. \\
Sia $ c : (a,b) \to \R \setminus \{0\} $ una curva integrale per $ X $: chiamando $ c(t) = x(t) $, abbiamo che

\begin{equation}
	\begin{cases}
		\dot{x}(t) = x^{2}(t) \\
		x(0) = p
	\end{cases}
\end{equation}

risolvendo la ODE

\begin{equation}
	\dv{x}{t} = x^{2} %
	\implies %
	\dfrac{\dd{x}}{x^{2}} = \operatorname{dt} %
	\implies %
	- \dfrac{1}{x} = t + c \qcomma c \in \R
\end{equation}

da cui

\begin{equation}
	\begin{cases}
		x(t) = - \dfrac{1}{t + c} \\ \\
		x(0) = p = - \dfrac{1}{c}
	\end{cases} %
	\implies %
	c = - \dfrac{1}{p} %
	\implies %
	x(t) = - \dfrac{1}{t - \dfrac{1}{p}}
\end{equation}

La soluzione è dunque la curva integrale

\begin{equation}
	x(t) = \dfrac{p}{1 - t p}
\end{equation}

Sappiamo, dal teorema, che esiste un intervallo massimale su cui è definita (che deve contenere 0): è necessario che $ 1-tp \neq 0 $ perciò

\begin{equation}
	\begin{cases}
		c : \left( -\infty, \dfrac{1}{p} \right) \to \R \setminus \{0\}, & p > 0 \\ \\
		c : \left( -\dfrac{1}{p}, +\infty \right) \to \R \setminus \{0\}, & p < 0
	\end{cases}
\end{equation}

siccome non è possibile estendere il dominio a tutto $ \R $, il campo di vettori $ X $ non è completo. \\
Considerando $ p>0 $, i.e. $ t \in (-\infty, 1/p) $, e $ q \in \R \setminus \{0\} $, otteniamo il flusso locale

\map{F}
	{\left( -\infty, \dfrac{1}{p} \right) \times \R \setminus \{0\}}{\R \setminus \{0\}}
	{(t,q)}{\dfrac{q}{1 - t q}}

Supponendo che $ t+s $ sia definito, verifichiamo ora che valga la legge di composizione dei flussi:

\begin{align}
	\begin{split}
		(F_{t} \circ F_{s})(q) &= F_{t} \left( \dfrac{q}{1 - s q} \right) \\ \\
		&= \dfrac{\dfrac{q}{1 - s q}}{1 - t \left( \dfrac{q}{1 - s q} \right)} \\ \\
		&= \dfrac{q}{(1 - s q) \left(1 - \dfrac{t q}{1 - s q} \right)} \\ \\
		&= \dfrac{q}{1 - s q - t q} \\ \\
		&= \dfrac{q}{1 - (t+s) q} \\ \\
		&= F_{t+s}(q)
	\end{split}
\end{align}

\subsection{Commutatore tra due campi di vettori}

Consideriamo le identificazioni tra i campi di vettori e le derivazioni $ \chi(M) = \der(C^{\infty}(M)) $, e tra una funzione liscia e la sua derivata tramite un campo di vettori

\begin{equation}
	f \in C^{\infty}(M) \to X f \in C^{\infty}(M)
\end{equation}

Presi $ X,Y \in \chi(M) = \der(C^{\infty}(M)) $, l'applicazione liscia

\map{X Y}
	{C^{\infty}(M)}{C^{\infty}(M)}
	{f}{X(Y(f))}

è $ \R $-lineare, in quanto

\begin{equation}
	X Y (\lambda f + \mu g) = X (\lambda Y f + \mu Y g) %
	= \lambda (XY)(f) + \mu (XY)(g)
\end{equation}

ma $ XY \notin \der(C^{\infty}(M)) $ perché non rispetta la regola di Leibniz

\begin{align}
	\begin{split}
		(XY)(fg) &= X ((Yf) g + f (Y g)) \\
		&= X(Y(f)) \, g + Y(f) \, X(g) + X(f) \, Y(g) + f \, X(Y(g)) \\
		&\neq X(Y(f)) \, g + f \, X(Y(g))
	\end{split}
\end{align}

Definiamo quindi il \textit{commutatore tra due campi di vettori} come\footnote{%
	Alternativamente si può scrivere
	
	\maps{[\cdot,\cdot]}
		{\der(C^{\infty}(M)) \times \der(C^{\infty}(M))}{\der(C^{\infty}(M))}
		{(X,Y)}{XY-YX}
	
	in quanto abbiamo identificato gli insiemi $ \chi(M) = \der(C^{\infty}(M)) $.%
}

\map{[\cdot,\cdot]}
	{\chi(M) \times \chi(M)}{\chi(M)}
	{(X,Y)}{XY-YX}

Il commutatore è $ \R $-lineare e rispetta anche la regola di Leibniz, dunque è un campo di vettori liscio:

\begin{equation}
	[X,Y] \doteq XY-YX \in \der(C^{\infty}(M))
\end{equation}

Di seguito, la verifica per la regola di Leibniz:

\begin{align}
	\begin{split}
		[X,Y](fg) &= (XY-YX)(fg) \\
		&= (XY)(fg) - (YX)(fg) \\
		&= ( \, X((Yf) g + f (Y g)) \, ) - ( \, Y((Xf) g + f (X g)) \, ) \\
		&= ( \, X(Y(f)) \, g + Y(f) \, X(g) + X(f) \, Y(g) + f \, X(Y(g)) \, ) + \\
		& \hal - ( \, Y(X(f)) \, g + X(f) \, Y(g) + Y(f) \, X(g) + f \, Y(X(g)) \, ) \\
		%
		&= X(Y(f)) \, g + Y(f) \, X(g) + X(f) \, Y(g) + f \, X(Y(g)) + \\
		& \hal - Y(X(f)) \, g - X(f) \, Y(g) - Y(f) \, X(g) - f \, Y(X(g)) \\
		%
		&= X(Y(f)) \, g - Y(X(f)) \, g + f \, X(Y(g)) - f \, Y(X(g)) \\
		&= (XY-YX)(f) \, g + f \, (XY-YX)(g) \\
		&= [X,Y](f) \, g + f \, [X,Y](g)
	\end{split}
\end{align}

A questo punto, considerati un aperto $ U \subset M $, un suo punto $ p \in U $, due campi di vettori lisci $ X,Y \in \chi(U) $ e una funzione liscia $ f \in C^{\infty}(U) $, possiamo definire l'azione del commutatore tra i due campi come

\begin{equation}
	[X,Y]_{p}(f) \doteq X_{p} (Y(f)) - Y_{p} (X(f)) \in C^{\infty}(U)
\end{equation}

dove $ Y(f), X(f) \in C^{\infty}(U) $ e $ [X,Y] \in \chi(U) $. \\
Il commutatore $ [X,Y]_{p}(f) $ è ancora una funzione liscia in quanto:

\begin{itemize}
	\item $ \R $-lineare poiché lo sono $ X_{p} $ e $ Y_{p} $
	
	\begin{equation}
		[X,Y]_{p}(\lambda f + \mu g) = \lambda [X,Y]_{p}(f) + \mu [X,Y]_{p}(g)
	\end{equation}
	
	%
	
	\item rispetta la regola di Leibniz perché la rispetta $ [X,Y] \in \der(C^{\infty}(U)) $
	
	\begin{align}
		\begin{split}
			[X,Y]_{p}(fg) &= [X,Y](fg)(p) \\
			&= ([X,Y](f) \, g + f \, [X,Y](g))(p) \\
			&= [X,Y]_{p}(f) \, g(p) + f(p) \, [X,Y]_{p}(g)
		\end{split}
	\end{align}
	
	%
	
	\item $ [X,Y] $ è un campo di vettori liscio in quanto è liscia la composizione $ XY-YX $
		
	\begin{equation}
		[X,Y]_{p}(f) = [X,Y](f)(p) %
		= (XY-YX)(f)(p) \in C^{\infty}(U) \\
	\end{equation}
\end{itemize}

\begin{definition}
	Siano una varietà differenziabile $ M $ di dimensione $ n $, due campi di vettori lisci $ X,Y \in \chi(M) $ e due funzioni lisce $ f,g \in C^{\infty}(M) $:
	
	\begin{itemize}
		\item Vale la proprietà
		
		\begin{equation}
			[f X, g Y] = f g [X,Y] + (f \, X(g)) \, Y - (g \, Y(f)) \, X
		\end{equation}
	
		dove $ (f \, X(g)), (g \, Y(f)) \in C^{\infty}(M) $;
	
		\item Presa una carta
		
		\begin{equation}
			(U,\varphi) = (U; x^{1},\dots,x^{n}) \in M
		\end{equation}
		
		e le forme locali dei campi tramite la carta
		
		\sbs{0.45}{%
					\begin{equation}
						\eval{X}_{U} = \sum_{j=1}^{n} a^{j} \, \pdv{x^{j}}
					\end{equation}
					}
			{0.45}{%
					\begin{equation}
						\eval{Y}_{U} = \sum_{j=1}^{n} b^{j} \, \pdv{x^{j}}
					\end{equation}
					}
	
		con $ a^{j},b^{j} \in C^{\infty}(U) $ per $ j=1,\dots,n $, abbiamo che la forma locale del commutatore tra i campi è la seguente
		
		\begin{equation}
			\eval{[X,Y]}_{U} = \sum_{j,k=1}^{n} \left( a^{j} \, \pdv{b^{k}}{x^{j}} - b^{j} \, \pdv{a^{k}}{x^{j}} \right) \pdv{x^{k}}
		\end{equation}
	\end{itemize}
\end{definition}

\begin{proof}
	Per la prima proprietà:
	
	\begin{align}
		\begin{split}
			[f X, g Y](h) &= f X(g \, Y(h)) - g Y(f \, X(h)) \\
			&= f ( \, X(g) \, Y(h) + g \, XY(h) \, ) - g ( \, Y(f) \, X(h) + f \, YX(h) \, ) \\
			&= f \, X(g) \, Y(h) + f g \, XY(h) - g \, Y(f) \, X(h) - g f \, YX(h) \\
			&= f g (XY-YX)(h) + f \, X(g) \, Y(h) - g \, Y(f) \, X(h) \\
			&= ( \, f g [X,Y] + (f \, X(g)) \, Y - (g \, Y(f)) \, X \, )(h)
		\end{split}
	\end{align}

	in quanto $ fg \equiv gf $. \\
	Per al seconda, usando la proprietà dimostrata sopra:
	
	\begin{align}
		\begin{split}
			\eval{[X,Y]}_{U} &= \left[ \sum_{j=1}^{n} a^{j} \, \pdv{x^{j}} , \sum_{k=1}^{n} b^{k} \, \pdv{x^{k}} \right] \\
			&= \sum_{j,k=1}^{n} a^{j} b^{k} \cancelto{0}{\left[ \pdv{x^{j}} , \pdv{x^{k}} \right]} + \sum_{j,k=1}^{n} a^{j} \, \pdv{b^{k}}{x^{j}} \pdv{x^{k}} - \sum_{j,k=1}^{n} b^{k} \, \pdv{a^{j}}{x^{k}} \pdv{x^{j}} \\
			&= \sum_{j,k=1}^{n} a^{j} \, \pdv{b^{k}}{x^{j}} \pdv{x^{k}} - \sum_{j,k=1}^{n} b^{j} \, \pdv{a^{k}}{x^{j}} \pdv{x^{k}} \\
			&= \sum_{j,k=1}^{n} \left( a^{j} \, \pdv{b^{k}}{x^{j}} - b^{j} \, \pdv{a^{k}}{x^{j}} \right) \pdv{x^{k}}
		\end{split}
	\end{align}

	dove il commutatore nel secondo passaggio è nullo in quanto le derivate parziali (applicate a una qualunque funzione) commutano e, sempre nello stesso passaggio, nell'ultima sommatoria possiamo scambiare gli indici $ j $ e $ k $ in quanto muti.
\end{proof}

\subsection{Distribuzioni e teorema di Frobenius}

Sia $ M $ una varietà differenziabile di dimensione $ n $, una \textit{distribuzione} $ r $\textit{-dimensionale} ($ r \leqslant n $) è un'assegnazione di ogni punto $ p \in M $ a un sottospazio $ D_{p} \subseteq T_{p}(M) $ tale che $ \dim(D_{p}) = r $, i.e. l'immagine di una distribuzione è una famiglia di sottospazi dello spazio tangente punto per punto. \\
Una distribuzione $ r $-dimensionale è \textit{liscia} se per qualunque punto $ p \in M $ esistono un intorno aperto $ U \subset M $ di $ p $ e un'insieme di $ r $ campi di vettori lisci $ X_{1},\dots,X_{r} \in \chi(M) $ tali che

\begin{equation}
	D_{p} = \ev{ (X_{1})_{p}, \dots, (X_{r})_{p} }_{\R}
\end{equation}

Una distribuzione è liscia perché varia in tale modo rispetto al punto a cui è legata. \\
Sia $ D $ una distribuzione liscia $ r $-dimensionale, una \textit{varietà integrale} di $ D $ è una varietà differenziabile $ S \subset M $ (non necessariamente una sottovarietà) tale che

\begin{equation}
	T_{p}(S) = D_{p} \qcomma \forall p \in S
\end{equation}

con $ \dim(S) = r $, cioè che ha, in ogni suo punto, come spazio tangente una distribuzione liscia. \\
Una distribuzione liscia $ r $-dimensionale $ D $ è \textit{completamente integrabile} se per qualsiasi $ q \in M $ esiste una varietà integrale $ S $ per $ D $ tale che $ q \in S $, cioè se esiste una varietà integrale per ogni punto della varietà. \\ \\
%
Ad esempio, sia un campo di vettori $ X \in \chi(M) $ non nullo in un aperto $ U $, i.e. $ X \neq 0 $ in $ U \subset M $, e consideriamo la distribuzione liscia $ 1 $-dimensionale

\begin{equation}
	D = \{ \lambda X \mid \lambda \in \R \} %
	\implies %
	D_{p} = \{ \lambda X_{p} \mid \lambda \in \R \}
\end{equation}

ciò significa che, per ogni punto, la distribuzione individua la retta nella direzione del vettore $ X_{p} $. \\
Questa distribuzione è completamente integrabile in quanto per qualsiasi $ q \in M $ esiste la curva integrale di $ X $ che passa per $ q $: questo significa che la curva integrale è anche una varietà integrale. \\ \\
%
Data una distribuzione $ D $, un campo di vettori $ X \in \chi(M) $ appartiene alla distribuzione se $ X_{p} \in D_{p} $ per qualsiasi $ p \in M $. Nella distribuzione dell'esempio precedente, $ f X \in D $ per qualsiasi $ f \in C^{\infty}(M) $ con $ f \neq 0 $.

\begin{theorem}[Frobenius]
	Sia $ D $ una distribuzione $ r $-dimensionale liscia su una varietà differenziabile $ M $, allora $ D $ è completamente integrabile se e solo se $ D $ è \textit{involutoria} (il commutatore di due campi di vettori appartiene alla distribuzione), i.e.
	
	\begin{equation}
		D \text{ completamente integrabile} %
		\iff %
		D \text{ involutoria} %
		\stackrel{\quad}{=\joinrel=} %
		[X,Y] \in D \qcomma \forall X,Y \in D
	\end{equation}
\end{theorem}

\begin{remark}
	Il teorema di Frobenius implica il teorema di esistenza di curve integrali per un campo di vettori.
\end{remark}

Se $ X_{p} = 0 $ allora la curva integrale per $ X $ che inizia in $ p $ è costante\footnote{%
	Vedi Esercizio \ref{exer2-32}.%
}. A questo punto, supponiamo che $ X_{p} \neq 0 $ e consideriamo un intorno $ U $ di $ p $ dove $ X_{q} \neq 0 $ per qualsiasi $ q \in U $ e la distribuzione liscia $ 1 $-dimensionale

\begin{equation}
	D = \{ \lambda X \mid \lambda \in \R \} \qcomma D_{p} = \ev{ X_{p} }_{\R}
\end{equation}

L'esistenza di una curva integrale che passa per $ p $ è verificata se $ D $ è involutoria, i.e. (per il teorema di Frobenius) $ D $ è completamente integrabile e quindi esiste la curva integrale ricercata. \\
Per dimostrare che $ D $ sia involutoria è necessario che, presi due campi di vettori della distribuzione $ Y,Z \in D $, il commutatore di questi appartenga ancora alla distribuzione, i.e. $ [Y,Z] \in D $: per la distribuzione considerata in precedenza

\begin{equation}
	Y,Z \in D \iff Y = f X \, \wedge \, Z = g X \qcomma f,g \in C^{\infty}(U)
\end{equation}

in quanto la moltiplicazione di un campo di vettori liscio per una funzione liscia è ancora un vettore liscio (nella stessa direzione). Calcolando ora il commutatore tra i due campi

\begin{align}
	\begin{split}
		[Y,Z] &= [fX,gX] \\
		&= fg [X,X] + (f \, X(g)) X - (g \, X(f)) X \\
		&= (f \, X(g) - g \, X(f)) X \\
		&\doteq h X \in D
	\end{split}
\end{align}

Questo implica dunque l'esistenza di una curva integrale passante per $ p $.

\subsubsection{\textit{Esempio}}

\paragraph{Distribuzione non integrabile}

Prendiamo $ \R^{3} $ e i campi di vettori lisci (con la loro identificazione)

\begin{equation}
	\begin{cases}
		X = z \, \dpdv{x} + \dpdv{z} \equiv (z,0,1) \\ \\
		Y = \dpdv{y} + \dpdv{z} \equiv (0,1,1)
	\end{cases}
\end{equation}

Consideriamo la distribuzione liscia 2-dimensionale

\begin{equation}
	D = \{ \lambda X + \mu Y \mid \lambda,\mu \in \R \}
\end{equation}

Se $ D $ fosse completamente integrabile, per ogni punto di $ \R^{3} $ esisterebbe una superficie il cui spazio tangente è generato da una combinazione lineare dei campi di vettori (definiti sopra) valutati in quello stesso punto. \\
Verifichiamo ora che non sia completamente integrabile: per farlo, è sufficiente esibire un controesempio, i.e. un commutatore tra i campi della distribuzione che non sia appartenente alla distribuzione. \\
Ricordando la proprietà

\begin{equation}
	[fX,gX] = fg [X,X] + (f \, X(g)) X - (g \, X(f)) X \qcomma \forall f,g \in C^{\infty}(M), \,  \forall X,Y \in \chi(M)
\end{equation}

calcoliamo il commutatore tra i campi $ X $ e $ Y $

\begin{align}
	\begin{split}
		[X,Y] &= \left[ z \, \pdv{x} + \pdv{z} , \pdv{y} + \pdv{z} \right] \\
		&= \left[ z \pdv{x}, \pdv{y} \right] + \left[ z \pdv{x}, \pdv{z} \right] + \cancel{\left[ \pdv{z}, \pdv{y} \right]} + \cancel{\left[ \pdv{z}, \pdv{z} \right]} \\
		&= \left[ z \pdv{x}, \pdv{y} \right] + \left[ z \pdv{x}, \pdv{z} \right] \\
		&= (z \cdot 1) \cancel{\left[ \pdv{x}, \pdv{y} \right]} + \left( z \, \cancel{ \pdv{1}{x} } \right) \pdv{y} - 1 \cdot \cancel{\left( \pdv{z}{y} \right)} \pdv{x} + \\
		& \hal + (z \cdot 1) \cancel{\left[ \pdv{x}, \pdv{z} \right]} + \left( z \, \cancel{\pdv{1}{x}} \right) \pdv{z} - 1 \cdot \cancelto{1}{\left( \pdv{z}{z} \right)} \pdv{x} \\
		%
		&= - \pdv{x} \notin D
	\end{split}
\end{align}

Questo prova che $ D $ non è involutoria e quindi, per il teorema di Frobenius, non è completamente integrabile.

\section{Pushforward di campi di vettori}

Siano due varietà differenziabili $ N $ e $ M $ di dimensione rispettivamente $ n $ ed $ m $, un'applicazione liscia $ F : N \to M $ e un campo di vettori liscio $ X \in \chi(N) $. Fissando un punto $ p \in N $, per associare un vettore del campo dallo spazio tangente di $ N $ in $ p $ a quello di $ M $ in $ F(p) $ possiamo usare il differenziale
%
\map{F_{*p}}
	{T_{p}(N)}{T_{F(p)}(M)}
	{X_{p}}{F_{*p}(X_{p})}

ma questo non può essere esteso a tutto il campo di vettori in quanto, se l'applicazione $ F $ non fosse iniettiva, esisterebbero due punti diversi $ p,q \in N $ tali che $ F(p) = F(q) $: a questo punto si presenta l'assegnazione del vettore in $ F(p) = F(q) $ e ci sarebbe ambiguità tra $ F_{*p}(X_{p}) $ e $ F_{*p}(X_{q}) $ e tra $ F_{*q}(X_{p}) $ e $ F_{*q}(X_{q}) $. Se l'applicazione fosse iniettiva ma non suriettiva, non esisterebbe un vettore $ X_{p} \in \chi(N) $ in $ p \in N $ la cui immagine attraverso il differenziale $ F_{*p} $ sia un vettore in un punto $ r \in M \setminus F(N) $ (dove $ F(N) \subsetneq M $ in quanto $ F $ non è suriettiva). \\
Supponiamo ora che l'applicazione $ F : N \to M $ sia un diffeomorfismo (quindi iniettiva e suriettiva) e definiamo il \textit{pushforward di} $ X $ tramite il diffeomorfismo $ F $ come

\begin{equation}
	(F_{*}(X))_{q} \doteq F_{*F^{-1}(q)} (X_{F^{-1}(q)}) \in T_{q}(M) \qcomma \forall q \in M
\end{equation}

La definizione utilizza $ F^{-1}(q) $ in quanto $ F $ è biettiva e dunque $ F^{-1}(q) $ è l'unico punto che viene mandato in $ q $ tramite $ F $.

\begin{definition}
	Il pushforward $ F_{*}(X) $ di un campo di vettori liscio $ X \in \chi(N) $ tramite il diffeomorfismo $ F : N \to M $ è un campo di vettori liscio su $ M $, i.e. $ F_{*}(X) \in \chi(M) $.
\end{definition}

\begin{proof}
	Siccome $ (F_{*} (X))_{q} \in T_{q}(M) $ per qualsiasi $ q \in N $, dobbiamo solo dimostrare che $ F_{*}(X) $ sia liscio, i.e.
	
	\begin{equation}
		(F_{*}(X))(g) \in C^{\infty}(M) \qcomma \forall g \in C^{\infty}(M)
	\end{equation}

	Prendendo $ q \in M $
	%
	\begin{align}
		\begin{split}
			(F_{*}(X))(g)(q) &= (F_{*}(X))_{q}(g) \\
			&\doteq F_{*F^{-1}(q)} (X_{F^{-1}(q)})(g) \\
			&\doteq X_{F^{-1}(q)} (g \circ F) \\
			&= X (g \circ F)(F^{-1}(q)) \\
			&= ((X (g \circ F)) \circ F^{-1})(q)
		\end{split}
	\end{align}

	dove nel secondo passaggio abbiamo usato la definizione di pushforward e nel terzo quella di differenziale; a questo punto $ (X (g \circ F)) \circ F^{-1} \in C^{\infty}(M) $ in quanto composizione di funzioni lisce.
\end{proof}

\subsection{Campi di vettori $ F $-related}

Siano $ F : N \to M $ una funzione liscia (non necessariamente un diffeomorfismo) e due campi di vettori lisci $ X \in \chi(N) $ e $ Y \in \chi(M) $, si dice che $ X $ è $ F $\textit{-related a} $ Y $ se

\begin{equation}
	F_{*p}(X_{p}) = Y_{F(p)} \qcomma p \in N
\end{equation}

\begin{remark}
	Se $ F $ è un diffeomorfismo, allora $ X $ è $ F $-related a $ Y $ se e solo se $ Y $ è il pushforward di $ X $, i.e. $ Y = F_{*}(X) $.
\end{remark}

\begin{proof}
	\begin{equation}
		Y_{F(p)} = F_{*p}(X_{p}) = (F_{*}(X))_{F(p)} \qcomma \forall p \in N %
		\implies%
		Y = F_{*}(X)
	\end{equation}
\end{proof}

\begin{theorem}
	Siano $ F : N \to M $ una funzione liscia e due campi di vettori lisci $ X \in \chi(N) $ e $ Y \in \chi(M) $, allora $ X $ è $ F $-related a $ Y $ se
	
	\begin{equation}
		X(g \circ F) = (Y g) \circ F \qcomma \forall g \in C^{\infty}(M)
	\end{equation}

	con $ g : M \to \R $ e quindi $ g \circ F \in C^{\infty}(N) $.
\end{theorem}

\begin{proof}
	Per definizione, $ X $ è $ F $-related a $ Y $ se e solo se
	
	\begin{equation}
		F_{*p}(X_{p}) = Y_{F(p)} \qcomma p \in N
	\end{equation}

	applicando la definizione a una funzione qualsiasi $ g \in C^{\infty}(M) $ e usando la definizione di differenziale, deve valere
	
	\begin{align}
		\begin{split}
			(F_{*p}(X_{p}))(g) &= Y_{F(p)}(g) \\
			X_{p}(g \circ F) &= (Y g)(F(p)) \\
			X(g \circ F)(p) &= ((Y g) \circ F)(p) \\
			X(g \circ F) &= (Y g) \circ F
		\end{split}
	\end{align}
\end{proof}

\begin{theorem}
	Siano $ F : N \to M $ una funzione liscia e i campi di vettori lisci $ X_{1},X_{2} \in \chi(N) $ e $ Y_{1},Y_{2} \in \chi(M) $ con $ X_{1} $ $ F $-related a $ Y_{1} $ e $ X_{2} $ $ F $-related a $ Y_{2} $, i.e.
	
	\begin{equation}
		\begin{cases}
			Y_{1} = F_{*}(X_{1}) \\
			Y_{2} = F_{*}(X_{2})
		\end{cases}
	\end{equation}

	allora il commutatore $ [X_{1},X_{2}] $ è $ F $-related a $ [Y_{1},Y_{2}] $.
\end{theorem}

In sostanza, questo teorema asserisce che il commutatore tra due campi viene preservato tramite campi $ F $-related.

\begin{proof}
	Usando il teorema precedente, $ [X_{1},X_{2}] $ è $ F $-related a $ [Y_{1},Y_{2}] $ se e solo se
	
	\begin{equation}
		[X_{1},X_{2}](g \circ F) = ([Y_{1},Y_{2}] g) \circ F \qcomma \forall g \in C^{\infty}(M)
	\end{equation}

	Sempre usando il teorema precedente
	%
	\begin{align}
		\begin{split}
			[X_{1},X_{2}](g \circ F) &= (X_{1} X_{2} - X_{2} X_{1})(g \circ F) \\
			&= (X_{1} X_{2})(g \circ F) - (X_{2} X_{1})(g \circ F) \\
			&= X_{1}((Y_{2} g) \circ F) - X_{2}((Y_{1} g) \circ F) \\
			&= Y_{1}(Y_{2} g) \circ F - Y_{1}(Y_{2} g) \circ F \\
			&= (Y_{1} Y_{2} - Y_{1} Y_{2})(g \circ F) \\
			&= [Y_{1},Y_{2}](g \circ F)
		\end{split}
	\end{align}
\end{proof}

\begin{corollary}\label{cor:f-rel-brack}
	Siano $ F : N \to M $ un diffeomorfismo e $ X_{1},X_{2} \in \chi(N) $ due campi di vettori lisci, allora il pushforward del commutatore è uguale al commutatore dei pushforward, i.e.
	
	\begin{equation}
		F_{*} ([X_{1},X_{2}]) = [F_{*} (X_{1}),F_{*} (X_{2})]
	\end{equation}

	Il pushforward commuta dunque con il commutatore.
\end{corollary}

\begin{proof}
	Per l'osservazione precedente, $ X_{i} $ è $ F $-related a $ F_{*}(X_{i}) $ per $ i=1,2 $ (poiché $ F $ è un diffeomorfismo). Per il teorema precedente $ [X_{1},X_{2}] $ è $ F $-related a $ [F_{*} (X_{1}),F_{*} (X_{2})] $ e ancora per l'osservazione precedente
	
	\begin{equation}
		F_{*} ([X_{1},X_{2}]) = [F_{*} (X_{1}),F_{*} (X_{2})]
	\end{equation}
\end{proof}


%

\chapter{Lie groups and algebras}
%\section{Gruppi di Lie}\label{sec:lie-groups}

Un gruppo $ (G,\mu) \equiv G $ è un \textit{gruppo di Lie} se:

\begin{itemize}
	\item L'insieme $ G $ del gruppo è una varietà differenziabile;
	
	\item $ (G,\mu) $ è un gruppo algebrico;
	
	\item sono lisce le sue operazioni di prodotto e inversione
\end{itemize}

\sbs{0.5}{%
			\map{\mu}
				{G \times G}{G}
				{(a,b)}{a b}
			}
	{0.5}{%
			\map{i}
				{G}{G}
				{a}{a^{-1}}
			}
		
Preso un elemento $ a \in G $, possiamo considerare delle restrizioni della moltiplicazione, chiamate traslazione a sinistra $ L_{a} $ e traslazione a destra $ R_{a} $:

\sbs{0.5}{%
			\map{L_{a}}
				{G}{G}
				{b}{a b}
			}
	{0.5}{%
			\map{R_{a}}
				{G}{G}
				{b}{b a}
			}

Queste sono lisce in quanto restrizioni di applicazioni lisce:

\begin{gather}
	L_{a} \doteq \mu(a,\cdot) = \eval{\mu}_{\{a\} \times G} \\
	R_{a} \doteq \mu(\cdot,a)  = \eval{\mu}_{G \times \{a\}}
\end{gather}

Sono inoltre diffeomorfismi, in quanto sono lisce le inverse:

\begin{gather}
	L_{a}^{-1} = L_{a^{-1}} \\
	R_{a}^{-1} = R_{a^{-1}}
\end{gather}

\subsubsection{\textit{Esempi}}

\paragraph{1) Gruppo lineare $ GL_{n}(\R) $}

Sia il gruppo delle matrici invertibili

\begin{equation}
	GL_{n}(\R) = \{ A \in M_{n}(\R) \mid \det(A) \neq 0 \} \subset M_{n}(\R) = \R^{n^{2}}
\end{equation}

Questo è aperto in $ M_{n}(\R) = \R^{n^{2}} $, dunque ha la struttura differenziale ereditata da quest'ultimo e perciò è un gruppo di Lie rispetto alla moltiplicazione e l'inversione definite come

\sbs{0.5}{%
			\map{\mu}
				{GL_{n}(\R) \times GL_{n}(\R)}{GL_{n}(\R)}
				{(A,B)}{A B}
			}
	{0.5}{%
			\map{i}
				{GL_{n}(\R)}{GL_{n}(\R)}
				{A}{A^{-1}}
		}

le quali sono lisce.

\paragraph{2) Gruppo lineare speciale $ SL_{n}(\R) $}

Il gruppo delle matrici con determinante unitario

\begin{equation}
	SL_{n}(\R) = \{ A \in M_{n}(\R) \mid \det(A) = 1 \} \subset M_{n}(\R) = \R^{n^{2}}
\end{equation}

è un gruppo di Lie rispetto alla moltiplicazione e l'inversione

\sbs{0.5}{%
			\map{\mu}
				{SL_{n}(\R) \times SL_{n}(\R)}{SL_{n}(\R)}
				{(A,B)}{A B}
			}
	{0.5}{%
			\map{i}
				{SL_{n}(\R)}{SL_{n}(\R)}
				{A}{A^{-1}}
			}

in quanto sottovarietà di $ GL_{n}(\R) $ (e quindi una varietà) di dimensione $ n^{2}-1 $ e gruppo algebrico rispetto alle sue operazioni, le quali sono lisce. \\
Dimostriamo ora che la moltiplicazione è liscia\footnote{%
	Questa dimostrazione è già stata fatta nell'Esempio \ref{example:slnr-smooth}.%
}: prendiamo la moltiplicazione (liscia) in $ GL_{n}(\R) $

\map{F}
	{GL_{n}(\R) \times GL_{n}(\R)}{GL_{n}(\R)}
	{(A,B)}{A B}

e l'inclusione liscia

\begin{equation}
	i \times i : SL_{n}(\R) \times SL_{n}(\R) \to GL_{n}(\R) \times GL_{n}(\R)
\end{equation}

L'applicazione $ G = F \circ (i \times i) $ è dunque liscia perché composizione di applicazione lisce. Siccome il prodotto di due matrici con determinante unitario (i.e. in $ SL_{n}(\R) $) è ancora una matrice con determinante unitario, vale per l'immagine di $ G $

\begin{equation}
	G(SL_{n}(\R) \times SL_{n}(\R)) \subset SL_{n}(\R)
\end{equation}

A questo punto, dato che $ SL_{n}(\R) $ è una sottovarietà di $ GL_{n}(\R) $ (ipotesi del Teorema \ref{thm:smooth-restriction-subman} che permette di mantenere l'applicazione liscia), possiamo considerare la restrizione del codominio della funzione $ G $

\map{\tilde{G}}
	{SL_{n}(\R) \times SL_{n}(\R)}{SL_{n}(\R)}
	{(A,B)}{A B}

la quale è identica a $ \mu $, dunque $ \mu \in C^{\infty}(SL_{n}(\R) \times SL_{n}(\R)) $. \\
Per dimostrare che l'inversione sia liscia, consideriamo l'inversione in $ GL_{n}(\R) $

\map{F}
	{GL_{n}(\R)}{GL_{n}(\R)}
	{A}{A^{-1}}

la quale è liscia e, analogamente per la moltiplicazione, definiamo $ G = F \circ i $

\map{G}
	{SL_{n}(\R)}{GL_{n}(\R)}
	{A}{A^{-1}}

e dunque la restrizione del suo codominio a $ SL_{n}(\R) $ (sottovarietà di $ GL_{n}(\R) $)

\map{\tilde{G}}
	{SL_{n}(\R)}{SL_{n}(\R)}
	{A}{A^{-1}}

funzione che coincide con l'inversione $ i $ in $ SL_{n}(\R) $, rendendola dunque liscia.

\paragraph{3) Gruppo ortogonale $ O(n) $}\label{example:on-group-lie}

Il gruppo delle matrici ortogonali

\begin{equation}
	O(n) = \{ A \in GL_{n}(\R) \mid A^{T} A = I \} \subset GL_{n}(\R)
\end{equation}

è una sottovarietà di $ GL_{n}(\R) $ (dal teorema della preimmagine di un'applicazione di rango costante). \\
Il ragionamento per cui $ O(n) $ sia un gruppo di Lie rispetto alla moltiplicazione e l'inversione

\sbs{0.5}{%
			\map{\mu}
				{O(n) \times O(n)}{O(n)}
				{(A,B)}{A B}
			}
	{0.5}{%
			\map{i}
				{O(n)}{O(n)}
				{A}{A^{-1}}
			}

è analogo a quello fatto per $ SL_{n}(\R) $. \\
Calcoliamo ora la dimensione di $ O(n) $ come varietà e il suo spazio tangente nell'identità $ T_{I}(O(n)) $. Consideriamo l'insieme delle matrici simmetriche di ordine $ n $

\begin{equation}
	S(n) = \{ B \in M_{n}(\R) \mid B^{T} = B \}
\end{equation}

e l'applicazione liscia

\map{f}
	{GL_{n}(\R)}{S(n)}
	{A}{A^{T} A}

dove

\begin{equation}
	(A^{T} A)^{T} = A^{T} A \implies %
	A^{T} A \in S(n) \qcomma \forall A \in GL_{n}(\R)
\end{equation}

$  $. Dimostrando che $ I \in \VR_{f} $, per il teorema della preimmagine di un valore regolare, avremo che $ O(n) $ è una sottovarietà di $ GL_{n}(\R) $ con dimensione

\begin{equation}
	\dim(O(n)) = \dim(GL_{n}(\R)) - \dim(S(n))
\end{equation}

dove $ S(n) $ è uno spazio vettoriale su $ \R $ e anche un sottospazio vettoriale di $ M_{n}(\R) $, la cui dimensione è data da $ n^{2} $ meno le condizioni necessarie al fine di determinare una matrice simmetrica, i.e.

\begin{equation}
	\dim(S(n)) = \dfrac{n^{2} - n}{2} + n = \dfrac{n (n+1)}{2}
\end{equation}

Siccome $ \dim(GL_{n}(\R)) = n^{2} $, otteniamo che

\begin{equation}
	\dim(O(n)) = n^{2} - \dfrac{n (n+1)}{2} = \dfrac{n (n-1)}{2}
\end{equation}

Verifichiamo ora che $ I \in \VR_{f} $: per fare ciò, dobbiamo calcolare il differenziale di $ f $ e controllare che tutti i punti che stanno nell'immagine di $ I $ attraverso $ f_{*} $ siano punti regolari

\begin{equation}
	f_{*A} : T_{A}(GL_{n}(\R)) \to T_{f(A)}(S(n))
\end{equation}

possiamo identificare $ T_{A}(GL_{n}(\R)) = M_{n}(\R) $ e $ T_{f(A)}(S(n)) = S(n) $ (in quanto $ S(n) $ è uno spazio vettoriale), perciò $ f_{*} $ porta matrici in matrici simmetriche. Calcoliamo dunque il differenziale $ f_{*A}(X) $ prendendo una curva che passi per $ A $ in 0 e il cui vettore tangente sia $ X $, i.e.

\begin{equation}
	\begin{cases}
		c : (-\varepsilon,\varepsilon) \to GL_{n}(\R) \\
		c(0) = A \\
		c'(0) = \dot{c}(0) = X
	\end{cases}
\end{equation}

dove

\begin{equation}
	T_{A}(GL_{n}(\R)) = M_{n}(\R) \implies c'(0) = \dot{c}(0)
\end{equation}

perciò

\begin{align}
	\begin{split}
		f_{*A}(X) &= \dv{t} (f \circ c) (0) \\
		&= \eval{ \dv{t} f(c(t)) }_{t=0} \\
		&= \eval{ \dv{t} \left( c(t)^{T} c(t) \right) }_{t=0} \\
		&= \dot{c}(0)^{T} c(0) + c(0)^{T} \dot{c}(0) \\
		&= X^{T} A + A^{T} X
	\end{split}
\end{align}

i.e.

\map{f_{*A}}
	{M_{n}(\R)}{S(n)}
	{X}{X^{T} A + A^{T} X}

Vale la condizione

\begin{equation}
	I \in \VR_{f} %
	\iff %
	f_{*A} : M_{n}(\R) \to S(n) \text{ suriettiva} \qcomma \forall A \in f^{-1}(I) = O_{n}
\end{equation}

Perché $ f_{*A} $ sia suriettiva

\begin{equation}
	\forall A \in O(n), \, \forall B \in S(n), \E X \in M_{n}(\R) \mid X^{T} A + A^{T} X = B
\end{equation}

è sufficiente dunque prendere $ X = A B / 2 $, i.e.

\begin{align}
	\begin{split}
		X^{T} A + A^{T} X &= \left( \dfrac{1}{2} A B \right)^{T} A + \dfrac{1}{2} A^{T} A B \\
		&= \dfrac{1}{2} B^{T} A^{T} A + \dfrac{1}{2} B \\
		&= \dfrac{1}{2} B^{T} + \dfrac{1}{2} B \\
		&= \dfrac{1}{2} B + \dfrac{1}{2} B \\
		&= B
	\end{split}
\end{align}

dunque $ I \in \VR_{f} $ e $ \dim(O(n)) = n(n-1)/2 $. Da questo ragionamento, otteniamo anche lo spazio tangente a $ O(n) $ in $ I $ poiché, sempre per il teorema della preimmagine di un valore regolare, vale la seguente uguaglianza

\begin{equation}
	T_{A}(f^{-1}(I)) = T_{A}(O(n)) = \ker(f_{*A}) \qcomma A \in f^{-1}(I)
\end{equation}

Siccome

\begin{equation}
	f_{*I}(X) = X^{T} + X
\end{equation}

abbiamo che

\begin{equation}
	T_{I}(O(n)) = \ker(f_{*I}) = \{ X \in M_{n}(\R) \mid X^{T} = -X \}
\end{equation}

ovvero lo spazio tangente di $ O(n) $ è formato dalle matrici antisimmetriche (spazio vettoriale), il quale ha dimensione esattamente $ \dim(T_{I}(O(n))) = n(n-1)/2 $.

\subsection{Omomorfismi e isomorfismi}

Un \textit{omomorfismo} di gruppi\footnote{%
	Omomorfismo e omeomorfismo sono due concetti differenti legati a due parti differenti della matematica.%
} è un'applicazione (non necessariamente liscia) che preserva le moltiplicazioni di un gruppo nell'altro, i.e. per un omomorfismo $ F : G \to H $ vale

\begin{equation}
	F(g_{1} g_{2}) = F(g_{1}) \, F(g_{2}) \in H \qcomma \forall g_{1},g_{2} \in G
\end{equation}

\begin{remark}
Siano $ F : G \to H $ un omomorfismo, $ e_{G} \in G $ ed $ e_{H} \in H $ gli elementi neutri dei rispettivi gruppi, allora $ F(e_{G}) = e_{H} $.
\end{remark}

\begin{definition}
	Sia $ F : G \to H $ un omomorfismo di gruppi, vale la relazione
	
	\begin{equation}
		F \circ L_{g} = L_{F(g)} \circ F \qcomma \forall g \in G
	\end{equation}
\end{definition}

\begin{proof}
	Essendo $ F $ un omomorfismo, è preservata la moltiplicazione, perciò presi due elementi qualsiasi $ g,h \in G $ vale
	
	\begin{align}
		\begin{split}
			F(g h) &= F(g) \, F(h) \\
			F(L_{g}(h)) &= L_{F(g)}(F(h)) \\
			(F \circ L_{g})(h) &= (L_{F(g)} \circ F)(h) \\
			F \circ L_{g} &= L_{F(g)} \circ F
		\end{split}
	\end{align}
\end{proof}

Un omomorfismo di due gruppi di Lie $ H $ e $ G $ è un'applicazione liscia $ F : H \to G $ tale che sia un omomorfismo di gruppi. \\
Un \textit{isomorfismo di Lie} è un omomorfismo di gruppi che sia anche un diffeomorfismo.

\subsection{Sottogruppi di Lie}

Siano $ G $ un gruppo di Lie e $ H \neq \emptyset $ un suo sottoinsieme, diremo che $ H $ è un \textit{sottogruppo di Lie} (immerso) di $ G $ se:

\begin{enumerate}
	\item $ H $ è un sottogruppo algebrico di $ G $, in notazione $ H < G $;
	
	\item $ H $ è una sottovarietà immersa di $ G $, i.e. $ H $ è l'immagine di un'immersione iniettiva che sia contenuta in $ G $ (la topologia di $ H $ non è necessariamente la topologia indotta da $ G $);
	
	\item Le operazioni di moltiplicazione e inversione su $ H $, indotte da $ G $, sono lisce.
\end{enumerate}

\begin{definition}\label{prop:lie-group-cond}
	Se $ H $ è un sottogruppo algebrico di $ G $ e una sottovarietà di $ G $, allora la terza condizione è superflua.
\end{definition}

\begin{proof}
	Consideriamo la moltiplicazione e l'inclusione in $ G $
	
	\begin{gather}
		\mu_{G} : G \times G \to G \\
		i : H \to G
	\end{gather}

	La loro composizione
	
	\begin{equation}
		\nu \doteq \mu_{G} \circ (i \times i) : H \times H \to G
	\end{equation}

	ha come immagine $ \nu (H \times H) \subset H $ perciò possiamo restringere il codominio di questa al solo insieme $ H $, ottenendo dunque la moltiplicazione per $ H $
	
	\begin{equation}
		\tilde{\nu} = \mu_{H} : H \times H \to H
	\end{equation}

	la quale è liscia per i teoremi sulle sottovarietà; il ragionamento è analogo per l'inversione.
\end{proof}

Se $ H < G $ è un  sottogruppo algebrico e $ H \subset G $ è una sottovarietà di $ G $, diremo che $ H $ è un \textit{sottogruppo di Lie embedded}. \\ \\
%
Ad esempio, $ SL_{n}(\R) $ e $ O(n) $ sono sottogruppi di Lie embedded di $ GL_{n}(\R) $, perché sottovarietà di quest'ultimo.

\begin{theorem}[Cartan]\label{thm:liesub-var}
	Siano $ G $ e $ H $ varietà differenziabili, se $ H < G $ sottogruppo algebrico e $ H $ è chiuso in $ G $ (come sottospazio topologico) allora $ H $ è un sottogruppo di Lie embedded di $ G $.
\end{theorem}

Da questo teorema, possiamo ancora derivare che $ SL_{n}(\R) $ e $ O(n) $ siano sottogruppi di Lie embedded di $ GL_{n}(\R) $ (e quindi sottovarietà) in quanto varietà e chiusi:

\begin{itemize}
	\item $ SL_{n}(\R) $ è un chiuso in $ GL_{n}(\R) $ in quanto controimmagine di 1 (chiuso in $ \R $) tramite l'applicazione continua (porta chiusi in chiusi)
	
	\map{f}
		{GL_{n}(\R)}{\R}
		{A}{\det(A)}
	
	\item $ O(n) $ è un chiuso in $ GL_{n}(\R) $ in quanto controimmagine di $ I $ (chiuso in $ S(n) $) tramite l'applicazione
	
	\map{f}
		{GL_{n}(\R)}{S(n)}
		{A}{A^{T} A}
\end{itemize}

\subsubsection{\textit{Esempio}}

Questo esempio è riferito a un sottogruppo di Lie immerso importante, il resto dei sottogruppi considerati in questo testo saranno sottogruppi di Lie embedded. \\
Consideriamo i gruppi di Lie $ (\R,+) $ e il toro\footnote{%
	Siccome $ \S^{1} $ è un gruppo di Lie, il prodotto diretto di due $ \S^{1} $ è ancora un gruppo di Lie; vedi Esercizio \ref{exer3-1}.%
} $ \T^{2} = \S^{1} \times \S^{1} $ con le operazioni naturali (moltiplicazione di $ \S^{1} $). Inoltre, prendiamo l'applicazione

\map{F}
	{\R}{\T^{2}}
	{t}{(e^{2 \pi i t},e^{2 \pi i \alpha t})}

dove $ \alpha \in \R \setminus \Q $, i.e. $ \alpha $ è irrazionale; questa applicazione è un'immersione iniettiva\footnote{%
	Vedi Esempio \ref{example:embed-torus}.%
} e un omomorfismo di gruppi, in quanto

\begin{align}
	\begin{split}
		F(t+s) &= (e^{2 \pi i (t+s)},e^{2 \pi i \alpha (t+s)}) \\
		&= (e^{2 \pi i t} e^{2 \pi i s},e^{2 \pi i \alpha t} e^{2 \pi i \alpha s}) \\
		&= (e^{2 \pi i t},e^{2 \pi i \alpha t}) (e^{2 \pi i s},e^{2 \pi i \alpha s}) \\
		&= F(t) \, F(s)
	\end{split}
\end{align}

L'immagine di $ \R $ tramite $ F $ è un sottogruppo di $ \T^{2} $, i.e. $ F(\R) < \T^{2} $, è una sottovarietà immersa di $ \T^{2} $ ($ F $ non è un embedding) e le operazioni su $ F(\R) $ sono lisce, dunque $ F(\R) $ è un sottogruppo di Lie (immerso) di $ \T^{2} $.

\section{Esponenziale di una matrice}

Sia una matrice quadrata $ X \in M_{n}(\R) $, definiamo l'\textit{esponenziale di matrice} come

\begin{equation}
	e^{X} = \sum_{i=0}^{+\infty} \dfrac{X^{i}}{i!} \qcomma X \in M_{n}(\R)
\end{equation}

In particolare abbiamo che $ X^{0} = I $. \\
La definizione dell'esponenziale di matrice segue la falsa riga delle serie di potenze

\begin{equation}
	e^{x} = \sum_{i=0}^{+\infty} \dfrac{x^{i}}{i!} \qcomma x \in \R
\end{equation}

Non è però chiaro se la serie di matrici considerata sia convergente e dunque se $ e^{X} \in M_{n}(\R) $: questo è facilmente dimostrabile se la matrice all'esponente è

\begin{itemize}
	\item la matrice nulla $ X = 0_{n} $ da cui $ e^{0_{n}} = I_{n} = I $;
	
	\item la matrice è un multiplo della matrice identità $ X = x I_{n} $, i.e. $ X^{i} = x^{i} I_{n} $, da cui $ e^{X} = e^{x} I_{n} $.
\end{itemize}

Per dimostrarlo in generale, facciamo una digressione su concetti di analisi utili alla dimostrazione.

\subsection{Spazi vettoriali e algebre normati}

\paragraph{Spazi vettoriali normati}

Uno spazio vettoriale $ V $ sui reali $ \R $ è detto \textit{normato} se esiste un'applicazione

\map{\norm{}}
	{V}{\R}
	{v}{\norm{v}}

chiamata \textit{norma} tale che siano soddisfatte le condizioni:

\begin{equation}
	\begin{cases}
		\begin{cases}
			\norm{v} \geqslant 0 \\
			\norm{v} = 0 \iff v = 0 \in V
		\end{cases} & \text{definita positiva} \\
		\norm{\lambda v} = \abs{\lambda} \norm{v} & \text{omogeneità} \\
		\norm{v + w} \leqslant \norm{v} + \norm{w} & \text{subadditività}
	\end{cases}
\end{equation}

per qualsiasi $ \lambda \in \R $ e qualsiasi $ v,w \in V $. \\
La condizione di subadditività deriva dalla \textit{disuguaglianza di Cauchy-Schwarz}:

\begin{equation}
	\norm{v \cdot w} \leqslant \norm{v} \norm{w} \qcomma \forall v,w \in \R^{n}
\end{equation}

Consideriamo in particolare lo spazio vettoriale delle matrici quadrate sui reali $ M_{n}(\R) $ di dimensione $ n^{2} $ e come norma

\map{\norm{}}
	{M_{n}(\R)}{\R}
	{X = [x_{ij}]_{i,j=1,\dots,n}}{\left( \sum_{i,j=1}^{n} x_{ij}^{2} \right)^{1/2}}
	
la quale corrisponde alla norma usuale in $ \R^{n^{2}} $: da ciò deriva che questa norma soddisfa le condizioni poste sopra, rendendo quindi $ M_{n}(\R) $ uno spazio vettoriale normato. \\

\paragraph{Algebre normate}

Una tripletta $ (V,\cdot,\norm{}) $ è un'\textit{algebra normata} se $ V $ è uno spazio vettoriale su $ \R $, $ (V,\cdot) $ è un'algebra su $ \R $ e $ (V,\norm{}) $ è uno spazio vettoriale normato tali che valga la proprietà di submoltiplicatività

\begin{equation}
	\norm{v \cdot w} \leqslant \norm{v} \norm{w} \qcomma v,w \in V
\end{equation}

che lega l'operazione dell'algebra $ \cdot $ e la norma $ \norm{} $. \\
La tripletta $ (M_{n}(\R),\cdot,\norm{}) $ con $ \cdot $ la moltiplicazione tra matrici e $ \norm{} $ la norma sopra definita per $ M_{n}(\R) $ è un'algebra normata: per verificarlo dobbiamo dimostrare che il prodotto tra matrici e la norma soddisfino la proprietà di submoltiplicatività (in quanto le altre condizioni sono già soddisfatte). \\
Siano le matrici quadrate $ X = [x_{ij}] $ e $ Y = [y_{ij}] $, utilizzando la disuguaglianza di Cauchy-Schwarz e la positività della norma, otteniamo la disuguaglianza

\begin{align}
	\norm{v \cdot w} &\leqslant \norm{v} \norm{w} \\
	\norm{v \cdot w}^{2} &\leqslant \norm{v}^{2} \norm{w}^{2} \\
	((X Y)_{ij})^{2} = \left( \sum_{k=1}^{n} x_{ik} y_{kj} \right)^{2} &\leqslant \left( \sum_{i,k=1}^{n} x_{ik}^{2} \right) \left( \sum_{j,k=1}^{n} y_{kj}^{2} \right)
\end{align}

Se consideriamo dunque la norma al quadrato

\begin{align}
	\begin{split}
		\norm{X Y}^{2} &= \sum_{i,j=1}^{n} ((X Y)_{ij})^{2} \\
		&= \sum_{i,j=1}^{n} \left( \sum_{k=1}^{n} x_{ik} y_{kj} \right)^{2} \\
		&\leqslant \sum_{i,j=1}^{n} \left( \sum_{k=1}^{n} x_{ik}^{2} \right) \left( \sum_{k=1}^{n} y_{kj}^{2} \right) \\
		&= \left( \sum_{i,k=1}^{n} x_{ik}^{2} \right) \left( \sum_{j,k=1}^{n} y_{kj}^{2} \right) \\
		&= \norm{X}^{2} \norm{Y}^{2}
	\end{split}
\end{align}

da cui $ \norm{X Y} \leqslant \norm{X} \norm{Y} $.

\begin{definition}
	Sia $ (V,\cdot,\norm{}) $ un'algebra normata, allora
	
	\begin{itemize}
		\item Se $ s_{n} \in V $ è una successione convergente, i.e. $ s_{n} \to s \in V $, allora
		
		\begin{equation}
			s_{n} \to s \implies a s_{n} \to a s \qcomma \forall a \in V
		\end{equation}
	
		\item Se consideriamo la serie
		
		\begin{equation}
			\sum_{n=0}^{+\infty} s_{n} = s %
			\implies %
			\begin{cases}
				\displaystyle \sum_{n=0}^{+\infty} a s_{n} = a s \\ \\
				\displaystyle \sum_{n=0}^{+\infty} s_{n} a = s a
			\end{cases}
			\qquad \forall a \in V
		\end{equation}
	
		dove $ a s $ ed $ s a $ implicano la moltiplicazione dell'algebra.
	\end{itemize}
\end{definition}

Per definizione, la notazione $ s_{n} \to s $ implica l'equazione

\begin{equation}
	\lim_{n \to \infty} \norm{s_{n} - s} = 0
\end{equation}

\begin{proof}
	Per la prima proprietà
	
	\begin{equation}
		\begin{cases}
			\norm{a s_{n} - a s} = \norm{a (s_{n} - s)} \\ \\
			\lim\limits_{n \to \infty} \norm{s_{n} - s} = 0
		\end{cases}
		\implies%
		\begin{cases}
			\norm{a s_{n} - a s} = \abs{a} \norm{s_{n} - s} \\ \\
			s_{n} \to s
		\end{cases}
		\implies%
		a s_{n} \to a s
	\end{equation}

	Per la seconda: per definizione, una serie converge se la successione delle somme parziali converge allo stesso valore, i.e.
	
	\begin{equation}
		\sum_{n=0}^{+\infty} s_{n} = s %
		\iff %
		\begin{cases}
			\displaystyle \tilde{s}_{k} = \sum_{n=0}^{k} s_{n} \\ \\
			\tilde{s}_{k} \to s
		\end{cases}
	\end{equation}

	dunque per la proprietà
	
	\begin{equation}
		\sum_{n=0}^{+\infty} a s_{n} = a s %
		\iff %
		\begin{cases}
			\displaystyle a \tilde{s}_{k} = \sum_{n=0}^{k} a s_{n} \\ \\
			a \tilde{s}_{k} \to a s
		\end{cases}
	\end{equation}

	dove per la prima proprietà
	
	\begin{equation}
		\sum_{n=0}^{+\infty} s_{n} = s %
		\iff%
		\tilde{s}_{k} \to s%
		\implies%
		a \tilde{s}_{n} \to a \tilde{s} %
		\iff%
		\sum_{n=0}^{+\infty} a s_{n} = a s
	\end{equation}

	Il ragionamento è analogo per la moltiplicazione per $ a $ a destra.
\end{proof}

Una serie è \textit{assolutamente convergente} se è convergente la stessa serie considerando le norme degli addendi, i.e.

\begin{equation}
	\sum_{n=0}^{+\infty} s_{n} \text{ assolutamente convergente} =\joinrel= \sum_{n=0}^{+\infty} \norm{s_{n}} \text{ convergente}
\end{equation}

Una successione $ s_{n} $ è una \textit{successione di Cauchy} se

\begin{equation}
	s_{n} \text{ di Cauchy} \iff \forall \varepsilon > 0, \E p,q,N_{\varepsilon} \in \N \mid \norm{s_{p} - s_{q}} \leqslant \epsilon \qcomma \forall p,q \geqslant N_{\varepsilon}
\end{equation}

\begin{remark}
	Una successione convergente è di Cauchy\footnote{%
		Per dimostrarlo è sufficiente prendere una differenza tra gli addendi arbitrariamente piccola e sfruttare la subadditività.%
	}, ma non è necessariamente vero che una successione di Cauchy sia convergente.
\end{remark}

\subsection{Spazi vettoriali e algebre completi}

Uno spazio normato $ (V,\norm{}) $ è detto \textit{completo} se ogni sua successione di Cauchy è convergente. Uno spazio normato e completo si chiama \textit{spazio di Banach}. \\
Analogamente, un'algebra normata $ (V,\cdot,\norm{}) $ è \textit{completa} se $ (V,\norm{}) $ è completo. Un'algebra normata e completa si chiama \textit{algebra di Banach}. \\ \\
%
Possiamo prendere come esempio di algebra di Banach $ (M_{n}(\R),\cdot,\norm{}) $, in quanto $ (M_{n}(\R),\norm{}) $ è uno spazio completo perché lo è $ \R^{n^{2}} $ e questo si identifica con lo spazio delle matrici quadrate $ M_{n}(\R) = \R^{n^{2}} $.

\begin{definition}
	Sia $ (V,\norm{}) $ uno spazio completo, se una serie è assolutamente convergente, allora è anche convergente (strettamente), i.e.
	
	\begin{equation}
		\sum_{n=0}^{+\infty} \norm{s_{n}} \text{ convergente} \implies \sum_{n=0}^{+\infty} s_{n} \text{ convergente}
	\end{equation}
\end{definition}

Come controesempio dell'implicazione inversa, la serie in $ \R $

\begin{equation}
	\sum_{n=0}^{+\infty} \dfrac{(-1)^{n}}{n}
\end{equation}

è convergente ma non assolutamente convergente se si prende la norma in $ \R $ (valore assoluto).

\begin{proof}
	Consideriamo la successione di serie parziali $ \tilde{s}_{k} \in V $ e $ p,q \in \N $ con $ p > q $. Per ipotesi la serie è assolutamente convergente perciò
	
	\begin{equation}
		\norm{\tilde{s}_{p} - \tilde{s}_{q}} = \norm{ \sum_{n=q+1}^{p} \tilde{s}_{n} } \leqslant \sum_{n=q+1}^{p} \norm{\tilde{s}_{n}} < \varepsilon %
		\qcomma \forall p,q > N_{\varepsilon} \in \N
	\end{equation}

	dove nel secondo passaggio abbiamo utilizzato la subadditività, dunque $ \tilde{s}_{k} $ è di Cauchy. Essendo $ V $ completo, la successione $ \tilde{s}_{k} $ è convergente strettamente dunque anche la serie è convergente.
\end{proof}

\subsection{Definizione di esponenziale di matrice}

Mostriamo ora che la serie
%
\begin{equation}
	e^{X} = \sum_{i=0}^{+\infty} \dfrac{X^{i}}{i!} \qcomma X \in M_{n}(\R)
\end{equation}

sia convergente in $ M_{n}(\R) $. \\
Essendo lo spazio $ M_{n}(\R) $ completo, è sufficiente verificare la serie sia assolutamente convergente. Utilizzando la submoltiplicatività

\begin{equation}
	\sum_{i=0}^{+\infty} \norm{ \dfrac{X^{i}}{i!} } = %
	\sum_{i=0}^{+\infty} \dfrac{1}{i!} \norm{X^{i}} \leqslant %
	\sum_{i=0}^{+\infty} \dfrac{1}{i!} \norm{X}^{i} = %
	e^{\norm{X}}
\end{equation}

dove $ \norm{X} \in \R $, dunque la serie converge. \\
Per un esempio di calcolo di esponenziale di matrice, vedi Esercizio \ref{exer3-3}.

\subsection{Proprietà dell'esponenziale di matrice}

Valgono le tre seguenti proprietà:

\begin{equation}
	\begin{cases}
		[A,B] = 0 \implies e^{A + B} = e^{A} e^{B} = e^{B} e^{A}, & \forall A,B \in M_{n}(\R) \\ \\
		e^{A} \in GL_{n}(\R), & \forall A \in M_{n}(\R) \\ \\
		\ddv{t}  e^{t X} = X e^{t X} = e^{t X} X, & \forall X\in M_{n}(\R), \, \forall t \in \R
	\end{cases}
\end{equation}

\begin{proof}[Dimostrazione (1)]
	Considerando che $ [A,B] = 0 $, possiamo utilizzare la formula binomiale, dunque
	
	\begin{align}
		\begin{split}
			e^{A + B} &= \sum_{k=0}^{+\infty} \dfrac{1}{k!} (A+B)^{k} \\
			&= \sum_{k=0}^{+\infty} \dfrac{1}{k!} \left( \sum_{j=0}^{k} \binom{k}{j} A^{k-j} B^{j} \right) \\
			&= \sum_{k=0}^{+\infty} \sum_{j=0}^{k} \dfrac{1}{k!} \dfrac{k!}{(k-j)! \, j!} A^{k-j} B^{j} \\
			&= \sum_{k=0}^{+\infty} \sum_{j=0}^{k} \dfrac{A^{k-j} B^{j}}{(k-j)! \, j!} \\
			&= \sum_{m=0}^{+\infty} \sum_{j=0}^{m} \dfrac{A^{m} B^{j}}{m! \, j!} \\
			&= \left( \sum_{m=0}^{+\infty} \dfrac{A^{m}}{m!} \right) \left( \sum_{j=0}^{+\infty} \dfrac{B^{j}}{j!} \right) \\
			&= e^{A} e^{B}
		\end{split}
	\end{align}

	dove nel secondo passaggio abbiamo usato la commutatività delle matrici, nel quinto abbiamo definito un nuovo indice $ m \doteq k-j $, e nel sesto abbiamo usato il prodotto di Cauchy\footnote{%
		Il prodotto di Cauchy è definito come la convoluzione discreta di due serie infinite; la sua formula generale è la seguente
		
		\begin{equation*}
			\left( \sum_{i=0}^{+\infty} a^{i} \right) \left( \sum_{j=0}^{+\infty} b^{j} \right) = \sum_{k=0}^{+\infty} c^{k} %
			\qq{dove} %
			c^{k} \doteq \sum_{m=0}^{k} a^{m} b^{k-m}
		\end{equation*}%
	}.
\end{proof}

Per un esempio di matrici che non rispettano questa proprietà, vedi Esercizio \ref{exer3-4}.

\begin{proof}[Dimostrazione (2)]
	La matrice $ e^{A} $ è invertibile con inversa $ e^{-A} $ (dalla prima proprietà) in quanto
	
	\begin{equation}
		e^{A} e^{-A} = e^{-A} e^{A} = e^{A-A} = e^{0_{n}} = I
	\end{equation}
\end{proof}

\begin{proof}[Dimostrazione (3)]
	\begin{align}
		\begin{split}
			\dv{t} e^{t X} &= \dv{t} \left( \sum_{j=0}^{+\infty} \dfrac{(t X)^{j}}{j!} \right) \\
			&= \sum_{j=0}^{+\infty} \dv{t} \left( \dfrac{(t X)^{j}}{j!} \right) \\
			&= \sum_{j=0}^{+\infty} \left( \dfrac{j X (t X)^{j-1}}{j!} \right) \\
			&= X \sum_{j=0}^{+\infty} \left( \dfrac{j (t X)^{j-1}}{j (j-1)!} \right) \\
			&= X \sum_{j=0}^{+\infty} \dfrac{(t X)^{j-1}}{(j-1)!} \\
			&= X \sum_{k=0}^{+\infty} \dfrac{(t X)^{k}}{k!} \\
			&= X e^{t X}
		\end{split}
	\end{align}
	
	dove nel quarto passaggio abbiamo usato il fatto che
	
	\begin{equation}
		s_{n} \to s \implies a s_{n} \to a s
	\end{equation}
\end{proof}

\begin{remark}
	Se consideriamo il campo dei numeri complessi $ \C $ al posto di $ \R $ per gli spazi vettoriali e dunque la norma di una matrice con entrate complesse come
	
	\map{\norm{}}
		{M_{n}(\C)}{\R}
		{X = [x_{ij}]_{i,j=1,\dots,n}}{\left( \sum_{i,j=1}^{n} \abs{x_{ij}}^{2} \right)^{1/2}}
		
	tutti i ragionamenti fatti in questa sezione sono validi, e.g. l'esponenziale di una matrice qualunque è ancora una matrice invertibile, i.e.
	
	\begin{equation}
		e^{X} \in GL_{n}(\C) \qcomma \forall X \in M_{n}(\C)
	\end{equation}
\end{remark}

\section{Richiami di algebra lineare}

\subsection{Prodotti scalari ed hermitiani}

Nell'algebra $ (\R^{n},\cdot) $, presi due vettori $ v,w \in \R^{n} $, il loro prodotto scalare $ v \cdot w \in \R $ è definito come

\begin{equation}
	v \cdot w = (v^{1},\dots,v^{n}) \cdot (w^{1},\dots,w^{n}) \doteq \sum_{j=1}^{n} v^{j} w^{j}
\end{equation}

Considerando tutti i vettori come matrici con $ n $ righe e una colonna, possiamo pensare al prodotto scalare come il prodotto di un vettore riga per un vettore colonna

\begin{equation}
	v \cdot w = v^{T} w = %
	\bmqty{ v^{1} & \cdots & v^{n} } \bmqty{ w^{1} \\ \vdots \\ w^{n} }
\end{equation}

Per vettori in campo complesso, i.e. nell'algebra $ (\C^{n},\cdot) $, definiamo il \textit{prodotto hermitiano} $ v \cdot w \in \C $ come

\begin{equation}
	v \cdot w \doteq \sum_{j=1}^{n} v^{j} \bar{w}^{j} = v^{T} \bar{w}
\end{equation}

con $ v,w \in \C^{n} $ e $ \bar{w} $ indica il coniugato di $ w $ (sia per la componente che per tutte le componenti dell'intero vettore colonna). Per il prodotto hermitiano vale

\begin{equation}
	v \cdot v = \sum_{j=1}^{n} v^{j} \bar{v}^{j} = \sum_{j=1}^{n} \abs{v}^{j}
\end{equation}

Essendo $ v \cdot v \in \R $ ha senso dire che $ v \cdot v \geqslant 0 $ e che

\begin{equation}
	v \cdot v = 0 \iff v = 0 \in \C^{n}
\end{equation}

cioè il prodotto hermitiano è definito positivo. \\
Il prodotto hermitiano non è bilineare come il prodotto scalare, ma lineare per la prima entrata e sesquilineare per la seconda, i.e.

\begin{equation}
	\begin{cases}
		(\lambda v_{1} + \mu v_{2}) \cdot w = \lambda (v_{1} \cdot w) + \mu (v_{2} \cdot w) \\
		v \cdot (\lambda w_{1} + \mu w_{2}) = \bar{\lambda} (v \cdot w_{1}) + \bar{\mu} (v \cdot w_{2})
	\end{cases} %
	\qquad \forall \lambda,\mu \in \C, \, \forall v,w \in \C^{n}
\end{equation}

dunque non è nemmeno simmetrico ma vale l'uguaglianza

\begin{equation}
	v \cdot w = \overline{w \cdot v}
\end{equation}

\subsection{Matrici ortogonali e unitarie}

Utilizzando come entrate i numeri reali, le matrici ortogonali $ O(n) $ sono definite come

\begin{equation}
	O(n) = \{ A \in M_{n}(\R) \mid A^{T} A = I \}
\end{equation}

In campo complesso, si parla invece di \textit{matrici unitarie}, definite come

\begin{equation}
	U(n) = \{ A \in M_{n}(\C) \mid A^{*} A = I \}
\end{equation}

dove $ A^{*} \doteq \bar{A}^{T} \equiv A ^{-1} $. \\
Analogamente come le matrici ortogonali hanno colonne ortogonali tra loro e di norma unitaria (entrambi rispetto al prodotto scalare), anche le matrici unitarie hanno colonne ortogonali tra loro e di norma unitaria (entrambi rispetto al prodotto hermitiano): svolgendo il prodotto matriciale nella definizione di $ U(n) $

\begin{equation}
	A^{*} A = %
	\bmqty{ %
			\bar{a}_{11} & \bar{a}_{21} & \cdots & \bar{a}_{n1} \\ %
			\bar{a}_{12} & \bar{a}_{22} & \cdots & \bar{a}_{n2} \\ %
			\vdots & \vdots & \ddots & \vdots \\ %
			\bar{a}_{1n} & \bar{a}_{2n} & \cdots & \bar{a}_{nn}
			}
	\bmqty{ %
			a_{11} & a_{12} & \cdots & a_{1n} \\ %
			a_{21} & a_{22} & \cdots& a_{2n} \\ %
			\vdots & \vdots & \ddots & \vdots \\ %
			a_{n1} & a_{n2} & \cdots & a_{nn}
			}
\end{equation}

se consideriamo solo la prima entrata della matrice prodotto, otteniamo

\begin{equation}
	\abs{a_{11}}^{2} + \abs{a_{21}}^{2} + \cdots + \abs{a_{n1}}^{2} = 1
\end{equation}

e così per il resto degli elementi nella diagonale, mentre tutti gli altri prodotti hermitiani tra il resto delle colonne è nullo, i.e. le colonne sono tra loro ortonormali.

\begin{definition}
	Le colonne di una matrice unitaria sono una base ortonormale per l'algebra $ (\C^{n},\cdot) $, i.e. prese due colonne $ v_{i} $ e $ v_{j} $ di una matrice unitaria, vale $ v_{i} \cdot v_{j} = \delta_{ij} $.
\end{definition}


\begin{definition}
	La dimensione di un sottospazio $ W \subset \C^{n} $ in $ \C^{n} $ è doppia rispetto alla dimensione che questo avrebbe sui numeri reali $ \R $, in quanto si identificano $ \C^{n} = \R^{2n} $, i.e.

	\begin{equation}
		\dim_{\C}(W) = 2 \dim_{\R}(W)
	\end{equation}
\end{definition}

\subsection{Matrici simili}

Siano due matrici $ A,B \in M_{n}(\C) $, diremo che $ A $ è \textit{simile} a $ B $ se esiste una matrice invertibile $ S \in GL_{n}(\C) $ tale che valga la relazione

\begin{equation}
	B = S^{-1} A S
\end{equation}

Le stesse due matrici sono \textit{unitariamente simili} se la matrice $ S $ della relazione è unitaria.

\subsection{Altre proprietà}

\begin{definition}
	Siano $ A,B \in M_{n}(\C) $, allora $ A $ è simile a $ B $ se e solo se esiste un'applicazione lineare $ T : \C^{n} \to \C^{n} $ per la quale $ A $ e $ B $ rappresentino l'immagine di $ T $ rispetto a due basi di $ \C^{n} $.
\end{definition}

\begin{definition}
	Siano $ A,B \in M_{n}(\C) $, allora $ A $ è unitariamente simile a $ B $ se e solo se esiste un'applicazione lineare $ T : \C^{n} \to \C^{n} $ per la quale $ A $ e $ B $ rappresentino l'immagine di $ T $ rispetto a due basi ortonormali di $ \C^{n} $.
\end{definition}

Questi due risultati derivano dal fatto che, prese due basi $ \mathcal{A} $ e $ \mathcal{B} $ per $ \C^{n} $ con la matrice $ S $ che rappresenta il cambio di base, i.e.

\begin{equation}
	\begin{cases}
		x' = S x \\
		[T(x)]_{\mathcal{B}} = S [T(x)]_{\mathcal{A}}
	\end{cases}
\end{equation}

abbiamo che l'applicazione lineare $ T $ applicata a un vettore $ x $ ha due immagini a seconda della base in cui sono espresse

\begin{align}
	\begin{split}
		T : \C^{n} &\to \C^{n} \\
		x &\stackrel{\mathcal{A}}{\mapsto} [T(x)]_{\mathcal{A}} = A x \\
		x' &\stackrel{\mathcal{B}}{\mapsto} [T(x)]_{\mathcal{B}} = B x'
	\end{split}
\end{align}

le quali sono legate da

\begin{align}
	\begin{split}
		[T(x)]_{\mathcal{B}} &= B x' \\
		S [T(x)]_{\mathcal{A}} &= B S x \\
		[T(x)]_{\mathcal{A}} &= S^{-1} B S x \\
		[T(x)]_{\mathcal{A}} &= A x
	\end{split}
\end{align}

da cui

\begin{equation}
	A = S^{-1} B S
\end{equation}

il quale rende le due matrici simili.

\begin{definition}
	Siano $ A,B \in M_{n}(\C) $, allora "$ A $ è (unitariamente) simile a $ B $" è una relazione di equivalenza in $ M_{n}(\C) $, i.e. $ A \sim B $.
\end{definition}

\begin{definition}\label{prop:similar-matrix-properties}
	Siano $ A,B \in M_{n}(\C) $, se $ A \sim B $ allora $ A $ e $ B $ hanno stessi traccia, rango, determinante e polinomio caratteristico.
\end{definition}

\begin{proof}
	\begin{itemize}
		\item Per quanto riguarda la traccia:

		\begin{equation}
			\tr(A) = \tr(S^{-1} B S) = \tr(B S^{-1} S) = \tr(B)
		\end{equation}

		\item Il rango di una matrice non cambia se la si moltiplica per una matrice invertibile

		\item Per quanto riguarda il determinante, per Binet:

		\begin{equation}
			\det(A) = \det(S^{-1} B S) %
			= \det(S^{-1}) \det(B) \det(S) %
			= \det(B) \det(S^{-1} S) %
			= \det(B)
		\end{equation}

		\item Per quanto riguarda il polinomio caratteristico, per Binet:
		
		\begin{align}
			\begin{split}
				P_{\lambda}(A) &= \det(A - \lambda I) \\
				&= \det(S^{-1} B S - \lambda I) \\
				&= \det(S^{-1} B S - S^{-1} S \lambda I) \\
				&= \det(S^{-1} B S - S^{-1} \lambda I S) \\
				&= \det(S^{-1} (B - \lambda I) S) \\
				&= \det(S^{-1}) \det(B - \lambda I) \det(S) \\
				&= \det(B - \lambda I) \\
				&= P_{\lambda}(B)
			\end{split}
		\end{align}
	
			dove nel settimo passaggio abbiamo usato la proprietà $ \det(S^{-1}) = \det(S)^{-1} $.
	\end{itemize}
\end{proof}

\subsection{Teorema di Schur e teorema spettrale}

\begin{theorem}(Teorema di Schur)
	Data una matrice $ A \in M_{n}(\C) $, esiste una matrice $ U \in U(n) $ tale che
	
	\begin{equation}
		U^{*} A U = T = %
		\bmqty{ %
				\dmat{ %
						\lambda_{1},
						& & K \\ & \ddots & \\ 0 & & ,
						\lambda_{n}
						}
				}
	\end{equation}

	con $ T $ matrice triangolare (superiore), i.e. una matrice che ha tutte entrate nulle sotto la diagonale principale ($ K \neq 0 $). \\
	In altre parole, una qualunque matrice quadrata $ A $ è unitariamente simile a una matrice triangolare (superiore), il che rende gli elementi della diagonale della matrice triangolare $ \lambda_{1},\dots,\lambda_{n} $ gli autovalori di $ A $.
\end{theorem}

\begin{proof}
	La dimostrazione di questo teorema è fatta per induzione su $ n $. \\
	Per $ n=1 $, le matrici diventano numeri e il teorema è dimostrato in quanto questi sono già "diagonali". \\
	Supponiamo dunque che sia vero per $ n-1 $ e dimostriamo per $ n $: dimostrare che $ A $ è simile a una matrice triangolare superiore è equivalente a trovare una base ortonormale $ \{v_{1},\dots,v_{n}\} $ (rispetto al prodotto hermitiano, i.e. $ v_{i}^{T} \bar{v}_{j} = \delta_{ij} $ con $ i,j=1,\dots,n $) di $ \C^{n} $ tale che $ A v_{k} $ sia combinazione lineare dei vettori di base. \\
	Siano $ \lambda_{1} \in \C $ un autovalore per $ A $ e $ v_{1} \neq 0 $ il corrispondente autovettore, i.e. $ A v_{1} = \lambda_{1} v_{1} $. Sia lo spazio dei vettori generati da $ v_{1} $
	
	\begin{equation}
		W = \ev{v_{1}}_{\C} = \{ \lambda v_{1} \mid \lambda \in \C \} \subset \C
	\end{equation}
	
	il quale avrà $ \dim_{\C}(W) = 1 $. \\
	Consideriamo ora il suo \textit{complemento ortogonale}
	
	\begin{equation}
		W^{\perp} = \{ v\in \C^{n} \mid v \cdot w = 0, \, \forall w \in W \} \subset \C^{n-1}
	\end{equation}

	con $ \dim_{\C}(W^{\perp}) = n-1 $ e la proiezione $ \pi_{W^{\perp}} : \C^{n} \to W^{\perp} $. Consideriamo infine l'applicazione lineare
	
	\map{\pi_{W^{\perp}} \circ A}
		{W^{\perp}}{W^{\perp}}
		{w}{\pi_{W^{\perp}}(A w)}
	
	Per ipotesi induttiva, esiste una base ortonormale $ \{v_{2},\dots,v_{n}\} $ di $ W^{\perp} $ tale che $ (\pi_{W^{\perp}} \circ A)(v_{k}) $ è combinazione lineare di $ v_{2},\dots,v_{k} $ con $ k=2,\dots,n $. \\
	A questo punto $ \{v_{1},\dots,v_{n}\} $ è una base ortonormale di $ \C^{n} $ tale che $ A v_{k} $ sia combinazione lineare dei vettori di base. \\
	Essendo la base ortonormale, la matrice $ A $ è dunque unitariamente simile a una matrice triangolare (superiore).
\end{proof}

\begin{remark}
	La base $ \{v_{1},\dots,v_{n}\} $ che appare nel teorema di Schur viene chiamata \textit{base di Schur}.
\end{remark}

Una matrice $ A \in M_{n}(\C) $ è detta \textit{normale} se $ A A^{*} = A^{*} A $, i.e. commuta con la sua trasposta coniugata: esempi di matrici normali sono

\begin{itemize}
	\item Le matrici unitarie, in quanto $ A^{*} A = I $;
	
	\item Le matrici ortogonali (le quali hanno entrate reali), perciò $ A^{*} A = A^{T} A = I $;
	
	\item Le matrici hermitiane, che coincidono con la loro trasposta coniugata, i.e. $ A = A^{*} $;
	
	\item Le matrici simmetriche $ A \in S(n) $ (con entrate reali), in quanto $ A = A^{T} $.
\end{itemize}

\begin{theorem}(Teorema spettrale)
	Ogni matrice $ A \in M_{n}(\C) $ normale è unitariamente simile a una matrice diagonale, i.e.
	
	\begin{equation}
		A \in M_{n}(\C) \text{ normale} %
		\implies %
		\E U \in U(n) \mid U^{*} A U = %
		\bmqty{ %
				\dmat{ %
						\lambda_{1},
						& & 0 \\ & \ddots & \\ 0 & & ,
						\lambda_{n}
						}
				}
	\end{equation}

	con $ \lambda_{j} $ gli autovalori di $ A $. \\
	In altre parole, una matrice normale è sempre diagonalizzabile e i suoi autovettori sono una base ortonormale di $ \C^{n} $. \\
	Ancora, una matrice $ A \in M_{n}(\C) $ è normale se e solo se esiste una base ortonormale di $ \C^{n} $ costituita dagli autovettori di $ A $.
\end{theorem}

\begin{proof}
	Se $ A \in M_{n}(\C) $, per il teorema di Schur esiste una matrice unitaria $ U \in U(n) $ tale che
	
	\begin{equation}
		U^{*} A U = T = %
		\bmqty{ %
				\dmat{ %
						\lambda_{1},
						& & K \\ & \ddots & \\ 0 & & ,
						\lambda_{n}
						}
				}
	\end{equation}

	cioè $ A $ sia simile a una matrice triangolare superiore. \\
	Essendo $ A $ normale, questo implica che anche $ U^{*} A U $ lo sia perché
	
	\begin{align}
		\begin{split}
			(U^{*} A U) (U^{*} A U)^{*} &= U^{*} A U U^{*} A^{*} U \\
			&= U^{*} A A^{*} U \\
			&= U^{*} A^{*} A U \\
			&= U^{*} A^{*} U U^{*} A U \\
			&= (U^{*} A U)^{*} (U^{*} A U)
		\end{split}
	\end{align}

	A questo punto, $ T $ è normale ma se una matrice è triangolare superiore e normale allora è diagonale: dimostriamo questo per induzione su $ n $. \\
	Per $ n=1 $, la matrice è "diagonale" in quanto numero. \\
	Supponiamo che sia vero per matrici di ordine $ n-1 $ e mostriamolo per matrici di ordine $ n $: riscriviamo la matrice normale triangolare superiore come
	
	\begin{equation}
		T = \bmqty{ \lambda_{1} & B \\ 0 & C } %
		\qcomma %
		\begin{cases}
			\lambda_{1} \in \C \\
			B \in M_{1,n-1}(\C) \\
			0 \in M_{n-1,1}(\C) \\
			C \in M_{n-1}(\C)
		\end{cases}
	\end{equation}

	Imponendo la condizione di matrice normale
	
	\begin{align}
		\begin{split}
			T T^{*} &= T^{*} T \\
			\bmqty{ %
					\lambda_{1} & B \\
					0 & C %
					} %
			\bmqty{ %
					\bar{\lambda}_{1} & 0 \\
					B^{*} & C^{*} %
					}
			&= %
			\bmqty{ %
						\bar{\lambda}_{1} & 0 \\
						B^{*} & C^{*} %
						}
			\bmqty{ %
					\lambda_{1} & B \\
					0 & C %
					} \\
			%
			\bmqty{ %
					\abs{\lambda}_{1} + B B^{*} & B C^{*} \\
					C B^{*} & C C^{*} %
					}
			&= %
			\bmqty{ %
					\abs{\lambda}_{1} & \bar{\lambda}_{1} B \\
					\lambda_{1} B^{*} & B^{*} B + C^{*} C %
					}
		\end{split}
	\end{align}

	otteniamo che
	
	\begin{equation}
		\begin{cases}
			B B^{*} = 0 \implies B = 0 \\
			C^{*} C = C C^{*}
		\end{cases}
	\end{equation}

	rendendo $ C $ normale oltre che triangolare superiore: per ipotesi induttiva, $ C $ è diagonale quindi lo è anche $ T $.
\end{proof}

\begin{corollary}[1. Teorema spettrale per matrici simmetriche]
	Sia $ A \in S(n) $ con entrate reali ($ A \in M_{n}(\R) $), allora esiste una matrice $ P \in O(n) $ tale che
	
	\begin{equation}
		P^{T} A P = D = %
		\bmqty{ %
				\dmat{ %
						\lambda_{1},
						& & 0 \\ & \ddots & \\ 0 & & ,
						\lambda_{n}
						}
				}
	\end{equation}
\end{corollary}

\begin{proof}
	Per dimostrare questo corollario, è sufficiente dimostrare che gli autovalori di $ A $ siano reali e dunque anche gli autovettori sono reali: considerando l'uguaglianza
	
	\begin{equation}
		U(n) \cap M_{n}(\R) = O(n)
	\end{equation}

	il resto della dimostrazione deriva dal teorema spettrale. \\
	Per dimostrare che gli autovalori di una matrice $ A \in S(n) \subset M_{n}(\C) $  a entrate reali siano reali, consideriamo la relazione
	
	\begin{equation}
		A v = \lambda v \qcomma \lambda \in \C, \, v \in \C^{n}
	\end{equation}

	da cui

	\begin{align}
		\begin{split}
			\lambda v &= A v \\
			\bar{v}^{T} \lambda v &= \bar{v}^{T} A v \\
			\lambda \abs{v}^{2} &= \bar{v}^{T} \bar{A}^{T} v \\
			&= \overline{A v}^{T} v \\
			&= \overline{\lambda v}^{T} v \\
			&= \bar{\lambda} v^{T} v \\
			&= \bar{\lambda} \abs{v}^{2}
		\end{split}
	\end{align}

	perciò
	
	\begin{equation}
		\bar{\lambda} = \lambda \implies \lambda \in \R
	\end{equation}
\end{proof}

\begin{corollary}[2]
	Siano una matrice normale $ A \in M_{n}(\C) $ e un suo autovalore $ \lambda $, allora la molteplicità algebrica di $ \lambda $ coincide con quella geometrica, dove la prima indica il grado della soluzione $ \lambda $ all'interno del polinomio caratteristico mentre la seconda indica la dimensione dell'autospazio associato a $ \lambda $.
\end{corollary}

\begin{proof}
	Ricordiamo che una matrice normale è sempre diagonalizzabile e una matrice è diagonalizzabile se e solo se la molteplicità algebrica dei suoi autovalori coincide con quella geometrica.
\end{proof}

\begin{corollary}[3]\label{cor:unit-matrix-similar}
	Sia $ A \in U(n) $, i.e. $ A^{*} A = I $, allora esiste una matrice unitaria $ U \in U(n) $ tale che
	
	\begin{equation}
		U^{*} A U = %
		\bmqty{ %
				\dmat{ %
						e^{i \theta_{1}} & & 0 \\
						& \ddots & \\
						0 & & e^{i \theta_{n}} %
						}
				} %
		\qcomma \theta_{j} \in \R, \, j=1,\dots,n
	\end{equation}

	In altre parole, una matrice unitaria è simile a una matrice diagonale con entrate di norma unitaria.
\end{corollary}

\begin{proof}
	Se $ A \in U(n) $ allora è normale, dunque esiste $ U \in U(n) $ tale che $ U^{*} A U $ sia diagonale, i.e.
	
	\begin{equation}
		U^{T} A U = %
		\bmqty{ %
				\dmat{ %
						\lambda_{1} & & 0 \\
						& \ddots & \\
						0 & & \lambda_{n} %
						}
				}
	\end{equation}

	quindi è sufficiente dimostrare che $ \abs{\lambda_{j}} = 1 $ per $ j=1,\dots,n $:
	
	\begin{align}
		\begin{split}
			\lambda v &= A v \\
			\overline{(A v)}^{T} \lambda v &= \overline{(A v)}^{T} A v \\
			\overline{(\lambda v)}^{T} \lambda v &= \bar{v}^{T} A^{*} A v \\
			\bar{\lambda} \lambda \bar{v}^{T} v &= \bar{v}^{T} v \\
			\abs{\lambda}^{2} \abs{v}^{2} &= \abs{v}^{2} \\
			\abs{\lambda} &= 1
		\end{split}
	\end{align}
\end{proof}

Consideriamo l'insieme delle \textit{matrici unitarie speciali}:

\begin{equation}
	SU(n) = \{ X \in U(n) \mid \det(X) = 1 \} %
	= \{ X \in M_{n}(\C) \mid X^{*} X = I \wedge \det(X) = 1 \}
\end{equation}

\begin{corollary}[4]
	Sia $ A \in SU(n) $, allora esiste una matrice unitaria $ U $ tale che
	
	\begin{equation}
		U^{*} A U = %
		\bmqty{ %
				\dmat{ %
						e^{i \theta_{1}} & & 0 \\
						& \ddots & \\
						0 & & e^{i \theta_{n}} %
						}
				} %
		\qcomma \theta_{j} \in \R, \, j=1,\dots,n
	\end{equation}

	con
	
	\begin{equation}
		\sum_{j=1}^{n} \theta_{j} = 2 k \pi \qcomma k \in \Z
	\end{equation}
\end{corollary}

\begin{proof}
	\begin{gather}
		1 = \det(A) = \det(U^{*} A U) = \prod_{j=1}^{n} e^{i \theta_{j}} = e^{i \sum_{j=1}^{n} \theta_{j}} \\
		\Downarrow \nonumber \\
		\sum_{j=1}^{n} \theta_{j} = 2 k \pi \qcomma k \in \Z
	\end{gather}
\end{proof}

\subsection{Forma canonica}

\begin{theorem}[Forma canonica ortogonale]
	Sia una matrice $ A \in O(n) $, i.e. $ A^{T} A = I $, allora esistono una matrice ortogonale $ P $, tre numeri naturali $ p,q \in \N $ e $ \theta_{j} \in (0,\pi) $ con $ j=1,\dots,k $ dove $ k = (n-p-q)/2 $ tali che\footnote{%
		Le entrate omesse nelle matrici implicano entrate nulle.%
	}

	\begin{equation}
		P^{T} A P = %
		\bmqty{ %
				\dmat{ %
					I_{p}, - I_{q}, R_{1}, \ddots, R_{n-p-q} %
					}
				}
	\end{equation}

	sia una matrice a blocchi dove
	
	\begin{gather}
		R_{j} = %
		\bmqty{ %
				\cos(\theta_{j}) & \sin(\theta_{j}) \\ \\
				- \sin(\theta_{j}) & \cos(\theta_{j}) %
				 } \\
		%
		\nonumber \\
		%
		\det(R_{j}) = 1 \\
		%
		\nonumber \\
		%
		\det(P^{T} A P) = (- 1)^{q}
	\end{gather}

	Se $ A \in SO(n) $ allora $ q $ è sempre pari.
\end{theorem}

\subsection{Matrici elementari e generatori del gruppo lineare speciale}

Siano $ a \in \R $, $ i,j=1,\dots,n $ con $ i \neq j $, una \textit{matrice elementare} $ M_{a}(i,j) \in M_{n}(\R) $ è definita nel seguente modo:

\begin{equation}
	M_{a}(i,j) \doteq I + a E_{ij}
\end{equation}

dove

\begin{equation}
	[E_{ij}]_{kl} = %
	\begin{cases}
		1, & k = i \wedge l = j \\
		0, & \text{altrimenti}
	\end{cases}
\end{equation}

in altre parole, una matrice elementare è una matrice che ha 1 nella diagonale principale, $ a $ nell'entrata identificata dalla riga $ i $ e dalla colonna $ j $ e 0 nelle altre entrate. \\
Il determinante di una matrice elementare è sempre unitario, i.e.

\begin{equation}
	\det(M_{a}(i,j)) = 1
\end{equation}

in quanto è sempre una matrice triangolare superiore o inferiore, dunque il determinante è dato dal prodotto degli elementi sulla diagonale. L'inversa di una matrice elementare è ancora una matrice elementare ed è data da

\begin{equation}
	(M_{a}(i,j))^{-1} = M_{-a}(i,j)
\end{equation}

in quanto

\begin{align}
	\begin{split}
		M_{a}(i,j) M_{-a}(i,j) &= (I + a E_{ij}) (I - a E_{ij}) \\
		&= I + a E_{ij} - a E_{ij} - a^{2} \cancelto{0}{E_{ij} E_{ij}} \\
		&= I
	\end{split}
\end{align}

\begin{theorem}
	Siano $ \K $ un campo e il gruppo lineare speciale
	
	\begin{equation}
		SL_{n}(\K) = \{ A \in M_{n}(\K) \mid \det(A) = 1 \} \qcomma n \geqslant 1
	\end{equation}

	allora $ SL_{n}(\K) $ è generato da matrici elementari, i.e.
	
	\begin{equation}
		\forall A \in SL_{n}(\K), \E M_{a_{1}}(i_{1},j_{1}), \dots, M_{a_{t}}(i_{t},j_{t}) \mid A = \prod_{p=1}^{t} M_{a_{p}}(i_{p},j_{p})
	\end{equation}

	cioè ogni matrice del gruppo lineare speciale può essere scritta come prodotto finito di matrici elementari.
\end{theorem}

\begin{proof}
	Se $ A \in M_{n}(\K) $ e $ M_{b}(i,j) $ è una matrice elementare, allora il prodotto $ M_{b}(i,j) A $ corrisponde alla seguente operazione elementare sulle righe $ R_{i} $ di $ A $:
	
	\begin{equation}
		R_{i} \to R_{i} + b R_{j}
	\end{equation}

	con
	
	\begin{equation}
		R_{i} = \bmqty{ a_{i1} & \cdots & a_{in} }
	\end{equation}

	in quanto
	
	\begin{align}
		\begin{split}
			M_{b}(i,j) A &= (I + b E_{ij}) A \\
			&= A + b E_{ij} A \\
			&= \sbmqty{ %
						a_{11} & \cdots & a_{1n} \\
						\vdots & \ddots & \vdots \\
						a_{n1} & \cdots & a_{nn} %
						} + %
				b \sbmqty{ %
							0 & \cdots & & & \cdots & & \cdots & 0 \\
							\vdots & \ddots & & & & & & \vdots \\
							& & \ddots & & & & & \\
							0 & \cdots & \cdots & 0 & \cdots & 1 & \cdots & 0 \\
							& & & & \ddots & & & \\
							\vdots & & & & & \ddots & & \vdots \\
							& & & & & & \ddots & \\
							0 & \cdots & & & \cdots & & \cdots & 0 %
							} %
				\sbmqty{ %
						a_{11} & \cdots & a_{1n} \\
						\vdots & \ddots & \vdots \\
						a_{n1} & \cdots & a_{nn} %
						} \\
			%
			&= \sbmqty{ %
						R_{1} \\
						\vdots \\
						R_{i-1} \\
						R_{i} \\
						R_{i+1} \\
						 \\
						\vdots \\
						 \\
						R_{n} %
						} + %
				\sbmqty{ %
						0 & \cdots & & & \cdots & & \cdots & 0 \\
						\vdots & \ddots & & & & & & \vdots \\
						& & \ddots & & & & & \\
						0 & \cdots & \cdots & 0 & \cdots & b & \cdots & 0 \\
						& & & & \ddots & & & \\
						\vdots & & & & & \ddots & & \vdots \\
						& & & & & & \ddots & \\
						0 & \cdots & & & \cdots & & \cdots & 0 %
						} %
				\sbmqty{ %
						R_{1} \\
						\vdots \\
						R_{i-1} \\
						R_{i} \\
						R_{i+1} \\
						 \\
						\vdots \\
						 \\
						R_{n} %
						} \\
			%
			&= \sbmqty{ %
						R_{1} \\
						\vdots \\
						R_{i-1} \\
						R_{i} \\
						R_{i+1} \\
						 \\
						\vdots \\
						 \\
						R_{n} %
						} + %
				\sbmqty{ %
						0 \\
						\vdots \\
						0 \\ %
						b R_{j} \\
						0 \\
						 \\
						\vdots \\
						 \\
						0 %
						} \\
			%
			&= \sbmqty{ %
						R_{1} \\
						\vdots \\
						R_{i-1} \\
						R_{i} + b R_{j} \\
						R_{i+1} \\
						 \\
						\vdots \\
						 \\
						R_{n} %
						}
		\end{split}
	\end{align}

	Analogamente, il prodotto $ A M_{b}(i,j) $ corrisponde alla seguente operazione elementare sulle colonne $ C_{i} $ di $ A $:
	
	\begin{equation}
		C_{j} \to C_{j} + b C_{i}
	\end{equation}

	con
	
	\begin{equation}
		C_{i} = \bmqty{ a_{1i} \\ \vdots \\ a_{ni} }
	\end{equation}
	
	in quanto
	
	\begin{align}
		\begin{split}
			A M_{b}(i,j) &= A (I + b E_{ij}) \\
			&= A + b A E_{ij} \\
			&= \sbmqty{ %
						a_{11} & \cdots & a_{1n} \\
						\vdots & \ddots & \vdots \\
						a_{n1} & \cdots & a_{nn} %
						} + %
			b \sbmqty{ %
						a_{11} & \cdots & a_{1n} \\
						\vdots & \ddots & \vdots \\
						a_{n1} & \cdots & a_{nn} %
						} %
			\sbmqty{ %
						0 & \cdots & & & \cdots & & \cdots & 0 \\
						\vdots & \ddots & & & & & & \vdots \\
						& & \ddots & & & & & \\
						0 & \cdots & \cdots & 0 & \cdots & 1 & \cdots & 0 \\
						& & & & \ddots & & & \\
						\vdots & & & & & \ddots & & \vdots \\
						& & & & & & \ddots & \\
						0 & \cdots & & & \cdots & & \cdots & 0 %
						} \\
			%
			&= \sbmqty{ %
						C_{1} & \cdots & C_{j-1} & C_{j} & C_{j+1} & \cdots & C_{n} %
						} + %
				\sbmqty{ %
							0 & \cdots & & & \cdots & & \cdots & 0 \\
							\vdots & \ddots & & & & & & \vdots \\
							& & \ddots & & & & & \\
							0 & \cdots & \cdots & 0 & \cdots & b & \cdots & 0 \\
							& & & & \ddots & & & \\
							\vdots & & & & & \ddots & & \vdots \\
							& & & & & & \ddots & \\
							0 & \cdots & & & \cdots & & \cdots & 0 %
							} %
				\sbmqty{ %
						C_{1} & \cdots & C_{j-1} & C_{j} & C_{j+1} & \cdots & C_{n} %
						} \\
			%
			&= \sbmqty{ %
						C_{1} & \cdots & C_{j-1} & C_{j} & C_{j+1} & \cdots & C_{n} %
						} + %
				\sbmqty{ %
						0 & \cdots & 0 & b \, C_{i} & 0 & \cdots & 0 %
						} \\
			%
			&= \sbmqty{ %
						C_{1} & \cdots & C_{j-1} & C_{j} + b \, C_{i} & C_{j+1} & \cdots & C_{n} %
						}
		\end{split}
	\end{align}

	Per dimostrare il teorema è dunque sufficiente mostrare che data $ A \in SL_{n}(\K) $ esistono un numero finito di operazioni elementari sulle righe e sulle colonne di $ A $ tali che trasformino $ A $ nella matrice identità: per fare ciò, usiamo l'algoritmo di Gauss-Jordan:
	
	\begin{enumerate}
		\item Sia $ A \in SL_{n}(\K) $ e consideriamo il suo elemento $ a_{12} $: se questo è diverso da zero, saltiamo al passo successivo; nel caso in cui non lo sia, esisterà un elemento della stessa riga non nullo (in quanto la matrice ha determinante diverso da zero), i.e. $ a_{1j} \neq 0 $, e applichiamo la seguente operazione elementare
		
		\begin{equation}
			C_{2} \to C_{2} + C_{j}
		\end{equation}
	
		che rende $ a_{12} \neq 0 $;
		
		\item Con l'operazione elementare
		
		\begin{equation}
			C_{1} \to C_{1} + \left( \dfrac{1-a_{11}}{a_{12}} \right) C_{2}
		\end{equation}
	
		otteniamo che $ a_{11} = 1 $, in quanto
		
		\begin{equation}
			a_{11} \to a_{11} + \left( \dfrac{1-a_{11}}{a_{12}} \right) a_{12} = a_{11} + 1 - a_{11} = 1
		\end{equation}
		
		\item Per tutti i $ j \geqslant 2 $, applichiamo l'operazione elementare
		
		\begin{equation}
			C_{j} \to C_{j} - a_{1j} C_{1}
		\end{equation}
	
		rendendo la prima riga $ R_{1} $ della matrice $ A $ pari a 
		
		\begin{equation}
			R_{1} = \bmqty{ 1 & 0 & \cdots & 0 }
		\end{equation}
		
		\item Per tutti i $ j \geqslant 2 $, applichiamo l'operazione elementare
		
		\begin{equation}
			R_{j} \to R_{j} - a_{j1} R_{1}
		\end{equation}
		
		rendendo la prima colonna $ C_{1} $ della matrice $ A $ pari a 
		
		\begin{equation}
			C_{1} = \bmqty{ 1 \\ 0 \\ \vdots \\ 0 }
		\end{equation}
	
		e dunque la matrice $ A $ sarà ora pari a
		
		\begin{equation}
			A = %
			\bmqty{ %
					1 & 0 & \cdots & 0 \\ %
					0 & & & \\ %
					\vdots & & B & \\ %
					0 & & & %
					}
		\end{equation}
	
		con $ B \in SL_{n-1}(\K) $, in quanto le operazioni elementari fatte non modificano il determinante della matrice $ A $ (e dunque della matrice $ B $);
		
		\item Per induzione, rifacendo gli stessi passaggi\footnote{%
			Le operazioni elementari su $ B $ non intaccano le operazioni precedenti perché gli elementi fuori da $ B $ e non nella diagonale sono tutti nulli.%
		} per $ B $, si ricava la matrice identità $ I $ mediante un numero finito di operazioni elementari sulle righe e sulle colonne.
	\end{enumerate}

	Una volta svolte tutte le operazioni elementari, scritte come moltiplicazioni a destra e a sinistra per matrici elementari, possiamo scrivere infine
	
	\begin{align}
		\begin{split}
			(M_{1} \cdots M_{q}) A (M_{q+1} \cdots M_{t}) &= I \\
			A &= (M_{1} \cdots M_{q})^{-1} (M_{q+1} \cdots M_{t})^{-1} \\
			&= M_{q}^{-1} \cdots M_{1}^{-1} \, M_{t}^{-1} \cdots M_{q+1}^{-1} \\
			&= M_{-q} \cdots M_{-1} \, M_{-t} \cdots M_{-(q+1)}
		\end{split}
	\end{align}
\end{proof}

\subsection{Traccia, determinante ed esponenziale di una matrice}

Le seguenti considerazioni e dimostrazioni valgono indifferentemente se si considera come campo su cui sono definiti gli spazi quello dei numeri reali o quello dei numeri complessi, i.e. $ \K = \R,\C $. Per semplicità, useremo nella trattazione il campo dei reali $ \R $.

\begin{definition}
	Sia un matrice $ X \in M_{n}(\R) $, allora
	
	\begin{equation}
		\tr(X) = \sum_{j=1}^{n} \lambda_{j}
	\end{equation}
	
	dove i $ \lambda_{j} $ sono gli autovalori di $ X $.
\end{definition}

La traccia di una matrice a entrare reali è un numero reale, i.e. $ \tr(X) \in \R $, nonostante ci possano essere autovalori complessi, poiché quest'ultimi saranno complessi coniugati tra loro e, sommandoli, si elidono a vicenda.

\begin{proof}
	Per il teorema di Schur, $ X $ è simile a una matrice triangolare superiore, i.e. esiste $ U \in U(n) $ tale che
	
	\begin{equation}
		U^{*} X U = T = %
		\bmqty{ %
				\lambda_{1}	& & A \\
				& \ddots & \\
				0 & & \lambda_{n} %
				}
	\end{equation}
	
	dove i $ \lambda_{j} $ sono gli autovalori di $ X $. Possiamo ora scrivere
	
	\begin{equation}
		\tr(X) = \tr(U^{*} X U) %
		= \tr(T) %
		= \sum_{j=1}^{n} \lambda_{j}
	\end{equation}
\end{proof}

\begin{definition}\label{prop:det-exp-tr}
	Sia una matrice $ X \in M_{n}(\R) $, allora
	
	\begin{equation}
		\det(e^{X}) = e^{\tr(X)}
	\end{equation}
\end{definition}

\begin{proof}
	Supponiamo che $ X $ sia triangolare superiore, i.e.
	
	\begin{equation}
		X = %
		\bmqty{ %
				\lambda_{1} & & A \\
				& \ddots & \\
				0 & & \lambda_{n} %
				}
	\end{equation}

	La potenza $ k $-esima di questa matrice avrà la forma
	
	\begin{equation}
		X^{k} = %
		\bmqty{ %
				\lambda_{1}^{k} & & A' \\
				& \ddots & \\
				0 & & \lambda_{n}^{k} %
				}
	\end{equation}

	perciò l'esponenziale sarà
	
	\begin{align}
		\begin{split}
			e^{X} &= \sum_{k=0}^{+\infty} \dfrac{X^{k}}{k!} \\
			&= \sum_{k=0}^{+\infty} %
				\bmqty{ %
						\lambda_{1}^{k} & & A' \\
						& \ddots & \\
						0 & & \lambda_{n}^{k} %
						} \\
			%
			&= \bmqty{ %
						\displaystyle\sum_{k=0}^{+\infty} \dfrac{\lambda_{1}^{k}}{k!} & & A'' \\
						& \ddots & \\
						0 & & \displaystyle\sum_{k=0}^{+\infty} \dfrac{\lambda_{n}^{k}}{k!} %
						} \\
			&= \bmqty{ %
						e^{\lambda_{1}} & & A'' \\
						& \ddots & \\
						0 & & e^{\lambda_{n}} %
						}
		\end{split}
	\end{align}

	A questo punto
	
	\begin{equation}
		\det(e^{X}) = \prod_{k=1}^{n} e^{\lambda_{k}} %
		= e^{\sum_{k=1}^{n} \lambda_{k}} %
		= e^{\tr(X)}
	\end{equation}

	Nel caso in cui la matrice $ X $ non fosse triangolare, per il teorema di Schur, $ X $ è simile a una matrice triangolare superiore, i.e. esiste $ U \in U(n) $ tale che
	
	\begin{equation}
		U^{*} X U = T = %
		\bmqty{ %
				\lambda_{1} & & A \\
				& \ddots & \\
				0 & & \lambda_{n} %
				}
	\end{equation}
	
	Per calcolare il determinante di $ e^{X} $, utilizziamo Binet e la dimostrazione fatta sopra:

	\begin{align}
		\begin{split}
			\det(e^{X}) &= \det(U^{*}) \det(e^{X}) \det(U) \\
			&= \det(U^{*} e^{X} U) \\
			&= \det(e^{U^{*} X U}) \\
			&= e^{\tr(U^{*} X U)} \\
			&= e^{\tr(X)}
		\end{split}
	\end{align}
	
	nel terzo passaggio abbiamo utilizzato la proprietà $ U^{*} e^{X} U = e^{U^{*} X U} $, che si dimostra come
	
	\begin{equation}
		e^{U^{*} X U} %
		= \sum_{k=0}^{+\infty} \dfrac{(U^{*} X U)^{k}}{k!} %
		= \sum_{k=0}^{+\infty} \dfrac{U^{*} X^{k} U}{k!} %
		= U^{*} \left( \sum_{k=0}^{+\infty} \dfrac{X^{k}}{k!} \right) U %
		= U^{*} e^{X} U
	\end{equation}

	dove $ (U^{*} X U)^{k} = U^{*} X^{k} U $ in quanto $ U^{*} U = I $ e, nel terzo passaggio, abbiamo usato le proprietà delle serie convergenti, i.e.
	
	\begin{equation}
		\sum_{n=0}^{+\infty} a s_{n} = a \sum_{n=0}^{+\infty} s_{n}
	\end{equation}
\end{proof}

\begin{corollary}\label{cor:det-exp-gl}
	Siccome la proposizione dà un metodo per calcolare il determinante di $ e^{X} $, ricaviamo dunque che l'esponenziale di una matrice è sempre invertibile\footnote{%
		Questo fatto è già stato dimostrato, questa è solo un'ulteriore conferma.%
	}, in quanto

	\begin{equation}
		e^{\tr(X)} \neq 0 \qcomma \forall X \in M_{n}(\K)
	\end{equation}
\end{corollary}

\begin{corollary}
	Sappiamo che l'esponenziale di una matrice è sempre invertibile, i.e.
	
	\begin{equation}
		e^{X} \in GL_{n}(\R) \qcomma \forall X \in M_{n}(\R)
	\end{equation}

	Sia dunque una matrice $ X \in M_{n}(\R) $, allora la curva
	
	\map{c}
		{\R \times M_{n}(\R)}{GL_{n}(\R)}
		{(t,X)}{e^{t X}}
	
	ha vettore tangente nell'origine esattamente $ X $, i.e. $ c'(0) = X $. \\
	Valgono le seguenti identificazioni
	
	\begin{equation}
		\begin{cases}
			c(0) = I \\ \\
			T_{I}(GL_{n}(\R)) = M_{n}(\R) \\ \\
			c'(0) = \eval{ \ddv{t} c }_{t=0}
		\end{cases}
	\end{equation}

	Più in generale, se $ A \in GL_{n}(\R) $ e $ B \in M_{n}(\R) $, allora per la curva
	
	\map{c}
		{\R \times GL_{n}(\R) \times M_{n}(\R)}{GL_{n}(\R)}
		{(t,A,B)}{A e^{t A^{-1} B}}
	
	valgono
	
	\begin{equation}
		\begin{cases}
			c(0) = A \\
			c'(0) = B
		\end{cases}
	\end{equation}
\end{corollary}

\begin{proof}
	Per il primo caso
	
	\begin{equation}
		c(0) = e^{0 X} = e^{0_{n}} = I
	\end{equation}

	e il vettore tangente
	
	\begin{equation}
		c'(0) = \eval{ \dv{t} e^{t X} }_{t=0} %
		= \eval{ X e^{t X} }_{t=0} %
		= X I %
		= X
	\end{equation}

	Più in generale
	
	\begin{equation}
		c(0) = A e^{0 A B} = A e^{0_{n}} = A
	\end{equation}

	e il vettore tangente
	
	\begin{equation}
		c'(0) = \eval{ \dv{t} A e^{t A^{-1} B} }_{t=0} %
		= A \eval{ \dv{t} e^{t A^{-1} B} }_{t=0} %
		= A \eval{ A^{-1} B e^{t A^{-1} B} }_{t=0} %
		= B I %
		= B
	\end{equation}
\end{proof}

\begin{corollary}\label{cor:det-diff}
	Sia l'applicazione determinante
	
	\map{\det}
		{GL_{n}(\R)}{\R}
		{A}{\det(A)}
		
	Tramite le identificazioni \footnote{%
		Lo spazio tangente può anche essere scritto come $ T_{A}(GL_{n}(\R)) = A \, T_{I}(GL_{n}(\R)) $, derivato dal differenziale della traslazione a sinistra nell'Esempio \ref{example:trasl-diff}.%
	}
	
	\begin{equation}
		\begin{cases}
			T_{A}(GL_{n}(\R)) = M_{n}(\R) \\
			T_{\det(A)}(\R) = \R
		\end{cases}
	\end{equation}

	il differenziale del determinante $ \det_{*A} : T_{A}(GL_{n}(\R)) \to T_{\det(A)}(\R) $ può essere scritto come

	\map{{\det}_{*A}}
		{M_{n}(\R)}{\R}
		{B}{\det(A) \tr(A^{-1} B)}
\end{corollary}

\begin{proof}
	Sia la curva
	
	\map{c}
		{\R \times GL_{n}(\R) \times M_{n}(\R)}{GL_{n}(\R)}
		{(t,A,B)}{A e^{t A^{-1} B}}
	
	con $ c(0) = A $ e $ c'(0) = B $. \\
	Usando le proprietà dell'esponenziale di matrice
	
	\begin{align}
		\begin{split}
			{\det}_{*A}(B) &= \dot{(\det \circ \, c)} (0) \\
			&= \eval{ \dv{t} \det(A e^{t A^{-1} B}) }_{t=0} \\
			&= \eval{ \dv{t} \det(A) \det(e^{t A^{-1} B}) }_{t=0} \\
			&= \det(A) \eval{ \dv{t} \det(e^{t A^{-1} B}) }_{t=0} \\
			&= \det(A) \eval{ \dv{t} e^{\tr(t A^{-1} B)} }_{t=0} \\
			&= \det(A) \eval{ \dv{t} e^{t \tr(A^{-1} B)} }_{t=0} \\
			&= \det(A) \tr(A^{-1} B) \eval{ e^{t \tr(A^{-1} B)} }_{t=0} \\
			&= \det(A) \tr(A^{-1} B)
		\end{split}
	\end{align}
\end{proof}

Come asserito all'inizio della sottosezione, queste proprietà valgono anche se si considera il campo dei numeri complessi $ \C $.

\section{Esempi di sottogruppi di Lie e loro topologia}

Alcuni gruppi di Lie sono $ SL_{n}(\K) $, $ O(n) $, $ U(n) $ e $ SU(n) $: in questa sezione vedremo come questi sono sottogruppi di Lie di $ GL_{n}(\K) $, anch'esso gruppo di Lie.

\subsection{$ SL_{n}(\R) $ come sottogruppo di Lie di $ GL_{n}(\R) $}\label{s-sec:slng-sublie}

Il gruppo di Lie $ SL_{n}(\R) \subset GL_{n}(\R) $ è un sottogruppo di Lie di $ GL_{n}(\R) $ di dimensione

\begin{equation}
	\dim(SL_{n}(\R)) = n^{2} - 1
\end{equation}

con spazio tangente alla matrice unità

\begin{equation}
	T_{I}(SL_{n}(\R)) = \{ X \in M_{n}(\R) \mid \tr(X) = 0 \}
\end{equation}

Inoltre, $ SL_{n}(\R) $ è connesso (e quindi connesso per archi in quanto varietà), chiuso in $ GL_{n}(\R) $ ma non compatto. \\ \\
%
Per dimostrare che sia un sottogruppo di Lie senza usare la proprietà di sottovarietà, dovremmo dimostrare che $ SL_{n}(\R) $ sia un sottogruppo algebrico di $ GL_{n}(\R) $, che sia una sottovarietà immersa e che le sue operazioni siano lisce; usando invece il teorema della preimmagine, possiamo dimostrare che $ SL_{n}(\R) $ è una sottovarietà\footnote{%
	Vedi Esempio \ref{example:sl-subman}.%
} di $ GL_{n}(\R) $ e dunque, per il Teorema \ref{thm:liesub-var}, che è un sottogruppo di Lie embedded di $ GL_{n}(\R) $. \\
Per (ri)dimostrare che $ SL_{n}(\R) $ sia una sottovarietà di $ GL_{n}(\R) $, consideriamo sempre la funzione determinante

\map{f \doteq \det}
	{GL_{n}(\R)}{\R}
	{A}{\det(A)}
	
da cui $ SL_{n}(\R) = f^{-1}(1) $ (il quale rende $ SL_{n}(\R) $ chiuso in quanto immagine di un chiuso tramite una funzione continua). Perché $ SL_{n}(\R) $ sia effettivamente una sottovarietà di $ GL_{n}(\R) $ è necessario\footnote{%
	Vedi Teorema \ref{thm:preimg}.%
} che $ 1 \in \VR_{f} $, i.e. il differenziale $ f_{*A} $ è suriettivo per qualsiasi $ A \in SL_{n}(\R) = f^{-1}(1) $. \\
Preso il differenziale

\map{f_{*A} = {\det}_{*A}}
	{T_{A}(GL_{n}(\R))}{T_{\det(A)}(\R) = \R}
	{B}{\det(A) \tr(A^{-1} B) = \tr(A^{-1} B)}
	
dove $ \det(A) = 1 $ in quanto $ A \in SL_{n}(\R) $, la condizione di suriettività è la seguente:

\begin{equation}
	\forall c \in \R, \E B \in T_{A}(GL_{n}(\R)) = M_{n}(\R) \mid \tr(A^{-1} B) = c
\end{equation}

questo si verifica per $ B = c A / n $ con $ n = \dim(M_{n}(\R)) $, in quanto

\begin{equation}
	\tr(A^{-1} B) = \tr(\dfrac{c A^{-1} A}{n}) %
	= \dfrac{c \tr(I_{n})}{n} %
	= \dfrac{c \, n}{n} %
	= c
\end{equation}

dunque $ SL_{n}(\R) $ è una sottovarietà di $ GL_{n}(\R) $ con dimensione (usando la funzione determinante)

\begin{equation}
	\dim(SL_{n}(\R)) = \dim(GL_{n}(\R)) - \dim(\R) = n^{2} - 1
\end{equation}

Il teorema della preimmagine asserisce inoltre che

\begin{equation}
	T_{I}(F^{-1}(c)) = \ker(F_{*p})
\end{equation}

da cui lo spazio tangente a $ SL_{n}(\R) $

\begin{equation}
	T_{I}(SL_{n}(\R)) = \ker({\det}_{*I}) %
	= \{ X \in M_{n}(\R) = T_{A}(GL_{n}(\R)) \mid \tr(X) = 0 \}
\end{equation}

il quale ha dimensione $ n^{2} - 1 $ (esattamente come $ SL_{n}(\R) $), in quanto le sue entrate sono vincolate dall'unica condizione di avere traccia nulla. Nel caso in cui lo spazio tangente fosse calcolato in un altro punto

\begin{equation}
	T_{A}(SL_{n}(\R)) = A \, T_{I}(SL_{n}(\R))
\end{equation}

Per quanto riguarda la topologia di $ SL_{n}(\R) $, considerando il teorema di Heine-Borel\footnote{%
	Il teorema di Heine-Borel asserisce che un insieme è compatto se e solo se è chiuso e limitato nel suo spazio ambiente.%
} con spazio ambiente lo spazio euclideo $ \R^{n^{2}} $, in quanto $ SL_{n}(\R) \subset GL_{n}(\R) \subset M_{n}(\R) = \R^{n^{2}} $, $ SL_{n}(\R) $ non è compatto (per $ n \geqslant 2 $) in $ GL_{n}(\R) $ in quanto chiuso ma non limitato: prendendo l'insieme di matrici

\begin{equation}
	\left\{ %
			\bmqty{\dmat{ k & 0 \\ 0 & \sfrac{1}{k}, I_{n-2} }} \qcomma k \in \N \setminus \{0\}
	\right\} %
	\subset SL_{n}(\R)
\end{equation}

questo è illimitato al variare di $ k $ in quanto non esiste una palla in $ GL_{n}(\R) $ che lo contenga. \\
Per uno spazio localmente euclideo (come $ SL_{n}(\R) $ in quanto varietà), vale che questo è connesso se e solo se è connesso per archi\footnote{%
	Per uno spazio non localmente euclideo, la connessione per archi implica la connessione ma non viceversa.%
}. Per dimostrare dunque che $ SL_{n}(\R) $ sia connesso, facciamo vedere che una matrice qualsiasi $ A \in SL_{n}(\R) $ può essere unita tramite un arco, i.e. un'applicazione continua, all'identità: usando questo procedimento e il suo inverso, ogni matrice sarà unita a ogni altra matrice di $ SL_{n}(\R) $ tramite archi ($ A \to I \to B $). Ricordiamo che le matrici appartenenti a $ SL_{n}(\R) $ possono essere scritte come prodotto finito di matrici elementari

\begin{equation}
	A = \prod_{k=1}^{s} M_{a_{k}}(i_{k},j_{k}) %
	= \prod_{k=1}^{s} (I + a_{k} E_{i_{k} j_{k}}) %
	\qcomma \forall A \in SL_{n}(\R)
\end{equation}

con $ E_{ij} $ la matrice che ha tutte entrate nulle tranne per quella nella riga $ i $-esima e colonna $ j $-esima dove è presente 1. \\
Sia la curva

\map{c}
	{[0,1]}{SL_{n}(\R)}
	{t}{\prod_{k=1}^{s} M_{t a_{k}}(i_{k},j_{k}) = \prod_{k=1}^{s} (I + t a_{k} E_{i_{k} j_{k}})}
	
un'applicazione continua con $ c(0) = I $ e $ c(1) = A $: questa curva funge da arco che collega la matrice identità ad $ A $, dimostrando dunque che $ SL_{n}(\R) $ è connesso.

\subsection{$ SL_{n}(\C) $ come sottogruppo di Lie di $ GL_{n}(\C) $}

Il gruppo di Lie $ SL_{n}(\C) \subset GL_{n}(\C) $ è un sottogruppo di Lie di $ GL_{n}(\C) $ di dimensione

\begin{equation}
	\dim_{\R}(SL_{n}(\C)) = 2 n^{2} - 2
\end{equation}

con spazio tangente alla matrice unità

\begin{equation}
	T_{I}(SL_{n}(\C)) = \{ X \in M_{n}(\C) \mid \tr(X) = 0 \}
\end{equation}

Inoltre, $ SL_{n}(\C) $ è connesso (e quindi connesso per archi in quanto varietà), chiuso in $ GL_{n}(\C) $ ma non compatto. \\ \\
%
Analogamente a $ SL_{n}(\R) $, l'insieme $ SL_{n}(\C) $ è una sottovarietà di $ GL_{n}(\C) $ in quanto preimmagine del valore regolare $ 1 \in \VR_{f} $ tramite la funzione

\map{f \doteq \det}
	{GL_{n}(\C)}{\C}
	{A}{\det(A)}

da cui la dimensione

\begin{equation}
	\dim_{\R}(SL_{n}(\C)) = \dim_{\R}(GL_{n}(\C)) - \dim_{\R}(\C) = 2 n^{2} - 2
\end{equation}

Per quanto riguarda lo spazio tangente, prendendo il nucleo del differenziale

\map{f_{*A} = \operatorname{det}_{*A}}
	{T_{A}(GL_{n}(\C))}{\C}
	{B}{\tr(A^{-1} B)}

si ottiene

\begin{equation}
	T_{I}(SL_{n}(\C)) = \ker(\operatorname{det}_{*I}) = \{ X \in M_{n}(\C) \mid \tr(X) = 0 \}
\end{equation}

il quale ha dimensione $ 2 n^{2} - 2 $ (esattamente come $ SL_{n}(\C) $) in quanto le sue entrate sono vincolate dalla condizione di avere traccia nulla (che vale come due condizioni in quanto le entrate sono complesse).

\subsection{$ O(n) $ come sottogruppo di Lie di $ GL_{n}(\R) $}\label{s-sec:o-n-subgroup-lie-glnr}

Il gruppo delle matrici ortogonali

\begin{equation}
	O(n) = \{ A \in GL_{n}(\R) \mid A^{T} A = I \}
\end{equation}

è un sottogruppo di Lie embedded di $ GL_{n}(\R) $, in quanto sottogruppo algebrico e sottovarietà di $ GL_{n}(\R) $. \\
È compatto in quanto chiuso e limitato in $ GL_{n}(\R) $: chiuso perché preimmagine di $ I \in \VR_{f} $ tramite l'applicazione continua

\map{f}
	{GL_{n}(\R)}{S(n)}
	{A}{A^{T} A}
	
e limitato in quanto le colonne delle matrici di $ O(n) $ hanno norma unitaria e quindi la somma delle colonne è limitata dalla dimensione $ n $ delle matrici. \\
Non è connesso per $ n \geqslant 1 $, ma costituito da due componenti connesse

\begin{equation}
	O(n) = SO(n) \sqcup SO^{-}(n)
\end{equation}

dove

\begin{gather}
	SO(n) = \{ A \in O(n) \mid \det(A) = 1 \} \\
	SO^{-}(n) = \{ A \in O(n) \mid \det(A) = -1 \}
\end{gather}

Questo perché, siccome

\begin{equation}
	\det(A^{T}) = \det(A) \qcomma \forall A \in M_{n}(\K)
\end{equation}

abbiamo che

\begin{equation}
	\begin{aligned}
		1 &= \det(I) \\
		&= \det(A^{T} A) \\
		&= \det(A^{T}) \det(A) \\
		&= (\det(A))^{2} \\
	\end{aligned} %
	\implies %
	\abs{\det(A)} = 1 \qcomma \forall A \in O(n)
\end{equation}

In particolare, per $ n=1 $ si ha l'insieme

\begin{equation}
	O(n) = \{ x \in \R \mid x^{2} = 1 \} = \{ \pm 1 \}
\end{equation}

La dimensione di questo gruppo di Lie è

\begin{equation}
	\dim(O(n)) = \dfrac{n(n-1)}{2}
\end{equation}

e il suo spazio tangente coincide con l'insieme delle matrici antisimmetriche\footnote{%
	Vedi Esempio \ref{example:on-group-lie}.%
}

\begin{equation}
	T_{I}(O(n)) = \{ X \in M_{n}(\R) \mid X^{T} = -X \}
\end{equation}

Sappiamo anche che $ SO(n) $ è un sottogruppo di Lie di $ O(n) $ della stessa dimensione (in quanto componente connessa) e con lo stesso spazio tangente di $ O(n) $. \\ \\
%
Per dimostrare che $ O(n) $ non sia connesso (e quindi non connesso per archi in quanto varietà), usiamo il fatto che non esiste un arco che connette due matrici che abbiano rispettivamente determinante $ 1 $ e $ -1 $. Siano la matrice identità $ I $ e la matrice

\begin{equation}
	A = \bmqty{\dmat{-1,1,\ddots,1}}
\end{equation}

con

\begin{equation}
	\begin{cases}
		\det(I) = 1 \\
		\det(A) = - 1
	\end{cases}
\end{equation}

Supponiamo per assurdo che esista una curva continua (arco) $ c $ tale che

\begin{equation}
	\begin{cases}
		c : [0,1] \to O(n) \\
		c(0) = I \\
		c(1) = A
	\end{cases}
\end{equation}

Per la curva vale

\begin{equation}
	c(t) \in O(n) \implies \det(c(t)) \neq 0 \qcomma \forall t \in [0,1]
\end{equation}

e abbiamo che

\begin{equation}
	\begin{cases}
		\det(c(0)) = \det(I) = 1 \\
		\det(c(1)) = \det(A) = - 1
	\end{cases}
\end{equation}

Per il teorema del valore medio, esiste un $ t_{0} \in [0,1] $ tale che $ \det(c(t_{0})) = 0 $, ma questo è assurdo in quanto $ c(t) \in O(n) $ e dunque $ O(n) $ non è connesso. \\
Verifichiamo ora che il sottoinsieme $ SO(n) $ sia connesso (per archi) tramite la forma canonica ortogonale:

\begin{equation}
	A \in SO(n) %
	\implies %
	\E P \in O(n) \mid P^{T} A P = %
	\bmqty{ %
			\dmat{ %
					I_{p}, %
					- I_{q}, %
					R_{1}(\theta_{1}), %
					\ddots, %
					R_{k}(\theta_{k}) %
					} %
			}
\end{equation}

con $ p,q,k \in \N $ e $ \theta_{j} \in (0,\pi) $ dove $ j=1,\dots,k $ e $ k = (n-p-q)/2 $, definendo $ R_{j}(\theta_{j}) $ come

\begin{equation}
	R_{j}(\theta_{j}) = %
	\bmqty{ %
			\cos(\theta_{j}) & \sin(\theta_{j}) \\ \\
			- \sin(\theta_{j}) & \cos(\theta_{j}) %
			}
\end{equation}

Siccome il teorema del valore medio vale per $ O(n) $, la condizione $ A \in SO(n) $ implica anche che $ \det(P^{T} A P) = 1 $ e quindi che $ \det(-I_{q}) = 1 $ in quanto $ \det(I_{p}) = \det(R_{j}) = 1 $, i.e. $ q $ deve essere pari. Consideriamo la curva continua (arco)

\map{c}
	{[0,1]}{SO(n)}
	{t}{\bmqty{ %
				\dmat{ %
						I_{p}, %
						R_{1}(\pi t), %
						\ddots, %
						R_{q}(\pi t), %
						R_{1}(t \theta_{1}), %
						\ddots, %
						R_{k}(t \theta_{k}) %
						}%
				} %
		}

con $ q $ pari, per cui vale

\begin{equation}
	\begin{cases}
		c(0) = I \\
		c(1) = P^{T} A P
	\end{cases}
\end{equation}

Considerando ora un'altra curva continua

\map{g \doteq P \, c \, P^{T}}%
	{[0,1]}{SO(n)}%
	{t}{ P \bmqty{ %
					\dmat{ %
							I_{p}, %
							R_{1}(\pi t), %
							\ddots, %
							R_{q}(\pi t), %
							R_{1}(t \theta_{1}), %
							\ddots, %
							R_{k}(t \theta_{k}) %
							}%
					} P^{T} %
	}

per cui vale

\begin{equation}
	\begin{cases}
		g(0) = I \\
		g(1) = A
	\end{cases}
\end{equation}

Abbiamo dunque costruito un arco che collega qualunque matrice di $ SO(n) $ alla matrice identità, dunque lo spazio è connesso (per archi). \\
L'insieme $ SO(n) $ è una componente connessa di $ O(n) $ anche perché se esistesse un insieme connesso $ M $ tale che

\begin{equation}
	SO(n) \subsetneq M \subset O(n)
\end{equation}

un elemento di $ SO(n) $ potrebbe essere unito a un elemento $ A \in M $ con $ \det(A) = -1 $ e, siccome $ M $ è connesso (per archi), esisterebbe un arco $ c $ tale che

\begin{equation}
	\begin{cases}
		c : [0,1] \to M \\
		c(0) = I \\
		c(1) = A
	\end{cases}
\end{equation}

Questo è assurdo, per il teorema del valore medio, perché il determinante di $ \det(c(t)) \neq 0 $ ma $ c $ è un'applicazione continua che passerebbe da $ \det(c(0)) = 1 $ a $ \det(c(1)) = -1 $. \\
Analogamente, l'insieme delle matrici con determinante pari a $ -1 $, i.e. $ SO^{-}(n) $, è l'altra componente connessa di $ O(n) $, per cui vale

\begin{equation}
	O(n) = SO(n) \sqcup SO^{-}(n)
\end{equation}

Consideriamo le due seguenti proposizioni:

\begin{definition}
	Presi uno spazio topologico $ X $ e una sua componente connessa $ C \subset X $, allora $ C $ è chiuso.
\end{definition}

\begin{proof}
	La chiusura $ \bar{C} $ di una componente connessa $ C $ è ancora connessa e anche chiusa, dunque $ \bar{C} \equiv C $, perciò $ C $ è chiusa.
\end{proof}

\begin{definition}\label{prop:conn-comp-open}
	Siano $ X $ uno spazio topologico localmente euclideo e $ C \subset X $ una sua componente connessa (per archi, in quanto $ X $ è localmente euclideo), allora $ C $ è aperto in $ X $.
\end{definition}

\begin{proof}
	Sia un punto $ p \in C \subset X $: essendo $ X $ localmente euclideo, esiste un intorno aperto $ U \ni p $ che sia connesso (per archi), in quanto la connessione (per archi) si preserva per omeomorfismi. \\
	Se, per assurdo, $ U \not\subset C $ allora $ U \cup C $ è un connesso tale che $ U \cup C \supsetneq C $ ma questo non è possibile perché avremmo trovato un connesso che contiene strettamente $ C $, in contrasto con il fatto che $ C $ sia una componente connessa. \\
	Questo rende il punto (arbitrario) $ p $ interno a $ C $ e dunque $ C $ è aperto.
\end{proof}

Queste proposizioni insieme asseriscono che uno spazio topologico localmente euclideo, e.g. una varietà differenziabile, ha le sue componenti connesse sia aperte che chiuse: nello specifico, possiamo dire che l'insieme $ SO(n) $ è sia chiuso che aperto in $ O(n) $. \\
Qualunque aperto di una varietà differenziabile è una sottovarietà\footnote{%
	Vedi Osservazione \ref{rem:subvar-open}.%
}, dunque $ SO(n) $ è una sottovarietà di $ O(n) $ e, per questo, è anche un sottogruppo di Lie embedded. La dimensione e lo spazio tangente di un aperto di una varietà sono gli stessi di quelli della varietà, perciò

\begin{gather}
	\dim(SO(n)) = \dfrac{n(n-1)}{2} \\
	\nonumber \\
	T_{I}(SO(n)) = T_{I}(O(n)) = \{ X \in M_{n}(\R) \mid X^{T} = -X \}
\end{gather}

Essendo chiuso e limitato (o, equivalentemente, chiuso nel compatto $ O(n) $), $ SO(n) $ è compatto. \\
Gli stessi ragionamenti sono applicabili all'altra componente connessa $ SO^{-}(n) $ di $ O(n) $. \\
Per piccoli valori di $ n $, abbiamo che:

\begin{itemize}
	\item $ SO(1) = \{ 1 \} $;
	
	\item $ SO(2) $ è diffeomorfo\footnote{%
		Vedi Esercizio \ref{BONUS3-1}.%
	} a $ \S^{1} $;
	
	\item $ SO(3) $ è diffeomorfo a $ \rp{3} $.
\end{itemize}

\subsection{$ U(n) $ come sottogruppo di Lie di $ GL_{n}(\C) $}

L'insieme delle matrici unitarie

\begin{equation}
	U(n) = \{ A \in GL_{n}(\C) \mid A^{*} A = I \}
\end{equation}

dove $ A^{*} \doteq \bar{A}^{T} \equiv A^{-1} $, è un sottogruppo di Lie embedded di $ GL_{n}(\C) $ di dimensione

\begin{equation}
	\dim_{\R}(U(n)) = n^{2}
\end{equation}

con spazio tangente l'insieme delle matrici antihermitiane

\begin{equation}
	T_{I}(U(n)) = \{ X \in M_{n}(\C) \mid X^{*} = -X \}
\end{equation}

L'insieme $ U(n) $ è anche compatto e connesso. \\
Per dimostrare che $ U(n) $ sia un sottogruppo di Lie embedded di $ GL_{n}(\C) $, è sufficiente mostrare che $ U(n) $ sia una sottovarietà di $ GL_{n}(\C) $. Consideriamo l'applicazione continua

\map{f}
	{GL_{n}(\C)}{H(n)}
	{A}{A^{*} A}

dove il codominio è l'insieme delle matrici hermitiane

\begin{equation}
	H(n) = \{ B \in M_{n}(\C) \mid B^{*} = B \}
\end{equation}

Abbiamo che $ U(n) = f^{-1}(I) $, dunque dobbiamo verificare che $ I \in \VR_{f} $ per poter usare il teorema della preimmagine di un valore regolare. Prendiamo quindi differenziale $ f_{*A}(B) $ con $ A \in GL_{n}(\C) $ e $ B \in T_{A}(GL_{n}(\C)) = M_{n}(\C) $ e verifichiamo che sia suriettivo nella restrizione a matrici unitarie: presa una curva liscia $ c $ tale che

\begin{equation}
	\begin{cases}
		c : (-\varepsilon,\varepsilon) \to GL_{n}(\C) \\
		c(0) = A \\
		c'(0) = B
	\end{cases}
\end{equation}

possiamo scrivere

\begin{align}
	\begin{split}
		f_{*A}(B) &= \eval{ \dv{t} (f \circ c(t)) }_{t=0} \\
		&= \eval{ \dv{t} f(c(t)) }_{t=0} \\
		&= \eval{ \dv{t} (c(t)^{*} c(t)) }_{t=0} \\
		&= \dot{c}(0)^{*} c(0) + c(0)^{*} \dot{c}(0) \\
		&= B^{*} A + A^{*} B
	\end{split}
\end{align}

da cui, tramite le identificazioni $ T_{A}(GL_{n}(\C)) = M_{n}(\C) $ e $ T_{f(A)}(H(n)) = H(n) $ ($ H(n) $ è uno spazio vettoriale) otteniamo

\map{f_{*A}}
	{M_{n}(\C)}{H(n)}
	{B}{B^{*} A + A^{*} B}

A questo punto abbiamo che

\begin{equation}
	I \in \VR_{f} \iff \E B \in M_{n}(\C) \mid f_{*A}(B) = C %
	\qcomma \forall C \in H(n), \, \forall A \in U(n) = f^{-1}(I)
\end{equation}

cioè $ f_{*A} $ è suriettiva per $ A \in U(n) $: questo si verifica per $ B = AC/2 $

\begin{align}
	\begin{split}
		f_{*A} \left( \dfrac{A C}{2} \right) &= \left( \dfrac{A C}{2} \right)^{*} A + A^{*} \left( \dfrac{A C}{2} \right) \\
		&= \dfrac{1}{2} (C^{*} A^{*} A + A^{*} A C) \\
		&= \dfrac{1}{2} (C^{*} + C) \\
		&= \dfrac{1}{2} (C + C) \\
		&= C
	\end{split}
\end{align}

in quanto $ C \in H(n) $, perciò $ I \in \VR_{f} $ e dunque $ U(n) $ è una sottovarietà di $ GL_{n}(\C) $. \\
La dimensione di $ U(n) $ deriva dal teorema della preimmagine\footnote{%
	La dimensione di $ H(n) $ deriva dal fatto che gli $ n $ elementi della diagonale sono reali mentre quelli sopra la diagonale determinano anche quelli sotto e sono complessi, da cui il termine $ 2 (n(n-1)/2) $.%
}

\begin{equation}
	\dim_{\R}(U(n)) = \dim_{\R}(GL_{n}(\C)) - \dim_{\R}(H(n)) %
	= 2 n^{2} - \left( n +\dfrac{n (n-1)}{2} \, 2 \right) %
	= 2 n^{2} - n^{2} %
	= n^{2}
\end{equation}

Sempre dal teorema della preimmagine, lo spazio tangente di $ U(n) $ è

\begin{equation}
	T_{I}(U(n)) = \ker(f_{*I}) = \{ X \in M_{n}(\C) \mid X^{*} = -X \} = H^{-}(n)
\end{equation}

ovvero lo spazio delle matrici antihermitiane. \\
La dimostrazione\footnote{%
	Vedi Esercizio \ref{exer3-5}.%
} che $ U(n) $ sia compatto è analoga a quella per $ O(n) $. Per dimostrare che sia connesso, sia $ A \in U(n) $, per il corollario del teorema spettrale esiste una matrice $ U \in U(n) $ tale che

\begin{equation}
	U^{-1} A U = %
	\bmqty{ %
			\dmat{ %
					e^{i \theta_{1}}, %
					\ddots, %
					e^{i \theta_{n}} %
					} %
			}
\end{equation}

con $ \theta_{j} \in \R $ per $ j=1,\dots,n $. Possiamo definire una curva continua

\map{c}
	{[0,1]}{U(n)}
	{t}{ %
			U \bmqty{ %
					\dmat{ %
							e^{i t \theta_{1}}, %
							\ddots, %
							e^{i t \theta_{n}} %
							} %
					} U^{-1} %
			}

dove

\begin{equation}
	\begin{cases}
		c(0) = I \\
		c(1) = A
	\end{cases}
\end{equation}

Esiste dunque un arco che connette ogni matrice unitaria alla matrice identità, perciò $ U(n) $ è connesso.

\subsection{$ SU(n) $ come sottogruppo di Lie di $ U(n) $}\label{s-sec:su-n-lie-subg}

L'insieme delle matrici unitarie speciali

\begin{align}
	\begin{split}
		SU(n) &= \{ A \in GL_{n}(\C) \mid A^{*} A = I \wedge \det(A) = 1 \} \\
		&= \{ A \in U(n) \mid \det(A) = 1 \}
	\end{split}
\end{align}

è un sottogruppo di Lie embedded di $ U(n) $ di dimensione $ \dim_{\R}(SU(n)) = n^{2} - 1 $ con spazio tangente l'insieme delle matrici antihermitiane a traccia nulla

\begin{align}
	\begin{split}
		T_{I}(SU(n)) &= \{ X \in M_{n}(\C) \mid X^{*} = -X \wedge \tr(X) = 0 \} \\
		&= \{ X \in T_{I}(U(n)) \mid \tr(X) = 0 \}
	\end{split}
\end{align}

L'insieme $ SU(n) $ è anche compatto e connesso. \\
Per dimostrare che sia un sottogruppo di Lie embedded di $ U(n) $, utilizziamo il teorema della preimmagine per dimostrare che ne sia una sottovarietà tramite l'applicazione

\map{f}
	{U(n)}{\S^{1}}
	{A}{\det(A)}

Questa applicazione ha come codominio $ \S^{1} $ perché per $ A \in U(n) $ vale\footnote{%
	È possibile dimostrare questo fatto anche nel seguente modo: siccome due matrici simili hanno stesso determinante (vedi Proposizione \ref{prop:similar-matrix-properties}) e siccome ogni matrice unitaria $ A \in U(n) $ è simile a una matrice del tipo
	
	\begin{equation*}
		U^{*} A U = %
		\bmqty{ %
				\dmat{ %
						e^{i \theta_{1}} & & 0 \\
						& \ddots & \\
						0 & & e^{i \theta_{n}} %
						}
				} %
		\qcomma \theta_{j} \in \R, \, j=1,\dots,n
	\end{equation*}
	
	con $ U \in U(n) $ (vedi Corollario \ref{cor:unit-matrix-similar}), possiamo asserire che
	
	\begin{equation*}
		\det(U) = \det(U^{-1} A U) %
		= e^{i \sum_{i=1}^{n} \theta_{i}} %
		\doteq e^{i \Theta} %
		\qcomma \Theta \in \R
	\end{equation*}
	%
}

\begin{equation}
	A^{*} A = I %
	\implies %
	\begin{aligned}
		\det(I) &= \det(A^{*} A) \\
		1 &= \det(A^{*}) \det(A) \\
		&= \overline{\det(A)} \det(A) \\
		&= \abs{\det(A)}^{2}
	\end{aligned} %
	\implies %
	\det(A) = e^{i \theta} \in \S^{1} \qcomma \theta \in \R
\end{equation}

A questo punto, $ SU(n) $ è la controimmagine di $ 1 \in \S^{1} \subset \C $ tramite il determinante, i.e. $ SU(n) = f^{-1}(1) $: perché sia una sottovarietà di $ U(n) $ è sufficiente che $ 1 \in \VR_{f} $. Prendiamo quindi il differenziale dell'applicazione determinante\footnote{%
	Vedi Corollario \ref{cor:det-diff}.%
}

\map{f_{*A}}
	{T_{A}(U(n))}{T_{\det(A)}(\S^{1})}
	{B}{\det(A) \tr(A^{-1} B)}
	
Siccome siamo interessati agli elementi di $ SU(n) $, abbiamo che $ \det(A) = 1 $, i.e. per $ A \in SU(n) $

\map{f_{*A}}
	{T_{A}(U(n))}{T_{1}(\S^{1})}
	{B}{\tr(A^{-1} B)}

A questo punto, dobbiamo studiare $ T_{1}(\S^{1}) $ con $ \S^{1} \subset \C $: possiamo trovare gli elementi di questo spazio tangente considerando la curva

\sbs{0.5}{%
			\map{c}
				{(-\varepsilon,\varepsilon)}{\S^{1}}
				{t}{e^{i a t}}
			}
	{0.5}{%
			\begin{equation}
				\begin{cases}
					a \in \R \\
					c(0) = 1 \\
					\dot{c}(0) = i a
				\end{cases}
			\end{equation}
			}

Dato che $ \dot{c}(0) \in T_{1}(\S^{1}) $, possiamo asserire che

\begin{equation}
	T_{1}(\S^{1}) = \ev{i}_{\R}
\end{equation}

cioè lo spazio tangente è generato da tutti i vettori paralleli all'asse immaginario che partano da $ 1 \in \C $. \\
A questo punto, perché $ 1 \in \VR_{f} $, è necessario che il differenziale sia suriettivo per qualsiasi $ A \in SU(n) $ o equivalentemente

\begin{equation}
	1 \in \VR_{f} \iff \forall a \in \R, \E B \in T_{A}(U(n)) \mid f_{*A} (B) = \tr(A^{-1} B) = i a \in T_{1}(\S^{1}) %
	\qcomma A \in SU(n)
\end{equation}

Abbiamo dimostrato che

\begin{equation}
	\begin{cases}
		T_{A}(U(n)) = A \, T_{I}(U(n)) \\
		T_{I}(U(n)) = H^{-}(n) & \text{ matrici antihermitiane}
	\end{cases}
\end{equation}

perciò

\begin{equation}
	T_{A}(U(n)) = \{ AX \in M_{n}(\C) \mid X^{*} = -X \}
\end{equation}

Per soddisfare la condizione, prendiamo $ B = AX $ con $ X = (i a/n) I $, da cui

\begin{equation}
	\tr(A^{-1} B) = \tr(A^{-1} A \, \dfrac{ia}{n} \, I) %
	= \dfrac{ia}{n} \tr(I) %
	= \dfrac{ia}{n} \, n %
	= i a
\end{equation}

Questo dimostra che $ 1 \in \VR_{f} $ perciò $ SU(n) $ è una sottovarietà e quindi un sottogruppo di Lie embedded di $ U(n) $ di dimensione

\begin{equation}
	\dim_{\R}(SU(n)) = \dim_{\R}(U(n)) - \dim_{\R}(\S^{1}) = n^{2} - 1
\end{equation}

Lo spazio tangente di $ SU(n) $ deriva ancora dal teorema della preimmagine

\begin{equation}
	T_{I}(SU(n)) = \ker({\det}_{*A}) %
	= \{ X \in T_{I}(U(n)) \mid \tr(X) = 0 \}
\end{equation}

ma sappiamo che

\begin{equation}
	T_{I}(U(n)) = \{ X \in M_{n}(\C) \mid X^{*} = - X \}
\end{equation}

perciò

\begin{equation}
	T_{I}(SU(n)) = \{ X \in M_{n}(\C) \mid X^{*} = -X \wedge \tr(X) = 0 \}
\end{equation}

Essendo $ SU(n) $ chiuso (perché controimmagine del chiuso $ 1 \in \C $ tramite l'applicazione continua determinante) nel compatto $ U(n) $, $ SU(n) $ è compatto\footnote{%
	Alternativamente, si può utilizzare il fatto che sia chiuso e limitato in $ U(n) $.%
}. \\
L'insieme $ SU(n) $ è connesso perché, presa una matrice $ A \in SU(n) $, per il corollario del teorema spettrale, esiste una matrice $ U \in U(n) $ tale che

\begin{equation}
	U^{-1} A U = %
	\bmqty{ %
			\dmat{ %
			 		e^{i \theta_{1}}, %
			 		\ddots, %
			 		e^{i \theta_{n}} %
		 			} %
	 		}
\end{equation}

con $ \theta_{j} \in \R $ per $ j=1,\dots,n $ e siccome il determinante deve essere unitario

\begin{equation}
	\sum_{j=1}^{n} \theta_{j} = 2 k \pi \qcomma k \in \Z
\end{equation}

Possiamo definire una curva continua

\map{c}
	{[0,1]}{SU(n)}
	{t}{ %
			U \bmqty{ %
						\dmat{ %
								e^{i t \theta_{1}}, %
								\ddots, %
								e^{i t \theta_{n}} %
								} %
						} U^{-1} %
			}

dove

\begin{equation}
	\begin{cases}
		c(0) = I \\
		c(1) = A
	\end{cases}
\end{equation}

e la curva appartiene a $ SU(n) $ per Binet. \\
Esiste dunque un arco che connette ogni matrice unitaria speciale alla matrice identità perciò $ SU(n) $ è connesso. \\
Notiamo anche che\footnote{%
	Vedi Esercizio \ref{BONUS3-2}.%
}

\begin{gather}
	SU(1) = U(1) = SO(2) \simeq \S^{1} \\
	SU(2) \simeq \S^{3}
\end{gather}

\subsection{Isomorfismo tra $ GL_{n}(\R) $ e $ SL_{n}(\R) \times \R \setminus \{0\} $}

Un gruppo di Lie è un gruppo algebrico con operazioni lisce che sia anche una varietà differenziabile. Due gruppi di Lie possono essere legati da un diffeomorfismo, in quanto varietà, ma non essere isomorfi: questo succede se il diffeomorfismo non è anche un omomorfismo di gruppi, i.e. non preserva la moltiplicazione tra i gruppi. \\
Sia l'applicazione

\map{f}
	{GL_{n}(\R)}{SL_{n}(\R) \times \R \setminus \{0\}}
	{A}{(A M_{1/\det(A)},\det(A))}

dove la matrice $ M_{r} $ è definita come

\begin{equation}
	M_{r} \doteq %
	\bmqty{ %
			r & 0_{1,n} \\ \\
			0_{n,1} & I_{n-1} %
			} = %
	\bmqty{ %
			\dmat{ %
					r, %
					1, %
					\ddots, %
					1 %
					} %
			} %
	\qcomma \qquad \det(M_{r}) = r \in \R
\end{equation}

e il codominio è un prodotto diretto tra i gruppi di Lie, i.e. il gruppo delle matrici invertibili con determinante unitario $ SL_{n}(\R) $ e il gruppo $ (\R \setminus \{0\},\cdot) $ con $ \cdot $ il prodotto tra numeri reali. \\
L'applicazione $ f $ è un diffeomorfismo ma non un isomorfismo. \\
L'immagine è ben definita, i.e. il primo termine appartiene a $ SL_{n}(\R) $ in quanto

\begin{equation}
	\det(A M_{1/\det(A)}) = \det(A) \det(M_{1/\det(A)}) = \det(A) \, \dfrac{1}{\det(A)} = 1
\end{equation}

e il secondo appartiene a $ \R \setminus \{0\} $, poiché $ \det(A) \neq 0 $ per qualsiasi $ A \in GL_{n}(\R) $. \\
L'applicazione $ f $ è liscia perché tutte le entrate sono lisce e perché, considerando l'applicazione liscia

\map{g}
	{GL_{n}(\R)}{GL_{n}(\R) \times \R \setminus \{0\}}
	{A}{(A M_{1/\det(A)},\det(A))}
 
il codominio di $ f $ è una restrizione del codominio di $ g $ a una sua sottovarietà. \\
L'inversa di $ f $ (ancora liscia anche perché restrizione di un'applicazione liscia) è

\map{f^{-1}}
	{SL_{n}(\R) \times \R \setminus \{0\}}{GL_{n}(\R)}
	{(S,r)}{S M_{r}}

in quanto

\begin{align}
	\begin{split}
		(f^{-1} \circ f)(A) &= f^{-1} \left( A M_{1/\det(A)},\det(A) \right) \\
		&= A M_{1/\det(A)} M_{\det(A)} \\
		&= A
	\end{split}
\end{align}

\begin{align}
	\begin{split}
		(f \circ f^{-1})(S,r) &= f(S M_{r}) \\
		&= \left( S M_{r} M_{{1/\det(S M_{r})}},\det(S M_{r}) \right) \\
		&= \left( S M_{r} M_{1/(\det(S) \det(M_{r}))},\det(S) \det(M_{r}) \right) \\
		&= \left( S M_{r} M_{1/r},r \right) \\
		&= (S,r)
	\end{split}
\end{align}

dove $ \det(S) = 1 $ poiché $ S \in SL_{n} (\R) $. Questo dimostra che $ f $ è un diffeomorfismo. \\
Perché sia un isomorfismo, oltre ad essere un diffeomorfismo, dovrebbe essere un omomorfismo e quindi preservare la moltiplicazione, ma

\begin{align}
	\begin{split}
		f(A B) &\neq f(A) \, f(B) \\
		\left( A B M_{1/\det(A B)}, \det(A B) \right) &\neq \left( A M_{1/\det(A)},\det(A) \right) \, \left( B M_{1/\det(B)},\det(B) \right) \\
		&\neq \left( A M_{1/\det(A)} B M_{1/\det(B)},\det(A) \det(B) \right) \\
		&\neq \left( A M_{1/\det(A)} B M_{1/\det(B)},\det(A B) \right)
	\end{split}
\end{align}

in quanto, in generale, $ [B, M_{1/\det(A)}] \neq 0 $. \\
Da questa analisi otteniamo che

\begin{equation}
	GL_{n}(\R) \stackrel{diff.}{\simeq} SL_{n}(\R) \times \R \setminus \{0\} %
	\quad \wedge \quad %
	GL_{n}(\R) \stackrel{iso.}{\not\simeq} SL_{n}(\R) \times \R \setminus \{0\}
\end{equation}

Inoltre, siccome $ \R \setminus \{0\} $ è costituito da due componenti connesse

\begin{equation}
	\R \setminus \{0\} = \R^{+} \sqcup \R^{-}
\end{equation}

dal diffeomorfismo $ f $, ricaviamo che anche $ GL_{n}(\R) $ è costituito da due componenti connesse (in quanto $ SL_{n}(\R) $ è connesso), i.e. le matrici con determinante positivo e quelle con determinante negativo

\begin{equation}
	GL_{n}(\R) = (SL_{n}(\R) \times \R^{+}) \sqcup (SL_{n}(\R) \times \R^{-}) %
	\doteq GL_{n}^{+}(\R) \sqcup GL_{n}^{-}(\R)
\end{equation}

In generale, $ GL_{n}(\R) $ e $ SL_{n}(\R) \times \R \setminus \{0\} $ non sono quindi isomorfi: questo rimane vero per $ n $ pari, ma per $ n $ dispari è possibile trovare un isomorfismo tra questi due gruppi. \\
Preso un gruppo algebrico $ G $, il suo \textit{centro} $ Z(G) $ è l'insieme  degli elementi di $ G $ che commutano con tutti gli altri elementi, i.e.

\begin{equation}
	Z(G) = \{ z \in G \mid zg = gz, \, \forall g \in G \}
\end{equation}

Il centro del gruppo lineare è l'insieme di tutte le matrici scalari, i.e.

\begin{equation}
	Z(GL_{n}(\R)) = \{ A \in M_{n}(\R) \mid A = c I, \, c \in \R \setminus \{0\} \}
\end{equation}

Questo significa che

\begin{equation}
	Z(GL_{n}(\R)) = \{I\} \times \R \setminus \{0\}
\end{equation}

Siccome il centro del prodotto di due gruppi è il prodotto del centro dei gruppi, i.e.

\begin{equation}
	Z(G) \times Z(H) = Z(G \times H) \qcomma \forall G,H \text{ gruppi}
\end{equation}

possiamo scrivere

\begin{equation}
	Z(SL_{n}(\R) \times \R \setminus \{0\}) = Z(SL_{n}(\R)) \times Z(\R \setminus \{0\})
\end{equation}

L'insieme dei reali è commutativo e, siccome $ Z(G) = G $ per $ G $ abeliano, abbiamo che coincide con il suo centro, mentre il gruppo lineare speciale ha centro diverso a seconda della dimensione delle matrici:

\begin{equation}
	Z(SL_{n}(\R)) = %
	\begin{cases}
		\{ \pm I \}, & n = 2 k \\
		\{ I \}, & n = 2 k + 1
	\end{cases} %
	\qquad k \in \N
\end{equation}

A questo punto, abbiamo che per $ n $ pari

\begin{align}
	\begin{split}
		Z(GL_{n}(\R)) &\neq Z(SL_{n}(\R) \times \R \setminus \{0\}) \\
		\{ I \} \times \R \setminus \{0\} &\neq Z(SL_{n}(\R)) \times Z(\R \setminus \{0\}) \\
		&\neq \{ \pm I \} \times \R \setminus \{0\}
	\end{split}
\end{align}

mentre per $ n $ dispari

\begin{equation}
	Z(GL_{n}(\R)) = Z(SL_{n}(\R) \times \R \setminus \{0\}) = \{ I \} \times \R \setminus \{0\}
\end{equation}

Questo prova che non sia possibile costruire un isomorfismo per $ n $ pari in quanto, se i gruppi fossero isomorfi, in particolare dovrebbero esserlo anche i loro centri, fatto che non si verifica in questo caso. \\
A questo punto, esibiamo un isomorfismo di Lie esplicito, per $ n $ dispari, tra $ GL_{n}(\R) $ e $ SL_{n}(\R) \times \R \setminus \{0\} $:

\map{h}
	{GL_{n}(\R)}{SL_{n}(\R) \times \R \setminus \{0\}}
	{A}{\left( \det(A)^{-1/n} \, A, \det(A) \right)}

Notiamo che la prima entrata è ben definita in quanto esiste sempre la radice $ n $-esima di un numero reale per $ n $ dispari, che l'applicazione è liscia, invertibile, e la sua inversa è liscia

\map{h^{-1}}
	{SL_{n}(\R) \times \R \setminus \{0\}}{GL_{n}(\R)}
	{(S,r)}{r^{1/n} S}
	
Verifichiamo che siano inverse:

\begin{align}
	\begin{split}
		(h^{-1} \circ h)(A) &= h^{-1} \left( \det(A)^{-1/n} \, A, \det(A) \right) \\
		&= \det(A)^{1/n} \det(A)^{-1/n} A \\
		&= A
	\end{split}
\end{align}

\begin{align}
	\begin{split}
		(h \circ h^{-1})(S,r) &= h \left( r^{1/n} S \right) \\
		&= \left( \det(r^{1/n} S)^{-1/n} \, r^{1/n} S, \det(r^{1/n} S) \right) \\
		&= \left( \left( (r^{1/n})^{n} \det(S) \right)^{-1/n} \, r^{1/n} S, (r^{1/n})^{n} \det(S) \right) \\
		&= \left( r^{-1/n} r^{1/n} S, r \right) \\
		&= (S,r)
	\end{split}
\end{align}

dove $ \det(S) = 1 $ poiché $ S \in SL_{n} (\R) $ e

\begin{equation}
	\det(c \, A) = c^{n} \det(A) \qcomma \forall c \in \R, \, \forall A \in M_{n}(\R)
\end{equation}

Verifichiamo ora che sia un omomorfismo:

\begin{align}
	\begin{split}
		h(A B) &= \left( \det(A B)^{-1/n} \, A B, \det(A B) \right) \\
		&= \left( \det(A)^{-1/n} \, A \, \det(B)^{-1/n} \, B, \det(A) \det(B) \right) \\
		&= \left( \det(A)^{-1/n} \, A, \det(A) \right) \, \left( \det(B)^{-1/n} \, B, \det(B) \right) \\
		&= h(A) \, h(B)
	\end{split}
\end{align}

Da cui otteniamo che $ h $ è un isomorfismo di gruppi di Lie tra $ GL_{n}(\R) $ e $ SL_{n}(\R) \times \R \setminus \{0\} $ per $ n $ dispari.

\section{Algebre di Lie}

Un'\textit{algebra di Lie su un campo} $ \K $ è uno spazio vettoriale\footnote{%
	Non necessariamente di dimensione finita.%
} $ V $ dotato di un'applicazione

\map{[,]}
	{V \times V}{V}
	{(x,y)}{[x,y]}

tale che:

\begin{itemize}
	\item sia bilineare, i.e.
	
	\begin{equation}
		\begin{cases}
			[ax+by,z] = a[x,z] + b[y,z], & \forall a,b \in \R, \, \forall x,y,z \in V \\
			[x,cz+dw] = c[x,z] + d[x,w], & \forall c,d \in \R, \, \forall x,z,w \in V
		\end{cases}
	\end{equation}
	
	\item sia antisimmetrica, i.e.
	
	\begin{equation}
		[x,y] = -[y,x] \qcomma \forall x,y \in V
	\end{equation}
	
	\item valga l'\textit{identità di Jacobi}
	
	\begin{equation}
		[x,[y,z]] + [y,[z,x]] + [z,[x,y]] = 0 \qcomma \forall x,y,z \in V
	\end{equation}
\end{itemize}

\subsubsection{\textit{Esempi}}

\paragraph{1) Algebra di Lie abeliana}

Se $ V $ è uno spazio vettoriale su $ \K $, possiamo definire il commutatore nullo

\begin{equation}
	[x,y] = 0 \qcomma \forall x,y \in V
\end{equation}

che definisce l'algebra di Lie \textit{abeliana} (gli elementi $ x $ e $ y $ commutano).

\paragraph{2) Insieme delle matrici quadrate}

Siano lo spazio vettoriale $ M_{n}(\K) $ delle matrici quadrate su un campo $ \K $ e il commutatore

\begin{equation}
	[A,B] = AB - BA \qcomma A,B \in M_{n}(\K)
\end{equation}

La verifica delle proprietà è immediata e dunque $ (M_{n}(\K),[,]) $ è un'algebra di Lie.

\paragraph{3) Commutatore}

Una qualunque algebra $ A $ su un campo $ \K $ con commutatore

\begin{equation}
	[a,b] = ab - ba \qcomma \forall a,b \in A
\end{equation}

è un'algebra di Lie su campo $ \K $.

\paragraph{4) Campi di vettori}

Siano $ M $ una varietà differenziabile, $ \chi(M) $ l'insieme dei campi di vettori lisci sulla varietà, e il commutatore tra campi

\begin{equation}
	[X,Y] = XY - YX \qcomma X,Y \in \chi(M)
\end{equation}

allora $ (\chi(M),[,]) $ è un'algebra di Lie su $ \R $. Lo spazio vettoriale $ \chi(M) $ ha dimensione infinita.

\subsection{Algebre e gruppi di Lie}

Siano un gruppo di Lie $ G $, il suo elemento neutro $ e $, e lo spazio vettoriale tangente a $ G $ in $ e $, i.e. $ T_{e}(G) $, il quale ha la stessa dimensione del gruppo di Lie

\begin{equation}
	\dim(T_{e}(G)) = \dim(G)
\end{equation}

Vogliamo definire una struttura di algebra di Lie su $ T_{e}(G) $ che sia legata al gruppo di Lie $ G $: per fare questo, definiamo un commutatore su $ T_{e}(G) $

\map{[,]}
	{T_{e}(G) \times T_{e}(G)}{T_{e}(G)}
	{(v,w)}{[v,w]}

che sia bilineare, antisimmetrico e che rispetti l'identità di Jacobi. \\
L'idea è usare il commutatore dei campi di vettori lisci sul gruppo di Lie $ G $, i.e. considerare l'algebra $ (\chi(G),[,]) $, per indurre un commutatore su $ T_{e}(G) $.

\subsubsection{Campi di vettori invarianti a sinistra}

Sia $ X $ un campo di vettori su $ G $ (non necessariamente liscio), diremo che $ X $ è \textit{invariante a sinistra} se

\begin{equation}
	(L_{g})_{*} X = X \qcomma \forall g \in G
\end{equation}

cioè il pushforward\footnote{%
	Sia un'applicazione $ F : N \to M $ un diffeomorfismo (quindi sia iniettiva che suriettiva), il pushforward di un campo di vettori $ X $ tramite $ F $ è definito come

	\begin{equation*}
		(F_{*} (X))_{q} = F_{*F^{-1}(q)} (X_{F^{-1}(q)})
	\end{equation*}%
} di $ X $ tramite la traslazione a sinistra è identico a sé stesso, dove la traslazione a sinistra è definita come

\map{L_{g}}
	{G}{G}
	{h}{g h}

la quale è un diffeomorfismo (quindi è possibile definire il pushforward in quanto $ L_{g} $ è sia iniettiva che suriettiva) con inversa $ L_{g}^{-1} = L_{g^{-1}} $. Equivalentemente, il campo di vettori $ X $ è invariante per traslazioni a sinistra se

\begin{equation}
	(L_{g})_{*h} (X_{h}) = X_{gh} \qcomma \forall g,h \in G
\end{equation}

ciò significa che $ X $ è $ L_{g} $-related\footnote{%
	Sia un diffeomorfismo $ F : N \to M $, un campo di vettori $ X $ è $ F $-related a un altro campo di vettori $ Y $ se
	
	\begin{equation*}
		F_{*p}(X_{p}) = Y_{F(p)} \qcomma \forall p \in N
	\end{equation*}

	i.e. $ Y $ è il pushforward di $ X $ tramite $ F $.%
} a sé stesso.

\begin{remark}
	Un campo di vettori invariante a sinistra $ X $ è determinato dal suo valore nell'elemento neutro $ X_{e} $ in quanto
	
	\begin{equation}
		X_{g} = (L_{g})_{*e} (X_{e})
	\end{equation}
\end{remark}

I campi di vettori invarianti a sinistra sono importanti perché esiste un isomorfismo (di spazi vettoriali) tra questi e lo spazio tangente a un gruppo di Lie.

\begin{theorem}[Proprietà dei campi di vettori invarianti a sinistra]
	Siano $ G $ un gruppo di Lie ed $ e \in G $ il suo elemento neutro. Valgono le seguenti proprietà:
	
	\begin{enumerate}
		\item L'applicazione
		%		
		\map{\hatapp}
			{T_{e}(G)}{L(G)}
			{A}{\hat{A}}
			
		dove $ L(G) $ indica l'insieme dei campi di vettori invarianti a sinistra di $ G $ e
		
		\begin{equation}
			\hat{A}_{g} \doteq (L_{g})_{*e}(A)
		\end{equation}
	
		con $ (L_{g})_{*e} : T_{e}(G) \to T_{g}(G) $, è un isomorfismo di spazi vettoriali;
		
		\item L'insieme dei campi di vettori invarianti a sinistra di $ G $ è contenuto nell'insieme dei campi di vettori lisci su $ G $, i.e. $ L(G) \subset \chi(G) $, quindi un campo di vettori invariante a sinistra è sempre liscio;
		
		\item Il commutatore di due campi di vettori invarianti a sinistra è ancora un campo di vettori invariante a sinistra, i.e.
		
		\begin{equation}
			\comm{X}{Y} \in L(G) \qcomma \forall X,Y \in L(G)
		\end{equation}
	
		quindi i campi di vettori invarianti a sinistra sono chiusi rispetto al commutatore.
	\end{enumerate}
\end{theorem}

\begin{proof}
	\begin{enumerate}
		\item Verifichiamo innanzitutto che $ \hat{A} \in L(G) $:
		
		\begin{gather}
			\hat{A} \in L(G) \nonumber \\
			\Updownarrow \nonumber \\
			(L_{g})_{*}(\hat{A}) = \hat{A} \qcomma \forall g \in G \\
			\Updownarrow \nonumber \\
			(L_{g})_{*h}(\hat{A}_{h}) = \hat{A}_{gh} \qcomma \forall g,h \in G \nonumber
		\end{gather}
	
		Possiamo dimostrare l'ultima equazione tramite la definizione di $ \hat{A}_{g} $
		
		\begin{equation}
			\hat{A}_{g} \doteq (L_{g})_{*e}(A)
		\end{equation}
	
		otteniamo dunque che
		
		\begin{align}
			\begin{split}
				(L_{g})_{*h}(\hat{A}_{h}) &= (L_{g})_{*h}((L_{h})_{*e}(A)) \\
				&= (L_{g} \circ L_{h})_{*e} (A) \\
				&= (L_{gh})_{*e} (A) \\
				&= \hat{A}_{gh}
			\end{split}
		\end{align}
		
		quindi $ \hat{A} \in L(G) $. \\
		Per dimostrare che $ \hatapp $ sia un isomorfismo tra spazi vettoriali dobbiamo far vedere che sia un omomorfismo invertibile. Per dimostrare che sia un omomorfismo dobbiamo mostrare che $ \hatapp $ sia un'applicazione lineare, i.e.
		
		\begin{equation}
			\widehat{\lambda A + \mu B} = \lambda \hat{A} + \mu \hat{B} %
			 \qcomma \forall \lambda,\mu \in \R, \, \forall A,B \in T_{e}(G)
		\end{equation}
	
		per fare ciò, applichiamo il primo membro a un qualunque elemento $ g \in G $:
	
		\begin{align}
			\begin{split}
				(\widehat{\lambda A + \mu B})_{g} &\doteq (L_{g})_{*e} (\lambda A + \mu B) \\
				&= \lambda (L_{g})_{*e} (A) + \mu (L_{g})_{*e} (B) \\
				&\doteq \lambda \hat{A}_{g} + \mu \hat{B}_{g} \\
				&= (\lambda \hat{A} + \mu \hat{B})_{g}
			\end{split}
		\end{align}
	
		L'applicazione inversa (lineare) è la "valutazione in $ e $"
		
		\map{\mid_{e}}
			{L(G)}{T_{e}(G)}
			{X}{X_{e}}
			
		in quanto
		
		\begin{equation}
			(\mid_{e} \circ \hatapp)(A) = \hat{A}_{e} = (L_{e})_{*e} (A) = A
		\end{equation}
		
		poiché, per le proprietà funtoriali del differenziale, vale l'implicazione
		
		\begin{equation}
			L_{e} = \id_{G} %
			\implies %
			(L_{e})_{*e} = \id_{T_{e}(G)}
		\end{equation}
	
		e viceversa verifichiamo che
		
		\begin{equation}
			(\hatapp \circ \mid_{e})(X) = \widehat{X_{e}} = X
		\end{equation}
	
		applicando il primo membro a un qualunque elemento $ g \in G $:
		
		\begin{equation}
			\eval{\widehat{X_{e}}}_{g} \doteq (L_{g})_{*e}(X_{e}) = X_{g} \qcomma \forall g \in G %
			\implies %
			\hat{X_{e}} = X
		\end{equation}
		
		\item Un campo di vettori è liscio se la derivata di una funzione liscia rispetto al campo di vettori è ancora una funzione liscia, i.e.
		
		\begin{equation}
			X \in \chi(G) \iff X f \in C^{\infty}(G) \qcomma \forall f \in C^{\infty}(G)
		\end{equation}
	
		cioè la derivata direzionale $ (X f)(g) = X_{g} f $ varia in modo liscio al variare di $ g \in G $: presa una curva liscia
		
		\begin{equation}
			\begin{cases}
				\gamma : (-\varepsilon,\varepsilon) \to G \\
				\gamma(0) = g \\
				\gamma'(0) = X_{g}
			\end{cases}
		\end{equation}
	
		abbiamo che
		
		\begin{equation}
			X_{g} f = \eval{ \dv{t} f(\gamma(t)) }_{t=0}
		\end{equation}
	
		Consideriamo dunque la curva $ \gamma = g \cdot c $ dove $ \cdot $ è il prodotto in $ G $ e $ c $ è una curva liscia
		
		\begin{equation}
			\begin{cases}
				c : (-\varepsilon,\varepsilon) \to G \\
				c(0) = e \\
				c'(0) = X_{e}
			\end{cases}
		\end{equation}
	
		che rispetta le condizioni poste su $ \gamma $:
		
		\begin{equation}
			\gamma(0) = g \, c(0)= g \, e = g
		\end{equation}
		%		
		\begin{align}
			\begin{split}
				\gamma'(0) &= \eval{ \dv{t} g \, c(t) }_{t=0} \\
				&= \eval{ \dv{t} L_{g}(c(t)) }_{t=0} \\
				&= (L_{g} \circ c)'(0) \\
				&= (L_{g})_{*c(0)} (c'(0)) \\
				&= (L_{g})_{*e} (X_{e}) \\
				&= X_{g}
			\end{split}
		\end{align}
	
		dove nel quarto passaggio abbiamo utilizzato
		
		\begin{equation}
			F_{*p}(X_{p}) = (F \circ c)'(0)
		\end{equation}
		
		e nell'ultimo abbiamo utilizzato il fatto che $ X $ è invariante a sinistra. \\
		A questo punto, per mostrare che $ X \in \chi(G) $ basta mostrare che sia liscia in $ G $ la derivata direzionale
		
		\begin{equation}
			X_{g} f = \eval{ \dv{t} f(\gamma(t)) }_{t=0} %
			= \eval{ \dv{t} f(g \cdot c(t)) }_{t=0}
		\end{equation}
	
		questo è vero perché $ f \in C^{\infty}(G) $ per ipotesi, $ c(t) $ è una curva liscia, il prodotto $ \cdot $ è liscio in quanto $ G $ è un gruppo di Lie e $ f(g \cdot c(t)) $ rimane liscia se derivata rispetto a $ t $.
		
		\item Siano due campi di vettori invarianti a sinistra $ X,Y \in L(G) \subset \chi(G) $, perché il loro commutatore appartenga a $ L(G) $ è necessario che
		
		\begin{equation}
			(L_{g})_{*}([X,Y]) = [X,Y] \qcomma \forall g \in G \iff [X,Y] \in L(G)
		\end{equation}
	
		Sappiamo che $ X $ e $ Y $ sono lisci e inoltre sono $ L_{g} $-related rispettivamente a $ (L_{g})_{*}(X) $ e $ (L_{g})_{*}(Y) $ dunque, per il Corollario \ref{cor:f-rel-brack}, abbiamo che
		
		\begin{equation}
			L_{g_{*}}([X,Y]) = [(L_{g})_{*}(X), (L_{g})_{*}(Y)]
		\end{equation}
	
		Essendo $ X $ e $ Y $ invarianti a sinistra, i.e.
		
		\begin{equation}
			\begin{cases}
				(L_{g})_{*}(X) = X \\
				(L_{g})_{*}(Y) = Y
			\end{cases}
		\end{equation}
		
		abbiamo dunque che
		
		\begin{equation}
			(L_{g})_{*}([X,Y]) = [(L_{g})_{*}(X), (L_{g})_{*}(Y)] = [X,Y]
		\end{equation}		
	\end{enumerate}
\end{proof}

\begin{corollary}
	La coppia $ (L(G),[,]) $ è una sottoalgebra\footnote{%
		Presi due spazi vettoriali $ V $ e $ W \subset V $ con $ (V,[,]) $ algebra di Lie, $ \left( W, \eval{[,]}_{W} \right) $ è una sottoalgebra di Lie se $ W $ è un sottospazio vettoriale e se vale
		
		\begin{equation*}
			[w_{1}, w_{2}] \in W \qcomma \forall w_{1},w_{2} \in W
		\end{equation*}%
	} di Lie di $ (\chi(G),[,]) $ e dunque essa stessa un'algebra di Lie finito-dimensionale, in quanto ha la stessa dimensione dello spazio tangente $ T_{e}(G) $ poiché $ L(G) \stackrel{iso.}{\simeq} T_{e}(G) $.
\end{corollary}

\subsection{Algebra di Lie su $ T_{e}(G) $}

Siano $ G $ un gruppo di Lie, $ e \in G $ il suo elemento neutro e $ T_{e}(G) $ lo spazio tangente a $ G $ nell'elemento $ e $. \\
Dati $ A,B \in T_{e}(G) $ definiamo il commutatore tra gli elementi dello spazio tangente mediante il commutatore tra i campi di vettori\footnote{%
	I due commutatori sono diversi in quanto hanno come entrate oggetti diversi (nel primo elementi dello spazio tangente a $ G $ mentre nel secondo campi di vettori invarianti a sinistra) e appartengono a spazi diversi, i.e. $ \comm{A}{B} \in T_{e}(G) $ mentre $ \comm{\hat{A}}{\hat{B}} \in L(G) $.%
}

\begin{equation}
	[A,B] \doteq \comm{\hat{A}}{\hat{B}}_{e}
\end{equation}

cioè tramite l'isomorfismo

\map{\hatapp}
	{T_{e}(G)}{L(G)}
	{A}{\hat{A}}
	
dove

\begin{equation}
	\hat{A}_{g} = (L_{g})_{*e}(A) \qcomma \forall g \in G
\end{equation}

\sbs{0.55}{%
			In sintesi, vengono mappati gli elementi dello spazio tangente $ A $ e $ B $ nei rispettivi campi di vettori invarianti a sinistra $ \hat{A} $ e $ \hat{B} $, di questi si calcola il commutatore (come campi di vettori), il quale viene infine valutato nell'elemento $ e $, i.e. utilizzando l'inversa dell'isomorfismo
			
			\map{\mid_{e}}
				{L(G)}{T_{e}(G)}
				{X}{X_{e}}
			
			ottenendo dunque il commutatore $ [A,B] $ come un elemento dello spazio tangente $ T_{e}(G) $.
			}
	{0.45}{%
			\diagr{%
					L(G) \times L(G) \arrow[rr, "{[,]}"]                         \&  \& L(G) \arrow[dd, "\mid_{e}"]                         \\
					\&  \&                                                     \\
					T_{e}(G) \times T_{e}(G) \arrow[uu, "\hatapp \times \hatapp"]                   \&  \& T_{e}(G)                                            \\
					                                                 \&  \&                                                     \\
					{(\hat{A},\hat{B})} \arrow[rr, "{[,]}", maps to] \&  \& {[\hat{A},\hat{B}]} \arrow[dd, "\mid_{e}", maps to] \\
					\&  \&                                                     \\
					{(A,B)} \arrow[uu, "\hatapp \times \hatapp", maps to]           \&  \& {[\hat{A},\hat{B}]_{e}}                                     
					}
			}

In questo modo, l'applicazione $ \hatapp $ è un isomorfismo di algebre di Lie tra $ (T_{e}(G),[,]) $ e $ (L(G),[,]) $ con i rispettivi commutatori; le proprietà di algebra di Lie sono rispettate da $ (T_{e}(G),[,]) $ in quanto isomorfo a $ (L(G),[,]) $, il quale le rispetta. \\
L'algebra $ (T_{e}(G),[,]) $ è chiamata \textit{algebra di Lie del gruppo} $ G $ e viene solitamente indicata con

\begin{equation}
	\g \doteq (T_{e}(G),[,])
\end{equation}

\begin{remark}
	Siano $ A,B \in T_{e}(G) $, allora il campo di vettori associato al loro commutatore tramite l'isomorfismo tra $ T_{e}(G) $ e $ L(G) $ è identico al commutatore tra i campi di vettori corrispondenti ad $ A $ e $ B $, i.e.
	
	\begin{equation}
		\widehat{[A,B]} = \widehat{\comm{\hat{A}}{\hat{B}}}_{e} = \comm{\hat{A}}{\hat{B}}
	\end{equation}

	dove nel primo passaggio abbiamo usato la definizione di commutatore nello spazio tangente e nel secondo abbiamo usato il fatto che l'isomorfismo $ \hatapp $ e la valutazione nell'elemento neutro $ \mid_{e} $ siano inversi tra loro.
\end{remark}

\subsubsection{Esempi di algebre di Lie su spazi tangenti a gruppi di Lie}

\paragraph{1) $ (\R^{n},+) $}

La coppia $ (\R^{n},+) $ è un gruppo di Lie (abeliano) con elemento neutro $ 0 \in \R^{n} $. Lo spazio tangente all'elemento neutro $ T_{0}(\R^{n}) $ può essere identificato con $ \R^{n} $ stesso:

\begin{equation}
	T_{0}(\R^{n}) = \left\{ \sum_{j=1}^{n} a^{j}(0) \eval{\pdv{x^{j}}}_{0} \right\} %
	= (a^{1}(0),\dots,a^{n}(0)) \in \R^{n}
\end{equation}

dove le $ a^{j} : \R^{n} \to \R $ sono applicazioni e lo spazio tangente è generato dalle derivate parziali

\begin{equation}
	T_{0}(\R^{n}) = \ev{ \eval{ \pdv{x^{j}} }_{0} }
\end{equation}

Consideriamo ora l'isomorfismo $ \hatapp $, in modo da analizzare i campi di vettori invarianti a sinistra

\map{\hatapp}
	{T_{0}(\R^{n})}{L(\R^{n})}
	{\eval{ \pdv{x^{j}} }_{0}}{\widehat{ \eval{ \pdv{x^{j}} }_{0} }}

Applicando l'isomorfismo alla base di $ T_{0}(\R^{n}) $ otteniamo i generatori dello spazio $ L(\R^{n}) $:

\begin{equation}
	L(\R^{n}) = \ev{ \widehat{ \eval{ \pdv{x^{j}} }_{0} } }
\end{equation}

Per esplicitare la forma dei campi di vettori invarianti a sinistra, valutiamo un rappresentante di $ L(\R^{n}) $ in un generico elemento $ g \in \R^{n} $

\begin{align}
	\begin{split}
		\left( \widehat{ \eval{ \pdv{x^{i}} }_{0} } \right)_{g} \in T_{g}(\R^{n}) %
		\implies %
		\left( \widehat{ \eval{ \pdv{x^{i}} }_{0} } \right)_{g} = \sum_{i=1}^{n} b^{ij} \eval{ \pdv{x^{j}} }_{g} \qcomma b^{ij} \in \R
	\end{split}
\end{align}

Al fine di trovare i coefficienti $ b^{ij} $, applichiamo entrambi i membri dell'equazione alla proiezione

\map{x^{k}}
	{\R^{n}}{\R}
	{(x^{1},\dots,x^{n})}{x^{k}}

Per il secondo membro

\begin{equation}
	\left( \sum_{i=1}^{n} b^{ij} \eval{ \pdv{x^{j}} }_{g} \right) (x^{k}) = \sum_{i=1}^{n} b^{ij} \eval{ \pdv{x^{k}}{x^{j}} }_{g} %
	= \sum_{i=1}^{n} b^{ij} \delta^{jk} %
	= b^{ik}
\end{equation}

mentre per il primo membro, considerando $ g = (g^{1},\dots,g^{n}) $ e la traslazione a sinistra

\map{L_{g}}
	{\R^{n}}{\R^{n}}
	{x}{g+x}

abbiamo che

\begin{align}
	\begin{split}
		\left( \widehat{ \eval{ \pdv{x^{i}} }_{0} } \right)_{g} (x^{k}) &= (L_{g})_{*0} \left( \eval{ \pdv{x^{i}} }_{0} \right) (x^{k}) \\
		&= \eval{ \pdv{x^{i}} }_{0} (x^{k} \circ L_{g}) \\
		&= \eval{ \pdv{x^{i}} }_{0} (g^{k} + x^{k}) \\
		&= \cancel{ \eval{ \pdv{g^{k}}{x^{i}} }_{0} } + \eval{ \pdv{x^{k}}{x^{i}} }_{0} \\
		&= \delta^{ik}
	\end{split}
\end{align}

perciò $ b^{ik} = \delta^{ik} $ e quindi

\begin{equation}
	\left( \widehat{ \eval{ \pdv{x^{i}} }_{0} } \right)_{g} = \sum_{i=1}^{n} \delta^{ij} \eval{ \pdv{x^{j}} }_{g} %
	= \eval{ \pdv{x^{i}} }_{g} \qcomma \forall g \in \R
\end{equation}

il che implica

\begin{equation}
	\widehat{ \eval{ \pdv{x^{i}} }_{0} } \equiv \pdv{x^{i}}
\end{equation}

Un campo di vettori invariante a sinistra può dunque essere scritto come

\begin{equation}
	X \in L(\R^{n}) \iff X = \sum_{i=1}^{n} a^{i} \pdv{x^{i}} \qcomma a^{i} \in \R
\end{equation}

mentre un campo di vettori qualunque può avere delle funzioni al posto dei coefficienti reali $ a^{i} $. \\
Presi due vettori $ A,B \in T_{0}(\R^{n}) $, il commutatore dell'algebra di Lie $ (T_{0}(\R^{n}),[,]) $ è definito come

\begin{align}
	\begin{split}
		\comm{A}{B} &\doteq \comm{\hat{A}}{\hat{B}}_{0} \\
		&= \comm{ \sum_{i=1}^{n} a^{i} \pdv{x^{i}} }{ \sum_{j=1}^{n} b^{j} \pdv{x^{j}} }_{0} \\
		&= \sum_{i,j=1}^{n} a^{i} b^{j} \cancel{ \comm{ \pdv{x^{i}} }{ \pdv{x^{j}} }_{0} } \\
		&= 0
	\end{split}
\end{align}

Essendo il commutatore nullo, l'algebra di Lie $ (T_{0}(\R^{n}),[,]) $ è abeliana.

\paragraph{2) $ (\S^{1},\cdot) $}

La coppia $ (\S^{1},\cdot) $ è un gruppo di Lie (abeliano) con elemento neutro $ 1 \in \S^{1} $, dove $ \S^{1} $ è una sottovarietà di $ \R^{2} = \C $. Lo spazio tangente all'elemento neutro $ T_{1}(\S^{1}) $ è generato dall'unità immaginaria\footnote{%
	Vedi Sottosezione \ref{s-sec:su-n-lie-subg}.%
}, i.e.

\begin{equation}
	T_{1}(\S^{1}) = \ev{i}_{\R} = \{ ia \mid a \in \R \}
\end{equation}

Prendendo la traslazione a sinistra

\map{L_{g}}
	{\S^{1}}{\S^{1}}
	{h}{g h}
	
e la curva liscia

\sbs{0.5}{%
			\map{c}
				{(-\varepsilon,\varepsilon)}{\S^{1}}
				{t}{e^{i t}}
			}
	{0.5}{%
			\begin{equation}
				\begin{cases}
					c(0) = 1 \\
					c'(0) = i
				\end{cases}
			\end{equation}
			}

l'isomorfismo verso i campi di vettori invarianti a sinistra è definito come

\begin{equation}
	\hat{i}_{g} \doteq (L_{g})_{*1} (i) %
	= \eval{ \dv{t} L_{g}(c(t)) }_{t=0} %
	= \eval{ \dv{t} (g \, e^{i t}) }_{t=0} %
	= i g
\end{equation}

dove $ L(\S^{1}) = \ev{\hat{i}}_{\R} $ e $ \hat{i}_{g} $ è il vettore perpendicolare alla normale al punto $ g \in \S^{1} $ (la rotazione di $ \pi/2 $ in senso antiorario è data da $ i $) \\
Anche in questo caso, il commutatore definito nello spazio tangente è nullo in quanto

\begin{equation}
	[ia,ib] \doteq \comm{\widehat{ia}}{\widehat{ib}}_{1} %
	= a b \cancel{ [i,i] } %
	= 0
\end{equation}

dunque anche l'algebra di Lie $ (T_{1}(\S^{1}),[,]) $ è abeliana.

\paragraph{3) $ (\T^{n},\cdot) $}

Il toro è definito come

\begin{equation}
	\T^{n} = \prod^{n} \S^{1} = \S^{1} \times \cdots \times \S^{1}
\end{equation}

La coppia $ (\T^{n},\cdot) $ è un gruppo di Lie (abeliano) con elemento neutro $ e = (1,\dots,1) $. Lo spazio tangente all'elemento neutro è dato da

\begin{equation}
	T_{e}(\T^{n}) = \prod^{n} T_{1}(\S^{1}) = T_{1}(\S^{1}) \times \cdots \times T_{1}(\S^{1})
\end{equation}

e il commutatore dell'algebra di Lie $ (\T^{n},\cdot) $ è nullo, dunque l'algebra è abeliana.

\begin{definition}
	Se un gruppo di Lie $ G $ di dimensione $ \dim(G) = n $ è abeliano, allora
	
	\begin{equation}
		G \stackrel{iso.}{\simeq} \R^{k} \times \T^{n-k} \qcomma k \in [0,n]
	\end{equation}

	dove $ \times $ indica il prodotto diretto di gruppi di Lie.
\end{definition}

\paragraph{4) $ (GL_{n}(\R),\cdot) $}

L'insieme delle matrici invertibili

\begin{equation}
	GL_{n}(\R) = \{ A \in M_{n}(\R) \mid \det(A) \neq 0 \}
\end{equation}

con il prodotto matriciale $ \cdot $ forma un gruppo di Lie $ (GL_{n}(\R),\cdot) $ con elemento neutro la matrice identità $ I $. \\
Lo spazio tangente all'elemento neutro è l'insieme delle matrici quadrate

\begin{equation}
	T_{I}(GL_{n}(\R)) = M_{n}(\R)
\end{equation}

Un elemento dello spazio tangente può essere scritto come sommatoria degli elementi di base delle matrici quadrate e poi identificato con la matrice dei coefficienti, i.e.

\begin{equation}
	A = \sum_{i,j=1}^{n} a_{ij} \eval{ \pdv{x_{ij}} }_{I} \equiv [a_{ij}] %
	\qcomma \forall A \in T_{I}(GL_{n}(\R))
\end{equation}

Consideriamo l'isomorfismo $ \hatapp $ tra lo spazio tangente $ T_{I}(GL_{n}(\R)) $ e l'insieme dei campi di vettori invarianti a sinistra $ L(GL_{n}(\R)) $: presa una matrice $ g = [g_{ij}] \in GL_{n}(\R) $ e la traslazione a sinistra

\map{L_{g}}
	{GL_{n}(\R)}{GL_{n}(\R)}
	{h}{g h}

un elemento di $ L(GL_{n}(\R)) $ è definito come\footnote{%
	Vedi Esempio \ref{example:trasl-diff}.%
}

\begin{equation}
	\hat{A}_{g} \doteq (L_{g})_{*I} (A) %
	= \sum_{i,j=1}^{n} (g A)_{ij} \eval{ \pdv{x_{ij}} }_{g}
\end{equation}

Il commutatore nello spazio tangente è definito tramite l'isomorfismo e avrà la forma

\begin{equation}
	[A,B] \doteq \comm{\hat{A}}{\hat{B}}_{I} = \sum_{i,j=1}^{n} c_{ij} \eval{ \pdv{x_{ij}} }_{I}
\end{equation}

può dunque essere identificato dalla matrice

\begin{equation}
	[A,B] \doteq C = [c_{ij}] \in M_{n}(\R)
\end{equation}

Definendo l'applicazione

\map{x_{ij}}
	{M_{n}(\R)}{\R}
	{X = [x_{ij}]}{x_{ij}}

e applicando entrambi i membri del commutatore a questa applicazione, abbiamo che

\begin{align}
	\begin{split}
		\left( \sum_{p,q=1}^{n} c_{pq} \eval{ \pdv{x_{pq}} }_{I} \right)(x_{ij}) &= \comm{\hat{A}}{\hat{B}}_{I} (x_{ij}) \\
		c_{ij} &= \hat{A}_{I} (\hat{B} \, x_{ij}) - \hat{B}_{I} (\hat{A} \, x_{ij}) \\
		&= A (\hat{B} \, x_{ij}) - B (\hat{A} \, x_{ij})
	\end{split}
\end{align}

per qualsiasi $ i,j = 1,\dots,n $, dove $ \hat{X}_{I} = X $ per definizione dell'isomorfismo e della sua inversa. \\
Per calcolare il secondo membro, lo applichiamo a una matrice $ g \in GL_{n}(\R) $ e utilizziamo il differenziale della traslazione a sinistra definito sopra, ottenendo (per la parte tra parentesi del secondo termine)

\begin{align}
	\begin{split}
		(\hat{A} \, x_{ij})(g) &= \hat{A}_{g} (x_{ij}) \\
		&= \left( \sum_{p,q=1}^{n} (g A)_{pq} \eval{ \pdv{x_{pq}} }_{g} \right)(x_{ij}) \\
		&= \left( \sum_{p,q=1}^{n} (g A)_{pq} \, \delta_{ip} \, \delta_{jq} \right) \\
		&= (g A)_{ij} \\
		&= \sum_{k=1}^{n} g_{ik} \, a_{kj} \\
		&= \sum_{k=1}^{n} a_{kj} \, x_{ik}(g)
	\end{split}
\end{align}

da cui, per analogia

\begin{equation}
	\begin{cases}
		\hat{A} \, x_{ij} = \displaystyle\sum_{k=1}^{n} a_{kj} \, x_{ik} \\ \\
		\hat{B} \, x_{ij} = \displaystyle\sum_{k=1}^{n} b_{kj} \, x_{ik}
	\end{cases}
\end{equation}

Sostituendo ora in $ c_{ij} $

\begin{align}
	\begin{split}
		c_{ij} &= A (\hat{B} \, x_{ij}) - B (\hat{A} \, x_{ij}) \\
		&= \left( \sum_{p,q=1}^{n} a_{pq} \eval{ \pdv{x_{pq}} }_{I} \right) \left( \sum_{k=1}^{n} b_{kj} \, x_{ik} \right) - \left( \sum_{p,q=1}^{n} b_{pq} \eval{ \pdv{x_{pq}} }_{I} \right) \left( \sum_{k=1}^{n} a_{kj} \, x_{ik} \right) \\
		&= \sum_{p,q,k=1}^{n} ( a_{pq} b_{kj} - b_{pq} a_{kj} ) \delta_{ip} \, \delta_{kq} \\
		&= \sum_{k=1}^{n} ( a_{ik} b_{kj} - b_{ik} a_{kj} )
	\end{split}
\end{align}

Tramite questo risultato e l'identificazione $ C = [c_{ij}] = [A,B] $, otteniamo dunque

\begin{equation}
	[A,B] = A B - B A
\end{equation}

cioè l'usuale commutatore tra matrici.

\subsection{Pushforward di campi di vettori invarianti a sinistra tramite omomorfismi di gruppi di Lie}

Siano $ G $ e $ H $ due gruppi di Lie; un omomorfismo di gruppi di Lie $ F : G \to H $ è un'applicazione liscia (ma non necessariamente invertibile), che preserva le operazioni dei gruppi e per la quale vale

\begin{equation}
	L_{F(g)} \circ F = F \circ L_{g} \qcomma \forall g \in G
\end{equation}

Per trasportare i campi di vettori invarianti a sinistra su $ G $ in quelli su $ H $, possiamo utilizzare gli isomorfismi tra questi spazi e gli spazi tangenti e comporli con il differenziale dell'omomorfismo.

\sbs{0.55}{%
			Consideriamo le seguenti applicazioni\footnotemark
			
			\begin{equation}
				\begin{cases}
					\mid_{e} \, : L(G) \to T_{e}(G) \\
					F_{*e} : T_{e}(G) \to T_{e}(H) \\
					\hatapp : T_{e}(H) \to L(H)
				\end{cases}
			\end{equation}
		
			
			}
	{0.45}{%
			\diagr{%
					T_{e}(G) \arrow[rr, "F_{*e}"]                   \&  \& T_{e}(H) \arrow[dd, "\hatapp"] \\
					\&  \&                                \\
					L(G) \arrow[uu, "\mid_{e}"] \arrow[rr, "F_{*}"] \&  \& L(H)                          
					}
			}

\footnotetext{%
	Al fine di non appesantire la notazione, il simbolo $ \hatapp $ verrà usato per entrambi gli isomorfismi
	
	\begin{equation*}
		\begin{cases}
			\hatapp : T_{e}(G) \to L_{G} \\
			\hatapp : T_{e}(H) \to L_{H}
		\end{cases}
	\end{equation*}
	
	e analogamente per il suo inverso; inoltre $ e $ indicherà l'elemento neutro per entrambi i gruppi di Lie, anche perché questi sono legati da un omomorfismo, il quale mappa $ e \in G $ in $ e \in H $.%
}

Definiamo il \textit{pushforward tramite l'omomorfismo} $ F $ come la composizione:

\begin{equation}
	F_{*} \doteq \hatapp \circ F_{*e} \circ \mid_{e}
\end{equation}

Applicandolo a un campo di vettori invariante a sinistra $ X \in L(G) $

\begin{equation}
	F_{*}(X) = \widehat{ F_{*e}(X_{e}) } \in L(G)
\end{equation}

Esplicitando questa definizione tramite la traslazione a sinistra

\begin{equation}
	(F_{*}(X))_{h} = (\widehat{ F_{*e}(X_{e}) })_{h} %
	\doteq (L_{h})_{*e} (F_{*e}(X_{e})) \qcomma \forall h \in H
\end{equation}

Analogamente, prendendo un elemento dello spazio tangente $ A \in T_{e}(G) $ e considerando il campo di vettori invariante a sinistra $ \hat{A} \in L(G) $ dato dall'isomorfismo, si può riscrivere il pushforward tramite l'omomorfismo $ F $ come

\begin{equation}
	F_{*} (\hat{A}) = \widehat{ F_{*e} (\hat{A}_{e}) } = \widehat{ F_{*e} (A) }
\end{equation}

\begin{definition}
	Sia $ F : G \to H $ un isomorfismo di gruppi di Lie, allora il pushforward di $ X \in L(G) $ tramite $ F $ visto come diffeomorfismo coincide al pushforward di $ X $ tramite $ F $ visto come omomorfismo.
\end{definition}

\begin{proof}
	Per notazione, in questa dimostrazione useremo
	
	\begin{itemize}
		\item $ \tilde{F}_{*}(X) $ per il pushforward di $ X $ tramite $ F $ visto come diffeomorfismo
		
		\item $ F_{*}(X) $ per il pushforward di $ X $ tramite $ F $ visto come omomorfismo
	\end{itemize}

	Per definizione, applicando il pushforward tramite il diffeomorfismo $ F : G \to H $ a un elemento $ F(g) = h \in H $, otteniamo
	
	\begin{align}
		\begin{split}
			(\tilde{F}_{*}(X))_{h} &\doteq F_{*F^{-1}(h)}(X_{F^{-1}(h)}) \\
			&= F_{*g}(X_{g}) \\
			&= F_{*g}((L_{g})_{*e}(X_{e})) \\
			&= (F \circ L_{g})_{*e} (X_{e}) \\
			&= (L_{F(g)} \circ F)_{*e} (X_{e}) \\
			&= (L_{h} \circ F)_{*e} (X_{e}) \\
			&= (L_{h})_{*F(e)} (F_{*e}(X_{e})) \\
			&= (L_{h})_{*e} (F_{*e}(X_{e})) \\
			&\doteq (\widehat{ F_{*e}(X_{e}) })_{h} \\
			&= (F_{*}(X))_{h}
		\end{split}
	\end{align}

	dove abbiamo usato le seguenti proprietà:
	
	\begin{itemize}
		\item nel primo passaggio, la definizione del pushforward;
		
		\item nel secondo, $ g = F^{-1}(h) $ in quanto $ F $ è un diffeomorfismo;
		
		\item nel terzo passaggio, il campo di vettori è invariante a sinistra, i.e. $ X \in L(G) $;
		
		\item nel quarto e nel settimo passaggio, la regola della catena;
		
		\item nel quinto passaggio, il fatto che $ F $ è un omomorfismo, i.e.
		%
		\begin{equation}
			L_{F(g)} \circ F = F \circ L_{g} \qcomma \forall g \in G
		\end{equation}		
		
		\item nel nono passaggio, la definizione dell'immagine dell'isomorfismo $ \hatapp $.
	\end{itemize}

	A questo punto
	
	\begin{equation}
		\tilde{F}_{*}(X) = F_{*}(X) \qcomma \forall X \in L(G)
	\end{equation}

	Essendo equivalenti, d'ora in poi verrà utilizzata la notazione $ F_{*}(X) $ per indicare il pushforward sia tramite diffeomorfismo che tramite omomorfismo.
\end{proof}

Anche nel contesto di omomorfismi, è possibile definire campi di vettori $ F $-related.

\begin{definition}
	Siano un omomorfismo di gruppi di Lie $ F : G \to H $ e un campo di vettori su $ G $ invariante a sinistra $ X \in L(G) $, allora $ X $ è $ F $-related \footnote{%
		Siano un'applicazione liscia $ F : N \to M $ e due campi di vettori lisci $ X \in \chi(N) $ e $ Y \in \chi(M) $, allora $ X $ è $ F $-related a $ Y $ se
		
		\begin{equation*}
			F_{*p} (X_{p}) = Y_{F(p)} \qcomma \forall p \in N
		\end{equation*}
	
		Nel caso in cui $ F $ sia un diffeomorfismo, un campo di vettori e il suo pushforward sono sempre $ F $-related tra loro.%
	} a $ F_{*}(X) $, i.e.

	\begin{equation}
		F_{*g} (X_{g}) = (F_{*}(X))_{F(g)} \qcomma \forall g \in G
	\end{equation}
\end{definition}

\begin{proof}
	Per dimostrare che $ X $ è $ F $-related a $ F_{*}(X) $ tramite $ F $ omomorfismo, usiamo la definizione di campo di vettori invariante a sinistra
	
	\begin{align}
		\begin{split}
			F_{*g} (X_{g}) &= F_{*g} ((L_{g})_{*e}(X_{e})) \\
			&= (F \circ L_{g})_{*e} (X_{e}) \\
			&= (L_{F(g)} \circ F)_{*e} (X_{e}) \\
			&= (L_{F(g)})_{*e} (F_{*e}(X_{e})) \\
			&= (\widehat{ F_{*e}(X_{e}) })_{F(g)} \\
			&= (F_{*}(X))_{F(g)}
		\end{split}
	\end{align}
\end{proof}

\begin{definition}
	Sia $ F : G \to H $ un omomorfismo di gruppi di Lie, allora il differenziale
	
	\begin{equation}
		F_{*e} : T_{e}(G) \to T_{e}(H)
	\end{equation}

	scritto anche come
	
	\begin{equation}
		F_{*e} : \g \to \h
	\end{equation}

	è un omomorfismo di algebre di Lie (con il commutatore usuale), i.e.
	
	\begin{equation}
		F_{*e} \comm{A}{B} = \comm{F_{*e} (A)}{F_{*e} (B)} \qcomma \forall A,B \in \g
	\end{equation}
\end{definition}

\begin{proof}
	Siano i campi di vettori su $ G $ invarianti a sinistra $ \hat{A},\hat{B} \in L(G) $. Per la seconda proposizione
	
	\begin{itemize}
		\item $ \hat{A} $ è $ F $ related a $ F_{*}(\hat{A}) $
		
		\item $ \hat{B} $ è $ F $ related a $ F_{*}(\hat{B}) $
	\end{itemize}

	Per il Corollario \ref{cor:f-rel-brack} e per l'identificazione tra il pushforward tramite il diffeomorfismo e quello tramite omomorfismo, se $ F $ è un'applicazione liscia allora $ [\hat{A},\hat{B}] $ è $ F $-related a $ [F_{*} (\hat{A}),F_{*} (\hat{B})] $, i.e.
	
	\begin{equation}
		F_{*g} \left( \comm{\hat{A}}{\hat{B}}_{g} \right) = \comm{ F_{*}(\hat{A}) }{ F_{*}(\hat{B}) }_{F(g)} %
		\qcomma \forall g \in G
	\end{equation}

	Se $ g = e $, abbiamo che $ F(e) = e $, dunque
	
	\begin{align}
		\begin{split}
			F_{*e} \left( \comm{\hat{A}}{\hat{B}}_{e} \right) &= \comm{ F_{*}(\hat{A}) }{ F_{*}(\hat{B}) }_{e} \\
			F_{*e} (\comm{A}{B}) &= \comm{ \widehat{ F_{*e}(A) } }{ \widehat{ F_{*e}(B) } }_{e} \\
			&= \comm{F_{*e} (A)}{F_{*e} (B)}
		\end{split}
	\end{align}
\end{proof}

\begin{corollary}\label{cor:incl-alg-lie-subg}
	Siano $ H $ un sottogruppo di Lie di un gruppo di Lie $ G $ e l'inclusione $ i : H \to G $, allora, chiamando le algebre di Lie degli spazi tangenti come
	
	\begin{equation}
		\begin{cases}
			\h = (T_{e}(H),[,]) \\
			\g = (T_{e}(G),[,])
		\end{cases}
	\end{equation}
	
	abbiamo che il differenziale dell'inclusione $ i_{*e} : \h \to \g $ soddisfa
	
	\begin{equation}
		i_{*e} (\comm{A}{B}) = \comm{i_{*e} (A)}{i_{*e} (B)} \qcomma \forall A,B \in T_{e}(H)
	\end{equation}

	Quindi il commutatore di $ \g $ induce il commutatore (e dunque l'algebra) di $ \h $.
\end{corollary}

In particolare, il commutatore dei sottogruppi (matriciali) di $ GL_{n}(\R) $ o $ GL_{n}(\C) $ è dato dal \textit{commutatore standard}

\begin{equation}
	[A,B] = A B - B A \qcomma A,B \in T_{I}(GL_{n}(\K)), \, \K = \R, \C
\end{equation}

in quanto questo è il commutatore definito nell'algebra dello spazio tangente all'insieme delle matrici invertibili, i.e. $ (T_{I}(GL_{n}(\R)),[,]) $ con l'identificazione $ T_{I}(GL_{n}(\R)) = M_{n}(\R) $. L'algebra di Lie dei sottogruppi matriciali è dunque data dalla coppia formata da spazio tangente all'identità del sottoinsieme e il commutatore standard.

\subsubsection{\textit{Esempi}}

\paragraph{1)}

Il gruppo di Lie delle matrici ortogonali $ O(n) $ è un sottogruppo di Lie di $ GL_{n}(\R) $ con spazio tangente quello delle matrici antisimmetriche

\begin{equation}
	T_{I}(O(n)) = \{ X \in M_{n}(\R) \mid X^{T} = -X \}
\end{equation}

L'algebra di Lie $ (T_{I}(O(n)),[,]) $ ha come commutatore, per il corollario sopraccitato, quello standard

\begin{equation}
	[X,Y] = X Y - Y X \qcomma X,Y \in T_{I}(O(n))
\end{equation}

Essendo un elemento dello spazio tangente, per il commutatore vale

\begin{equation}
	[X,Y]^{T} = - [X,Y]
\end{equation}

\paragraph{2)}

Il gruppo di Lie delle matrici unitarie speciali $ SU(n) $ è un sottogruppo di Lie di $ GL_{n}(\C) $ con spazio tangente quello delle matrici antihermitiane a traccia nulla

\begin{equation}
	T_{I}(SU(n)) = \{ X \in M_{n}(\C) \mid X^{*} = -X \wedge \tr(X) = 0 \}
\end{equation}

L'algebra di Lie dello spazio tangente a $ SU(n) $

\begin{equation}
	\mathfrak{su}(n) \doteq (T_{I}(SU(n)),[,])
\end{equation}

ha come commutatore, per il corollario sopraccitato, quello standard

\begin{equation}
	[X,Y] = X Y - Y X \qcomma X,Y \in T_{I}(SU(n))
\end{equation}

Essendo un elemento dello spazio tangente, per il commutatore valgono

\begin{equation}
	\begin{cases}
		[X,Y]^{*} = - [X,Y] \\
		\tr([X,Y]) = 0
	\end{cases}
\end{equation}

\paragraph{3)}

Per la verifica che il commutatore standard definisca un'algebra di Lie nei gruppi matriciali studiati in precedenza, vedi Esercizio \ref{exer3-9}.

\subsection{Applicazione esponenziale di un gruppo di Lie}

Siano $ G $ un gruppo di Lie, $ e \in G $ il suo elemento neutro, $ \g = (T_{e}(G),[,]) $ la sua algebra di Lie e $ \xi,\eta \in \g $ due elementi di quest'algebra. Vogliamo costruire un'applicazione

\begin{equation}
	f : \g \to G
\end{equation}

che leghi l'algebra di Lie di $ G $, costruita mediante i campi di vettori invarianti a sinistra sul gruppo, al gruppo stesso.

\subsubsection{Flusso di un campo di vettori invariante a sinistra}

Sia $ \hat{\xi} \in L(G) $ il campo di vettori invariante a sinistra associato al campo di vettori dell'algebra di Lie $ \xi \in \g $ con

\begin{equation}
	\hat{\xi}_{g} = (L_{g})_{*e}(\xi)
\end{equation}

Possiamo considerare il flusso locale di $ \hat{\xi} $ nel punto $ g \in U $ con $ U \subset G $ aperto, i.e. $ \sigma_{t}^{\xi}(g) $ con $ t \in (-\delta,\delta) $ e $ \delta \in \R $. La condizione per cui questo oggetto sia un flusso è che, considerando $ \sigma_{t}^{\xi}(g) $ come una curva al variare di $ t $ con $ g $ fissato, la curva inizi in $ g $ e il vettore tangente a questa curva in ogni suo punto coincida con il campo di vettori $ \hat{\xi} $ calcolato nello stesso punto, i.e.

\begin{equation}
	\begin{cases}
		\sigma_{0}^{\xi}(g) = g \\ \\
		\ddv{t} \sigma_{t}^{\xi}(g) = \hat{\xi}_{\sigma_{t}^{\xi}(g)}
	\end{cases}
\end{equation}

Equivalentemente, $ \sigma_{t}^{\xi}(g) $ è la curva integrale del campo di vettori $ \hat{\xi} $. \\
Dai teoremi sulle curve integrali e sui flussi di campi di vettori\footnote{%
	Vedi Teorema \ref{thm:flux-var}.%
}, $ \sigma_{t}^{\xi}(g) $ dipende in modo liscio da $ t $ e da $ g $; inoltre vale la relazione

\begin{equation}
	\sigma_{t+s}^{\xi} = \sigma_{t}^{\xi} \circ \sigma_{s}^{\xi}
\end{equation}

dove sono definite entrambe le applicazioni. \\
Prendendo l'elemento neutro dell'algebra di Lie $ 0 \in \g $, abbiamo che

\begin{equation}
	\dv{t} \sigma_{t}^{0}(g) = \hat{0}_{\sigma_{t}^{\xi}(g)} = 0
\end{equation}

dunque $ \sigma_{t}^{0}(g) $ è costante in $ t $, in particolare in $ t=0 $ possiamo usare la condizione

\begin{equation}
	\sigma_{0}^{\xi}(g) = g
\end{equation}

ottenendo

\begin{equation}
	\sigma_{t}^{0}(g) = \sigma_{0}^{0}(g) = g
\end{equation}

dunque il flusso locale dell'elemento neutro dell'algebra di Lie è costante. \\
Riassumendo per $ \sigma_{t}^{\xi}(g) $

\begin{equation}
	\begin{cases}
		\sigma_{0}^{\xi}(g) = g \\ \\
		\ddv{t} \sigma_{t}^{\xi}(g) = \hat{\xi}_{\sigma_{t}^{\xi}(g)} \\ \\
		\sigma_{t+s}^{\xi} = \sigma_{t}^{\xi} \circ \sigma_{s}^{\xi} \\ \\
		\sigma_{t}^{0}(g) = g, & \forall t \in (-\delta,\delta)
	\end{cases}
\end{equation}

Definiamo ora la curva liscia

\map{\gamma^{\xi}}
	{(-\delta,\delta)}{G}
	{t}{\sigma_{t}^{\xi}(e)}
	
dove $ e \in G $ è l'elemento neutro del gruppo. Calcolandone il punto in cui inizia e il vettore tangente, otteniamo:

\begin{gather}
	\gamma^{\xi}(0) = \sigma_{0}^{\xi}(e) = e \\
	\nonumber \\
	\dv{t} \gamma^{\xi}(t) = \dv{t} \sigma_{t}^{\xi}(e) %
	= \hat{\xi}_{\sigma_{t}^{\xi}(e)} %
	= \hat{\xi}_{\gamma^{\xi}(t)}
\end{gather}

dunque $ \gamma^{\xi}(t) $ è la curva integrale di $ \hat{\xi} $ che inizia in $ e $. \\
In particolare

\begin{equation}
	\eval{ \dv{t} \gamma^{\xi}(t) }_{t=0} = \eval{ \hat{\xi}_{\gamma^{\xi}(t)} }_{t=0} %
	= \hat{\xi}_{\gamma^{\xi}(0)} %
	= \hat{\xi}_{e} %
	= \xi
\end{equation}

\begin{definition}
	Il flusso locale $ \sigma_{t}^{\xi}(g) $ può essere scritto come
	
	\begin{equation}
		\sigma_{t}^{\xi}(g) = g \cdot \gamma^{\xi}(t)
	\end{equation}

	dove $ \cdot $ indica il prodotto in $ G $.
\end{definition}

Ricordiamo che una curva integrale che inizia in un punto è l'unica curva integrale (dove definita) che inizi nello stesso punto.

\begin{proof}
	Per definizione
	
	\begin{equation}
		\begin{cases}
			\sigma_{0}^{\xi}(g) = g \\ \\
			\ddv{t} \sigma_{t}^{\xi}(g) = \hat{\xi}_{\sigma_{t}^{\xi}(g)}
		\end{cases}
	\end{equation}

	Perché valga $ \sigma_{t}^{\xi}(g) = g \cdot \gamma^{\xi}(t) $, dobbiamo dimostrare che
	
	\begin{equation}
		\begin{cases}
			g \cdot \gamma^{\xi}(0) = g \\ \\
			\ddv{t} (g \cdot \gamma^{\xi}(t)) = \hat{\xi}_{g \cdot \gamma^{\xi}(t)}
		\end{cases}
	\end{equation}

	Per la prima equazione
	
	\begin{equation}
		g \cdot \gamma^{\xi}(0) = g \cdot e = g
	\end{equation}

	Per la seconda
	
	\begin{align}
		\begin{split}
			\dv{t} (g \cdot \gamma^{\xi}(t)) &= \dv{t} L_{g}(\gamma^{\xi}(t)) \\
			&= (L_{g})_{*\gamma^{\xi}(t)} \left( \dv{t} \gamma^{\xi}(t) \right) \\
			&= (L_{g})_{*\gamma^{\xi}(t)} \left( \hat{\xi}_{\gamma^{\xi}(t)} \right) \\
			&= \hat{\xi}_{L_{g}(\gamma^{\xi}(t))} \\
			&= \hat{\xi}_{g \cdot \gamma^{\xi}(t)}
		\end{split}
	\end{align}

	dove nel secondo passaggio abbiamo usato la definizione di differenziale tramite curve e nel quarto il fatto che $ \hat{\xi} $ sia un campo di vettori invariante a sinistra. \\
	Per unicità delle curve integrali abbiamo dunque che
	
	\begin{equation}
		\sigma_{t}^{\xi}(g) = g \cdot \gamma^{\xi}(t)
	\end{equation}

	in quanto le due curve hanno stesso punto d'inizio e stesso vettore tangente in ogni punto.
\end{proof}

Data questa forma del flusso integrale, possiamo scrivere

\begin{align}
	\begin{split}
		\gamma^{\xi}(t+s) &= \sigma_{t+s}^{\xi}(e) \\
		&= (\sigma_{t}^{\xi} \circ \sigma_{s}^{\xi})(e) \\
		&= \sigma_{t}^{\xi}(\sigma_{s}^{\xi}(e)) \\
		&= \sigma_{t}^{\xi}(\gamma^{\xi}(s)) \\
		&= \gamma^{\xi}(s) \cdot \gamma^{\xi}(t)
	\end{split}
\end{align}

Riassumendo per $ \gamma^{\xi}(t) $:

\begin{equation}
	\begin{cases}
		\gamma^{\xi}(0) = e \\ \\
		\ddv{t} \gamma^{\xi}(t) = \hat{\xi}_{\gamma^{\xi}(t)} \\ \\
		\eval{ \ddv{t} \gamma^{\xi}(t) }_{t=0} = \xi \\ \\
		\gamma^{\xi}(t+s) = \gamma^{\xi}(s) \cdot \gamma^{\xi}(t)
	\end{cases}
\end{equation}

\begin{definition}
	Il campo di vettori invariante a sinistra $ \hat{\xi} $ è completo, i.e. $ t \in \R $ e $ g \in G $.
\end{definition}

\begin{proof}
	Perché $ \hat{\xi} $ sia completo, è necessario che il flusso $ \sigma_{t}^{\xi}(g) $ sia globale: per dimostrare questo, è sufficiente dimostrare che $ g \cdot \gamma^{\xi}(t) $ sia globale in quanto, per la proposizione precedente, abbiamo che $ \sigma_{t}^{\xi}(g) = g \cdot \gamma^{\xi}(t) $. \\
	Supponiamo, per assurdo, che $ (-\delta,\delta) $ con $ \delta < \infty $ sia il dominio massimale di definizione di $ \gamma^{\xi} $. Definiamo ora la curva
	
	\map{\gamma}
		{(-\delta,\delta + t_{0})}{G}
		{t}{%
			\begin{cases}
				\gamma^{\xi}(t), & t \in (-\delta,\delta) \\
				\gamma^{\xi}(t_{0}) \cdot \gamma^{\xi}(t - t_{0}), & t \in (t_{0},\delta + t_{0})
			\end{cases}%
			}
		
	dove $ t_{0} \in (0,\delta) $; questa curva è liscia perché liscia nelle due condizioni e queste coincidono nell'intersezione aperta $ (t_{0},\delta) $ in quanto
	
	\begin{equation}
		\gamma^{\xi}(t_{0}) \cdot \gamma^{\xi}(t - t_{0}) %
		= \gamma^{\xi}(t_{0} + t - t_{0}) %
		= \gamma^{\xi}(t) %
		\qcomma \forall t \in (t_{0},\delta)
	\end{equation}

	Abbiamo che
	
	\begin{equation}
		\gamma(0) = \gamma^{\xi}(0) = e
	\end{equation}
	
	inoltre, per $ t \in (-\delta,\delta) $
	
	\begin{equation}
		\dv{t} \gamma(t) = \dv{t} \gamma^{\xi}(t) %
		= \hat{\xi}_{\gamma^{\xi}(t)} %
		= \hat{\xi}_{\gamma(t)}
	\end{equation}

	e per $ t \in (\delta,\delta + t_{0}) $
	
	\begin{align}
		\begin{split}
			\dv{t} \gamma(t) &= \dv{t} ( \gamma^{\xi}(t_{0}) \cdot \gamma^{\xi}(t - t_{0}) ) \\
			&= \dv{t} L_{\gamma^{\xi}(t_{0})} ( \gamma^{\xi}(t - t_{0}) ) \\
			&= (L_{\gamma^{\xi}(t_{0})})_{*\gamma^{\xi}(t - t_{0})} \left( \dv{t} \gamma^{\xi}(t - t_{0}) \right) \\
			&= (L_{\gamma^{\xi}(t_{0})})_{*\gamma^{\xi}(t - t_{0})} \left( \hat{\xi}_{\gamma^{\xi}(t - t_{0})} \right) \\
			&= \hat{\xi}_{\gamma^{\xi}(t_{0}) \cdot \gamma^{\xi}(t - t_{0})}
		\end{split}
	\end{align}
	
	Da questi risultati otteniamo
		
	\begin{equation}
		\begin{cases}
			\gamma(0) = e \\ \\
			\ddv{t} \gamma(t) = \hat{\xi}_{\gamma(t)}
		\end{cases}
	\end{equation}	
	
	cioè $ \gamma $ è una curva integrale per $ \hat{\xi} $ che inizia in $ e $, dunque la condizione $ \delta < \infty $ è contraddittoria e quindi il parametro $ t $ può assumere qualunque valore reale, i.e. $ t \in \R $.
\end{proof}

A questo punto, la curva integrale $ \gamma^{\xi}(t) \in G $ può essere scritta come

\map{\gamma^{\xi}}
	{\R}{G}
	{t}{\sigma_{t}^{\xi}(e)}

\subsubsection{Definizione e proprietà dell'esponenziale di un gruppo di Lie}

Siano $ G $ un gruppo di Lie, $ e \in G $ il suo elemento neutro, $ \xi \in T_{e}(G) $ un elemento dell'algebra di Lie $ \g $ di $ G $ e $ \gamma^{\xi}(t) $ con $ t \in \R $ la curva integrale del campo di vettori invariante a sinistra $ \hat{\xi} $ che inizia in $ e $, definiamo l'applicazione \textit{esponenziale} come

\map{\exp}
	{T_{e}(G)}{G}
	{\xi}{\gamma^{\xi}(1) = \sigma_{1}^{\xi}(e)}

Questa applicazione è il riassunto di alcuni passaggi intermedi:

\begin{enumerate}
	\item Prendiamo un campo di vettori $ \xi \in \g $ dell'algebra di Lie di $ G $;
	
	\item associamo a questo il campo di vettori invariante a sinistra $ \hat{\xi} \in L(G) $;
	
	\item consideriamo il flusso globale di questo $ \sigma_{t}^{\xi}(g) $;
	
	\item consideriamo la curva integrale $ \gamma^{\xi}(t) = \sigma_{t}^{\xi}(e) $, ottenuta valutando il flusso nell'elemento neutro $ e \in G $;
	
	\item valutiamo questa curva in $ t=1 $.
\end{enumerate}

A questo punto, possiamo dire che $ \exp(\xi) \in G $ è il valore in $ t=1 $ della curva integrale del campo di vettori invariante a sinistra $ \hat{\xi} $ che inizia nell'elemento neutro $ e \in G $. \\
Fin'ora sappiamo solo che $ \sigma_{t}^{\xi}(g) $ dipende in modo liscio da $ t $ e da $ g $ ma non abbiamo informazioni sulla dipendenza da $ \xi $.

\begin{definition}[Proprietà dell'esponenziale di un gruppo di Lie]\hfill\break
	\begin{enumerate}
		\item L'esponenziale $ \exp : T_{e}(G) \to G $ è liscio rispetto a $ \xi $;
		
		\item $ \exp(0) = e $;
		
		\item $ \exp(t \xi) = \gamma^{\xi}(t) $.
	\end{enumerate}
\end{definition}

\begin{proof}\hfill\break
	\begin{enumerate}
		\item Sia $ \tilde{G} = G \times T_{e}(G) $ il prodotto diretto dei gruppi di Lie $ (G,\cdot) $ e $ (T_{e}(G),+) $, da cui il gruppo di Lie $ (\tilde{G},\cdot) $ con quest'ultima operazione definita come
		
		\map{\cdot}
			{\tilde{G} \times \tilde{G}}{\tilde{G}}
			{((g_{1},\xi_{1}),(g_{2},\xi_{2}))}{(g_{1} \cdot g_{2},\xi_{1} + \xi_{2})}

		Tramite il gruppo di Lie $ \tilde{G} $, consideriamo $ \xi $ non come parametro ma come coordinata di un punto $ (g,\xi) \in \tilde{G} $. \\
		Lo spazio tangente di $ \tilde{G} $ nell'elemento neutro $ (e,0) $, ricordando che $ T_{e}(G) $ è uno spazio vettoriale quindi coincide con il suo spazio tangente, è pari a
		
		\begin{equation}
			T_{(e,0)}(\tilde{G}) = T_{e}(G) \times T_{0}(T_{e}(G)) %
			= T_{e}(G) \times T_{e}(G)
		\end{equation}
		
		dove abbiamo usato l'isomorfismo tra lo spazio tangente del prodotto diretto e il prodotto diretto degli spazi tangenti\footnote{%
			Vedi Esercizio \ref{exer2-11}.%
		}. \\
		Prendiamo un elemento $ \tilde{\xi} $ di questo spazio tangente definito come
		
		\begin{equation}
			\tilde{\xi}_{(g,\xi)} = (\hat{\xi}_{g},0) \in T_{e}(G) \times T_{e}(G)
		\end{equation}
	
		dove $ \hat{\xi} \in L(G) $ è il campo di vettori invarianti a sinistra associato a $ \xi $ e $ 0 \in T_{e}(G) $ è il campo di vettori nullo; $ \tilde{\xi} $ è un campo di vettori invariante a sinistra, i.e. $ \tilde{\xi} \in L(\tilde{G}) $, in quanto
		
		\begin{equation}
			L_{(g,\xi)_{*}} \left( \tilde{\xi} \right) = \left( L_{(g,\xi)_{*}} \left( \hat{\xi}_{g} \right), L_{(g,\xi)_{*}} (0) \right) %
			= \left( \hat{\xi}_{g},0 \right) %
			= \tilde{\xi}_{(g,\xi)}
		\end{equation}
	
		Per il flusso globale di $ \tilde{\xi} $ scritto come $ \tilde{\sigma}(t,g,\xi) $ valgono le seguenti relazioni
		
		\begin{equation}
			\begin{cases}
				\tilde{\sigma} : \R \times \tilde{G} \to \tilde{G} \\ \\
				\ddv{t} \tilde{\sigma}(t,g,\xi) = \tilde{\xi}_{\tilde{\sigma}(t,g,\xi)} \\ \\
				\tilde{\sigma}(0,g,\xi) = (g,\xi)
			\end{cases}
		\end{equation}
	
		Essendo $ (g,\xi) \in \tilde{G} $, il flusso globale $ \tilde{\sigma}(t,g,\xi) $ dipende in modo liscio da $ \xi $ per il Teorema \ref{thm:flux-var}. \\
		Considerando la curva $ (\sigma_{t}^{\xi}(g),\xi) \in \tilde{G} $, questa comincia in
		
		\begin{equation}
			(\sigma_{0}^{\xi}(g),\xi) = (g,\xi)
		\end{equation}
	
		e il suo vettore tangente è
		
		\begin{equation}
			\dv{t} (\sigma_{t}^{\xi}(g),\xi) = \left( \hat{\xi}_{\sigma_{t}^{\xi}(g)}, 0 \right) = \tilde{\xi}_{(\sigma_{t}^{\xi}(g),\xi)}
		\end{equation}
	
		Dunque è una curva integrale per $ \tilde{\xi} $ che inizia in $ (g,\xi) $: per il teorema sull'unicità della curva integrale, otteniamo che
		
		\begin{equation}
			\tilde{\sigma}(t,g,\xi) = (\sigma_{t}^{\xi}(g),\xi) %
			\qcomma \forall t \in \R, \, \forall g \in G, \, \forall \xi \in \g
		\end{equation}
	
		Questo prova che anche $ \sigma_{t}^{\xi}(g) $ è liscia in $ \xi $, da cui $ \gamma^{\xi}(t) = \sigma_{t}^{\xi}(e) $ è liscia in $ \xi $ e infine abbiamo che $ \exp(\xi) = \gamma^{\xi}(1) $ è liscio in $ \xi $.
		
		\item Per definizione
		
		\begin{equation}
			\exp(0) \doteq \gamma^{0}(1) %
			= \sigma_{1}^{0}(e) %
			= e
		\end{equation}
	
		in quanto $ \sigma_{t}^{0}(g) = g $ per qualsiasi $ t \in \R $.
		
		\item Per definizione
		
		\begin{equation}
			\exp(t \xi) = \gamma^{t \xi}(1)
		\end{equation}
	
		e vorremmo dimostrare che
		
		\begin{equation}
			\exp(t \xi) = \gamma^{\xi}(t)
		\end{equation}
	
		Più in generale, dimostriamo che
		
		\begin{equation}
			\gamma^{t \xi}(s) = \gamma^{\xi}(t s) \qcomma \forall t,s \in \R
		\end{equation}
	
		Questa proprietà è verificata per $ t = 0 $
		
		\begin{equation}
			\gamma^{0}(s) = \sigma_{s}^{0}(e) %
			= e %
			= \gamma^{\xi}(0)
		\end{equation}
	
		Supponiamo che $ t \neq 0 $ e sfruttiamo l'unicità delle curve integrali: definendo $ u = t s $, vogliamo che
		
		\begin{equation}
			\gamma^{t \xi} \left( \dfrac{u}{t} \right) = \gamma^{\xi}(u)
		\end{equation}
		
		Calcoliamo dunque il vettore tangente del secondo membro, ottenendo
		
		\begin{equation}
			\begin{cases}
				\gamma^{\xi}(0) = e \\ \\
				\ddv{u} \gamma^{\xi}(u) = \hat{\xi}_{\gamma^{\xi}(u)}
			\end{cases}
		\end{equation}
	
		mentre per il primo membro, sapendo che $ \dd{u} = t \dd{s} $ e che l'isomorfismo $ \hatapp $ è lineare, otteniamo
		
		\begin{align}
			\begin{split}
				\dv{u} \gamma^{t \xi} \left( \dfrac{u}{t} \right) &= \dfrac{1}{t} \dv{s} \gamma^{t \xi}(s) \\
				&= \dfrac{1}{t} \left( \widehat{t \xi} \right)_{\gamma^{t \xi}(s)} \\
				&= \dfrac{1}{t} \, t \, \left( \hat{\xi}_{\gamma^{t \xi}(s)} \right) \\
				&= \hat{\xi}_{\gamma^{t \xi}(s)} \\
				&= \hat{\xi}_{\gamma^{t \xi}(u/t)}
			\end{split}
		\end{align}
	
		dunque
		
		\begin{equation}
			\begin{cases}
				\gamma^{t \xi}(0) = e \\ \\
				\ddv{u} \gamma^{t \xi} \left( \dfrac{u}{t} \right) = \hat{\xi}_{\gamma^{t \xi}(u/t)}
			\end{cases}
		\end{equation}
	
		perciò, per unicità delle curve integrali
		
		\begin{equation}
			\gamma^{\xi}(u) = \gamma^{t \xi} \left( \dfrac{u}{t} \right)
		\end{equation}
	
		Dato questo, abbiamo dimostrato la proprietà
		
		\begin{equation}
			\gamma^{t \xi}(s) = \gamma^{\xi}(t s)
		\end{equation}
	
		e quindi anche
		
		\begin{equation}
			\exp(t \xi) = \gamma^{t \xi}(1) = \gamma^{\xi}(t)
		\end{equation}
	\end{enumerate}
\end{proof}

\begin{corollary}
	Valgono le seguenti proprietà:
	
	\begin{enumerate}
		\item Il flusso globale di $ \hat{\xi} $ si può scrivere come
		
		\begin{equation}
			\sigma_{t}^{\xi}(g) = g \cdot \exp(t \xi)
		\end{equation}
		
		\item $ \exp((t+s) \xi) = \exp(t \xi) \cdot \exp(s \xi) %
		\qcomma \forall t,s \in \R, \, \forall \xi \in \g $
		
		\item L'inversa dell'esponenziale è pari a
		\begin{equation}
			\exp(\xi)^{-1} = \exp(-\xi) \qcomma \forall \xi \in \g
		\end{equation}
	\end{enumerate}
\end{corollary}

\begin{proof}\hfill\break
	\begin{enumerate}
		\item $ \sigma_{t}^{\xi}(g) = g \cdot \gamma^{\xi}(t) = g \cdot \exp(t \xi) $
		
		\item $ \exp((t+s) \xi) = \gamma^{\xi}(t+s) %
		= \gamma^{\xi}(t) \cdot \gamma^{\xi}(s) %
		= \exp(t \xi) \cdot \exp(s \xi) $
		
		\item Per $ t = 1 $ e $ s = -1 $
		%
		\begin{align}
			\begin{split}
				\exp(t \xi) \cdot \exp(s \xi) &= \exp((t+s) \xi) \\
				\exp(\xi) \cdot \exp(-\xi) &= \exp(0) \\
				&= e
			\end{split}
		\end{align}
	\end{enumerate}
\end{proof}

\begin{remark}
	Si dimostra che, presi due elementi qualunque dell'algebra di Lie $ \xi,\eta \in \g $ con $ [\xi,\eta] = 0 $, vale
	
	\begin{equation}
		\exp(\xi + \eta) = \exp(\xi) \cdot \exp(\eta)
	\end{equation}

	In generale, l'esponenziale della somma non è uguale al prodotto degli esponenziali\footnote{%
		Vedi \textit{Formula di Baker–Campbell–Hausdorff}.%
	}.
\end{remark}

\subsection{Sottogruppi a un parametro di un gruppo di Lie}

Sia $ G $ un gruppo di Lie, un'applicazione liscia $ \varphi : \R \to G $ è chiamata \textit{sottogruppo a un parametro di} $ G $ se vale

\begin{equation}
	\varphi(t+s) = \varphi(t) \cdot \varphi(s) \qcomma \forall t,s \in \R
\end{equation}

cioè $ \varphi $ è un omomorfismo di gruppi di Lie tra $ (\R,+) $ e $ G $. L'applicazione $ \varphi $ è chiamata così poiché l'immagine $ \varphi(\R) $ è un sottogruppo algebrico di $ G $ isomorfo al gruppo $ (\R,+) $. \\
Ad esempio, $ \gamma^{\xi}(t) = \exp(t \xi) $ definisce un sottogruppo a un parametro di $ G $.

\begin{definition}
	Sia $ \varphi : \R \to G $ un sottogruppo a un parametro di un gruppo di Lie $ G $ tale che
	
	\begin{equation}
		\varphi'(0) = \eval{ \dv{t} \varphi(t) }_{t=0} %
		= \xi \in T_{e}(G)
	\end{equation}
	
	allora
	
	\begin{equation}
		\varphi(t) = \exp (t \xi) \qcomma \forall t \in \R
	\end{equation}
\end{definition}

\begin{remark}
	Da questa proposizione, deriviamo che esiste una corrispondenza biunivoca tra gli elementi dell'algebra di Lie $ \g $ di $ G $ e l'insieme dei sottogruppi a un parametro; a questo punto, essendo presente l'isomorfismo $ \hatapp $ tra lo spazio tangente a $ G $ e l'insieme dei campi di vettori invarianti a sinistra, i.e. $ T_{e}(G) \stackrel{iso.}{\simeq} L(G) $, si possono descrivere gli elementi dell'algebra di Lie $ \g $ sia come sottogruppi a un parametro di $ G $ sia come campi di vettori invarianti a sinistra.
\end{remark}

\begin{proof}
	Sia la curva
	
	\map{q}
		{\R}{G}
		{t}{\varphi(t) \exp(-t \xi)}
		
	Osserviamo che $ \varphi(0) = e $ in quanto omomorfismo di gruppi di Lie (manda l'elemento neutro di $ \R $ in quello di $ G $) e dunque
	
	\begin{equation}
		q(0) \doteq \varphi(0) \exp(0) = e
	\end{equation}
	
	Calcoliamo il vettore tangente alla curva $ q(t) $
	
	\begin{align}
		\begin{split}
			\dv{t} q(t) &= \dv{t} \varphi(t) \exp(-t \xi) \\
			&= \eval{ \dv{s} \varphi(t+s) \exp(- (t+s) \xi) }_{s=0} \\
			&= \eval{ \dv{s} \varphi(t) \varphi(s) \exp(-s \xi) \exp(-t \xi) }_{s=0}
		\end{split}
	\end{align}

	Definendo l'applicazione
	
	\map{Q}
		{\R \times G \times T_{e}(G)}{G}
		{(t,g,\xi)}{Q^{\xi}_{t}(g)}
	
	dove
	
	\begin{equation}
		Q^{\xi}_{t}(g) \doteq (L_{\varphi(t)} \circ R_{\exp(-t \xi)})(g) = \varphi(t) \, g \, \exp(-t \xi)
	\end{equation}
	
	possiamo scrivere, usando la definizione di differenziale e la moltiplicazione $ \mu $ di $ G $
	
	\begin{align}
		\begin{split}
			\dv{t} q(t) &= \eval{ \dv{s} \varphi(t) \varphi(s) \exp(-s \xi) \exp(-t \xi) }_{s=0} \\
			&= \eval{ \dv{s} Q^{\xi}_{t} (\varphi(s) \exp(-s \xi)) }_{s=0} \\
			&= (Q^{\xi}_{t})_{*e} \left( \eval{ \dv{s} \varphi(s) \exp(-s \xi) }_{s=0} \right) \\
			&= (Q^{\xi}_{t})_{*e} \left( \eval{ \dv{s} \mu(\varphi(s), \exp(-s \xi)) }_{s=0} \right) \\
			&= (Q^{\xi}_{t})_{*e} \left( \mu_{*(e,e)} \left( \eval{ \dv{s} \varphi(s) }_{s=0}, \eval{ \dv{s} \exp(-s \xi) }_{s=0} \right) \right)
		\end{split}
	\end{align}

	Ricordando che
	
	\begin{equation}
		\mu_{*(e,e)} (X_{e},Y_{e}) = X_{e} + Y_{e} \qcomma X_{e},Y_{e} \in T_{e}(G)
	\end{equation}

	mostriamo che l'argomento di $ (Q^{\xi}_{t})_{*e} $ è nullo:
	
	\begin{align}
		\begin{split}
			\mu_{*(e,e)} \left( \eval{ \dv{s} \varphi(s) }_{s=0}, \eval{ \dv{s} \exp(-s \xi) }_{s=0} \right) &= \left( \xi, \eval{ \dv{s} \gamma^{-\xi}(s) }_{s=0} \right) \\
			&= \mu_{*(e,e)} (\xi, -\xi) \\
			&= \xi - \xi \\
			&= 0
		\end{split}
	\end{align}

	A questo punto, otteniamo che
	
	\begin{equation}
		\dv{t} q(t) = (Q^{\xi}_{t})_{*e} (0) = 0 \qcomma \forall t \in \R \implies q(t) = e \qcomma \forall t \in \R
	\end{equation}

	cioè la curva $ q(t) $ è costante (usiamo il suo valore in $ q(0) = e $), quindi
	
	\begin{equation}
		\varphi(t) \exp(-t \xi) = q(t) = e \qcomma \forall t \in \R \implies \varphi(t) = \exp(t \xi)
	\end{equation}

	in quanto $ \exp(t \xi) = \exp(-t \xi)^{-1} $.
\end{proof}

\subsection{Omomorfismi ed esponenziale di gruppi di Lie}

\begin{definition}\label{prop:exp-loc-diffeo}
	L'esponenziale $ \exp : T_{e}(G) \to G $ è un diffeomorfismo locale in $ 0 \in T_{e}(G) $, i.e. il differenziale dell'esponenziale nell'origine
	
	\begin{equation}
		\exp_{*0} : T_{0}(T_{e}(G)) = T_{e}(G) \to T_{e}(G)
	\end{equation}

	è un isomorfismo di spazi vettoriali.
\end{definition}

\begin{remark}
	In generale, l'esponenziale non è un diffeomorfismo locale in un elemento $ \xi \in T_{e}(G) $ con $ \xi \neq 0 $.
\end{remark}

\begin{proof}
	Preso un elemento $ \xi \in T_{e}(G) = T_{0}(T_{e}(G)) $, per calcolare il differenziale dell'esponenziale $ \exp_{*0} $, consideriamo la curva
	
	\map{c}
		{\R}{T_{e}(G)}
		{t}{t \xi}
	
	dove
	
	\begin{equation}
		\begin{cases}
			c(0) = 0 \in T_{e}(G) \\
			c'(0) = \xi
		\end{cases}
	\end{equation}
	
	dunque, per le proprietà del differenziale
	
	\begin{equation}
		\exp_{*0} (\xi) = \dot{(\exp \circ \, c)}(0) %
		= \eval{ \dv{t} \exp(t \xi) }_{0} %		
		= \xi
	\end{equation}

	e quindi
	
	\begin{equation}
		\exp_{*0} = \id_{T_{0}(T_{e}(G))}
	\end{equation}
	
	Il differenziale è dunque l'identità dello spazio $ T_{0}(T_{e}(G)) $, perciò invertibile e, in particolare, isomorfismo di spazi vettoriali: per il teorema della funzione inversa\footnote{%
		Vedi Teorema \ref{thm:ift}, e Teorema \ref{thm:diffeo-loc-iso-ift} per la formulazione tramite differenziale.%
	}, questo rende l'esponenziale un diffeomorfismo locale intorno a $ 0 \in T_{e}(G) $.
\end{proof}

\begin{definition}\label{prop:homo-lie-exp-comm}
	Siano un omomorfismo di gruppi di Lie $ F : G \to H $ e gli esponenziali dei rispettivi gruppi di Lie
	
	\sbs{0.6}{%
				\begin{equation}
					\begin{cases}
						\exp_{G} : T_{e}(G) \to G \\
						\exp_{H} : T_{e}(H) \to H
					\end{cases}
				\end{equation}
				
				allora il diagramma a lato è commutativo.
				}
		{0.4}{%
				\diagr{%
						T_{e}(G) \arrow[rr, "F_{*e}"] \arrow[dd, "\exp_{G}"] \&  \& T_{e}(H) \arrow[dd, "\exp_{H}"] \\
						\&  \&                                 \\
						G \arrow[rr, "F"]                                    \&  \& H                              %
						}
				}
	
	Questo significa che l'esponenziale commuta con gli omomorfismi di gruppi di Lie, i.e.
	
	\begin{equation}
		F \circ \exp_{G} = \exp_{H} \circ F_{*e}
	\end{equation}
\end{definition}

\begin{proof}
	Per dimostrare questo fatto, facciamo vedere che
	
	\begin{equation}
		F(\exp_{G}(t \xi)) = \exp_{H}(F_{*e}(t \xi)) \qcomma \forall t \in \R, \, \forall \xi \in T_{e}(G)
	\end{equation}

	Chiamiamo l'applicazione nel primo membro $ \varphi_{1} $ e quella nel secondo $ \varphi_{2} $ e osserviamo che queste applicazioni
	
	\sbs{0.5}{%
				\map{\varphi_{1}}
					{\R}{H}
					{t}{F(\exp_{G}(t \xi))}
				}
		{0.5}{%
				\map{\varphi_{2}}
					{\R}{H}
					{t}{\exp_{H}(F_{*e}(t \xi))}
				}
	
	sono sottogruppi a un parametro di $ H $, i.e.
	
	\begin{equation}
		\varphi_{j}(t+s) = \varphi_{j}(t) \, \varphi_{j}(s) \qcomma \forall t,s \in \R, \, j = 1,2
	\end{equation}
	
	tali che il loro vettore tangente in $ 0 $ sia
	
	\begin{equation}
		\varphi'_{1}(0) = \varphi'_{2}(0) = F_{*e}(\xi) \in T_{e}(H)
	\end{equation}

	Per verificarlo, utilizziamo il fatto che $ F $ sia un omomorfismo, i.e. preservi la moltiplicazione, dunque
	
	\sbs{0.5}{%
				\begin{align}
					\begin{split}
						\varphi_{1}(t+s) &= F(\exp_{G}((t+s) \xi)) \\
						&= F(\exp_{G}(t \xi) \exp_{G}(s \xi)) \\
						&= F(\exp_{G}(t \xi)) \, F(\exp_{G}(s \xi)) \\
						&= \varphi_{1}(t) \, \varphi_{1}(s)
					\end{split}
				\end{align}
				}
		{0.5}{%
				\begin{align}
					\begin{split}
						\varphi_{2}(t+s) &= \exp_{H}(F_{*e}((t+s) \xi)) \\
						&= \exp_{H}((t+s) F_{*e}(\xi)) \\
						&= \exp_{H}(t F_{*e}(\xi)) \exp_{H}(s F_{*e}(\xi)) \\
						&= \exp_{H}(F_{*e}(t \xi)) \exp_{H}(F_{*e}(s \xi)) \\
						&= \varphi_{2}(t) \, \varphi_{2}(s)
					\end{split}
				\end{align}
				}
	
	con vettori tangenti, calcolati usando i differenziali tramite curve e la linearità del differenziale, pari a
	
	\sbs{0.5}{%
				\begin{align}
					\begin{split}
						\varphi'_{1}(0) &= \eval{ \dv{t} }_{0} F(\exp_{G}(t \xi)) \\
						&= F_{*e} \left( \eval{ \dv{t} }_{0} \exp_{G}(t \xi) \right) \\
						&= F_{*e}(\xi)
					\end{split}
				\end{align}
				}
		{0.5}{%
				\begin{align}
					\begin{split}
						\varphi'_{2}(0) &= \eval{ \dv{t} }_{0} \exp_{H}(F_{*e}(t \xi)) \\
						&= \eval{ \dv{t} }_{0} \exp_{H}(t F_{*e}(\xi)) \\
						&= F_{*e}(\xi)
					\end{split}
				\end{align}
				}
	
	in quanto il vettore tangente in $ 0 \in T_{e}(G) $ dell'esponenziale $ \exp(t \xi) $ è pari a $ \xi $. \\
	A questo punto $ \varphi_{1} $ e $ \varphi_{2} $ sono due sottogruppi a un parametro su $ G $ entrambi costruiti a partire da $ F_{*e}(\xi) $, perciò coincidono, il che prova
	
	\begin{equation}
		F(\exp_{G}(t \xi)) = \exp_{H}(F_{*e}(t \xi)) %
		\qcomma \forall t \in \R, \, \forall \xi \in T_{e}(H)
	\end{equation}

	e in particolare per $ t=1 $, da cui otteniamo
	
	\begin{equation}
		F \circ \exp_{G} = \exp_{H} \circ F_{*e}
	\end{equation}

	cioè l'esponenziale commuta con gli omomorfismi di gruppi di Lie.
\end{proof}

\begin{corollary}\label{cor:om-cont-smooth}
	Sia $ F : G \to H $ un omomorfismo (algebrico) di gruppi di Lie continuo, allora $ F $ è liscio. \\
	In particolare, un sottogruppo a un parametro di $ G $ si può definire come un'applicazione continua $ \varphi : \R \to G $, in quanto questa sarà automaticamente liscia.
\end{corollary}

\begin{proof}
	Consideriamo il seguente diagramma
	
	\img{0.6}{img55}
	
	Mostriamo innanzitutto che l'omomorfismo $ F $ sia liscio in un intorno $ U $ dell'origine $ e \in G $: l'esponenziale è un diffeomorfismo locale intorno all'origine $ 0 \in T_{e}(G) $, dunque la sua restrizione all'intorno $ \tilde{U} \ni 0 $
	
	\begin{equation}
		\eval{\exp}_{\tilde{U}} : \tilde{U} \to \exp(\tilde{U})
	\end{equation}

	è un diffeomorfismo. A questo punto, la restrizione $ \eval{F}_{U} $, dove $ U \doteq \exp \!\; (\tilde{U}) $, è liscia poiché può essere scritta come
	
	\begin{equation}
		\eval{F}_{U} = \exp \circ F_{*e} \circ \left( \eval{\exp}_{\tilde{U}} \right)^{-1}
	\end{equation}

	questa composizione è liscia perché l'esponenziale è liscio e il differenziale è lineare (dunque liscio). \\
	Consideriamo ora un qualunque punto (che non sia l'origine) $ g \in G $ e l'immagine dell'aperto $ U $ tramite la traslazione a sinistra $ L_{g}(U) $, il quale è aperto in quanto $ L_{g} $ è un diffeomorfismo (porta aperti in aperti). Verifichiamo ora che la restrizione $ \eval{F}_{L_{g}(U)} $ sia liscia, rendendo $ F $ liscio dappertutto in quanto $ g $ è arbitrario. \\
	Essendo $ F $ un omomorfismo, vale la proprietà
	
	\begin{equation}
		F \circ L_{g} = L_{F(g)} \circ F \qcomma \forall g \in G
	\end{equation}

	siccome
	%	
	\begin{gather}
		L_{h}^{-1} = L_{h^{-1}} \\
		L_{h^{-1}} (L_{h}(U)) = U
	\end{gather}

	possiamo scrivere
	
	\begin{align}
		\begin{split}
			F \circ L_{g} &= L_{F(g)} \circ F \\
			F &= L_{F(g)} \circ F \circ L_{g^{-1}} \\
			\eval{F}_{L_{g}(U)} &= \eval{ (L_{F(g)} \circ F \circ L_{g^{-1}}) }_{L_{g}(U)} \\
			&= \eval{L_{F(g)}}_{F(U)} \circ \eval{F}_{U} \circ \eval{ (L_{g^{-1}}) }_{L_{g}(U)}
		\end{split}
	\end{align}

	il che rende $ \eval{F}_{L_{g}(U)} $ liscia in quanto composizione di applicazioni lisce.
\end{proof}

\begin{corollary}\label{cor:subgr-lie-exp-tang}
	Sia $ H \subset G $ un sottogruppo di Lie di $ G $, allora l'esponenziale del sottogruppo di Lie $ \exp_{H} : T_{e}(H) \to H $ è la restrizione dell'esponenziale di $ G $ allo spazio tangente in $ H $, i.e.
	
	\begin{equation}
		\exp_{H} = \eval{\exp_{G}}_{T_{e}(H)}
	\end{equation}
\end{corollary}

\begin{proof}\hfill\break

	\sbs{0.6}{%
				Siccome l'inclusione $ i : H \to G $ è un omomorfismo di gruppi di Lie (le operazioni sono quelle indotte), il diagramma commutativo a lato mostra che\footnotemark
				
				\begin{equation}
					i \circ \exp_{H} = \exp_{G} \circ \, i_{*e}
				\end{equation}
			
				e dunque che
	
				\begin{equation}
					\exp_{H} = \eval{\exp_{G}}_{T_{e}(H)}
				\end{equation}
				}
		{0.4}{%
				\diagr{%
					T_{e}(H) \arrow[rr, "i_{*e}"] \arrow[dd, "\exp_{H}"] \&  \& T_{e}(G) \arrow[dd, "\exp_{G}"] \\
					\&  \&                                 \\
					H \arrow[rr, "i"]                                    \&  \& G                              %
					}
				}
	\footnotetext{%
		Vedi Proposizione \ref{prop:homo-lie-exp-comm}.%
	}
\end{proof}

\subsection{Esponenziale di gruppi di Lie matriciali}

Ricordiamo che l'esponenziale per matrici è definito come

\map{e}
	{M_{n}(\R)}{GL_{n}(\R)}
	{A}{e^{A} = \sum_{k=0}^{+\infty} \dfrac{A^{k}}{k!}}

Possiamo identificare $ M_{n}(\R) = T_{I}(GL_{n}(\R)) $, quindi l'esponenziale di gruppi di Lie ha stesso dominio e codominio dell'esponenziale per matrici, i.e.

\map{\exp}
	{T_{I}(GL_{n}(\R))}{GL_{n}(\R)}
	{A}{\exp(A)}

\begin{corollary}
	L'esponenziale di matrici coincide con l'esponenziale di gruppi di Lie matriciali.
\end{corollary}

\begin{proof}
	Sia una matrice $ A \in M_{n}(\R) $, chiamiamo
	
	\begin{equation}
		\begin{cases}
			\varphi_{1}(t) \doteq e^{t A} \\
			\varphi_{2}(t) \doteq \exp(t A)
		\end{cases}
	\end{equation}

	dove, tramite l'identificazione $ M_{n}(\R) = T_{I}(GL_{n}(\R)) $
	
	\begin{equation}
		\varphi_{j} : \R \times T_{I}(GL_{n}(\R)) \to GL_{n}(\R) \qcomma j=1,2
	\end{equation}

	Fissando $ A \in M_{n}(\R) $, le applicazioni $ \varphi_{1}(t) $ e $ \varphi_{2}(t) $ sono sottogruppi di Lie a un parametro di $ GL_{n}(\R) $, i.e.
	
	\begin{equation}
		\varphi_{j} : \R \to GL_{n}(\R) \qcomma j=1,2
	\end{equation}
	
	e il loro vettore tangente nell'origine è la matrice $ A $, i.e.
	
	\begin{equation}
		\varphi'_{1}(0) = \varphi'_{2}(0) = A
	\end{equation}

	in quanto valgono
	
	\begin{equation}
		\begin{cases}
			e^{(t+s) A} = e^{t A} e^{s A} \\
			\exp((t+s) A) = \exp(t A) \exp(s A)
		\end{cases} %
		\qquad \forall t,s \in \R, \, \forall A \in M_{n}(\R)
	\end{equation}

	e per i vettori tangenti
	
	\begin{equation}
		\eval{ \dv{t} e^{t A} }_{t=0} = \eval{ \dv{t} \exp(t A) }_{t=0} = A %
		\qcomma \forall t \in \R, \, \forall A \in M_{n}(\R)
	\end{equation}

	Due sottogruppi a un parametro che hanno lo stesso vettore tangente nell'origine (essendo completamente determinati da questo) coincidono, dunque
	
	\begin{equation}
		e \equiv \exp
	\end{equation}

	per gruppi di Lie matriciali.
\end{proof}

\begin{corollary}
	Sia $ G \subset GL_{n}(\R) $ un sottogruppo di Lie matriciale di $ GL_{n}(\R) $, allora
	
	\begin{equation}
		\exp_{G} = \eval{e}_{T_{I}(G)}
	\end{equation}
\end{corollary}

\begin{proof}
	Dal Corollario \ref{cor:subgr-lie-exp-tang} e dall'identificazione $ e \equiv \exp $, possiamo scrivere
	
	\begin{equation}
		\exp_{G} = \eval{ \exp_{GL_{n}(\R)} }_{T_{I}(G)} %
		= \eval{e}_{T_{I}(G)}
	\end{equation}
\end{proof}

\subsubsection{\textit{Esempio}}

Sia il gruppo delle matrici lineari speciali $ SL_{n}(\R) \subset GL_{n}(\R) $, l'esponenziale in questo spazio è definito come

\map{e}
	{T_{I}(SL_{n}(\R))}{SL_{n}(\R)}
	{X}{e^{X}}

cioè coincide con l'esponenziale standard ma con $ X \in T_{I}(SL_{n}(\R)) $, i.e. $ \tr(X) = 0 $: se la traccia di $ X $ è nulla, l'esponenziale avrà determinante unitario e quindi apparterrà a $ SL_{n}(\R) $, in quanto\footnote{%
	Vedi Proposizione \ref{prop:det-exp-tr}.%
}

\begin{equation}
	\det(e^{X}) = e^{\tr(X)} = 1
\end{equation}

Per altri esempi di gruppi matriciali, vedi Esercizio \ref{exer3-10}.

\subsection{Sulla suriettività dell'esponenziale}

Dai risultati precedenti, dato un gruppo di Lie $ G $, possiamo costruire un'algebra di Lie $ \g = (T_{e}(G),[,]) $ e da questa possiamo tornare indietro tramite l'esponenziale: questo dimostra l'esistenza di un funtore, chiamato \textit{funtore di Lie} $ \mathcal{L} $, dalla categoria dei gruppi di Lie $ \mathfrak{G} $ alla categoria delle algebre di Lie $ \mathfrak{A} $ che porta omomorfismi di gruppi di Lie $ F : G \to H $ in omomorfismi\footnote{%
	Gli isomorfismi di algebre di Lie sono omomorfismi invertibili, per i morfismi della categoria delle algebre di Lie questo non è necessario.%
} di algebre di Lie $ F_{*e} : \g \to \h $

\begin{equation}
	\begin{cases}
		\mathcal{L}(G) = \g, & \forall G \in \ob(\mathfrak{G}), \, \g \in \ob(\mathfrak{A}) \\
		\mathcal{L}(F) = F_{*e}, & \forall F \in \mor(G,H), \, F_{*e} \in \mor(\g,\h)
	\end{cases}
\end{equation}

Ci chiediamo ora quali siano le condizioni sotto cui l'esponenziale di gruppi di Lie sia suriettivo, in modo tale da poter legare un gruppo di Lie alla sua algebra di Lie. \\
In generale, l'esponenziale non è suriettivo, a parte per un intorno dell'origine, per il quale è un diffeomorfismo locale\footnote{%
	Vedi Proposizione \ref{prop:exp-loc-diffeo}; l'essere un diffeomorfismo implica la suriettività.%
}. \\
Ad esempio, considerando l'insieme

\begin{equation}
	SL_{2}(\R) = \{ A \in GL_{2}(\R) \mid \det(A) = 1 \} \subset GL_{2}(\R)
\end{equation}

l'esponenziale

\map{\exp}
	{T_{I}(SL_{2}(\R))}{SL_{2}(\R)}
	{X}{e^{X}}

non è suriettivo, i.e. esistono delle matrici in $ SL_{2}(\R) $ che non hanno controimmagine tramite l'esponenziale in $ T_{I}(SL_{2}(\R)) $. In generale, nemmeno

\map{\exp}
	{T_{I}(GL_{2}(\R))}{GL_{2}(\R)}
	{X}{e^{X}}

è suriettivo. \\
Prendendo una matrice con determinante unitario

\begin{equation}
	A = \bmqty{ \sfrac{-1}{2} & 0 \\ \\ 0 & -2 } \in SL_{2}(\R)
\end{equation}

per questa matrice non esiste $ B \in T_{I}(SL_{2}(\R)) $ tale che $ e^{B} = A $. \\
Supponiamo, per assurdo, che questa matrice esista e dividiamo il problema nei casi in cui la matrice $ B $ sia diagonalizzabile oppure no:

\begin{itemize}
	\item La matrice $ B $ è diagonalizzabile: sia $ \lambda \in \R $ un suo autovalore, allora
	
	\begin{equation}
		B v = \lambda v \qcomma v \in \R^{2} \setminus {(0,0)}
	\end{equation}

	ed $ e^{\lambda} $ è un autovalore per $ A = e^{B} $ perché
	
	\begin{equation}
		A v = e^{B} v %
		= \left( \sum_{j=0}^{+\infty} \dfrac{B^{j}}{j!} \right) (v) %
		= \sum_{j=0}^{+\infty} \dfrac{B^{j} v}{j!} %
		= \sum_{j=0}^{+\infty} \dfrac{\lambda^{j} v}{j!} %
		= \left( \sum_{j=0}^{+\infty} \dfrac{\lambda^{j}}{j!} \right) (v) %
		= e^{\lambda} v
	\end{equation}

	dunque se $ \lambda_{1} $ e $ \lambda_{2} $ sono due autovalori di $ B $ allora $ e^{\lambda_{1}} $ ed $ e^{\lambda_{2}} $ sono due autovalori di $ A = e^{B} $, ma questo è assurdo perché gli autovalori di $ A $ sono negativi mentre $ e^{\lambda} > 0 $ per qualsiasi $ \lambda \in \R $;
	
	\item La matrice $ B $ non è diagonalizzabile: questo significa che gli autovalori di $ B $ non sono reali quindi, per il teorema fondamentale dell'algebra, avremmo che la matrice avrà degli autovalori $ \alpha, \bar{\alpha} \in \C $ complessi coniugati (in quanto il polinomio caratteristico ha coefficienti reali); anche in questo caso $ e^{\alpha} $ ed $ e^{\bar{\alpha}} $ sono due autovalori di $ A = e^{B} $ ma
	
	\begin{equation}
		\abs{e^{\alpha}} = \abs{e^{x + i y}} = \abs{e^{x} e^{i y}} = e^{x} %
		\implies %
		\abs{e^{\alpha}} = \abs{e^{\bar{\alpha}}}
	\end{equation}

	invece
	
	\begin{equation}
		\abs{- \dfrac{1}{2}} \neq \abs{2}
	\end{equation}

	dunque arriviamo a una contraddizione anche in questo caso, i.e. non esiste $ B $ tale che $ e^{B} = A $.
\end{itemize}

\subsubsection{Condizioni di suriettività}

Introduciamo i due seguenti lemmi:

\begin{lemma}
	Sia $ S $ un sottogruppo di Lie di $ G $ con $ S $ aperto e $ G $ connesso, allora $ S = G $.
\end{lemma}

\begin{proof}
	È sufficiente dimostrare che $ S $ sia chiuso: un insieme aperto e chiuso (diverso dal vuoto) contenuto in un insieme connesso, coincide con quest'ultimo. \\
	L'insieme $ S $ unito all'unione disgiunta delle classi laterali $ g S $, definite come
	
	\begin{equation}
		g \, S = L_{g}(S) \qcomma g \notin S
	\end{equation}
	
	quindi aperte, è uguale a $ G $, i.e.
	
	\begin{equation}
		G = S \cup \left( \bigsqcup_{g \notin S} g S \right) %
		\implies %
		S = G \setminus \left( \bigsqcup_{g \notin S} g S \right)
	\end{equation}
	
	dove
	
	\begin{equation}
		g_{1} S \cap g_{2} S = \emptyset %
		\qcomma \forall g_{1},g_{2} \in G \setminus S
	\end{equation}
	
	Questo rende $ S $ il complementare di un aperto e dunque un chiuso, dimostrando che $ S = G $.
\end{proof}

\begin{lemma}\label{lemma:lie-gr-neut-int}
	Siano $ G $ un gruppo di Lie connesso ed $ e \in G $ il suo elemento neutro, allora $ G $ è generato da un qualunque intorno $ U \subset G $ dell'elemento neutro $ e \in G $, i.e.
	
	\begin{equation}
		\forall \, U \subset G \mid U \ni e \implies g = \prod_{j=1}^{k} g_{j} %
		\qcomma \forall g \in G, \, g_{j} \in U, \, j=1,\dots,k %
		\qq{con} k \in \N
	\end{equation}
\end{lemma}

Per le proprietà della componente connessa attorno all'elemento neutro, vedi Esercizio \ref{exer3-6}.

\begin{proof}
	Dato un qualunque intorno $ U \ni e $, consideriamo l'intersezione di tutti gli elementi $ g $ di $ U $ con l'insieme $ U^{-1} $ dei suoi inversi $ g^{-1} $, i.e. $ U \cap U^{-1} $,  questo è ancora un aperto che contiene l'elemento neutro perché l'inversione
	
	\map{i}
		{U}{U^{-1}}
		{g}{g^{-1}}
	
	è un diffeomorfismo; sappiamo inoltre che $ U \cap U^{-1} \subset U $. \\
	Dimostriamo ora che il seguente insieme
	
	\begin{equation}
		S = \left\{ \prod_{j=1}^{k} g_{j} \in G \st g_{j} \in U \cap U^{-1}, \, j=1,\dots,k \qq{con} k \in \N \right\}
	\end{equation}
	
	coincide con $ G $. Possiamo dedurre che $ S \neq \emptyset $ e che sia un sottogruppo algebrico di Lie di $ G $, i.e. $ S \leqslant G $, in quanto
	
	\begin{equation}
		\left( \prod_{i=1}^{k} g_{i} \right) \left( \prod_{j=1}^{l} g_{j}^{-1} \right) \in S
	\end{equation}
	
	Inoltre, $ S $ è aperto in $ G $ perché qualunque punto di $ S $ è interno: sia $ g \in S $ e sia l'insieme di tutti gli elementi di $ U \cap U^{-1} $ moltiplicati a sinistra per $ g $, i.e.
	
	\begin{equation}
		g \, U \cap U^{-1} = L_{g}(U \cap U^{-1})
	\end{equation}
	
	questo insieme è aperto in quanto $ L_{g} $ è un diffeomorfismo (porta aperti in aperti) ed è un sottoinsieme di $ S $ perché ogni elemento di $ g \, U \cap U^{-1} $ è della forma
	
	\begin{equation}
		g \, U \cap U^{-1} = \left\{ \prod_{j=1}^{k+1} g_{j} \right\} \subset S
	\end{equation}
	
	Dunque $ S $ è aperto e, per il lemma precedente, questo prova che $ S = G $ e dunque che $ G $ è generato da un qualunque intorno del suo elemento neutro.
\end{proof}

Quest'ultimo lemma permette di dimostrare l'ultimo punto del teorema seguente:

\begin{theorem}
	Siano un gruppo di Lie connesso\footnote{%
		Lo spazio tangente all'identità $ T_{e}(G) $ è connesso e l'esponenziale è continuo quindi l'immagine dell'esponenziale $ \exp(T_{e}(G)) \subseteq G $ è connessa: se $ G $ non è connesso (e.g. $ O(n) $), l'esponenziale non potrà essere suriettivo.%
	} $ G $ e l'esponenziale $ \exp : T_{e}(G) \to G $. \\
	Valgono le seguenti affermazioni:
	
	\begin{enumerate}
		\item se $ G $ è anche compatto, l'esponenziale è suriettivo;
		
		\item se $ G $ è anche abeliano, l'esponenziale è suriettivo;
		
		\item ogni elemento del gruppo può essere scritto come prodotto finito di esponenziali, i.e. 
		%
		\begin{equation}
			g = \prod_{j=1}^{k} \exp(\xi_{j}) \qcomma \forall g \in G, \, \xi_{j} \in T_{e}(G), \, j=1,\dots,k \qq{con} k \in \N
		\end{equation}
	\end{enumerate}
\end{theorem}

Nel caso del gruppo di Lie matriciale connesso $ GL_{n}(\C) $, l'esponenziale

\begin{equation}
	\exp : T_{I}(GL_{n}(\C)) = M_{n}(\C) \to GL_{n}(\C)
\end{equation}

è suriettivo; la dimostrazione si avvale della \textit{forma canonica di Jordan}.

\begin{proof}\hfill\break
	\begin{enumerate}
		\item Per dimostrare che l'esponenziale sia suriettivo se $ G $ è connesso e compatto si usa la geometria riemanniana.
		
		\item Per dimostrare che l'esponenziale sia suriettivo se $ G $ è connesso e abeliano usiamo
		
		\begin{equation}
			G \stackrel{iso.}{\simeq} \R^{k} \times \T^{n-k} \qcomma k \in [0,n], \, \dim(G) = n
		\end{equation}
		
		Abbiamo che $ T_{0}(\R) = \R $, dunque l'esponenziale per $ \R $
		
		\begin{equation}
			\exp = \id_{\R} : \R \to \R
		\end{equation}
		
		mentre, siccome $ T_{1}(\S^{1}) = i \R $, l'esponenziale\footnote{%
			Questo esponenziale vale anche per $ \T $, in quanto il prodotto diretto di gruppi di Lie connessi e abeliani è ancora un gruppo di Lie connesso e abeliano.%
		} per $ \S^{1} $
		
		\map{\exp}
			{i \R}{\R}
			{t}{e^{i t}}
		
		da cui si ricava immediatamente che sono suriettivi.
		
		\item Per dimostrare che ogni elemento del gruppo possa essere scritto come prodotto finito di esponenziali, usiamo che un gruppo di Lie $ G $ è generato da un qualunque intorno del suo elemento neutro\footnote{%
			Vedi Lemma \ref{lemma:lie-gr-neut-int}.%
		}: scegliamo l'intorno che sia diffeomorfo a $ \g $ tramite l'esponenziale\footnote{%
			Vedi Proposizione \ref{prop:exp-loc-diffeo}.%
		}, quindi l'esponenziale è suriettivo perciò ogni elemento di $ \g $ è mappato in un solo elemento di $ G $ intorno all'elemento neutro.
	\end{enumerate}
\end{proof}

\begin{remark}
	Nel teorema dimostrato sopra, il punto 3 implica il punto 2, i.e. il poter scrivere ogni elemento di $ G $ come prodotto finito di esponenziali implica il fatto che se $ G $ è connesso e abeliano allora l'esponenziale è suriettivo: essendo $ G $ abeliano, vale l'implicazione
	
	\begin{equation}
		[\xi,\eta] = 0 \implies \exp(\xi + \eta) = \exp(\xi) \cdot \exp(\eta) %
		\qcomma \forall \xi,\eta \in \g
	\end{equation}

	Possiamo dunque scrivere un generico elemento del gruppo come
	
	\begin{equation}
		g = \prod_{j=1}^{k} \exp(\xi_{j}) = \exp( \sum_{j=1}^{k} \xi_{j} ) %
		\qcomma \forall g \in G, \, \xi_{j} \in \g, \, j=1,\dots,k \qq{con} k \in \N
	\end{equation}

	dove
	
	\begin{equation}
		\sum_{j=1}^{k} \xi_{j} \in \g %
		\qcomma \xi_{j} \in \g, \, j=1,\dots,k \qq{con} k \in \N
	\end{equation}
	
	quindi l'esponenziale è suriettivo.
\end{remark}

\subsection{Teorema di corrispondenza di Lie}

Sia $ \g $ un'algebra di Lie (finito-dimensionale) su $ \R $, vogliamo trovare un gruppo di Lie $ G $ tale che la sua algebra di Lie $ (T_{e}(G),[,]) $, che chiameremo $ Lie(G) $, coincida con $ \g $. \\
Tramite il seguente teorema, otteniamo le condizioni per cui questo sia possibile:

\begin{theorem}[Teoremi di corrispondenza di Lie]\hfill\break
	\begin{enumerate}
		\item Per ogni algebra di Lie $ \g $ (finito-dimensionale) su $ \R $ esiste un unico gruppo di Lie semplicemente connesso\footnote{%
			Uno spazio topologico è semplicemente connesso se è connesso per archi e se ogni curva chiusa può essere deformata fino a ridursi a un singolo punto; dato un gruppo di Lie qualunque, è sempre possibile costruire un gruppo di Lie semplicemente connesso a partire dal primo.%
		} $ G $ tale che l'algebra di Lie associata a $ G $ coincida con $ \g $, i.e. $ Lie(G) = \g $;
		
		\item Siano $ G $ e $ H $ due gruppi di Lie con $ G $ semplicemente connesso, e $ Lie(G) $ e $ Lie(H) $ le corrispondenti algebre di Lie, se esiste un omomorfismo di algebre di Lie $ \varphi : Lie(G) \to Lie(H) $ allora esiste un unico omomorfismo di gruppi di Lie $ F : G \to H $ tale che il suo differenziale sia uguale all'omomorfismo di algebre di Lie, i.e. $ F_{*e} = \varphi $;
		
		\item Siano $ G $ un gruppo di Lie connesso e $ \h $ una sottoalgebra dell'algebra di Lie $ Lie(G) $, allora esiste un unico sottogruppo di Lie connesso $ H $ di $ G $ tale che la sua algebra di Lie sia $ \h $, i.e. $ Lie(H) = \h $.
	\end{enumerate}
\end{theorem}

\begin{remark}
	In generale, se due gruppi di Lie sono isomorfi, le algebre di Lie relative sono isomorfe. I primi due punti di questo teorema provano l'implicazione opposta. Segue dunque da questi due punti che due gruppi di Lie $ G $ e $ H $ semplicemente connessi hanno le loro algebre di Lie $ Lie(G) $ e $ Lie(H) $ isomorfe se e solo se i due gruppi di Lie sono isomorfi.
\end{remark}

\begin{theorem}
	Considerando l'insieme dei gruppi di Lie semplicemente connessi quozientato per la relazione di equivalenza relativa all'isomorfismo di gruppi di Lie (due gruppi di Lie sono equivalenti se e solo se isomorfi), questo spazio quoziente è in corrispondenza biunivoca con l'insieme delle algebre di Lie finito-dimensionali sui reali quozientato per la relazione di equivalenza relativa all'isomorfismo di algebre di Lie, i.e.
	
	\begin{equation}
		\dfrac{\text{gruppi di Lie semplicemente connessi}}{\sim_{iso.gr.}} %
		\leftrightarrow %
		\dfrac{\text{algebre di Lie finito-dimensionali su } \R}{\sim_{iso.al.}}
	\end{equation}
	
	quindi si possono usare risultati di natura topologico-geometrico differenziabile (gruppi di Lie) per trarre conclusioni di natura puramente algebrica (algebre di Lie) e viceversa, a meno di isomorfismi.
\end{theorem}

\begin{remark}
	In termini di teoria delle categorie, si dimostra che la categoria dei gruppi di Lie semplicemente connessi è equivalente alla categoria delle algebre di Lie finito-dimensionali su $ \R $.
\end{remark}

Consideriamo un esempio in cui i gruppi di Lie non sono isomorfi ma le loro algebre di Lie lo sono.

\paragraph{\textit{Esempio}}

Siano i gruppi di Lie

\begin{equation}
	\begin{cases}
		G = (\R^{n},+) \\
		H = (\T^{n},\cdot)
	\end{cases}
\end{equation}

questi non sono isomorfi perché, in particolare, dovrebbero essere diffeomorfi ma questo non è possibile perché $ \T^{n} $ è compatto mentre $ \R^{n} $ non lo è. \\
Nonostante non siano isomorfi, le loro algebre sono identiche e dunque isomorfe

\begin{equation}
	Lie(G) = Lie(H) = (\R^{n},[,]_{0}) %
	\implies %
	Lie(G) \stackrel{iso.}{\simeq} Lie(H)
\end{equation}

dove $ [,]_{0} $ indica il commutatore nullo (e dunque un'algebra abeliana), ricordando che $ T_{1}(\S^{1}) = i \R $ è abbinato a $ [,]_{0} $ nell'algebra di $ \S^{1} $. \\
Questo non è in contraddizione con il teorema di corrispondenza di Lie in quanto $ \R^{n} $ è semplicemente connesso mentre il toro non lo è.

\subsubsection{Connessione tra $ SU(2) $ e $ SO(3) $ e le loro algebre di Lie}

Anche in questo caso, i gruppi di Lie $ SU(2) $ e $ SO(3) $ non sono isomorfi ma le loro algebre di Lie $ \su(2) $ e $ \so(3) $ lo sono: questo non contraddice il teorema di corrispondenza di Lie in quanto $ SO(3) $ non è semplicemente connesso. \\
Entrambi sono compatti e connessi, e hanno la stessa dimensione

\begin{equation}
	\begin{cases}
		\dim(SU(n)) = n^{2} - 1 \\
		\dim(SO(n)) = n(n-1)/2
	\end{cases}
	\implies %
	\dim(SO(3)) = \dim(SU(2)) = 3
\end{equation}

Analizziamo ora $ SU(2) $: l'insieme è definito come

\begin{equation}
	SU(2) = \{ A \in GL_{2}(\C) \mid A^{*} A = I_{2} \, \wedge \, \det(A) = 1 \}
\end{equation}

Possiamo ricavare delle condizioni sulle entrate delle matrici del gruppo: esplicitando quest'ultime

\begin{equation}
	A = \bmqty{ a & b \\ c & d } \qcomma a,b,c,d \in \C
\end{equation}

con le condizioni date nella definizione, dunque

\begin{equation}
	\begin{cases}
		A^{*} = \bmqty{ \bar{a} & \bar{c} \\ \bar{b} & \bar{d} } %
		= A^{-1} %
		= \cancel{\dfrac{1}{\det(A)}} \bmqty{ d & -b \\ -c & a } \\ \\
		\det(A) = a d - b c = 1
	\end{cases}
	\implies %
	\begin{cases}
		\bar{a} = d \\
		c = - \bar{b} \\
		a d - b c = 1
	\end{cases}
	\implies %
	\begin{cases}
		A = \bmqty{ a & b \\ - \bar{b} & \bar{a} } \\ \\
		\abs{a}^{2} + \abs{b}^{2} = 1
	\end{cases}
\end{equation}

dunque possiamo riscrivere la definizione

\begin{equation}
	SU(2) = \left\{ A = \bmqty{ a & b \\ - \bar{b} & \bar{a} } \in GL_{2}(\C) \st \abs{a}^{2} + \abs{b}^{2} = 1 \right\}
\end{equation}

Esiste un diffeomorfismo\footnote{%
	Vedi Esercizio \ref{BONUS3-2}.%
} tra $ SU(2) $ e $ \S^{3} \subset \R^{4} = \C^{2} $

\map{\varphi}
	{SU(2)}{\S^{3}}
	{\bmqty{ a & b \\ - \bar{b} & \bar{a} }}{(a,b) \in \C^{2}}

dove il punto $ (a,b) $ appartiene alla sfera $ \S^{3} $ grazie alla condizione $ \abs{a}^{2} + \abs{b}^{2} = 1 $. \\
Incidentalmente, tramite questo diffeomorfismo, possiamo dotare la sfera della struttura di gruppo di Lie mediante l'operazione associata a $ SU(2) $. \\
Lo spazio tangente di $ SU(2) $ è composto dalle matrici antihermitiane a traccia nulla

\begin{equation}
	T_{I}(SU(2)) = \{X \in M_{2}(\C) \mid X^{*} = - X \, \wedge \, \tr(X) = 0 \}
\end{equation}

Abbinandolo al commutatore matriciale standard otteniamo l'algebra di Lie associata

\begin{equation}
	\su(2) = Lie(SU(2)) = (T_{I}(SU(2)),[,])
\end{equation}

Per lo spazio tangente, le matrici hanno la forma:

\begin{equation}
	X = \bmqty{ %
				i u_{1} 			& u_{2} + i u_{3} \\ \\
				- u_{2} + i u_{3} 	& - i u_{1} %
				} %
	\qcomma u_{1},u_{2},u_{3} \in \R
\end{equation}

La base standard\footnote{%
	La base di $ \su(2) $ è legata alle \textit{matrici di Pauli} $ \sigma_{j} $ da $ \B_{\su(2)} = \{- i \sigma_{j}\}_{j = 1,2,3} $.%
} di $ \su(2) $ è la seguente

\begin{equation}
	\B_{\su(2)} = \left\{ %
		\bmqty{ 0 & 1 \\ -1 & 0 }, %
		\bmqty{ i & 0 \\ 0 & -i }, %
		\bmqty{ 0 & i \\ i & 0 } %
		\right\}
\end{equation}

Esiste un isomorfismo di spazi vettoriali tra $ T_{I}(SU(2)) $ ed $ \R^{3} $, rappresentato dall'applicazione

\map{\psi}
	{T_{I}(SU(2))}{\R^{3}}
	{\bmqty{ i u_{1} & u_{2} + i u_{3} \\ \\ - u_{2} + i u_{3} & - i u_{1} }}{u = \bmqty{ u_{1} \\ u_{2} \\ u_{3} }}

L'inversa di questa applicazione associa a un vettore $ u \in \R^{3} $ una matrice $ M_{u} $ che appartiene a $ T_{I}(SU(2)) $, i.e.

\map{\psi^{-1} = M}
	{\R^{3}}{T_{I}(SU(2))}
	{u = \bmqty{ u_{1} \\ u_{2} \\ u_{3} }}
	{M_{u} = \bmqty{ i u_{1} & u_{2} + i u_{3} \\ \\ - u_{2} + i u_{3} & - i u_{1} }}

Considerando il commutatore di $ \su(2) $ e il prodotto vettoriale\footnote{%
	Siano due vettori in $ T(\R^{3}) $ rappresentati tramite la loro identificazione con $ \R^{3} $
	%
	\begin{equation*}
		\begin{cases}
			u = (u_{1},u_{2},u_{3}) \\
			v = (v_{1},v_{2},v_{3})
		\end{cases}
	\end{equation*}
	%
	Il prodotto vettoriale tra questi può essere calcolato tramite il determinante tra le componenti dei vettori e i versori $ \hat{i},\hat{j},\hat{k} \in T(\R^{3}) $:
	%
	\begin{equation*}
		u \times v = %
		\mdet{ %
				\hat{i} & \hat{j} & \hat{k} \\
				u_{1} & u_{2} & u_{3} \\
				v_{1} & v_{2} & v_{3} %
				} %
		= (u_{2} v_{3} - u_{3} v_{2}, u_{3} v_{1} - u_{1} v_{3}, u_{1} v_{2} - u_{2} v_{1})
	\end{equation*} %
} $ \times $ di vettori di $ \R^{3} $, otteniamo

\begin{equation}
	[M_{u},M_{v}] = M_{u} M_{v} - M_{v} M_{u} = 2 M_{u \times v}
\end{equation}

questo induce un isomorfismo di algebre di Lie

\begin{equation}
	(\R^{3}, 2 \times) \stackrel{iso.}{\simeq} \su(2)
\end{equation}

in quanto il prodotto vettoriale soddisfa l'antisimmetria, la bilinearità e l'identità di Jacobi, i.e.

\begin{equation}
	(u \times v) \times w + (v \times w) \times u + (w \times u) \times v = 0
\end{equation}

Inoltre, considerando il prodotto scalare in $ \R^{3} $

\begin{equation}
	u \cdot v = - \dfrac{1}{2} \tr(M_{u} M_{v})
\end{equation}

questo induce un isomorfismo di gruppi di spazi euclidei

\begin{equation}
	(\R^{3},\cdot) \stackrel{iso.}{\simeq} \left( T_{I}(SU(2)), -\dfrac{1}{2} \tr(\cdot \, \cdot) \right)
\end{equation}

Per quanto riguarda $ SO(3) $: l'insieme è definito come

\begin{equation}
	SO(2) = \{ B \in GL_{3}(\R) \mid B^{T} B = I_{3} \, \wedge \, \det(B) = 1 \}
\end{equation}

La condizione di ortogonalità può essere scritta come

\begin{equation}
	B^{T} B = I_{3} %
	\iff %
	(B u) \cdot (B v) = u \cdot v %
	\qcomma \forall B \in O(3), \, \forall u,v \in \R^{3}
\end{equation}

i.e. la moltiplicazione di due vettori per matrici ortogonali speciali rispetta il prodotto scalare tra i due vettori.

\begin{lemma}
	Siano $ A \in SU(2) $ e $ M_{v} \in \su(2) $, allora $ A M_{v} A^{-1} \in \su(2) $.
\end{lemma}

\begin{proof}
	Siccome
	
	\begin{equation}
		\begin{cases}
			A \in SU(2) \implies A^{-1} = A^{*} \\
			M_{v} \in \su(2) \implies M_{v}^{*} = - M_{v}
		\end{cases}
	\end{equation}

	possiamo scrivere che
	
	\begin{equation}
		(A M_{v} A^{-1})^{*} = (A M_{v} A^{*})^{*} %
		= A M_{v}^{*} A^{*} %
		= - A M_{v} A^{*} %
		= - A M_{v} A^{-1}
	\end{equation}

	e considerando che
	
	\begin{gather}
		M_{v} \in \su(2) \implies \tr(M_{v}) = 0 \\
		\tr(A B) = \tr(B A) \qcomma \forall A,B \in M_{n}(\K)
	\end{gather}
	
	abbiamo che
	
	\begin{equation}
		\tr(A M_{v} A^{-1}) = \tr(M_{v} A^{-1} A) %
		= \tr(M_{v}) %
		= 0
	\end{equation}

	dunque
	
	\begin{equation}
		\begin{cases}
			(A M_{v} A^{-1})^{*} = - A M_{v} A^{-1} \\
			\tr(A M_{v} A^{-1}) = 0
		\end{cases}
		\implies %
		A M_{v} A^{-1} \in \su(2)
	\end{equation}
\end{proof}

Questo lemma implica che

\begin{equation}
	A M_{v} A^{-1} = M_{w} = M_{R(A) \, v}
\end{equation}

dove la matrice $ R(A) $ indica una trasformazione lineare; a priori, sappiamo solo che $ R(A) $ sia invertibile, i.e. $ R(A) \in GL_{3}(\R) $. \\
Facendo un passo oltre, mostriamo che $ R(A) \in O(3) $, i.e.

\begin{equation}
	(R(A) \, u) \cdot (R(A) \, v) = u \cdot v \qcomma u,v \in \R^{3}
\end{equation}

tramite l'isomorfismo

\begin{equation}
	(\R^{3},\cdot) \stackrel{iso.}{\simeq} \left( T_{I}(SU(2)), -\dfrac{1}{2} \tr(\cdot \, \cdot) \right)
\end{equation}

nel seguente modo:

\begin{align}
	\begin{split}
		(R(A) \, u) \cdot (R(A) \, v) &= - \dfrac{1}{2} \tr( M_{R(A) \, u} \, M_{R(A) \, v} ) \\
		&= - \dfrac{1}{2} \tr( A M_{u} A^{-1} A M_{v} A^{-1} ) \\
		&= - \dfrac{1}{2} \tr( A M_{u} M_{v} A^{-1} ) \\
		&= - \dfrac{1}{2} \tr( M_{u} M_{v} A^{-1} A ) \\
		&= - \dfrac{1}{2} \tr( M_{u} M_{v} ) \\
		&= u \cdot v
	\end{split}
\end{align}

\begin{theorem}
	L'applicazione
	
	\map{R}
		{SU(2)}{SO(3)}
		{A}{R(A)}
	%
	dove
	
	\begin{equation}
		 M_{R(A) \, v} = A M_{v} A^{-1} \in \su(2) %
		 \qcomma v \in \R^{3}, \, A \in SU(2), \, M_{v} \in \su(2)
	\end{equation}

	è un omomorfismo di gruppi di Lie suriettivo tale che
	
	\begin{equation}
		\ker(R) = \{ \pm I_{2} \}
	\end{equation}

	Conseguentemente, i seguenti gruppi di Lie sono isomorfi:
	
	\begin{equation}
		\sfrac{SU(2)}{\{ \pm I_{2}\}} \stackrel{iso.}{\simeq} SO(3)
	\end{equation}

	Inoltre il differenziale di $ R $ nell'identità 
	
	\begin{equation}
		R_{*I_{2}} : \su(2) \to \so(3)
	\end{equation}

	è un isomorfismo di algebre di Lie, i.e.
	
	\begin{equation}
		\su(2) \stackrel{iso.}{\simeq} \so(3)
	\end{equation}
\end{theorem}

\begin{proof}
	Dimostriamo innanzitutto che $ R(A) \in SO(3) $: l'applicazione $ R $ è continua (per costruzione) e $ SU(2) = \S^{3} $ è connesso dunque la sua immagine deve essere contenuta in $ SO(3) $
	
	\begin{equation}
		R(SU(2)) \subset SO(3)
	\end{equation}
	
	in quanto $ SO(3) $ è la componente connessa di $ O(3) $ che contiene l'identità ed essendo $ R $ un omomorfismo
	
	\begin{equation}
		R(\id_{SU(2)}) = \id_{O(3)} \in SO(3)
	\end{equation}

	Siccome un omomorfismo continuo è anche liscio\footnote{%
		Vedi Corollario \ref{cor:om-cont-smooth}.%
	}, è sufficiente mostrare che $ R $ sia un omomorfismo perché sia anche un omomorfismo di gruppi di Lie. \\
	Per mostrare che $ R $ sia un omomorfismo è necessario che
	
	\begin{equation}
		R(A B) = R(A) \, R(B) \qcomma \forall A,B \in SU(2)
	\end{equation}

	Usando la definizione di $ M_{R(A) \, v} $, otteniamo che
	
	\begin{equation}
		M_{R(A B) \, v} = (A B) M_{v} (A B)^{-1} %
		= A B M_{v} B^{-1} A^{-1} %
		= A M_{R(B) \, v} A^{-1} %
		= M_{R(A) \, R(B) \, v}
	\end{equation}

	essendo $ M $ un isomorfismo, i.e.
	
	\map{M}
		{\R^{3}}{T_{I}(SU(2))}
		{u = \bmqty{ u_{1} \\ u_{2} \\ u_{3} }}
		{M_{u} = \bmqty{ i u_{1} & u_{2} + i u_{3} \\ \\ - u_{2} + i u_{3} & - i u_{1} }}
	
	la condizione per cui $ R $ sia un omomorfismo è verificata. \\
	Il nucleo\footnote{%
		Per i gruppi, il nucleo di un'applicazione è dato dagli elementi del dominio che hanno come immagine l'elemento neutro del codominio.%
	} dell'omomorfismo $ R $ è dato da
	
	\begin{equation}
		\ker(R) = \{ A \in SU(2) \mid R(A) = I_{3} \}
	\end{equation}

	per cui
	
	\begin{equation}
		A \in \ker(R) %
		\implies %
		M_{R(A) \, v} = M_{v} = A M_{v} A^{-1} %
		\implies %
		[M_{v},A] = 0 \qcomma \forall v \in \R^{3}
	\end{equation}
	
	in quanto
	
	\begin{equation}
		[M_{v},A] = M_{v} A - A M_{v} %
		= A M_{v} A^{-1} A - A M_{v} %
		= A M_{v} - A M_{v} %
		= 0
	\end{equation}
	
	Le uniche matrici di $ SU(2) $ che commutano con $ M_{v} $ sono $ \pm I_{2} $ dunque $ \ker(R) = \{ \pm I_{2} \} $. \\
	Per dimostrare che $ R_{*I_{2}} : \su(2) \to \so(3) $ è un isomorfismo di algebre di Lie, scriviamo il differenziale di una matrice $ M_{u} \in \su(2) $ con $ u \in \R^{3} $ tramite la curva\footnote{%
		In questo caso, essendo il gruppo di Lie in questione un gruppo matriciale, l'applicazione esponenziale $ \exp $ coincide con l'esponenziale di matrici.%
	} $ \exp(t M_{u}) $
	
	\begin{equation}
		R_{*I_{2}}(M_{u}) = \dot{(R \circ \exp(t M_{u}))}(0) %
		= \eval{ \dv{t} }_{0} R( \exp(t M_{u}) )
	\end{equation}

	Siccome $ M_{v} $ è lineare e non dipende da $ t $, possiamo scrivere

	\begin{equation}
		\eval{ \dv{t} }_{0} M_{R(\exp(t M_{u})) \, v} = M_{\eval{ \dv{t} }_{0} R(\exp(t M_{u})) \, v} %
		= M_{R_{*I_{2}}(M_{u}) (v)}
	\end{equation}
	
	Tramite la definizione di $ M_{R(A) \, v} $, otteniamo che
	
	\begin{equation}
		M_{R(\exp(t M_{u})) \, v} = \exp(t M_{u}) \, M_{v} \, \exp(-t M_{u}) %
		\qcomma \forall t \in \R
	\end{equation}
	
	Facendo la derivata rispetto a $ t $ in $ t = 0 $ di entrambi i membri dell'uguaglianza sopra, possiamo anche scrivere
	
	\begin{align}
		\begin{split}
			\eval{ \dv{t} }_{0} M_{R(\exp(t M_{u})) \, v} &= \eval{ \dv{t} }_{0} \exp(t M_{u}) \, M_{v} \, \exp(-t M_{u}) \\
			&= \left( \eval{ \dv{t} }_{0} \exp(t M_{u}) \right) M_{v} \cancelto{I_{2}}{ \left( \eval{ \exp(-t M_{u}) }_{t=0} \right) } + \\
			& \hal + \cancelto{I_{2}}{ \left( \eval{ \exp(t M_{u}) }_{t=0} \right) } M_{v} \left( \eval{ \dv{t} \exp(-t M_{u}) }_{t=0} \right) \\
			%
			&= M_{u} M_{v} - M_{v} M_{u} \\
			&= [M_{u},M_{v}] \\
			&= 2 M_{u \times v}
		\end{split}
	\end{align}
	
	A questo punto, essendo $ M $ un isomorfismo, otteniamo che
	
	\begin{gather}
		\eval{ \dv{t} }_{0} M_{R(\exp(t M_{u})) \, v} = M_{R_{*I_{2}}(M_{u}) (v)} %
		= 2 M_{u \times v} \nonumber \\
		\Downarrow \\
		R_{*I_{2}}(M_{u}) (v) = 2 \, u \times v \nonumber
	\end{gather}
	
	Il prodotto vettoriale tra due vettori $ u,v \in \R^{3} $ può essere scritto come
	
	\begin{equation}
		u \times v = %
		\bmqty{ 0 & - u_{3} & u_{2} \\ u_{3} & 0 & - u_{1} \\ - u_{2} & u_{1} & 0 } %
		\bmqty{ v_{1} \\ v_{2} \\ v_{3} }
	\end{equation}
	
	dunque
	
	\begin{equation}
		R_{*I_{2}}(M_{u}) = %
		R_{*I_{2}} \left( \bmqty{ i u_{1} & u_{2} + i u_{3} \\ \\ - u_{2} + i u_{3} & - i u_{1} } \right) = %
		2 \bmqty{ 0 & - u_{3} & u_{2} \\ u_{3} & 0 & - u_{1} \\ - u_{2} & u_{1} & 0 } \in \so(3)
	\end{equation}
	
	il che rende il differenziale
	
	\map{R_{*I_{2}}}
		{\su(2)}{\so(3)}
		{M_{u} = \bmqty{ i u_{1} & u_{2} + i u_{3} \\ \\ - u_{2} + i u_{3} & - i u_{1} }}
		{2 \bmqty{ 0 & - u_{3} & u_{2} \\ u_{3} & 0 & - u_{1} \\ - u_{2} & u_{1} & 0 }}
	
	lineare e invertibile, dunque $ R_{*I_{2}} $ è un isomorfismo di algebre di Lie. \\
	Avendo calcolato il nucleo dell'applicazione (il quale non possiede solo l'elemento neutro del dominio), sappiamo che $ R $ non è iniettiva, però possiamo dimostrare che sia suriettiva: consideriamo il fatto che $ R $ sia un omomorfismo, i.e.
	
	\begin{equation}
		R \circ L_{A} = L_{R(A)} \circ R \qcomma \forall A \in SU(2)
	\end{equation}

	se differenziamo entrambi i membri in $ I_{2} $ e applichiamo la regola della catena
	
	\begin{align}
		\begin{split}
			(R \circ L_{A})_{*I_{2}} &= (L_{R(A)} \circ R)_{*I_{2}} \\
			R_{*L_{A}(I_{2})} \circ (L_{A})_{*I_{2}} &= (L_{R(A)})_{*R(I_{2})} \circ R_{*I_{2}} \\
			R_{*A} \circ (L_{A})_{*I_{2}} &= (L_{R(A)})_{*I_{3}} \circ R_{*I_{2}} \\
			R_{*A} &= (L_{R(A)})_{*I_{3}} \circ R_{*I_{2}} \circ ((L_{A})_{*I_{2}})^{-1}
		\end{split}
	\end{align}

	in quanto
	
	\begin{equation}
		\ker(R) = \{\pm I_{2}\} \subset SU(2) %
		\implies %
		R(\pm I_{2}) = I_{3} \in SO(3)
	\end{equation}
	
	Questo rende il differenziale in un punto qualsiasi $ A \in SU(2) $ un isomorfismo in quanto composizione di isomorfismi: a questo punto $ R $ è una sommersione (in quanto il suo differenziale, essendo un isomorfismo, è suriettivo) da cui, per il corollario del teorema di sommersione locale\footnote{%
		Vedi Corollario \ref{cor:sub-open}.%
	}, abbiamo che $ R $ è un'applicazione aperta\footnote{%
		Alternativamente, per il teorema della funzione inversa (vedi Teorema \ref{thm:ift}) $ R $ è un diffeomorfismo locale e dunque un'applicazione aperta.%
	}. Questa applicazione però è anche chiusa per il lemma dell'applicazione chiusa\footnote{%
		Vedi Lemma \ref{lemma:clos-app}.%
	}, in quanto bigezione continua, $ SU(2) $ compatto ed $ SO(3) $ di Hausdorff: l'immagine di $ SU(2) $ tramite un'applicazione sia aperta che chiusa è sia aperta che chiusa e, siccome $ SO(3) $ è connesso, l'immagine del dominio deve coincidere con il codominio, i.e.

	\begin{equation}
		R(SU(2)) = SO(3)
	\end{equation}
	
	dunque $ R : SU(2) \to SO(3) $ è suriettiva.
\end{proof}

\begin{remark}
	Questo teorema prova l'esistenza di un isomorfismo tra le algebre di Lie dei gruppi di Lie $ SU(2) $ e $ SO(3) $, nonostante questi non siano essi stessi isomorfi tra loro in quanto $ SO(3) $ non è semplicemente connesso (per il teorema di corrispondenza di Lie) e anche perché il loro centro\footnote{%
		Il centro $ Z(G) $ di un gruppo $ G $ è l'insieme degli elementi del gruppo che commutano con tutti gli altri elementi.%
	} ha un numero diverso di elementi
	
	\begin{equation}
		Z(SU(2)) = \{ \pm I_{2} \} \neq \{ I_{3} \} = Z(SO(3))
	\end{equation}
\end{remark}

\subsubsection{Il toro e sua algebra di Lie}

L'ultimo punto del teorema di corrispondenza di Lie asserisce che, presi un gruppo di Lie connesso $ G $ e la sua algebra di Lie $ \g = Lie(G) $, allora per ogni sottoalgebra di Lie $ \h < \g $ di $ \g $ esiste un sottogruppo di Lie connesso $ H \subset G $ di $ G $ tale che $ Lie(H) = \h $. Segue che, dati un gruppo di Lie connesso $ G $ e la sua algebra di Lie $ \g $, esiste una corrispondenza biunivoca tra sottoalgebre di Lie di $ \g $ e i sottogruppi connessi di $ G $: sappiamo che un sottogruppo di Lie $ H $ di $ G $ induce una sottoalgebra $ \h $ di $ \g $ tramite il suo spazio tangente, i.e.

\begin{equation}
	H \subset G \longrightarrow T_{e}(H) \subset T_{e}(G) \longrightarrow \h < \g
\end{equation}

in quanto l'inclusione è un omomorfismo di gruppi di Lie, ma il teorema di corrispondenza di Lie asserisce il viceversa (con la condizione che i gruppi di siano connessi). \\ \\
%
Consideriamo ora il gruppo di Lie del toro bidimensionale $ \T^{2} = \S^{1} \times \S^{1} $ e la sua algebra di Lie $ (\R^{2},[,]) $: le sottoalgebre di $ (\R^{2},[,]) $ hanno come insieme dei sottospazi vettoriali di $ \R^{2} $, dunque tutte le rette passanti per l'origine (oltre $ \R^{2} $ stesso, di dimensione 2, e l'origine, di dimensione 0). \\
Consideriamo l'esponenziale

\map{\exp}
	{\R^{2}}{\T^{2}}
	{(t,s)}{(e^{i t}, e^{i s})}

La retta identificata dai punti $ (0,s) \simeq \R $ viene mappata in $ (1,e^{i s}) \in \S^{1} \subset \T^{2} $: tutte le altre rette possono essere descritte dai punti $ (t,\alpha t) $ con $ \alpha \in \R $, la cui immagine

\begin{equation}
	\exp(t,\alpha t) = (e^{i t}, e^{i \alpha t}) = S_{\alpha}
\end{equation}

dipende da $ \alpha $ nel seguente modo

\begin{equation}
	\begin{cases}
		S_{\alpha} = \S^{1} & \qq*{se} \alpha \in \Q \\
		S_{\alpha} = \R & \qq*{se} \alpha \in \R \setminus \Q
	\end{cases}
\end{equation}

cioè se $ \alpha $ è razionale allora $ S_{\alpha} $ torna al punto di partenza e si chiude, mentre se è irrazionale $ S_{\alpha} $ non si chiude e sarà isomorfa a una retta. \\
Questi sottoinsiemi del toro sono sottovarietà immerse in quanto l'esponenziale è un'immersione\footnote{%
	Il differenziale dell'esponenziale è l'identità dello spazio tangente al gruppo di Lie, dunque iniettivo.%
}: quando si definiscono i sottogruppi di Lie, si richiede che questi siano solo sottovarietà immerse (e non embedded) del gruppo di Lie associato in modo tale da non escludere sottogruppi di Lie (e.g. quelli isomorfi a $ \R $) di alcuni gruppi di Lie, come il toro. \\ \\
%
L'ultima parte della dimostrazione (non riportata in questo testo) del teorema di corrispondenza di Lie utilizza il seguente teorema:

\begin{theorem}[Teorema di Ado]
	Una qualunque algebra di Lie è isomorfa a una sottoalgebra delle matrici quadrate a entrate reali, i.e.
	
	\begin{equation}
		\forall \g, \E  \g' \mid \g \simeq \g' \leqslant (M_{n}(\R),[,])
	\end{equation}
\end{theorem}

In linea teorica, studiare l'algebra di Lie delle matrici è dunque sufficiente per ottenere risultati su ogni algebra di Lie, a patto di conoscere l'isomorfismo tra l'algebra considerata e una sottoalgebra di $ M_{n}(\R) $, il quale potrebbe essere arbitrariamente complicato (il teorema di Ado non dà alcuna indicazione su come trovare l'isomorfismo in questione).


%

\appendix

\makeatletter
\renewcommand{\@chapapp}{Exercises}
\makeatother

\chapter{Exercises: Differential geometry in euclidean spaces}
\exer{Funzione $ C^{k}(\R) $ ma non $ C^{k+1}(\R) $}
{exer1-1}
{%
Per ogni numero naturale $ k \in \N $ costruire una funzione $ C^{k}(\R) $ ma non $ C^{k+1}(\R) $. %
}
{
Per la funzione

\map{f_{k}}
	{\R}{\R}
	{x}{\alpha x^{k(k+2)/(k+1)} + \beta}

per qualsiasi $ \alpha,\beta \in \R $ e con $ k \in \N $, valgono

\begin{equation}
	f_{k} \in C^{k}(\R) \; \wedge \; f_{k} \notin C^{k+1}(\R)
\end{equation}
}

%=======================================================================================

\exer{Funzione liscia ma non reale analitica}
{exer1-2}
{
Dimostrare che la funzione

\map{f}
	{\R}{\R}
	{x}{%
		\begin{cases}
			e^{\sfrac{-1}{x^{2}}} & x \neq 0 \\
			0 & x = 0
		\end{cases}
	}

risulta essere liscia ma non reale analitica.
}
{
La funzione $ f $ è liscia in quanto, perché lo sia, è necessario che

\begin{equation}
	\pdv[k]{f}{x} \, (0) = \pdv[k]{x} e^{\sfrac{-1}{x^{2}}} \, (0) = 0 %
	\qcomma \forall k \in \N
\end{equation}

e questo è vero poiché

\begin{equation}
	\lim_{x \to 0} \left( \dfrac{e^{\sfrac{-1}{x^{2}}}}{x^{p}} \right) = 0 \qcomma \forall p \in \N %
	\implies %
	\lim_{x \to 0} \left( \pdv[k]{x} \left( e^{\sfrac{-1}{x^{2}}} \right) \right) = 0 \qcomma \forall k \in \N
\end{equation}

La funzione non è però reale analitica perché, in un intervallo aperto qualunque di 0 non coincide con il suo sviluppo di Taylor: lo sviluppo di Taylor per la parte dei reali positivi è diversa da 0 per qualunque valore di $ x $ non nullo mentre la parte per i reali negativi è identicamente nulla, i.e. preso $ U $ un qualunque intorno di 0

\begin{equation}
	\sum_{k=0}^{+\infty} \left( \pdv[k]{x} \left( e^{\sfrac{-1}{x^{2}}} \right) \right) \dfrac{x^{k}}{k!} \neq 0 \qcomma \forall x \in U \setminus \{0\}
\end{equation}
}

%=======================================================================================

\exer{Intervalli diffeomorfi a $ \R $}
{exer1-3}
{
Siano $ a,b,c,d \in \R $ tale che $ a<b $. Dimostrare che i seguenti intervalli sono tutti diffeomorfi tra loro e diffeomorfi a $ \R $:

\begin{equation}
	\begin{cases}
		(a,b) \\
		(c,+\infty) \\
		(-\infty,d)
	\end{cases}
\end{equation}
}
{
Consideriamo le applicazioni:

\sbs{0.5}{%
			\map{f}
				{(a,b)}{(0,1)}
				{x}{\dfrac{x-a}{b-a}}
			
			\map{g}
				{(0,1)}{(c,+\infty)}
				{x}{\dfrac{c}{x}}
			}
	{0.5}{%
			\map{h}
				{(0,1)}{(-\infty,d)}
				{x}{\ln(x)-d}
	
			\map{i}
				{(c,+\infty)}{\R}
				{x}{\ln(x-c)}
			}

Queste sono diffeomorfismi in quanto bigezioni lisce con inversa liscia, dunque le loro composizioni sono ancora diffeomorfismi. Le seguenti composizioni delle applicazioni sopraccitate inducono i seguenti diffeomorfismi:

\begin{align}
	\begin{split}
		g \circ f &\implies (a,b) \simeq (c,+\infty) \\
		h \circ f &\implies (a,b) \simeq (-\infty,d) \\
		i \circ g \circ f &\implies (a,b) \simeq \R \\
		h \circ g^{-1} &\implies (c,+\infty) \simeq (-\infty,d) \\
		i &\implies (c,+\infty) \simeq \R \\
		i \circ g \circ h^{-1} &\implies (-\infty,d) \simeq \R
	\end{split}
\end{align}
}

%=======================================================================================

\exer{Diffeomorfismo tra $ B_{r}(c) $ e $ \R^{n} $}
{exer1-4}
{
Dimostrare che l'applicazione

\map{h}
	{B_{1}(0)}{\R^{n}}
	{x}{\left( \dfrac{x^{1}}{\sqrt{1 - \norm{x}^{2}}}, \cdots, \dfrac{x^{n}}{\sqrt{1 - \norm{x}^{2}}} \right)}

definisce un diffeomorfismo tra la palla aperta unitaria centrata nell'origine di $ \R^{n} $ ed $ \R^{n} $. Dedurre che la palla aperta di centro $ c \in \R^{n} $ e raggio $ r > 0 $ in $ \R^{n} $ è diffeomorfa a $ \R^{n} $.
}
{
L'applicazione $ h $ è una bigezione liscia in quanto ogni sua componente è liscia poiché

\begin{equation}
	\pdv[k]{(x^{i})} \left( \dfrac{x^{i}}{\sqrt{ 1-\norm{x}^{2} }} \right) < \infty %
	\qcomma \forall k \in \N, \, \forall x \in B_{1}(0), \, \forall i=1,\dots,n
\end{equation}

La sua inversa

\map{h^{-1}}
	{\R^{n}}{B_{1}(0)}
	{x}{\left( \dfrac{x^{1}}{\sqrt{1 + \norm{x}^{2}}}, \cdots, \dfrac{x^{n}}{\sqrt{1 + \norm{x}^{2}}} \right)}

è ancora liscia per lo stesso motivo, dunque $ h $ induce il diffeomorfismo $ B_{1}(0) \simeq \R^{n} $. \\
Se consideriamo l'applicazione lineare (dunque liscia con inversa liscia e perciò diffeomorfismo)

\map{g}
	{B_{r}(c)}{B_{1}(0)}
	{x}{\dfrac{x-c}{r}}

con $ c = (c^{1},\dots,c^{n}) $, e la componiamo con $ h $, otteniamo

\map{f = h \circ g}
	{B_{r}(c)}{\R^{n}}
	{x}{\left( \dfrac{\dfrac{x^{1} - c^{1}}{r}}{\sqrt{1 + \norm{\dfrac{x-c}{r}}^{2}}}, \cdots, \dfrac{\dfrac{x^{n} - c^{n}}{r}}{\sqrt{1 + \norm{\dfrac{x-c}{r}}^{2}}} \right)}

L'applicazione $ f $ è un diffeomorfismo in quanto composizione di diffeomorfismi, dunque $ f $ induce il diffeomorfismo $ B_{r}(c) \simeq \R^{n} $.
}

%=======================================================================================

\exer{Teorema di Taylor con resto per funzione a due variabili}
{exer1-5}
{
Sia $ f \in C^{\infty}(\R^{2}) $. Usando il teorema di Taylor con resto, dimostrare che esistono $ g_{11},g_{12},g_{22} \in C^{\infty}(\R^{2}) $ tali che

\begin{equation}
	f(x,y) = f(0,0) + x \, \dfrac{\partial f}{\partial x} (0,0) + y \, \dfrac{\partial f}{\partial y} (0,0) + x^{2} \, g_{11}(x,y) + x y \, g_{12}(x,y) + y^{2} \, g_{22}(x,y)
\end{equation}
}
{
Dal teorema di Taylor con resto, se $ f \in C^{\infty} (\R^{2}) $ ($ \R^{2} $ è stellato rispetto all'origine), abbiamo che

\begin{equation}
	\E g_{i_{1} \cdots i_{k}} \in C^{\infty} (\R^{2})
\end{equation}

definite come

\begin{equation}
	g_{i_{1} \cdots i_{k}} (0,0) \doteq \dfrac{1}{k!} \dfrac{\partial^{k} f}{\partial x^{i_{1}} \cdots \partial x^{i_{k}}} (0,0)
\end{equation}

tali che

\begin{equation}
	f(x,y) = f(0,0) + \sum_{m=1}^{k} \sum_{\substack{i_{1},\dots,i_{k}=1 \\ {i_{k}} > \cdots > i_{1}}}^{m} g_{i_{1} \cdots i_{k}} (x,y) \prod_{j=1}^{k} x^{i_{j}} %
	\qcomma \forall k \in \N
\end{equation}

Espandendo quest'ultima forma per $ k=1 $ otteniamo

\begin{align}
	\begin{split}
		f(x,y) &= f(0,0) + x \, g_{1} (x,y) + y \, g_{2} (x,y) \\
		&= f(0,0) + x \, g_{1} (0,0) + y \, g_{2} (0,0) + x^{2} \, g_{11}(x,y) + x y \, g_{12}(x,y) + y^{2} \, g_{22}(x,y) \\
		&= f(0,0) + x \, \dfrac{\partial f}{\partial x} (0,0) + y \, \dfrac{\partial f}{\partial y} (0,0) + x^{2} \, g_{11}(x,y) + x y \, g_{12}(x,y) + y^{2} \, g_{22}(x,y)
	\end{split}
\end{align}

dove gli ultimi tre termini indicano il resto.
}

%=======================================================================================

\exer{Funzione liscia tramite incollamento}
{exer1-6}
{
Sia $ f \in C^{\infty} (\R^{2}) $ tale che

\begin{equation}
	f(0,0) = \pdv{f}{x} \, (0,0) = \pdv{f}{y} \, (0,0) = 0
\end{equation}

Sia l'applicazione

\map{g}
	{\R^{2}}{\R}
	{(t,u)}{%
			\begin{cases}
				\dfrac{f(t,tu)}{t} & t \neq 0 \\ \\
				0 & t = 0
			\end{cases}
			}

Dimostrare che $ g \in C^{\infty}(\R^{2}) $.
}
{
Per il Teorema \ref{thm:taylor}, esistono due applicazioni $ h_{1},h_{2} \in C^{\infty}(\R^{2}) $ tali che

\begin{equation}
	\begin{cases}
		f(x,y) = f(0,0) + h_{1}(x,y) + h_{2}(x,y) \\\\
		h_{1}(0,0) = \dpdv{f}{x} \, (0,0) \\\\
		h_{2}(0,0) = \dpdv{f}{y} \, (0,0)
	\end{cases}
\end{equation}

Dalle ipotesi, possiamo scrivere

\begin{gather}
	f(x,y) = h_{1}(x,y) + h_{2}(x,y) \\
	 h_{1}(0,0) = h_{2}(0,0) = 0
\end{gather}

Considerando l'applicazione $ g $, possiamo dividere la trattazione in due casi:

\begin{itemize}
	\item $ t \neq 0 $:
	%
	\begin{align}
		\begin{split}
			g(t,u) &= \dfrac{1}{t} f(t,tu) \\
			&= \dfrac{1}{t} ( t h_{1}(t,tu) + t u \, h_{2}(t,tu) ) \\
			&= h_{1}(t,tu) + u \, h_{2}(t,tu)
		\end{split}
	\end{align}

	\item $ t = 0 $:
	%
	\begin{equation}
		g(0,u) = \cancelto{0}{h_{1}(0,0)} + u \, \cancelto{0}{h_{2}(0,0)} = 0
	\end{equation}
\end{itemize}

dunque

\begin{equation}
	g(t,u) = h_{1}(t,tu) + u \, h_{2}(t,tu) \qcomma \forall (t,u) \in \R^{2}
\end{equation}

Questo dimostra che $ g \in C^{0}(\R^{2}) $. Per dimostrare che sia liscia, consideriamo la derivata di $ g(t,u) $ rispetto a $ t $:

\begin{align}
	\begin{split}
		\dv{g(t,u)}{t} &= \pdv{g(t,u)}{x} \pdv{x}{t} + \pdv{g(t,u)}{y} \pdv{y}{t} \\
		&= \pdv{h_{1}(t,tu)}{x} \pdv{t}{t} + \pdv{h_{1}(t,tu)}{y} \pdv{(tu)}{t} + u \left( \pdv{h_{2}(t,tu)}{x} \pdv{t}{t} + \pdv{h_{2}(t,tu)}{y} \pdv{(tu)}{t} \right) \\
		&= \pdv{h_{1}(t,tu)}{x} + u \, \pdv{h_{1}(t,tu)}{y} + u \left( \pdv{h_{2}(t,tu)}{x} + u \, \pdv{h_{2}(t,tu)}{y} \right)\\
		&= \pdv{h_{1}(t,tu)}{x} + u \, \pdv{h_{1}(t,tu)}{y} + u \, \pdv{h_{2}(t,tu)}{x} + u^{2} \, \pdv{h_{2}(t,tu)}{y}
	\end{split}
\end{align}

Questa applicazione è liscia in quanto composizione liscia di applicazioni lisce (questo ragionamento si estende alle derivate di grado maggiore), dunque $ g \in C^{\infty}(\R^{2}) $.
}

%=======================================================================================

\exer{$ C_{p}^{\infty}(\R^{n}) $ come algebra commutativa e unitaria}
{exer1-7}
{
Dimostrare che l'insieme $ C_{p}^{\infty}(\R^{n}) $ dei germi delle funzioni lisce intorno a $ p \in \R^{n} $ con le operazioni di somma e di prodotto definite negli appunti è un'algebra commutativa e unitaria.
}
{
L'algebra $ A = (C_{p}^{\infty}(\R^{n}),+,\cdot) $ ha le operazioni definite come segue:

\map{+}
	{A \times A}{A}
	{([(f,U)],[(g,V)])}{[(f+g,U \cap V)]}

\map{\cdot}
	{A \times A}{A}
	{([(f,U)],[(g,V)])}{[(f g,U \cap V)]}

Perché sia effettivamente un'algebra, verifichiamo che sia distributiva e omogenea. \\
Per la distributività sinistra:

\begin{align}
	\begin{split}
		([(f,U)] + [(g,V)]) \cdot [(h,W)] &= [(f+g,U \cap V)] \cdot [(h,W)] \\
		&= [((f+g) h,U \cap V \cap W)] \\
		&= [(fh + gh,U \cap V \cap W)] \\
		&= [(fh,U \cap W)] + [(gh,V \cap W)] \\
		&= [(f,U)] \cdot [(h,W)] + [(g,V)] \cdot [(h,W)]
	\end{split}
\end{align}

per qualsiasi $ [(f,U)], [(g,V)], [(h,W)] \in C_{p}^{\infty}(\R^{n}) $. \\
La distributività destra deriva immediatamente dalla distributività sinistra e dalla commutatività (condizione non necessaria per un'algebra): quest'ultima può essere verificata tramite i seguenti passaggi

\sbs{0.5}{%
			\begin{align}
				\begin{split}
					[(f,U)] + [(g,V)] &= [(f+g,U \cap V)] \\
					&= [(g+f,V \cap U)] \\
					&= [(g,V)] + [(f,U)]
				\end{split}
			\end{align}
			}
	{0.5}{%
			\begin{align}
				\begin{split}
					[(f,U)] \cdot [(g,V)] &= [(fg,U \cap V)] \\
					&= [(gf,V \cap U)] \\
					&= [(g,V)] \cdot [(f,U)]
				\end{split}
			\end{align}
			}

per qualsiasi $ [(f,U)], [(g,V)] \in C_{p}^{\infty}(\R^{n}) $. \\
Per l'omogeneità:

\begin{align}
	\begin{split}
		\lambda ([(f,U)] \cdot [(g,V)]) &= \lambda [(fg,U \cap V)] \\
		&= [(\lambda fg,V \cap U)] \\
		&= [(\lambda f,U)] \cdot [(g,V)] \\
		&= [(f,U)] \cdot [(\lambda g,V)]
	\end{split}
\end{align}

per qualsiasi $ [(f,U)], [(g,V)] \in C_{p}^{\infty}(\R^{n}) $ e qualsiasi $ \lambda \in \R $. \\
Infine l'unitarietà, i.e.

\begin{equation}
	\E e = [(1,U)] \in C_{p}^{\infty}(\R^{n}) \mid [(f,U)] \cdot e = e \cdot [(f,U)] = [(f,U)] %
	\qcomma \forall [(f,U)] \in C_{p}^{\infty}(\R^{n})
\end{equation}

può essere verificata tramite i seguenti passaggi:

\begin{align}
	\begin{split}
		[(f,U)] \cdot [(1,U)] &= [(f \cdot 1,U \cap U)] \\
		&= [(1 \cdot f,U \cap U)] \\
		&= [(1,U)] \cdot [(f,U)] \\
		&= [(f,U)]
	\end{split}
\end{align}
}

%=======================================================================================

\exer{$ \der_{p}(C_{p}^{\infty}(\R^{n})) $ come spazio vettoriale su $ \R $}
{exer1-8}
{
Dimostrare che l'insieme $ \der_{p}(C_{p}^{\infty}(\R^{n})) $ delle derivazioni puntuali con le operazioni definite negli appunti è uno spazio vettoriale su $ \R $.
}
{
Per dimostrare che $ \der_{p}(C_{p}^{\infty}(\R^{n})) $ sia uno spazio vettoriale su $ \R $ è necessario che le operazioni di somma tra derivazioni e moltiplicazione per scalari rispettino i seguenti 8 assiomi:

\begin{equation}
	\begin{cases}
		D_{v} + (D_{w} + D_{x}) = (D_{v} + D_{w}) + D_{x} & \text{ 1. associatività (somma) } \\
		%
		D_{v} + D_{w} = D_{w} + D_{v} & \text{ 2. commutatività (somma) } \\
		%
		\E 0 \in \der_{p}(C_{p}^{\infty}(\R^{n})) \, \mid \, D_{v} + 0 = D_{v} & \text{ 3. elemento neutro (somma) } \\
		%
		\E - D_{v} \in \der_{p}(C_{p}^{\infty}(\R^{n})) \, \mid \, D_{v} + (- D_{v}) = 0 & \text{ 4. inverso (somma) } \\
		%
		\alpha (\beta D_{v}) = (\alpha \beta) D_{v} & \text{ 5. compatibilità (moltiplicazione) } \\
		%
		\E 1 \in \R \, \mid \, 1 D_{v} = D_{v} & \text{ 6. elemento neutro (moltiplicazione) } \\
		%
		(\alpha + \beta) D_{v} = \alpha D_{v} + \beta D_{v} & \text{ 7. distributività (somma vettoriale) } \\
		%
		\alpha (D_{v} + D_{w}) = \alpha D_{v} + \alpha D_{w} & \text{ 8. distributività (somma scalare) }
	\end{cases}
\end{equation}

per qualsiasi $ D_{v}, D_{w}, D_{x} \in \der_{p}(C_{p}^{\infty}(\R^{n})) $ e qualsiasi $ \alpha, \beta \in \R $. \\
Per dimostrare queste proprietà consideriamo un qualsiasi $ [(f,U)] \in C_{p}^{\infty}(\R^{n}) $ e applichiamo a questo le derivazioni:

\begin{enumerate}
	\item Associatività (somma)
	
	\begin{align}
		\begin{split}
			( D_{v} + (D_{w} + D_{x}) ) ([(f,U)]) &= D_{v} ([(f,U)]) + (D_{w} + D_{x}) ([(f,U)]) \\
			&= D_{v} ([(f,U)]) + D_{w} ([(f,U)]) + D_{x} ([(f,U)]) \\
			&= (D_{v} + D_{w}) ([(f,U)]) + D_{x} ([(f,U)]) \\
			&= ( (D_{v} + D_{w}) + D_{x} ) ([(f,U)])
		\end{split}
	\end{align}
	
	\item Commutatività (somma)
	
	\begin{align}
		\begin{split}
			(D_{v} + D_{w}) ([(f,U)]) &= D_{v} ([(f,U)]) + D_{w} ([(f,U)]) \\
			&= D_{w} ([(f,U)]) + D_{v} ([(f,U)]) \\
			&= (D_{w} + D_{v}) ([(f,U)])
		\end{split}
	\end{align}
	
	dove nel secondo passaggio abbiamo usato la commutatività della somma in $ \R $
	
	\item Elemento neutro (somma)
	
	\map{0}
		{C_{p}^{\infty}(\R^{n})}{\R}
		{[(f,U)]}{0}
	
	per qualsiasi $ [(f,U)] \in C_{p}^{\infty}(\R^{n}) $, dunque
	
	\begin{align}
		\begin{split}
			(D_{v} + 0) ([(f,U)]) &= D_{v} ([(f,U)]) + 0 ([(f,U)]) \\
			&= D_{v} ([(f,U)]) + 0 \\
			&= D_{v} ([(f,U)])
		\end{split}
	\end{align}
	
	\item Inverso (somma)
	
	\map{- D_{v}}
		{C_{p}^{\infty}(\R^{n})}{\R}
		{[(f,U)]}{- \sum_{i=1}^{n} \pdv{f}{x^{i}} \, (p) \, v^{i}}
	
	dunque
	
	\begin{align}
		\begin{split}
			(D_{v} + (- D_{v})) ([(f,U)]) &= D_{v} ([(f,U)]) + (- D_{v}) ([(f,U)]) \\
			&= \sum_{i=1}^{n} \pdv{f}{x^{i}} \, (p) \, v^{i} + \left( - \sum_{i=1}^{n} \pdv{f}{x^{i}} \, (p) \, v^{i} \right) \\
			&= 0
		\end{split}
	\end{align}
	
	\item Compatibilità (moltiplicazione)
	
	\begin{align}
		\begin{split}
			(\alpha (\beta D_{v})) ([(f,U)]) &= \alpha (\beta D_{v}) ([(f,U)]) \\
			&= \alpha D_{v} ([(\beta f,U)]) \\
			&= \alpha \beta D_{v} ([(f,U)]) \\
			&= (\alpha \beta) D_{v} ([(f,U)])
		\end{split}
	\end{align}
	
	\item Elemento neutro (moltiplicazione)
	
	\begin{align}
		\begin{split}
			(1 D_{v}) ([(f,U)]) &= D_{v} ([(1 f,U)]) \\
			&= D_{v} ([(f,U)])
		\end{split}
	\end{align}
	
	\item Distributività (somma vettoriale)
	
	\begin{align}
		\begin{split}
			(\alpha + \beta) D_{v} ([(f,U)]) &= D_{v} ([((\alpha + \beta) f,U)]) \\
			&= D_{v} ([(\alpha f + \beta f,U)]) \\
			&= \alpha D_{v} ([(f,U)]) + \beta D_{v} ([(f,U)]) \\
		\end{split}
	\end{align}
	
	\item Distributività (somma scalare)
	
	\begin{align}
		\begin{split}
			\alpha (D_{v} + D_{w}) ([(f,U)]) &= (D_{v} + D_{w}) ([(\alpha f,U)]) \\
			&= D_{v} ([(\alpha f,U)]) + D_{w} ([(\alpha f,U)]) \\
			&= \alpha D_{v} ([(f,U)]) + \alpha D_{w} ([(f,U)])
		\end{split}
	\end{align}
\end{enumerate}

Tutte queste proprietà sono valide per qualsiasi $ D_{v}, D_{w}, D_{x} \in \der_{p}(C_{p}^{\infty}(\R^{n})) $ e qualsiasi $ \alpha, \beta \in \R $.
}

%=======================================================================================

\exer{$ \chi(U) $ come spazio vettoriale su $ \R $ e $ C^{\infty} $-modulo}
{exer1-9}
{
Dimostrare che l'insieme dei campi di vettori lisci $ \chi(U) $ su un aperto $ U \subset \R^{n} $ con le operazioni definite negli appunti è uno spazio vettoriale su $ \R $ e un $ C^{\infty} $-modulo.
}
{
\paragraph{Spazio vettoriale su $ \R $}

Per dimostrare che $ \chi(U) $ sia uno spazio vettoriale su $ \R $ è necessario che le operazioni di somma tra campi di vettori e moltiplicazione per scalari rispettino i seguenti 8 assiomi:

\begin{equation}
	\begin{cases}
		X + (Y + Z) = (X + Y) + Z & \text{ 1. associatività (somma) } \\
		%
		X + Y = Y + X & \text{ 2. commutatività (somma) } \\
		%
		\E 0 \in \chi(U) \, \mid \, X + 0 = X & \text{ 3. elemento neutro (somma) } \\
		%
		\E - X \in \chi(U) \, \mid \, X + (- X) = 0 & \text{ 4. inverso (somma) } \\
		%
		\alpha (\beta X) = (\alpha \beta) X & \text{ 5. compatibilità (moltiplicazione) } \\
		%
		\E 1 \in \R \, \mid \, 1 X = X & \text{ 6. elemento neutro (moltiplicazione) } \\
		%
		(\alpha + \beta) X = \alpha X + \beta X & \text{ 7. distributività (somma vettoriale) } \\
		%
		\alpha (X + Y) = \alpha X + \alpha Y & \text{ 8. distributività (somma scalare) }
	\end{cases}
\end{equation}

per qualsiasi $ X, Y, Z \in \chi(U) $ e qualsiasi $ \alpha, \beta \in \R $. \\
Ricordiamo che le operazioni sono definite come:

\begin{gather}
	(X + Y)_{p} \doteq X_{p} + Y_{p} \\
	(\alpha X)_{p} \doteq \alpha X_{p}
\end{gather}

per qualsiasi $ X, Y \in \chi(U) $, qualsiasi $ \alpha \in \R $ e qualsiasi $ p \in U \subset \R^{n} $, dove i campi di vettori saranno:

\begin{equation}
	X = \sum_{i=1}^{n} a^{i} \pdv{x^{i}} %
	\qcomma Y = \sum_{i=1}^{n} b^{i} \pdv{x^{i}} %
	\qcomma a_{i}, b_{i} \in C^{\infty}(U), \, \forall i = 1, \dots, n
\end{equation}

Per dimostrare queste proprietà, valutiamo i campi di vettori in un qualsiasi $ p \in U \subset \R^{n} $:

\begin{enumerate}
	\item Associatività (somma)
	
	\begin{align}
		\begin{split}
			(X + (Y + Z))_{p} &= (X + Y)_{p} + Z_{p} \\
			&= X_{p} + Y_{p} + Z_{p} \\
			&= X_{p} + (Y + Z)_{p} \\
			&= (X + (Y + Z))_{p}
		\end{split}
	\end{align}
	
	\item Commutatività (somma)
	
	\begin{align}
		\begin{split}
			(X + Y)_{p} &= X_{p} + Y_{p} \\
			&= \sum_{i=1}^{n} a^{i} (p) \eval{ \pdv{x^{i}} }_{p} + \sum_{i=1}^{n} b^{i} (p) \eval{ \pdv{x^{i}} }_{p} \\
			&= \sum_{i=1}^{n} b^{i} (p) \eval{ \pdv{x^{i}} }_{p} + \sum_{i=1}^{n} a^{i} (p) \eval{ \pdv{x^{i}} }_{p} \\
			&= Y_{p} + X_{p} \\
			&= (Y + X)_{p}
		\end{split}
	\end{align}
	
	\item Elemento neutro (somma)
	
	\map{0}
		{U}{\bigsqcup_{p \in U} T_{p} (\R^{n})}
		{p}{%
			0_{p} = \sum_{i=1}^{n} 0 \eval{ \pdv{x^{i}} }_{p} \\
			&\stackrel{\R^{n}}{\mapsto} (0, \dots, 0)%
			}
	
	dunque
	
	\begin{align}
		\begin{split}
			(X + 0)_{p} &= X_{p} + 0_{p} \\
			&= \sum_{i=1}^{n} a^{i} (p) \eval{ \pdv{x^{i}} }_{p} + \sum_{i=1}^{n} 0 \eval{ \pdv{x^{i}} }_{p} \\
			&= \sum_{i=1}^{n} (a^{i} (p) + 0) \eval{ \pdv{x^{i}} }_{p} \\
			&= \sum_{i=1}^{n} a^{i} (p) \eval{ \pdv{x^{i}} }_{p} \\
			&= X_{p}
		\end{split}
	\end{align}
	
	\item Inverso (somma)
	
	\map{-X}
		{U}{\bigsqcup_{p \in U} T_{p} (\R^{n})}
		{p}{\sum_{i=1}^{n} (- a^{i} (p)) \eval{ \pdv{x^{i}} }_{p}}
	
	dunque
	
	\begin{align}
		\begin{split}
			(X + (- X))_{p} &= X_{p} + (- X)_{p} \\
			&= \sum_{i=1}^{n} a^{i} (p) \eval{ \pdv{x^{i}} }_{p} + \sum_{i=1}^{n} (- a^{i} (p)) \eval{ \pdv{x^{i}} }_{p} \\
			&= \sum_{i=1}^{n} (a^{i} (p) - a^{i} (p)) \eval{ \pdv{x^{i}} }_{p} \\
			&= \sum_{i=1}^{n} 0 \eval{ \pdv{x^{i}} }_{p} \\
			&= 0_{p}
		\end{split}
	\end{align}
	
	\item Compatibilità (moltiplicazione)
	
	\begin{align}
		\begin{split}
			(\alpha (\beta X))_{p} &= \alpha (\beta X)_{p} \\
			&= \alpha \beta X_{p} \\
			&= (\alpha \beta) X_{p} \\
			&= ((\alpha \beta) X)_{p}
		\end{split}
	\end{align}
	
	\item Elemento neutro (moltiplicazione)
	
	\begin{align}
		\begin{split}
			(1 X)_{p} &= 1 X_{p} \\
			&= 1 \sum_{i=1}^{n} a^{i} (p) \eval{ \pdv{x^{i}} }_{p} \\
			&= \sum_{i=1}^{n} (1 a^{i} (p)) \eval{ \pdv{x^{i}} }_{p} \\
			&= \sum_{i=1}^{n} a^{i} (p) \eval{ \pdv{x^{i}} }_{p} \\
			&= X_{p}
		\end{split}
	\end{align}
	
	\item Distributività (somma vettoriale)
	
	\begin{align}
		\begin{split}
			(\alpha (X + Y))_{p} &= \alpha (X + Y)_{p} \\
			&= \alpha (X_{p} + Y_{p}) \\
			&= \alpha \left( \sum_{i=1}^{n} a^{i} (p) \eval{ \pdv{x^{i}} }_{p} + \sum_{i=1}^{n} b^{i} (p) \eval{ \pdv{x^{i}} }_{p} \right) \\
			&= \alpha \sum_{i=1}^{n} a^{i} (p) \eval{ \pdv{x^{i}} }_{p} + \alpha \sum_{i=1}^{n} b^{i} (p) \eval{ \pdv{x^{i}} }_{p} \\
			&= \alpha X_{p} + \alpha Y_{p} \\
			&= (\alpha X)_{p} + (\alpha Y)_{p} \\
			&= (\alpha X + \alpha Y)_{p}
		\end{split}
	\end{align}
	
	\item Distributività (somma scalare)
	
	\begin{align}
		\begin{split}
			((\alpha + \beta) X)_{p} &= (\alpha + \beta) X_{p} \\
			&= (\alpha + \beta) \sum_{i=1}^{n} a^{i} (p) \eval{ \pdv{x^{i}} }_{p} \\
			&= \alpha \sum_{i=1}^{n} a^{i} (p) \eval{ \pdv{x^{i}} }_{p} + \beta \sum_{i=1}^{n} a^{i} (p) \eval{ \pdv{x^{i}} }_{p} \\
			&= \alpha X_{p} + \beta X_{p} \\
			&= (\alpha X)_{p} + (\beta X)_{p} \\
			&= (\alpha X + \beta X)_{p} 
		\end{split}
	\end{align}
\end{enumerate}

Tutte queste proprietà sono valide per qualsiasi $ X, Y, Z \in \chi(U) $ e qualsiasi $ \alpha, \beta \in \R $.

\paragraph{$ C^{\infty} $-modulo}

Sia l'applicazione

\map{\cdot}
	{C^{\infty}(U) \times \chi(U)}{\chi(U)}
	{(f,X)}{f X}

Per dimostrare che $ (\chi(U),+) $ sia un $ C^{\infty} $-modulo è necessario che siano verificate queste proprietà sia a sinistra che a destra:

\begin{equation}
	\begin{cases}
		1_{C^{\infty}(U)} X = X & \text{ 1. elemento neutro (somma) } \\
		f (g X) = (f g) X & \text{ 2. compatibilità (moltiplicazione) } \\
		f (X+Y) = f X + f Y & \text{ 3. distributività (somma vettoriale) } \\
		(f+g) X = f X + g X & \text{ 4. distributività (somma scalare) }
	\end{cases}
\end{equation}

per qualsiasi $ f,g \in C^{\infty}(U) $ e qualsiasi $ X,Y \in \chi(U) $. Siccome la moltiplicazione per funzione è commutativa, è sufficiente dimostrare che $ (\chi(U),+) $ sia un $ C^{\infty}(U) $-modulo sinistro (o destro) per dimostrare che sia $ C^{\infty}(U) $-modulo. \\
Dimostriamo dunque le proprietà riportate sopra, ancora una volta valutando i campi di vettori in un qualsiasi $ p \in U \subset \R^{n} $:

\begin{enumerate}
	\item Elemento neutro (somma)
	
	\map{1_{C^{\infty}(U)}}
		{U}{\R}
		{p}{1}
	
	dunque
	
	\begin{align}
		\begin{split}
			(1_{C^{\infty}(U)} X)_{p} &= 1_{C^{\infty}(U)} (p) X_{p} \\
			&= 1 \sum_{i=1}^{n} a^{i} (p) \eval{ \pdv{x^{i}} }_{p} \\
			&= \sum_{i=1}^{n} (1 a^{i} (p)) \eval{ \pdv{x^{i}} }_{p} \\
			&= \sum_{i=1}^{n} a^{i} (p) \eval{ \pdv{x^{i}} }_{p} \\
			&= X_{p}
		\end{split}
	\end{align}
	
	\item Compatibilità (moltiplicazione)
	
	\begin{align}
		\begin{split}
			(f (g X))_{p} &= f (p) (g X)_{p} \\
			&= f (p) g (p) X_{p} \\
			&= (f g) (p) X_{p} \\
			&= ((f g) X)_{p}
		\end{split}
	\end{align}
	
	\item Distributività (somma vettoriale)
	
	\begin{align}
		\begin{split}
			(f (X + Y))_{p} &= f (p) (X + Y)_{p} \\
			&= f (p) (X_{p} + Y_{p}) \\
			&= f (p) X_{p} + f (p) Y_{p} \\
			&= (f X)_{p} + (f Y)_{p} \\
			&= (f X + f Y)_{p}
		\end{split}
	\end{align}
	
	\item Distributività (somma scalare)
	
	\begin{align}
		\begin{split}
			((f + g) X)_{p} &= (f + g) (p) X_{p} \\
			&= (f (p) + g (p)) X_{p} \\
			&= f (p) X_{p} + g (p) X_{p} \\
			&= (f X)_{p} + (g X)_{p} \\
			&= (f X + g X)_{p}
		\end{split}
	\end{align}
\end{enumerate}

Tutte queste proprietà sono valide per qualsiasi $ f,g \in C^{\infty}(U) $ e qualsiasi $ X,Y \in \chi(U) $.
}

%=======================================================================================

\exer{$ \der(A) $ come spazio vettoriale su $ \K $}
{exer1-10}
{
Sia $ A $ un'algebra su un campo $ \K $. Dimostrare che le operazioni

\begin{equation}
	\begin{cases}
		(D_{1}+D_{2})(a) = D_{1}(a) + D_{2}(a) \\
		(\lambda D)(a) = \lambda D(a)
	\end{cases} %
	\qquad \forall \lambda \in \K, \, \forall D_{1},D_{2},D \in \der(A)
\end{equation}

dotano $ \der(A) $ della struttura di spazio vettoriale su $ \K $.
}
{
Per dimostrare che $ \der(A) $ sia uno spazio vettoriale su un campo $ \K $ è necessario che le operazioni di somma tra derivazioni e moltiplicazione per scalari rispettino i seguenti 8 assiomi:

\begin{equation}
	\begin{cases}
		D_{1} + (D_{2} + D_{3}) = (D_{1} + D_{2}) + D_{3} & \text{ 1. associatività (somma) } \\
		%
		D_{1} + D_{2} = D_{2} + D_{1} & \text{ 2. commutatività (somma) } \\
		%
		\E 0 \in \der(A)) \, \mid \, D + 0 = D & \text{ 3. elemento neutro (somma) } \\
		%
		\E - D \in \der(A) \, \mid \, D + (- D) = 0 & \text{ 4. inverso (somma) } \\
		%
		\alpha (\beta D) = (\alpha \beta) D & \text{ 5. compatibilità (moltiplicazione) } \\
		%
		\E \eta \in \K \, \mid \, \eta D = D & \text{ 6. elemento neutro (moltiplicazione) } \\
		%
		(\alpha + \beta) D = \alpha D + \beta D & \text{ 7. distributività (somma vettoriale) } \\
		%
		\alpha (D_{1} + D_{2}) = \alpha D_{1} + \alpha D_{2} & \text{ 8. distributività (somma scalare) }
	\end{cases}
\end{equation}

per qualsiasi $ D_{1}, D_{2}, D_{3}, D \in \der(A) $ e qualsiasi $ \alpha, \beta \in \K $. \\
Per dimostrare queste proprietà consideriamo un qualsiasi $ a \in A $ e applichiamo a questo le derivazioni:

\begin{enumerate}
	\item Associatività (somma)
	
	\begin{align}
		\begin{split}
			( D_{1} + (D_{2} + D_{3}) ) (a) &= D_{1} (a) + (D_{2} + D_{3}) (a) \\
			&= D_{1} (a) + D_{2} (a) + D_{3} (a) \\
			&= (D_{1} + D_{2}) (a) + D_{3} (a) \\
			&= ( (D_{1} + D_{2}) + D_{3} ) (a)
		\end{split}
	\end{align}
	
	\item Commutatività (somma)
	
	\begin{align}
		\begin{split}
			(D_{1} + D_{2}) (a) &= D_{1} (a) + D_{2} (a) \\
			&= D_{2} (a) + D_{1} (a) \\
			&= (D_{2} + D_{1}) (a)
		\end{split}
	\end{align}
	
	dove nel secondo passaggio abbiamo usato la commutatività della somma dell'algebra ereditata dallo spazio vettoriale che la compone
	
	\item Elemento neutro (somma)
	
	\map{0}
		{A}{A}
		{a}{0}
	
	dove
	
	\begin{equation}
		a + 0 = a \qcomma \forall a \in A
	\end{equation}
	
	dunque
	
	\begin{align}
		\begin{split}
			(D + 0) (a) &= D (a) + 0 (a) \\
			&= D (a) + 0 \\
			&= D (a)
		\end{split}
	\end{align}
	
	\item Inverso (somma)
	
	\map{- D}
		{A}{A}
		{a}{- D (a)}
	
	dunque
	
	\begin{align}
		\begin{split}
			(D + (- D)) (a) &= D (a) + (- D) (a) \\
			&= D (a) + - D (a) \\
			&= 0
		\end{split}
	\end{align}
	
	\item Compatibilità (moltiplicazione)
	
	\begin{align}
		\begin{split}
			(\alpha (\beta D)) (a) &= \alpha (\beta D) (a) \\
			&= \alpha \beta D (a) \\
			&= (\alpha \beta) D (a) \\
			&= ((\alpha \beta) D) (a)
		\end{split}
	\end{align}
	
	\item Elemento neutro (moltiplicazione)
	
	\begin{align}
		\begin{split}
			(\eta D) (a) &= \eta D (a) \\
			&= D (\eta a) \\
			&= D (a)
		\end{split}
	\end{align}
	
	dove abbiamo usato il fatto che lo spazio vettoriale che compone l'algebra è sullo stesso campo $ \K $ rispetto a quest'ultima
	
	\item Distributività (somma vettoriale)
	
	\begin{align}
		\begin{split}
			((\alpha + \beta) D) (a) &= (\alpha + \beta) D (a) \\
			&= \alpha D (a) + \beta D (a) \\
			&= (\alpha D) (a) + (\beta D) (a) \\
			&= (\alpha D + \beta D) (a)
		\end{split}
	\end{align}
	
	\item Distributività (somma scalare)
	
	\begin{align}
		\begin{split}
			(\alpha (D_{1} + D_{2})) (a) &= \alpha (D_{1} + D_{2}) (a) \\
			&= \alpha (D_{1} (a) + D_{2} (a)) \\
			&= \alpha D_{1} (a) + \alpha D_{2} (a) \\
			&= (\alpha D_{1}) (a) + (\alpha D_{2}) (a) \\
			&= (\alpha D_{1} + \alpha D_{2}) (a)
		\end{split}
	\end{align}
\end{enumerate}

Tutte queste proprietà sono valide per qualsiasi $ D_{1}, D_{2}, D_{3}, D \in \der(A) $ e qualsiasi $ \alpha, \beta \in \K $.
}

%=======================================================================================

\exer{Commutatore come derivazione}
{exer1-11}
{
Siano $ D_{1} $ e $ D_{2} $ due derivazioni di un'algebra $ A $ su un campo $ \K $, i.e. $ D_{1},D_{2} \in \der(A) $. Mostrare che $ D_{1} \circ D_{2} $ non è necessariamente una derivazione di $ A $ mentre

\begin{equation}
	D_{1} \circ D_{2} - D_{2} \circ D_{1} \in \der(A)
\end{equation}
}
{
Perché $ D_{1} \circ D_{2} $ sia una derivazione deve, in particolare, soddisfare la regola di Leibniz, ma questo non è verificato:

\begin{align}
	\begin{split}
		(D_{1} \circ D_{2})(a \cdot b) &= D_{1}(D_{2}(a \cdot b)) \\
		&= D_{1}( D_{2}(a) \cdot b + a \cdot D_{2}(b) ) \\
		&= D_{1}(D_{2}(a)) \cdot b + D_{2}(a) \cdot D_{1}(b) + D_{1}(a) \cdot D_{2}(b) + a \cdot D_{1}(D_{2}(b)) \\
		&= (D_{1} \circ D_{2})(a) \cdot b + a \cdot (D_{1} \circ D_{2})(b) + D_{2}(a) \cdot D_{1}(b) + D_{1}(a) \cdot D_{2}(b) \\
		&\neq (D_{1} \circ D_{2})(a) \cdot b + a \cdot (D_{1} \circ D_{2})(b)
	\end{split}
\end{align}

Mentre per la combinazione $ D_{1} \circ D_{2} - D_{2} \circ D_{1} $ questa prescrizione è verificata:

\begin{align}
	\begin{split}
		(D_{1} \circ D_{2} - D_{2} &\circ D_{1})(a \cdot b) = D_{1}(D_{2}(a \cdot b)) - D_{2}(D_{1}(a \cdot b)) \\
		&= (D_{1} \circ D_{2})(a) \cdot b + a \cdot (D_{1} \circ D_{2})(b) + D_{2}(a) \cdot D_{1}(b) + D_{1}(a) \cdot D_{2}(b) + \\
		& \hal - ((D_{2} \circ D_{1})(a) \cdot b + a \cdot (D_{2} \circ D_{1})(b) + D_{1}(a) \cdot D_{2}(b) + D_{2}(a) \cdot D_{1}(b)) \\
		%
		&= (D_{1} \circ D_{2})(a) \cdot b + a \cdot (D_{1} \circ D_{2})(b) + \cancel{ D_{2}(a) \cdot D_{1}(b) } + \cancel{ D_{1}(a) \cdot D_{2}(b) } + \\
		& \hal - (D_{2} \circ D_{1})(a) \cdot b - a \cdot (D_{2} \circ D_{1})(b) - \cancel{ D_{1}(a) \cdot D_{2}(b) } - \cancel{ D_{2}(a) \cdot D_{1}(b) } \\
		%
		&= (D_{1} \circ D_{2} - D_{2} \circ D_{1})(a) \cdot b + a \cdot (D_{1} \circ D_{2} - D_{2} \circ D_{1})(b)
	\end{split}
\end{align}

dunque $ D_{1} \circ D_{2} - D_{2} \circ D_{1} \in \der(A) $.
}


%

\chapter{Exercises: Differential manifolds}
%\exer{Unione disgiunta di spazi topologici come spazio topologico}
{BONUS2-1}
{
Verificare che l'unione disgiunta di spazi topologici

\begin{equation}
	A = \bigsqcup_{j \in J} A_{j} \equiv \bigcup_{j \in J} A_{j} \times \{j\}
\end{equation}

è uno spazio topologico, sapendo che $ U $ è aperto in $ A $ se e solo se $ U \cap A_{j} $ è aperto in $ A_{j} $ per qualsiasi $ j \in J $.
}
{
Per dimostrare che l'unione disgiunta di spazi topologici $ A $ sia ancora uno spazio topologico dobbiamo dimostrare che:

\begin{itemize}
	\item L'intersezione di due aperti in $ A $ sia ancora aperto in $ A $
	
	\item L'unione di un numero qualunque di aperti in $ A $ sia ancora aperto in $ A $
\end{itemize}

Per la prima, prendiamo due aperti $ U $ e $ V $ in $ A $: questo significa che, per ipotesi, $ U \cap A_{j} $ e $ V \cap A_{j} $ saranno aperti in $ A_{j} $ per qualsiasi $ j \in J $. Siccome gli $ A_{j} $ sono spazi topologici, l'intersezione di due aperti è ancora un aperto, dunque

\begin{equation}
	U \cap V \cap A_{j} \text{ aperto in } A_{j} \qcomma \forall j \in J
\end{equation}

A questo punto vale l'implicazione inversa, perciò $ U \cap V $ è aperto in $ A $. \\
Analogamente, per la seconda, prendiamo una famiglia infinita di aperti $ \{V_{i}\}_{i \in \N} $ in $ A $: questo significa che, per ipotesi, $ V_{i} \cap A_{j} $ saranno aperti in $ A_{j} $ per qualsiasi $ j \in J $ e qualsiasi $ i \in \N $. Siccome gli $ A_{j} $ sono spazi topologici, l'unione di un numero qualunque di aperti è ancora un aperto; definendo
%
\begin{equation}
	V \doteq \bigcup_{i=1}^{\infty} V_{i}
\end{equation}

possiamo dire che $ V \cap A_{j} $ è aperto in $ A_{j} $ per qualsiasi $ j \in J $. A questo punto vale l'implicazione inversa, perciò $ V $ è aperto in $ A $.
}

%=======================================================================================

\exer{Atlante differenziabile per $ \S^{n} $ con $ 2(n+1) $ carte}
{exer2-1}
{
Sia $ \S^{n} $ la sfera unitaria in $ \R^{n+1} $. Trovare un atlante differenziabile di $ \S^{n} $ con $ 2(n+1) $ carte.
}
{
Definiamo\footnote{%
	La funzione $ \ceil*{x} $ \textit{ceiling} (soffitto) mappa un numero reale nel successivo numero naturale più vicino (i.e. arrotonda per eccesso), e.g. $ \ceil*{\frac{3}{2}} = 2 $.%
} i $ 2(n+1) $ aperti

\begin{equation}
	U_{i} = \left\{ (x^{1},\dots,x^{n+1}) \in \R^{n+1} \st (-1)^{i} x^{\ceil*{\frac{i}{2}}} > 0 \right\} \subset \R^{n+1} %
	\qcomma i = 1, \dots, 2(n+1)
\end{equation}

i quali formano un ricoprimento per $ \S^{n} $, i.e.

\begin{equation}
	\S^{n} = \bigcup_{i=1}^{2(n+1)} U_{i}
\end{equation}

Definiamo anche gli aperti per $ j = 1,\dots,n+1 $

\begin{equation}
	D_{j} = \{ (x^{1},\dots,x^{j-1},x^{j+1},\dots,x^{n+1}) \in \R^{n} \times \{0\} = \R^{n} \, \mid \, \norm{x} < 1 \} = B_{1}(0) \simeq \R^{n}
\end{equation}

i quali, in sostanza, sono palle di raggio unitario centrate nell'origine nel piano $ \R^{n} \times \{0\} $, dove la \textit{norma} è definita come

\begin{equation}
	\norm{x} \doteq \sqrt{ \sum_{i=1}^{n} (x^{i})^{2} }
\end{equation}

Infine definiamo le applicazioni lisce

\map{\varphi_{i}}
	{U_{i}}{D_{\ceil*{\frac{i}{2}}}}
	{(x^{1},\dots,x^{i-1},x^{i},x^{i+1},\dots,x^{n+1})}{(x^{1},\dots,x^{i-1},x^{i+1},\dots,x^{n+1})}

con inverse

\map{\varphi_{i}^{-1}}
	{D_{\ceil*{\frac{i}{2}}}}{U_{i}}
	{(x^{1},\dots,x^{i-1},x^{i},\dots,x^{n})}{\left( x^{1},\dots,x^{i-1},(-1)^{i} \sqrt{1-\norm{x}^{2}},x^{i},\dots,x^{n} \right)}

La famiglia di coppie $ \mathfrak{U} = \{ (U_{i},\varphi_{i}) \}_{i=1,\dots,2(n+1)} $ definisce un atlante differenziabile per $ \S^{n} $ in quanto i cambi di carte sono lisci: la dimostrazione è analoga a quella fatta nell'Esempio \ref{example:diff-man-unit-sph} per $ \S^{2} $.
}

%=======================================================================================

\exer{Equivalenza tra strutture differenziabili su $ \S^{n} $}
{exer2-2}
{
Dimostrare che la struttura differenziabile su $ \S^{n} $ definita nell’esercizio precedente e quella definita dalle proiezioni stereografiche coincidono.
}
{
Per dimostrare che le due strutture differenziabili coincidano è sufficiente mostrare che le carte di ognuna siano $ C^{\infty} $-compatibili con quelle dell'altra. \\
Considerando le intersezioni:

\begin{align}
	U_{N} \cap U_{i} &= %
	\begin{cases}
		U_{i}, & i \neq 2(n+1) \\
		U_{i} \setminus \{ N \}, & i = 2(n+1)
	\end{cases} \\
	%
	\nonumber \\
	%
	U_{S} \cap U_{i} &= %
	\begin{cases}
		U_{i}, & i \neq 2(n+1) - 1 = 2 n + 1 \\
		U_{i} \setminus \{ S \}, & i = 2n+1
	\end{cases}
\end{align}

possiamo scrivere i cambi di carta

\begin{gather}
	\pi_{N} \circ \varphi_{i}^{-1} : \varphi_{i}(U_{N} \cap U_{i}) \to \pi_{N}(U_{N} \cap U_{i}) \\
	\pi_{S} \circ \varphi_{i}^{-1} : \varphi_{i}(U_{S} \cap U_{i}) \to \pi_{S}(U_{S} \cap U_{i}) \\
	\nonumber \\
	\varphi_{i} \circ \pi_{N}^{-1} : \pi_{N}(U_{N} \cap U_{i}) \to \varphi_{i}(U_{N} \cap U_{i}) \\
	\varphi_{i} \circ \pi_{S}^{-1} : \pi_{N}(U_{S} \cap U_{i}) \to \varphi_{i}(U_{S} \cap U_{i})
\end{gather}

esplicitamente abbiamo che

\begin{align}
	\begin{split}
		(\pi_{N} \circ \varphi_{i}^{-1})(x^{1},\dots,x^{n}) &= \pi_{N} \left( x^{1}, \dots, x^{i-1}, (-1)^{i} \sqrt{1 - \norm{x}^{2}}, x^{i}, \dots, x^{n} \right) \\
		&= \left( \dfrac{2 x^{1}}{1-x^{n}}, \dots, \dfrac{2 x^{i-1}}{1-x^{n}}, \dfrac{2 (-1)^{i} \sqrt{1 - \norm{x}^{2}}}{1-x^{n}}, \dfrac{2 x^{i}}{1-x^{n}}, \dots, \dfrac{2 x^{n-1}}{1-x^{n}} \right)
	\end{split}
\end{align}
%
\begin{align}
	\begin{split}
		(\varphi_{i} \circ \pi_{N}^{-1})(x^{1},\dots,x^{n}) &= \varphi_{i} \left( \dfrac{x^{1}}{1 + \norm{x}^{2}}, \dots, \dfrac{x^{n}}{1 + \norm{x}^{2}}, \dfrac{1 - \norm{x}^{2}}{1 + \norm{x}^{2}} \right) \\
		&= \left( \dfrac{x^{1}}{1 + \norm{x}^{2}}, \dots, \dfrac{x^{i-1}}{1 + \norm{x}^{2}}, \dfrac{x^{i+1}}{1 + \norm{x}^{2}}, \dots, \dfrac{x^{n}}{1 + \norm{x}^{2}}, \dfrac{1 - \norm{x}^{2}}{1 + \norm{x}^{2}} \right)
	\end{split}
\end{align}

e analogamente per $ \pi_{S} $: i cambi di carta sono lisci dunque le strutture differenziabili sono equivalenti.
}

%=======================================================================================

\exer{Grassmanniana come spazio topologico connesso e compatto}
{exer2-5}
{
Dimostrare che la Grassmanniana $ G(k,n) $ è uno spazio topologico connesso e compatto.
}
{
\paragraph{Connessione}

Siccome il quoziente preserva la proprietà di connessione, è sufficiente mostrare che lo spazio delle matrici di rango $ k $, i.e. $ F(k,n) $, sia connesso per dimostrare che lo sia $ G(k,n) $. \\
Per semplificare il calcolo, consideriamo il caso per $ k = 2 $ e $ n = 4 $: possiamo riscrivere lo spazio in esame come

\begin{equation}
	F(2,4) = \left\{ A = \bmqty{ a_{11} & a_{12} \\ a_{21} & a_{22} \\ a_{31} & a_{32} \\ a_{41} & a_{42} } \in M_{4,2}(\R) \st \rank(A) = 2 \right\}
\end{equation}

Possiamo associare a questo spazio un atlante $ \{(V_{ij},\psi_{ij})\} $ dove

\begin{equation}
	V_{ij} = \left\{ A \in F(2,4) \st A_{ij} \in GL_{2}(\R) \right\} %
	\qcomma \{i,j\} = \{1,2,3,4\}
\end{equation}

\map{\psi_{ij}}
	{V_{ij}}{M_{2}(\R) = \R^{4}}
	{A}{A_{lm} (A_{ij})^{-1}}

per $ \{i,j,l,m\} = \{1,2,3,4\} $. \\
Essendo $ \psi_{ij} $ un omeomorfismo, i.e.

\begin{equation}
	V_{ij} \stackrel{omeo}{\simeq} \R^{4} \qcomma \forall \{i,j\} \in \{1,2,3,4\}
\end{equation}

ogni carta $ V_{ij} $ è connessa in quanto $ \R^{4} $ è connesso e l'omeomorfismo preserva la connessione. \\
Siccome le carte dell'atlante formano un ricoprimento per $ F(2,4) $ e queste carte hanno sempre almeno un punto in comune, possiamo scrivere

\begin{equation}
	\begin{cases}
		V_{ij} \stackrel{omeo}{\simeq} \R^{4}, & \forall \{i,j\} \in \{1,2,3,4\} \\
		V_{ij} \cap V_{lm} \neq \emptyset, & \forall \{i,j,l,m\} \in \{1,2,3,4\}
	\end{cases} %
	\implies %
	F(2,4) \text{ connesso}
\end{equation}

in quanto l'unione di due aperti connessi che hanno almeno un punto in comune è ancora un insieme connesso. \\
Il ragionamento è analogo per qualsiasi valore di $ k $ ed $ n $.

\paragraph{Connessione (dimostrazione alternativa)}

% https://math.stackexchange.com/questions/2862992/connected-components-of-real-matrices-in-m-n-mathbb-r-with-constant-rank-k?noredirect=1&lq=1

Siccome il quoziente preserva la proprietà di connessione, è sufficiente mostrare che lo spazio delle matrici di rango $ k $, i.e. $ F(k,n) $, sia connesso per dimostrare che lo sia $ G(k,n) $. \\
Uno spazio topologico connesso per archi è anche connesso, dunque dimostriamo che per qualunque coppia di matrici di $ F(k,n) $ sia possibile costruire una funzione continua che le colleghi, i.e. preso $ I = [0,1] \subset \R $

\begin{gather}
	F(k,n) \text{ connesso per archi} \nonumber \\
	\Updownarrow \\
	\forall A, B \in F(k,n), \E f : I \to F(k,n) \text{ continua} %
	\, \mid \, f(0) = A \, \wedge \, f(1) = B \nonumber
\end{gather}

Siano due matrici invertibili $ P \in GL_{n}(\R) $ e $ Q \in GL_{k}(\R) $, una qualunque matrice $ A \in F(k,n) $ può essere scritta come

\begin{equation}
	A = P J Q \qcomma J \doteq \bmqty{I_{k} \\ 0_{n-k,k}}
\end{equation}

Consideriamo ora le seguenti applicazioni continue:

\sbs{0.5}{%
			\map{P_{t}}
				{I}{GL_{n}(\R)}
				{t}{P_{t}}
			
			\begin{equation}
				\begin{cases}
					P_{t}(0) = P_{0} = P \\
					P_{t}(1) = P_{1} = \diag(1,\dots,1,\det(P))
				\end{cases}
			\end{equation}
			}
	{0.5}{%
			\map{Q_{t}}
				{I}{GL_{n}(\R)}
				{t}{Q_{t}}
			
			\begin{equation}
				\begin{cases}
					Q_{t}(0) = Q_{0} = Q \\
					Q_{t}(1) = Q_{1} = \diag(1,\dots,1,\det(Q))
				\end{cases}
			\end{equation}
			}

Queste sono continue perché ognuna risiede in una delle componenti connesse dell'insieme delle matrici invertibili, i.e. in una delle seguenti due parti

\begin{equation}
	GL_{n}(\R) = \{ A \in GL_{n}(\R) \mid \det(A) > 0 \} \sqcup \{ A \in GL_{n}(\R) \mid \det(A) < 0 \}
\end{equation}

A questo punto, consideriamo l'applicazione

\map{f}
	{I}{F(k,n)}
	{t}{P_{t} J Q_{t}}

dove

\begin{equation}
	\begin{cases}
		f(0) = P_{0} J Q_{0} = A \\
		f(1) = P_{1} J Q_{1} = J
	\end{cases}
\end{equation}

La continuità di questa funzione deriva dalla continuità di $ P_{t} $ e $ Q_{t} $. \\
Collegando ogni matrice di $ F(k,n) $ alla matrice $ J $ (anch'essa in $ F(k,n) $ in quanto di rango $ k $) è possibile collegare queste matrici tra loro, dimostrando che $ F(k,n) $ è connesso e dunque lo è anche la Grassmanniana.

\paragraph{Compattezza}

Per Gram-Schmidt, possiamo considerare una base ortonormale di vettori di $ \R^{n} $ per uno spazio vettoriale di dimensione $ k $, i.e. $ \{ w_{1}, \dots, w_{k} \} $. Consideriamo ora l'insieme di matrici che abbiano come colonne i vettori ortonormali descritti sopra:

\begin{equation}
	H(k,n) = \{ W = \bmqty{ w_{1} & \cdots & w_{k} } \in M_{n,k}(\R) \mid w_{i} \cdot w_{j} = \delta_{ij}, \, i,j = 1, \dots, k \}
\end{equation}

Se pensiamo a questi vettori come a una matrice di $ M_{n,k}(\R) $, questa appartiene a $ F(k,n) $ in quanto avrà sempre rango $ k $, i.e.

\begin{gather}
	W = \bmqty{ w_{1} & \cdots & w_{k} } \in F(k,n) %
	\; \because \; %
	\rank(W) = k \nonumber \\
	\Downarrow \\
	H(k,n) \subseteq F(k,n) \nonumber
\end{gather}

Prendendo la seguente applicazione continua

\map{f_{ij}}
	{M_{n,k}(\R)}{\R}
	{A = \bmqty{ a_{1} & \cdots & a_{k} }}{a_{i} \cdot a_{j}}

possiamo notare che $ H(k,n) $ è chiuso in quanto intersezione di chiusi (controimmagini di chiusi mediante applicazione continua):

\begin{equation}
	H(k,n) = \left( \bigcap_{\substack{ i,j = 1 \\ i \neq j }}^{k} f_{ij}^{-1}(0) \right) \cap \left( \bigcap_{i = 1}^{k} f_{ii}^{-1}(1) \right)
\end{equation}

L'insieme $ H(k,n) $ è anche limitato in quanto, se prendiamo la norma per matrici definita tramite l'applicazione

\map{\norm{}}
	{M_{n,k}(\R)}{\R}
	{A = [a_{ij}]_{\substack{ i = 1, \dots, n \\ j = 1, \dots, k }}}{\left( \sum_{i=1}^{n} \sum_{j=1}^{k} \abs{a_{ij}}^{2} \right)^{1/2}}

otteniamo che

\begin{equation}
	\norm{A} = \sqrt{k} < + \infty \qcomma \forall A \in H(k,n)
\end{equation}

Essendo chiuso e limitato in $ M_{n,k}(\R) = \R^{nk} $, $ H(k,n) $ è compatto per il teorema di Heine-Borel. \\
Due matrici di $ H(k,n) $ generano lo stesso spazio vettoriale se e solo se esiste una matrice ortogonale\footnote{%
	In generale, è sufficiente che la matrice sia invertibile, ma se la base è ortonormale allora la matrice deve essere ortogonale.%
} che le lega

\begin{equation}
	W' = W O \in H(k,n) \qcomma \forall W \in H(k,n), \, O \in O(k)
\end{equation}

dunque è naturale considerare la seguente relazione di equivalenza:

\begin{equation}
	A \sim_{O} B \iff \E O \in O(n) \mid A = B O \qcomma \forall A, B \in H(k,n)
\end{equation}

da cui possiamo ricavare il quoziente

\begin{equation}
	J(k,n) \doteq \dfrac{H(k,n)}{\sim_{O}}
\end{equation}

il quale è ancora compatto in quanto il quoziente preserva la compattezza. \\
Consideriamo ora il seguente diagramma:

\sbs{0.5}{%
			\diagr{%
					{H(k,n)} \arrow[rrdd, "\varphi"] \arrow[dd, "\pi_{O}"] \arrow[rr, "i"] \&  \& {F(k,n)} \arrow[dd, "\pi_{g}"] \\
					\&  \&                                \\
					{J(k,n)} \arrow[rr, "\psi"]                                            \&  \& {G(k,n)}                      
					}
			}
	{0.5}{%
			\diagr{%
					A \arrow[rrdd, "\varphi", maps to] \arrow[dd, "\pi_{O}", maps to] \arrow[rr, "i", maps to] \&  \& A \arrow[dd, "\pi_{g}", maps to] \\
					\&  \&                                  \\
					{[A]_{O}} \arrow[rr, "\psi", maps to]                                                      \&  \& {[A]_{g}}                       
					}
			}

L'applicazione $ \psi $ è ben definita in quanto

\begin{equation}
	A \sim_{O} B \implies A \sim_{g} B %
	\; \because \; %
	O \in GL_{k}(\R) \qcomma \forall O \in O(k)
\end{equation}

Questa applicazione è anche continua in quanto composizione di funzioni continue:

\begin{equation}
	\psi = \varphi \circ \pi_{O}^{-1} = \pi_{g} \circ i \circ \pi_{O}^{-1}
\end{equation}

Consideriamo ora una classe della Grassmanniana

\begin{equation}
	[B]_{g} \in G(k,n) \qq{dove} B = \bmqty{ b_{1} & \cdots & b_{k} } \in F(k,n) \qcomma b_{i} \in \R^{n}
\end{equation}

Tramite Gram-Schmidt, possiamo trovare un'altra matrice che generi lo stesso spazio vettoriale ma che abbia le colonne ortonormali tra loro, i.e.

\begin{equation}
	A = \bmqty{ a_{1} & \cdots & a_{k} } \in H(k,n) \qcomma a_{i} \cdot a_{j} = \delta_{ij}
\end{equation}

Siccome generano lo stesso spazio, sono legate da una matrice invertibile, i.e.

\begin{equation}
	A = B g \qcomma g \in GL_{k}(\R) %
	\implies %
	[A]_{g} = [B]_{g}
\end{equation}

Questo mostra che $ \psi $ è suriettiva, in quanto

\begin{equation}
	\psi([A]_{O}) = [A]_{g} = [B]_{g}
\end{equation}

i.e. per qualsiasi classe di $ G(k,n) $ esiste una classe di $ J(k,n) $ la cui immagine tramite $ \psi $ coincida con la prima perché tutte le immagini di classi di $ J(k,n) $ generano lo stesso spazio. \\
A questo punto $ \Im(\psi) = G(k,n) $ e dunque la Grassmanniana è compatta in quanto immagine di un compatto tramite un'applicazione continua.

\paragraph{Compattezza (soluzione alternativa)}

Consideriamo una \textit{varietà differenziale di Stiefel} $ V_{k}(\R^{n}) $: questa varietà è l'insieme di tutte le basi ortonormali di dimensione $ k $, i.e. le $ k $-uple di vettori linearmente indipendenti e normalizzati; questa varietà è un sottoinsieme di $ \R^{n} $. Possiamo pensare alla Grassmanniana come quoziente di una varietà differenziale di Stiefel

\begin{equation}
	G(k,n) = \dfrac{V_{k}(\R^{n})}{\sim}
\end{equation}

dove due basi di $ V_{k}(\R^{n}) $ sono equivalenti se generano lo stesso $ k $-spazio. \\
Siccome il quoziente preserva anche la proprietà di compattezza, è sufficiente mostrare che la varietà di Stiefel sia compatta per dimostrare che lo sia la Grassmanniana. \\
Sappiamo che se un sottoinsieme di $ \R^{n} $ è chiuso e limitato allora è compatto: dato che una varietà di Stiefel $ V_{k}(\R^{n}) \subset \R^{n} $ è chiusa e limitata, allora è compatta e dunque lo è anche la Grassmanniana.
}

%=======================================================================================

\exer{Restrizione di funzione liscia su varietà}
{exer2-7}
{
Sia $ \S^{1} $ il cerchio unitario di $ \R^{2} $. Dimostrare che una funzione liscia $ f : \R^{2} \to \R $ si restringe a una funzione liscia $ \eval{f}_{\S^{1}} : \S^{1} \to \R $.
}
{
Dalla Sottosezione \ref{s-sec:smooth-app-on-man}, una funzione su una varietà è liscia se è liscia in ogni punto della varietà la sua composizione con l'inverso della carta che contiene il punto, i.e.

\begin{equation}
	\begin{cases}
		\eval{f}_{\S^{1}} : \S^{1} \to \R \\
		\eval{f}_{\S^{1}} \in C^{\infty}(\S^{1})
	\end{cases} %
	\iff %
	\begin{cases}
		\eval{f}_{\S^{1}} \circ \varphi_{i}^{-1} : U_{i} \to \R \\
		\eval{f}_{\S^{1}} \circ \varphi_{i}^{-1} \in C^{\infty}(U_{i}), & \forall i=1,\dots,4
	\end{cases}
\end{equation}

dove l'atlante $ \{(U_{i},\varphi_{i})\}_{i=1,\dots,4} $ è preso dall'Esempio \ref{example:diff-man-s1}. \\
Calcoliamo esplicitamente queste composizioni:

\sbs{0.5}{%
			\map{\eval{f}_{\S^{1}} \circ \varphi_{1}^{-1}}
				{(-1,1)_{x}}{\R}
				{x}{f \left( x,\sqrt{1-x^{2}} \right)}
			
			\map{\eval{f}_{\S^{1}} \circ \varphi_{2}^{-1}}
				{(-1,1)_{x}}{\R}
				{x}{f \left(x,-\sqrt{1-x^{2}} \right)}
			}
	{0.5}{%
			\map{\eval{f}_{\S^{1}} \circ \varphi_{3}^{-1}}
				{(-1,1)_{y}}{\R}
				{y}{f \left(\sqrt{1-y^{2}},y \right)}
			
			\map{\eval{f}_{\S^{1}} \circ \varphi_{4}^{-1}}
				{(-1,1)_{y}}{\R}
				{y}{f \left(-\sqrt{1-y^{2}},y \right)}
			}

Possiamo dunque vedere che sono tutte lisce in quanto composizioni di $ f \in C^{\infty}(\R^{2}) $ e applicazioni lisce, poiché le derivate degli argomenti di $ f $ nelle immagini delle composizioni hanno delle radici quadrate al denominatore che però non si annullano nei punti del dominio. A questo punto, la restrizione $ \eval{f}_{\S^{1}} $ è liscia in qualsiasi punto del cerchio $ \S^{1} $, perciò $ \eval{f}_{\S^{1}} \in C^{\infty}(\S^{1}) $.

% SOLUZIONE ALTERNATIVA (NON SICURA)

%Presa una carta $ (U,\varphi) $ della struttura differenziabile di $ \R^{2} $, perché $ f $ sia liscia è necessario che sia liscia la composizione
%
%\begin{equation}
%	f \circ \varphi^{-1} : \varphi(U) \to \R
%\end{equation}
%
%con $ U \subset \R^{2} $ aperto. \\
%Siccome la definizione non dipende dalla carta scelta e dal fatto che un diffeomorfismo
%
%\begin{equation}
%	g : U \to g(U) \subset \R^{2}
%\end{equation}
%
%definisce una carta della struttura differenziale $ (U,g) $\footnote{%
%	Vedi Proposizione \ref{prop:diffeo-map}.%
%}, possiamo prendere il diffeomorfismo
%
%\map{g}
%	{\R^{2}}{\S^{1}}
%	{(x,y)}{\left( \dfrac{x}{\sqrt{x^{2} + y^{2}}},\dfrac{y}{\sqrt{x^{2} + y^{2}}} \right)}
%
%e dunque la carta $ (\S^{1},g) $ della struttura differenziale di $ \R^{2} $ per riscrivere la definizione di funzione liscia, dunque
%
%\begin{equation}
%	f \circ g^{-1} : g(\R^{2}) = \S^{1} \to \R
%\end{equation}
%
%la quale può essere riscritta come
%
%\begin{equation}
%	\eval{f}_{\S^{1}} : \S^{1} \to \R
%\end{equation}
}

%=======================================================================================

\exer{Inclusione come funzione liscia tra varietà}
{exer2-6}
{
Siano $ M $ e $ N $ due varietà differenziabili e $ q_{0} \in N $. Dimostrare che

\map{i_{q_{0}}}
	{M}{M \times N}
	{p}{(p,q_{0})}

è un'applicazione liscia.
}
{
Prendiamo gli atlanti differenziabili $ \{(U,\varphi)\} \in M $ e $ \{(U \times V,\varphi \times \psi)\} \in M \times N $: perché $ i_{q_{0}} $ sia liscia questa deve essere continua (lo è in quanto inclusione) e la composizione

\map{(\varphi \times \psi) \circ i_{q_{0}} \circ \varphi^{-1}}
	{\varphi( i_{q_{0}}^{-1}(U \times V) \cap U )}{\R^{m+n}}
	{\varphi(p)}{(\varphi^{1}(p),\dots,\varphi^{m}(p),\psi^{1}(q_{0}),\dots,\psi^{n}(q_{0}))}

deve essere liscia. Per dimostrare che lo sia, dalla Proposizione \ref{prop:map-comp}, basta mostrare che le sue componenti siano lisce: presa la proiezione sulla $ k $-esima componente

\map{r^{k}}
	{\R^{n}}{\R}
	{(x^{1},\dots,x^{n})}{x^{k}}

le componenti della composizione sono date dalla seguente applicazione

\map{r^{k} \circ (\varphi \times \psi) \circ i_{q_{0}} \circ \varphi^{-1}}
	{\varphi( i_{q_{0}}^{-1}(U \times V) \cap U )}{\R}
	{\varphi(p)}{%
					\begin{cases}
						\varphi^{k}(p), & k \in [1,m] \\
						\psi^{k-m}(q_{0}), & k \in [m+1,m+n]
					\end{cases}%
					}

le quali sono tutte lisce, dunque anche l'inclusione è liscia tra varietà.
}

%=======================================================================================

\exer{Coefficienti campo di vettori}
{exer2-8}
{
Siano un punto $ p = (x,y) \in \R^{2} $ e un'applicazione

\map{F}
	{\R^{2}}{\R^{3}}
	{(x,y)}{(x,y,xy)}

Trovare $ a,b,c \in \R $ tali che:

\begin{equation}
	F_{*p} \left( \eval{ \pdv{x} }_{p} \right) = a \eval{ \pdv{u} }_{F(p)} + b \eval{ \pdv{v} }_{F(p)} + c \eval{ \pdv{w} }_{F(p)}
\end{equation}
}
{
Per trovare i coefficienti dell'immagine del differenziale ricordiamo che, per una funzione tra varietà differenziabili $ F : N \to M $ con carte

\begin{equation}
	\begin{cases}
		(U,\varphi) \in N, & \varphi = (x^{1},\dots,x^{n}) \\
		(V,\psi) \in M, & \psi = (y^{1},\dots,y^{m})
	\end{cases}
\end{equation}

vale

\begin{equation}
	F_{*p} \left( \eval{\pdv{x^{j}}}_{p} \right) = \sum_{k=1}^{m} \pdv{F^{k}}{x^{j}}\ (p) \left( \eval{\pdv{y^{k}}}_{F(p)} \right) \qcomma j = 1,\dots,n
\end{equation}

dove $ F^{k} = y^{k} \circ F $. \\
Essendo il differenziale un'applicazione lineare, considerando $ X_{p} \in T_{p}(N) $, possiamo scrivere

\begin{equation}
	F_{*p} (X_{p}) = J(F)(p) X_{p} = Y_{F(p)} \in T_{F(p)}(M)
\end{equation}

dove

\begin{equation}
	J(F)(p) \doteq \left[ \pdv{F^{k}}{x^{j}} \, (p) \right]
\end{equation}

Per la funzione considerata, abbiamo che

\begin{equation}
	J(F)(x,y) = \bmqty{ 1 & 0 \\ 0 & 1 \\ y & x }
\end{equation}

Siccome possiamo identificare $ T_{p}(\R^{k}) = \R^{k} $, è possibile scrivere

\begin{gather}
	X_{p} = \mu \eval{\pdv{x}}_{p} + \nu \eval{\pdv{y}}_{p} \equiv \bmqty{ \mu \\ \nu } \\
	Y_{F(p)} = a \eval{\pdv{u}}_{F(p)} + b \eval{\pdv{v}}_{F(p)} + c \eval{\pdv{w}}_{F(p)} \equiv \bmqty{ a \\ b \\ c }
\end{gather}

A questo punto

\begin{equation}
	F_{*p} \left( \bmqty{ \mu \\ \nu } \right) = \bmqty{ 1 & 0 \\ 0 & 1 \\ y & x } \bmqty{ \mu \\ \nu } %
	= \bmqty{ a \\ b \\ c } %
	= \bmqty{ \mu \\ \nu \\ y \mu + x \nu }
\end{equation}

perciò abbiamo che

\begin{equation}
	F_{*p} \left( \bmqty{ 1 \\ 0 } \right) = \bmqty{ 1 & 0 \\ 0 & 1 \\ y & x } \bmqty{ 1 \\ 0 } %
	= \bmqty{ 1 \\ 0 \\ y } %
	\equiv \eval{\pdv{u}}_{F(p)} + y \eval{\pdv{w}}_{F(p)}
\end{equation}

i.e. $ (a,b,c) = (1,0,y) $.
}

%=======================================================================================

\exer{Coefficienti cambio di base}
{exer2-9}
{
Siano $ x $ e $ y $ le coordinate standard su $ \R^{2} $ e $ U = \R^{2} \setminus \{(0,0)\} $. In $ U $ le coordinate polari $ (\rho, \theta) $ con $ \rho > 0 $ e $ \theta \in (0,2\pi) $ sono definite come

\begin{equation}
	\begin{cases}
		x = \rho \cos(\theta) \\
		y = \rho \sin(\theta)
	\end{cases}
\end{equation}

Si scrivano $ \pdv*{\rho} $ e $ \pdv*{\theta} $ in funzione di $ \pdv*{x} $ e $ \pdv*{y} $.
}
{
L'equazione che lega i vettori di una base dello spazio tangente

\begin{equation}
	T_{p}(\R^{2} \setminus \{(0,0)\}) = \R^{2} \setminus \{(0,0)\}
\end{equation}

a un'altra base è la seguente:

\begin{equation}
	\eval{\pdv{u^{j}}}_{p} = \sum_{k=1}^{2} \pdv{v^{k}}{u^{j}} \, (p) \left( \eval{\pdv{v^{k}}}_{p} \right)
\end{equation}

dove $ u = (\rho,\theta) $ e $ v = (x,y) $. \\
Siccome

\begin{equation}
	\begin{cases}
		x = \rho \cos(\theta) \\
		y = \rho \sin(\theta)
	\end{cases}%
	\implies %
	\begin{cases}
		\rho = \sqrt{x^{2} + y^{2}} \\
		\cos(\theta) = x / \rho \\
		\sin(\theta) = y / \rho
	\end{cases}
\end{equation}

Possiamo scrivere la matrice di trasformazione $ T $ come

\begin{equation}
	T \doteq \left[ \pdv{v^{k}}{u^{j}} \, (p) \right] = \bmqty{ \dpdv{x}{\rho} & \dpdv{x}{\theta} \\ \\ \dpdv{y}{\rho} & \dpdv{y}{\theta} } %
	= \bmqty{ \cos(\theta) & - \rho \sin(\theta) \\ \\ \sin(\theta) & \rho \cos(\theta) } %
	= \bmqty{ \dfrac{x}{\sqrt{x^{2} + y^{2}}} & - y \\ \\ \dfrac{y}{\sqrt{x^{2} + y^{2}}} & x }
\end{equation}

dunque

\begin{gather}
	\eval{\pdv{\rho}}_{p} = \dfrac{1}{\sqrt{x^{2} + y^{2}}} \left( x \eval{\pdv{x}}_{p} + y \eval{\pdv{y}}_{p} \right) %
	\quad \lor \quad %
	\rho \eval{\pdv{\rho}}_{p} = x \eval{\pdv{x}}_{p} + y \eval{\pdv{y}}_{p} \\
	\nonumber \\
	\eval{\pdv{\theta}}_{p} = - y \eval{\pdv{x}}_{p} + x \eval{\pdv{y}}_{p}
\end{gather}
}

%=======================================================================================

\exer{Vettore tangente a una curva}
{exer2-10}
{
Sia $ p = (x,y) $ un punto di $ \R^{2} $. Allora

\begin{equation}
	c_{p}(t) = \bmqty{ \cos(2t) & - \sin(2t) \\ \\ \sin(2t) & \cos(2t) } \bmqty{ x \\ y }
\end{equation}

è una curva liscia in $ \R^{2} $ che inizia in $ p $. Calcolare $ c'(0) $.
}
{
La curva considerata è la seguente:

\map{c}
	{[0, 2 \pi)}{\R^{2}}
	{t}{%
		\bmqty{ \cos(2t) & - \sin(2t) \\ \\ \sin(2t) & \cos(2t) } \bmqty{ x \\ y } %
		= \bmqty{ \cos(2t) x - \sin(2t) y \\ \\ \sin(2t) x + \cos(2t) y }%
		}

Dalla Proposizione \ref{prop:loc-exp-tan-cur} considerando $ \R^{2} $ come varietà immagine, abbiamo che il vettore tangente alla curva in un punto $ c(t_{0}) $ con $ t_{0} = 0 $ è dato da

\begin{equation}
	c'(0) = \sum_{i=1}^{n} \dot{c}_{i}(0) \eval{ \pdv{r^{i}} }_{c(0)} %
	= \dot{c}_{1}(0) \eval{ \pdv{x} }_{p} + \dot{c}_{2}(0) \eval{ \pdv{y} }_{p}
\end{equation}

in quanto la curva inizia in $ p $, i.e. $ c(0) = p $.

Calcoliamo dunque le componenti del vettore tangente:

\begin{align}
	\begin{split}
		\dot{c}(0) &= \eval{ \dv{t} }_{0} c(t) \\
		&= \eval{ \dv{t} }_{0} \left( \bmqty{ \cos(2t) x - \sin(2t) y \\ \\ \sin(2t) x + \cos(2t) y } \right) \\
		&= \eval{ \bmqty{ - 2 \sin(2t) x - 2 \cos(2t) y \\ \\ 2 \cos(2t) x - 2 \sin(2t) y } }_{0} \\
		&= \bmqty{ - 2 y \\ 2x }
	\end{split}
\end{align}

da cui

\begin{equation}
	c'(0) = - 2 y \eval{ \pdv{x} }_{0} + 2 x \eval{ \pdv{y} }_{0}
\end{equation}
}

%=======================================================================================

\exer{Isomorfismo prodotto spazi tangenti}
{exer2-11}
{
Siano $ N $ e $ M $ varietà differenziabili e $ \pi_{N} : N \times M \to N $ e $ \pi_{M} : N \times M \to M $ le proiezioni naturali. Dimostrare che per $ (p,q) \in N \times M $ l'applicazione

\begin{equation}
	(\pi_{N_{*p}},\pi_{M_{*q}}) : T_{(p,q)}(N \times M) \to T_{p}(N) \times T_{q}(M)
\end{equation}

è un isomorfismo.
}
{
% https://math.stackexchange.com/questions/413766/tangent-space-of-product-manifold

Al fine di dimostrare questo isomorfismo, definiamo $ f \doteq \left( (\pi_{N})_{*p},(\pi_{M})_{*p} \right) $ e scriviamo la sua azione su un vettore del dominio

\map{f}
	{T_{(p,q)}(N \times M)}{T_{p}(N) \times T_{q}(M)}
	{X_{(p,q)}}{\left( (\pi_{N})_{*(p,q)} (X_{(p,q)}), (\pi_{M})_{*(p,q)} (X_{(p,q)}) \right)}

Questa applicazione è lineare in quanto composizione di differenziali, i quali sono essi stessi lineari. \\
Consideriamo ora le inclusioni seguenti:

\sbs{0.5}{%
			\map{i_{N}}
				{N}{N \times M}
				{p}{(p,q)}
			}
	{0.5}{%
			\map{i_{M}}
				{M}{N \times M}
				{q}{(p,q)}
			}

le cui azioni sugli spazi sono

\begin{gather}
	i_{N}(N) = N \times \{q\} \\
	i_{M}(M) = \{p\} \times M
\end{gather}

Definiamo inoltre l'applicazione

\map{g}
	{T_{p}(N) \times T_{q}(M)}{T_{(p,q)}(N \times M)}
	{(X_{p}, Y_{q})}{(i_{N})_{*p} (X_{p}) + (i_{M})_{*q} (Y_{q})}

Consideriamo inoltre le seguenti composizioni:

\begin{gather}
	(\pi_{N} \circ i_{N}) (p) = \pi_{N} (i_{N} (p)) = \pi_{N} (p,q) = p \\
	(\pi_{M} \circ i_{M}) (q) = \pi_{M} (i_{M} (q)) = \pi_{M} (p,q) = q
\end{gather}

da cui

\begin{equation}
	\begin{cases}
		\pi_{N} \circ i_{N} = \id_{N} \\
		\pi_{M} \circ i_{M} = \id_{M}
	\end{cases}
\end{equation}

Il differenziale di queste composizioni risulta quindi nell'identità dei relativi spazi tangenti in quanto $ (\id_{N})_{*p} = \id_{T_{p}(N)} $. Le composizioni relative a spazi differenti, i.e. $ \pi_{N} \circ i_{M} $ e $ \pi_{M} \circ i_{N} $, sono delle mappe costanti quindi il loro differenziale è nullo. \\
Calcoliamo ora la seguente composizione delle funzioni $ f $ e $ g $:

\begin{align}
	\begin{split}
		&(f \circ g) (X_{p}, Y_{q}) = f ( (i_{N})_{*p} (X_{p}) + (i_{M})_{*q} (Y_{q}) ) \\
		&= ( (\pi_{N})_{*(p,q)} ( (i_{N})_{*p} (X_{p}) + (i_{M})_{*q} (Y_{q}) ), (\pi_{M})_{*(p,q)} ( (i_{N})_{*p} (X_{p}) + (i_{M})_{*q} (Y_{q}) ) ) \\
		&= ( (\pi_{N})_{*(p,q)} ( (i_{N})_{*p} (X_{p}) ) + (\pi_{N})_{*(p,q)} ( (i_{M})_{*q} (Y_{q}) ), (\pi_{M})_{*(p,q)} ( (i_{N})_{*p} (X_{p})) + (\pi_{M})_{*(p,q)} ( (i_{M})_{*q} (Y_{q}) ) ) \\
		&= \left( (\pi_{N} \circ i_{N})_{*p} (X_{p}) + \cancelto{0}{ (\pi_{N} \circ i_{M})_{*q} (Y_{q}) }, \cancelto{0}{ (\pi_{M} \circ i_{N})_{*p} (X_{p}) } + (\pi_{M} \circ i_{M})_{*q} (Y_{q}) \right) \\
		&=(X_{p}, Y_{q})
	\end{split}
\end{align}

dove i due $ 0 $ appartengono rispettivamente a $ T_{p}(N) $ e $ T_{q}(M) $, dunque

\begin{equation}
	f \circ g = \id_{T_{p}(N) \times T_{q}(M)}
\end{equation}

questo significa che la funzione $ f $ ha un'inversa destra dunque è suriettiva: la suriettività comporta che l'immagine della funzione coincida con il codominio, da cui

\begin{equation}
	\dim(\Im(f)) = \dim(T_{p}(N) \times T_{q}(M))
\end{equation}

Ricordando che la dimensione degli spazi tangenti considerati è la stessa, i.e.

\begin{equation}
	\dim(T_{(p,q)}(N \times M)) = \dim(T_{p}(N) \times T_{q}(M))
\end{equation}

possiamo invocare il teorema della dimensione\footnote{%
	Data un'applicazione lineare $ T : V \to W $, abbiamo che
	
	\begin{equation*}
		\dim(T(V)) + \dim(\ker(T)) = \dim(V)
	\end{equation*}
	
	dove $ \dim(T(V)) \doteq \rank(T) $ viene anche chiamato rango di $ T $.%
}

\begin{align}
	\begin{split}
		\dim(\Im(f)) + \dim(\ker(f)) &= \dim(T_{(p,q)}(N \times M)) \\
		\cancel{ \dim(T_{p}(N) \times T_{q}(M)) } + \dim(\ker(f)) &= \cancel{ \dim(T_{(p,q)}(N \times M)) } \\
		\dim(\ker(f)) &= 0
	\end{split}
\end{align}

perciò la funzione $ f = \left( (\pi_{N})_{*p},(\pi_{M})_{*p} \right) $ è anche iniettiva dunque è un isomorfismo, i.e.

\begin{equation}
	T_{(p,q)}(N \times M) \simeq T_{p}(N) \times T_{q}(M)
\end{equation}
}

%=======================================================================================

\exer{Prodotto di sottovarietà}
{exer2-12}
{
Siano $ S $ e $ P $ due sottovarietà di due varietà differenziabili $ N $ e $ M $ rispettivamente. Dimostrare che $ S \times P $ è una sottovarietà di $ N \times M $.
}
{
Poniamo le dimensioni delle varietà $ N $ e $ M $ pari a $ n $ e $ m $ rispettivamente. \\
Essendo $ S $ e $ P $ sottovarietà rispettivamente di $ N $ e $ M $, possiamo scrivere

\begin{gather}
	S \text{ sottovarietà di } N \nonumber \\
	\Updownarrow \\
	\forall s \in S, \, \E (U,\varphi) = (U; x^{1},\dots,x^{n}) \in N \mid (U,\varphi) \ni s \, \wedge \nonumber \\
	\wedge \, U \cap S = \{ r \in U \mid x^{k+1}(r) = \dots = x^{n}(r) = 0 \} \nonumber
\end{gather}
%
\begin{gather}
	P \text{ sottovarietà di } M \nonumber \\
	\Updownarrow \\
	\forall p \in P, \, \E (V,\psi) = (V; y^{1},\dots,y^{m}) \in M \mid (V,\psi) \ni p \, \wedge \nonumber \\
	\wedge \, V \cap P = \{ t \in V \mid y^{j+1}(t) = \dots = y^{m}(t) = 0 \} \nonumber
\end{gather}

dove quindi

\begin{equation}
	\begin{cases}
		\dim(S) = n - k \\
		\dim(P) = m - j
	\end{cases} %
	\iff %
	\begin{cases}
		\operatorname{cod}_{N}(S) = k \\
		\operatorname{cod}_{M}(P) = j
	\end{cases}
\end{equation}

Essendo $ N \times M $ ancora una varietà differenziabile, il prodotto delle carte considerate sopra è ancora una carta per la struttura differenziale di $ N \times M $, i.e.

\begin{equation}
	\begin{cases}
		\{(U_{\alpha},\varphi_{\alpha})\}_{\alpha \in A} \in N \\
		\{(V_{\beta},\psi_{\beta})\}_{\beta \in B} \in M
	\end{cases} %
	\implies %
	\{(U_{\alpha} \times V_{\beta}, \varphi_{\alpha} \times \psi_{\beta})\}_{\alpha \in A, \beta \in B} \in N \times M
\end{equation}

A questo punto, possiamo usare queste carte per definire una carta adattata di $ N \times M $ intorno a un qualunque punto $ (s,p) $ relativamente a $ S \times P $, rendendo perciò $ S \times P $ una sottovarietà di $ N \times M $; di seguito la condizione:

\begin{gather}
	\forall (s,p) \in S \times P, \, \E (U \times V, \varphi \times \psi) = (U \times V; x^{1},\dots,x^{n}, y^{1},\dots,y^{m}) \in N \times M \mid \nonumber \\
	\begin{cases}
		(U \times V, \varphi \times \psi) \ni (s,p) \\
		(U \times V) \cap (S \times P) = \{ (r,t) \in U \times V \mid x^{k+1}(r) = \dots = x^{n}(r) = y^{j+1}(t) = \dots = y^{m}(t) = 0 \}
	\end{cases}
\end{gather}

La dimensione e codimensione della sottovarietà prodotto sono dunque pari a

\begin{gather}
	\dim(S \times P) = (n - k) + (m - j) = (n + m) - (k + j) \\
	\operatorname{cod}_{N \times M}(S \times P) = k + j
\end{gather}
}

%=======================================================================================

\exer{Preimmagine di applicazione come sottovarietà}
{exer2-13}
{
Sia l'applicazione

\map{F}
	{\R^{2}}{\R}
	{(x,y)}{x^{2}-6xy+y^{2}}

Trovare i $ c \in \R $ tali che $ F^{-1}(c) $ sia una sottovarietà di $ \R^{2} $.
}
{
Tramite il teorema della preimmagine\footnote{%
	Vedi Teorema \ref{thm:preimg}.%
}, perché $ F^{-1}(c) $ sia una sottovarietà di $ \R^{2} $ è necessario che $ c \in \VR_{F} \cap \Im(F) $: questa condizione è equivalente a dire che almeno una delle derivate parziali di $ F $ non si annulli se calcolata nei punti che formano la controimmagine di $ c $ attraverso $ F $ (e dunque che tutti questi punti siano punti regolari per $ F $), i.e.

\begin{gather}
	c \in \VR_{F} \cap \Im(F) \nonumber \\
	\Updownarrow \nonumber \\
	\E \pdv{F}{x^{i}} \, (p) \neq 0 \qcomma \forall p \in F^{-1}(c), \, i=1,2 \\
	\Updownarrow \nonumber \\
	p \in \PR_{F} \qcomma \forall p \in F^{-1}(c) \nonumber
\end{gather}

dove $ (x^{1},x^{2}) = (x,y) $. \\
Siccome $ \PC_{F} \cap \PR_{F} = \emptyset $, escludendo i punti critici per $ F $ rimangono quelli regolari tra i quali individueremo la controimmagine di $ c $. Per trovare i punti critici, cerchiamo dunque i punti $ p = (x,y) $ che annullano contemporaneamente le derivate di $ F $:

\begin{equation}
	\begin{cases}
		\dpdv{F}{x} \, (x,y) = 2 x - 6 y = 0 \\ \\
		\dpdv{F}{y} \, (x,y) = - 6 x + 2 y = 0
	\end{cases} %
	\implies %
	(x,y) = (0,0)
\end{equation}

A questo punto possiamo derivare l'insieme dei punti regolari:

\begin{equation}
	\PC_{F} = \{(0,0)\} %
	\implies %
	\PR_{F} = \R^{2} \setminus \{(0,0)\}
\end{equation}

L'unico punto che ha come controimmagine il punto $ (0,0) $ è $ 0 $, i.e. $ F^{-1}(0) = (0,0) $ dunque $ \VR_{F} = \R \setminus \{0\} $: questo permette di scegliere per $ c $ un qualunque numero reale non nullo. \\
Il grafico di $ F $ è un paraboloide iperbolico e le controimmagini dei valori regolari di $ F $ sono intersezioni di questo con piani perpendicolari all'asse $ z $ e dunque iperboli.
}

%=======================================================================================

\exer{Sottovarietà tramite condizioni}
{exer2-14}
{
Dire se le soluzioni del sistema

\begin{equation}
	\begin{cases}
		x^{3} + y^{3} + z^{3} = 1 \\
		z = xy
	\end{cases}
\end{equation}

costituiscono una sottovarietà di $ \R^{3} $.
}
{
La soluzione di questo esercizio segue l'Esempio \ref{example:subvar-cond}. \\
Il sistema in esame può essere riscritto come l'insieme seguente

\begin{equation}
	S = \{ (x,y,z) \in \R^{3} \mid x^{3} + y^{3} + z^{3} = 1 \, \wedge \, z = xy \}
\end{equation}

Costruiamo ora l'applicazione
%
\map{F}
	{\R^{3}}{\R^{2}}
	{(x,y,z)}{(x^{3} + y^{3} + z^{3} - 1, z - xy)}

la quale porta all'equazione $ S = F^{-1}(0,0) $. \\
Perché $ S $ sia dunque una sottovarietà di $ \R^{3} $ verifichiamo che $ (0,0) \in \VR_{F} $: per fare ciò, dimostriamo che tutti i punti della preimmagine di $ (0,0) $ siano punti regolari, i.e. $ S \cap \PC_{F} = \emptyset $. \\
Siccome $ \PC_{F} \cap \PR_{F} = \emptyset $, cerchiamo i punti critici di $ F $ per escluderli dalla preimmagine: questi sono i punti per cui il differenziale $ F_{*(x,y,z)} $  non è suriettivo dunque, usando lo jacobiano, poniamo le condizioni per cui il rango di $ F $ sia minore del massimo (i.e. 2).

\begin{equation}
	J(F)(x,y,z) = \bmqty{ %
							3 x^{2} & 3 y^{2} & 3 z^{2} \\ \\
							- y 	& - x	  & 1 %
							}
\end{equation}

A questo punto, possiamo scrivere le condizioni:

\begin{equation}
	\rank(F_{*(x,y,z)}) < 2 %
	\implies %
	\begin{cases}
		- 3 x^{3} + 3 y^{3} = 0 \\
		3 x^{2} + 3 y z^{2} = 0 \\
		3 y^{2} + 3 x z^{2} = 0
	\end{cases}
\end{equation}

Consideriamo quindi l'intersezione tra i punti critici di $ F $ e l'insieme $ S $:

\begin{equation}
	\begin{cases}
		- 3 x^{3} + 3 y^{3} = 0 \\
		3 x^{2} + 3 y z^{2} = 0 \\
		3 y^{2} + 3 x z^{2} = 0 \\
		x^{3} + y^{3} + z^{3} = 1 \\
		z = xy
	\end{cases}
\end{equation}

Questo sistema non ha soluzioni\footnote{%
	La soluzione del sistema è stata trovata utilizzando un calcolatore online (\href{https://www.wolframalpha.com/input?i=solve+\%7B+-+3+x\%5E3+\%2B+3+y\%5E3+\%3D+0+\%2C+3+x\%5E2+\%2B+3+y+z\%5E2+\%3D+0+\%2C+3+y\%5E2+\%2B+3+x+z\%5E2+\%3D+0+\%2C+x\%5E3+\%2B+y\%5E3+\%2B+z\%5E3+\%3D+1+\%2C+z+\%3D+xy+\%7D}{link a WolframAlpha}).%
}, i.e. $ S \cap \PC_{F} = \emptyset $, dunque $ (0,0) \in \VR_{F} $ e $ S $ è una sottovarietà di $ \R^{3} $ con $ \dim(S) = 1 $ (in quanto le condizioni dell'insieme sono due).
}

%=======================================================================================

\exer{Spazio tangente a sottovarietà}
{BONUS2-3}
{
Sia la sottovarietà di $ \R^{3} $

\begin{equation}
	S = \{ (x,y,z) \in \R^{3} \mid x^{3} + y^{3} + z^{3} = 1 \, \wedge \, x + y + z = 0 \} \subset \R^{3}
\end{equation}

Calcolare lo spazio tangente $ T_{p}(S) $ con $ p \in S $.
}
{
Costruiamo l'applicazione

\map{F}
	{\R^{3}}{\R^{2}}
	{(x,y,z)}{(x^{3} + y^{3} + z^{3} - 1, x + y + z)}

tale che $ S = F^{-1}(0,0) $. \\
Tramite il teorema della preimmagine\footnote{%
	Vedi Teorema \ref{thm:preimg}.%
}, sappiamo che

\begin{equation}
	T_{p}(S) = \ker(F_{*p})
\end{equation}

Dunque calcoliamo quali sono i vettori che hanno immagine il vettore nullo tramite il differenziale

\begin{equation}
	F_{*p} : T_{p}(\R^{3}) \to T_{F(p)}(\R^{2})
\end{equation}

Consideriamo una curva liscia $ \gamma (t) = (x(t), y(t), z(t)) $ che rispetta le condizioni:

\begin{equation}
	\begin{cases}
		\gamma (- \varepsilon, \varepsilon) \to S \\
		\gamma (0) = p = (x(0), y(0), z(0)) \\
		\gamma' (0) = (\dot{x}(0), \dot{y}(0), \dot{z}(0)) = X_{p}
	\end{cases}
\end{equation}

Consideriamo inoltre le basi per $ T_{p}(\R^{3}) $ e $ T_{F(p)}(\R^{2}) $

\begin{gather}
	\B_{T_{p}(\R^{3})} = \left\{ \eval{ \pdv{x} }_{p}, \eval{ \pdv{y} }_{p}, \eval{ \pdv{z} }_{p} \right\} \\
	\B_{T_{F(p)}(\R^{2})} = \left\{ \eval{ \pdv{u} }_{F(p)}, \eval{ \pdv{v} }_{F(p)} \right\}
\end{gather}

A questo punto, calcoliamo l'immagine del differenziale:

\begin{align}
	\begin{split}
		F_{*p}(X_{p}) &= (F \circ \gamma)' (0) \\
		&= ((x(t))^{3} + (y(t))^{3} + (z(t))^{3} - 1, x(t) + y(t) + z(t))' (0) \\
		&= \dot{ ((x(t))^{3} + (y(t))^{3} + (z(t))^{3} - 1) } (0) \eval{ \pdv{u} }_{F(p)} + \dot{ (x(t) + y(t) + z(t)) } (0) \eval{ \pdv{v} }_{F(p)} \\
		&= (3 (x(0))^{2} \dot{x}(0) + 3 (y(0))^{2} \dot{y}(0) + 3 (z(0))^{2} \dot{z}(0)) \eval{ \pdv{u} }_{F(p)} + (\dot{x}(0) + \dot{y}(0) + \dot{z}(0)) \eval{ \pdv{v} }_{F(p)} \\
		&= ( 3 (x(0), y(0), z(0))^{2} \cdot (\dot{x}(0), \dot{y}(0), \dot{z}(0)) \eval{ \pdv{u} }_{F(p)} + ( (1,1,1) \cdot (\dot{x}(0), \dot{y}(0), \dot{z}(0)) ) \eval{ \pdv{v} }_{F(p)} \\
		&= 3 ( p^{2} \cdot X_{p} ) \eval{ \pdv{u} }_{F(p)} + ( (1,1,1) \cdot X_{p} ) \eval{ \pdv{v} }_{F(p)}
	\end{split}
\end{align}

Ponendo

\begin{equation}
	X_{p} = a \eval{ \pdv{x} }_{p} + b \eval{ \pdv{y} }_{p} + c \eval{ \pdv{z} }_{p} %
	\equiv (a,b,c)
\end{equation}

possiamo scrivere

\begin{equation}
	F_{*p}(a,b,c) = ( 3 (a x^{2} + b y^{2} + c z^{2}), a + b + c ) \in T_{F(p)}(\R^{2})
\end{equation}

da cui lo spazio tangente

\begin{equation}
	T_{p}(S) = \left\{ (a,b,c) \in T_{p}(\R^{3}) \st p = (x,y,z) \, \wedge \, %
	\begin{cases}
		a x^{2} + b y^{2} + c z^{2} = 0 \\
		a + b + c = 0
	\end{cases} %
	\right\}
\end{equation}
}

%=======================================================================================

\exer{Sottovarietà e spazio dei polinomi omogenei}
{exer2-15}
{
Un polinomio $ F(x_{1},\dots,x_{n}) \in \R[x_{1},\dots,x_{n}] $ è omogeneo di grado $ k $ se è combinazione lineare di monomi $ x_{1}^{i_{1}} \cdots x_{n}^{i_{m}} $ di grado $ k $ tale che $ \sum_{j=1}^{m} i_{j} = k $. Dimostrare che

\begin{equation}
	\sum_{i=1}^{n} x_{i} \, \pdv{F}{x_{i}} = k F
\end{equation}

Dedurre che $ F^{-1}(c) $ con $ c \neq 0 $ è una sottovarietà di $ \R^{n} $ di dimensione $ n-1 $. Dimostrare inoltre che per $ c,d>0 $ si ha che $ F^{-1}(c) \stackrel{diff}{\simeq} F^{-1}(d) $ e lo stesso vale per $ c,d<0 $. \\
\textit{Suggerimento per la prima parte: usare l'uguaglianza}

\begin{equation}
	F(\lambda x_{1},\dots,\lambda x_{n}) = \lambda^{k} F(x_{1},\dots,x_{n}) %
	\qcomma \forall \lambda \in \R
\end{equation}
}
{
\paragraph{Dimostrazione formula}

Scriviamo un polinomio omogeneo di grado $ k $ nel seguente modo

\begin{equation}
	F(x_{1},\dots,x_{n}) = \sum_{j=1}^{p} b_{j} \prod_{i=1}^{n} (x_{i})^{a_{ij}} %
	\qquad %
	\begin{cases}
		p \in \N \\
		b = [b_{j}] \in \R^{p} \\
		A = [a_{ij}] \in M_{n,p}(\R) \\
		\displaystyle\sum_{i=1}^{m} a_{ij} = k, & \forall j \in [0,p]
	\end{cases}
\end{equation}

Usando l'uguaglianza

\begin{equation}
	F(\lambda x_{1},\dots,\lambda x_{n}) = \lambda^{k} F(x_{1},\dots,x_{n}) %
	\qcomma \forall \lambda \in \R
\end{equation}

possiamo scrivere

\begin{align}
	\begin{split}
		\sum_{q=1}^{n} x_{q} \, \pdv{F}{x_{q}} &= \sum_{q=1}^{n} x_{q} \, \pdv{x_{q}} \left( \sum_{j=1}^{p} b_{j} \prod_{i=1}^{n} (x_{i})^{a_{ij}} \right) \\
		&= \sum_{q=1}^{n} \sum_{j=1}^{p} b_{j} x_{q} \, \pdv{x_{q}} \prod_{i=1}^{n} (x_{i})^{a_{ij}} \\
		&= \sum_{q=1}^{n} \sum_{j=1}^{p} b_{j} x_{q} a_{qj} \left( (x_{1})^{a_{1j}} \cdots (x_{q})^{a_{qj} - 1} \cdots (x_{n})^{a_{nj}} \right) \\
		&= \sum_{q=1}^{n} \sum_{j=1}^{p} b_{j} a_{qj} \prod_{i=1}^{n} (x_{i})^{a_{ij}} \\
		&= \left( \sum_{q=1}^{n} a_{qj} \right) \left( \sum_{j=1}^{p} b_{j} \prod_{i=1}^{n} (x_{i})^{a_{ij}} \right) \\
		&= k F
	\end{split}
\end{align}

\paragraph{Sottovarietà}

Dal Teorema \ref{thm:preimg}, perché $ F^{-1}(c) $ con $ c \neq 0 $ sia una sottovarietà di $ \R^{n} $, è necessario che l'applicazione $ F $ sia liscia e che $ c \in \VR_{F} \cap \Im(F) $: l'applicazione

\map{F}
	{\R^{n}}{\R}
	{(x_{1},\dots,x_{n})}{F(x_{1},\dots,x_{n}) \in \R[x_{1},\dots,x_{n}]}

dove $ \R[x_{1},\dots,x_{n}] $ indica lo spazio dei polinomi in $ n $ variabili con coefficienti reali, è liscia in quanto l'immagine è un polinomio; per la seconda condizione, consideriamo le seguenti condizioni equivalenti

\begin{gather}
	c \in \VR_{F} \nonumber \\
	\Updownarrow \nonumber \\
	p \in \PR_{F} \qcomma \forall p \in \R^{n} \\
	\Updownarrow \nonumber \\
	\E \pdv{F}{x^{i}} \, (p) \neq 0 \qcomma i=1,\dots,n \nonumber
\end{gather}

Siccome vale la relazione

\begin{equation}
	\sum_{i=1}^{n} x_{i} \, \pdv{F}{x_{i}} = k F
\end{equation}

esisterà una derivata di $ F $ non nulla per qualunque punto che sia controimmagine di un polinomio non nullo: considerando le condizioni

\begin{equation}
	\begin{cases}
		c \notin \R[0,\dots,0] \\
		k \neq 0
	\end{cases} %
	\implies %
	k F(p) \neq 0 \qcomma \forall p \in F^{-1}(c)
\end{equation}

dunque

\begin{equation}
	\begin{cases}
		\displaystyle \sum_{i=1}^{n} x_{i} \, \pdv{F}{x_{i}} \, (p) \neq 0 \\ \\
		p = (x^{1},\dots,x^{n}) \neq (0,\dots,0)
	\end{cases} %
	\implies %
	\E \pdv{F}{x^{i}} \, (p) \neq 0 \qcomma i=1,\dots,n
\end{equation}

e perciò $ F^{-1}(c) $ è una sottovarietà di $ \R^{n} $ di dimensione

\begin{equation}
	\dim(F^{-1}(c)) = \dim(\R^{n}) - \dim(\R) = n - 1
\end{equation}

\paragraph{Diffeomorfismi}

Considerando ancora una volta l'uguaglianza

\begin{equation}
	F(\lambda x_{1},\dots,\lambda x_{n}) = \lambda^{k} F(x_{1},\dots,x_{n}) %
	\qcomma \forall \lambda \in \R
\end{equation}

e scegliendo $ \lambda $ in modo tale da avere

\begin{equation}
	c = \lambda^{k} d \qcomma c,d \in \R^{+} \lor c,d \in \R^{-}
\end{equation}

possiamo definire l'applicazione

\map{\varphi}
	{F^{-1}(c)}{F^{-1}(d)}
	{x = (x_{1},\dots,x_{n})}{\lambda x}

L'applicazione $ \varphi $ è liscia, invertibile e con inversa liscia, dunque è un diffeomorfismo, i.e. $ F^{-1}(c) \stackrel{diff}{\simeq} F^{-1}(d) $.
}

%=======================================================================================

\exer{Gruppo lineare speciale complesso come sottovarietà}
{exer2-16}
{
Dimostrare che

\begin{equation}
	SL_{n}(\C) = \{ A \in M_{n}(\C) \mid \det(A) = 1 \} \subset M_{n}(\C)
\end{equation}

è una sottovarietà di $ M_{n}(\C) $ con $ \dim(SL_{n}(\C)) = 2 n^{2} - 2 $.
}
{
Il procedimento è analogo a quello presentato nell'Esempio \ref{example:sl-subman} con le differenze introdotte dal campo dei numeri complessi. \\
In particolare, la funzione usata per considerare la preimmagine del valore $ 1 = 1 + 0 i \in \C $ è la seguente

\map{f}
	{GL_{n}(\C)}{\C}
	{A}{\det(A) = \sum_{i=1}^{n} (-1)^{i+j} \, a_{ij} \, m_{ij}}

dove $ A = (a_{ij}) $ con $ i,j=1,\dots,n $, gli $ m_{ij} = \det(A_{ij}) $ sono i minori di $ A $ e le $ A_{ij} $ sono le sottomatrici ricavate da $ A $ rimuovendo la $ i $-esima riga e la $ j $-esima colonna. \\
Siccome la dimensione del gruppo lineare speciale in campo complesso è doppia rispetto al caso reale e dato che il valore $ 1 $ in $ f^{-1}(1) = SL_{n}(\C) $ è complesso e quindi di dimensione 2 in campo reale (i.e. è come se rappresentasse due condizioni in campo reale), otteniamo che

\begin{gather}
	\begin{cases}
		\dim(GL_{n}(\C)) = 2 \dim(GL_{n}(\R)) = 2 n^{2} \\
		\dim(1) = \dim(1 + 0 i) = 2
	\end{cases} \nonumber \\
	\Downarrow \\ %
	\dim(SL_{n}(\C)) = 2 \dim(SL_{n}(\R)) = 2(n^{2} - 1) = 2 n^{2} - 2 \nonumber
\end{gather}
}

%=======================================================================================

\exer{Punti regolari come aperto del dominio}
{exer2-17}
{
Sia $ F : N \to M $ un'applicazione liscia tra varietà differenziabili. Dimostrare che l'insieme $ \PR_{F} $ dei punti regolari di $ F $ è un aperto di $ N $.
}
{
% https://math.stackexchange.com/questions/2324431/the-set-of-all-regular-points-of-a-smooth-map-is-open

Siano le dimensioni delle varietà

\begin{equation}
	\begin{cases}
		\dim(N) = n \\
		\dim(M) = m
	\end{cases}
\end{equation}

Se un punto $ p \in N $ è un punto regolare per la funzione $ F $, le seguenti affermazioni sono equivalenti

\begin{itemize}
	\item $ F $ è una sommersione in $ p $
	
	\item $ F_{*p} $ è suriettiva
	
	\item  $ \rank(F_{*p}) = m = \min \{n,m\} $
\end{itemize}

Consideriamo le seguenti carte e condizioni:

\begin{equation}
	\begin{cases}
		(U,\varphi) \in N, & (U,\varphi) \ni p \\
		(V,\psi) \in M, & (V,\psi) \ni F(p) \\
		F(U) \subseteq V
	\end{cases}
\end{equation}

Consideriamo inoltre l'applicazione continua\footnote{%
	Siano due spazi vettoriali topologici $ V $ e $ W $, una funzione lineare $ L : V \to W $ è continua se $ V $ è finito dimensionale e di Hausdorff; in questo caso, $ G $ è lineare perché un differenziale, e dominio e codominio rispettano le condizioni in quanto varietà differenziabili.%
}

\map{G}
	{\varphi(U)}{M_{n \times m}(\R)}
	{x}{(\psi \circ F \circ \varphi^{-1})_{*x}}

Siccome

\begin{equation}
	G(x) = (\psi \circ F \circ \varphi^{-1})_{*x} %
	= \psi_{F(\varphi^{-1}(x))} \circ F_{*\varphi^{-1}(x)} \circ (\varphi^{-1})_{*x}
\end{equation}

se prendiamo $ x = \varphi(p) $ con $ p \in \PR_{F} $, abbiamo che

\begin{equation}
	G(\varphi(p)) = \psi_{F(p)} \circ F_{*p} \circ (\varphi^{-1})_{*\varphi(p)}
\end{equation}

dove i differenziali $ \psi_{F(p)} $ e $ (\varphi^{-1})_{*\varphi(p)} $ sono isomorfismi perché le applicazioni delle carte sono diffeomorfismi e $ F_{*p} $ è suriettiva perché $ p $ è un punto regolare: queste condizioni implicano che $ G $ sia suriettiva in $ \varphi(p) $ e dunque che abbia rango massimo, i.e.

\begin{equation}
	\rank(G(\varphi(p))) = k
\end{equation}

Questo implica anche che esiste una sottomatrice $ g(\varphi(p)) $ di dimensione $ k \times k $ di $ G(\varphi(p)) $ che abbia determinante diverso non nullo, i.e.

\begin{equation}
	\E g(\varphi(p)) \in M_{k \times k}(\R) \mid \det(g(\varphi(p))) \neq 0
\end{equation}

Consideriamo ora l'applicazione continua $ H $ e lo schema riassuntivo delle applicazioni utilizzate:

\sbs{0.4}{%
			\map{H}
				{M_{k \times k}(\R)}{\R}
				{A}{\det(A)}
			}
	{0.6}{%
			\diagr{%
					N \arrow[dd, "\varphi"] \arrow[rr, "F"]                               \&  \& M \arrow[dd, "\psi"] \\
					\&  \&                      \\
					\R^{n} \arrow[dd, "G"] \arrow[rr, "\psi \circ F \circ \varphi^{-1}"'] \&  \& \R^{m}               \\
					\&  \&                      \\
					M_{n \times m}(\R) \supset M_{k \times k}(\R) \arrow[rr, "H"]      \&  \& \R                  
					}
			}

A questo punto, considerando l'insieme

\begin{equation}
	(H \circ G \circ \varphi)^{-1} (\R \setminus \{0\}) \subset N
\end{equation}

questo è un intorno aperto di $ p $ in cui $ G(\varphi(p)) $ è suriettiva: prendendo l'unione degli intorni aperti di tutti i punti regolari, otteniamo un aperto che coincide con l'insieme dei punti regolari di $ F $, i.e.

\begin{equation}
	\bigcup_{p \in \PR_{F}} (H \circ G \circ \varphi)^{-1} (\R \setminus \{0\}) = \PR_{F}
\end{equation}
}

%=======================================================================================

\exer{Valori regolari come aperto del codominio}
{exer2-18}
{
Sia $ F : N \to M $ un'applicazione liscia tra varietà differenziabili. Dimostrare che se $ F $ è chiusa allora l'insieme $ \mathcal{VR}_{F} $ dei valori regolari di $ F $ è un aperto in $ M $.
}
{
Tramite l'Esercizio \ref{exer2-17}, sappiamo che l'insieme dei punti regolari di $ F $ è aperto in $ N $: questo implica che l'insieme dei punti critici sia chiuso, i.e.

\begin{equation}
	\begin{cases}
		\PR_{F} \text{ aperto} \\
		N \setminus \PR_{F} = \PC_{F}
	\end{cases} %
	\implies %
	\PC_{F} \text{ chiuso}
\end{equation}

Siccome vale la relazione

\begin{equation}
	F(\PC_{F}) = \VC_{F}
\end{equation}

se supponiamo che l'applicazione $ F $ sia chiusa (porta chiusi in chiusi), l'immagine del chiuso $ \PC_{F} $ sarà un chiuso, i.e. $ \VC_{F} \subset M $ è chiuso. \\
Sappiamo ora che l'insieme dei valori regolari è il complementare dell'insieme dei valori critici, dunque

\begin{equation}
	\begin{cases}
		\VC_{F} \text{ chiuso} \\
		M \setminus \VC_{F} = \VR_{F}
	\end{cases} %
	\implies %
	\VR_{F} \text{ aperto}
\end{equation}
}

%=======================================================================================

\exer{Embedding liscio (1)}
{exer2-19}
{
Dimostrare che l'applicazione

\map{F}
	{\R}{\R^{3}}
	{t}{(t,t^{2},t^{3})}

è un embedding liscio e scrivere $ F(\R) $ come zero di funzioni.
}
{
Perché l'applicazione $ F $ sia un embedding liscio è necessario dimostrare che sia un embedding topologico e un'immersione.

\paragraph{Embedding topologico}

Consideriamo l'applicazione indotta da $ F $

\map{f}
	{\R}{F(\R)}
	{t}{(t,t^{2},t^{3})}

dove

\begin{equation}
	F(\R) = \{ (a,b,c) \in \R^{3} \mid a \in \R , \, b = a^{2} \, \wedge \, c = a^{3} \} \subset \R^{3}
\end{equation}

Perché $ F $ sia un embedding topologico è necessario che $ f $ sia un omeomorfismo, i.e. continua, invertibile e con inversa continua: è continua in quanto prodotto diretto di polinomi; l'inversa è data dall'applicazione

\map{f^{-1}}
	{F(\R)}{\R}
	{(a,b,c)}{a}

in quanto

\begin{gather}
	(f^{-1} \circ f) (t) = f^{-1}(t,t^{2},t^{3}) = t \\
	(f \circ f^{-1}) (a,b,c) = f(a) = (a,a^{2},a^{3}) \equiv (a,b,c)
\end{gather}

dove nella dimostrazione per l'inversa destra vale la seguente implicazione

\begin{equation}
	(a,b,c) \in F(\R) %
	\implies %
	b = a^{2} \, \wedge \, c = a^{3} %
	\implies %
	(a,a^{2},a^{3}) \equiv (a,b,c)
\end{equation}

Anche l'inversa è continua (proiezione), dunque $ f $ è un omeomorfismo e $ F $ un embedding topologico.

\paragraph{Immersione}

Per dimostrare che $ F $ sia un'immersione, consideriamo la derivata rispetto a $ t $ di $ F $

\begin{equation}
	\dot{F}(t) = (1, 2 t, 3 t^{2}) \neq (0,0,0) \qcomma \forall t \in \R
\end{equation}

Non annullandosi mai la derivata, non esistono punti critici per l'applicaizone:

\begin{equation}
	\PC_{F} = \emptyset \implies \PR_{F} = \R
\end{equation}

dunque il differenziale dell'applicazione è suriettivo in ogni punto del dominio $ T_{\R} $, i.e. l'applicazione è un'immersione.

\paragraph{Embedding liscio}

Essendo un embedding topologico e un'immersione, l'applicazione $ F $ è un embedding liscio: questo porta a due conclusioni riguardanti l'immagine dell'applicazione

\begin{gather}
	F(\R) \stackrel{diff}{\simeq} \R \\
	F(\R) \text{ sottovarietà di } \R^{3} \qcomma \operatorname{cod}_{\R^{3}} (F(\R)) = 2
\end{gather}

\paragraph{Zero di funzione}

Possiamo riscrivere l'insieme $ F(\R) $ come controimmagine dell'origine di $ \R^{2} $ tramite l'applicazione

\map{G}
	{\R^{3}}{\R^{2}}
	{(a,b,c)}{(b - a^{2}, c - a^{3})}

i.e. $ F(\R) = G^{-1}(0,0) $.
}

%=======================================================================================

\exer{Embedding liscio (2)}
{exer2-20}
{
Dimostrare che l'applicazione

\map{F}
	{\R}{\R^{2}}
	{t}{(\cosh(t),\sinh(t))}

è un embedding liscio e che

\begin{equation}
	F(\R) = \{ (x,y) \in \R^{2} \mid x^{2}-y^{2} = 1 \}
\end{equation}
}
{
Perché l'applicazione $ F $ sia un embedding liscio è necessario dimostrare che sia un embedding topologico e un'immersione.

\paragraph{Embedding topologico}

Consideriamo la forma esponenziale delle funzioni iperboliche \\

\sbs{0.5}{%
			\begin{equation}
				\cosh(t) = \dfrac{e^{t} + e^{-t}}{2}
			\end{equation}
			}
	{0.5}{%
			\begin{equation}
				\sinh(t) = \dfrac{e^{t} - e^{-t}}{2}
			\end{equation}
			}

Tramite queste, scriviamo l'applicazione indotta da $ F $ come

\map{f}
	{\R}{F(\R)}
	{t}{\dfrac{1}{2} (e^{t} + e^{-t},e^{t} - e^{-t})}

dove

\begin{equation}
	F(\R) = \{ (x,y) \in \R^{2} \mid x^{2} - y^{2} = 1 \} \subset \R^{2}
\end{equation}

in quanto

\begin{equation}
	\begin{cases}
		x = \cosh(t) \\
		y = \sinh(t)
	\end{cases} %
	\implies %
	\begin{cases}
		2 x = e^{t} + e^{-t} \\
		2 y = e^{t} - e^{-t}
	\end{cases} %
	\implies %
	\begin{cases}
		e^{t} = 2 x - e^{-t} \\
		e^{t} = 2 y + e^{-t}
	\end{cases} %
	\implies %
	x - y = e^{-t}
\end{equation}

\begin{equation}
	\begin{cases}
		x = \cosh(t) \\
		y = \sinh(t)
	\end{cases} %
	\implies %
	\begin{cases}
		2 x = e^{t} + e^{-t} \\
		2 y = e^{t} - e^{-t}
	\end{cases} %
	\implies %
	\begin{cases}
		e^{-t} = 2 x - e^{t} \\
		e^{-t} = e^{t} - 2 y
	\end{cases} %
	\implies %
	x + y = e^{t}
\end{equation}

da cui

\begin{align}
	\begin{split}
		(x - y)(x + y) &= e^{-t} e^{t} \\
		x^{2} - y^{2} &= 1
	\end{split}
\end{align}

Perché $ F $ sia un embedding topologico è necessario che $ f $ sia un omeomorfismo, i.e. continua, invertibile e con inversa continua: è continua in quanto prodotto diretto di somme di esponenziali; l'inversa è data dall'applicazione

\map{f^{-1}}
	{F(\R)}{\R}
	{(x,y)}{\operatorname{arccosh}(x) = \ln(x + \sqrt{x^{2} - 1})}

in quanto

\begin{gather}
	(f^{-1} \circ f) (t) = f^{-1}(\cosh(t),\sinh(t)) = \operatorname{arccosh}(\cosh(t)) = t \\
	(f \circ f^{-1}) (x,y) = f \left( \ln(x + \sqrt{x^{2} - 1}) \right) = \left( x, x - \dfrac{1}{x + \sqrt{x^{2} - 1}} \right) \equiv (x,y)
\end{gather}

dove nella dimostrazione per l'inversa destra vale la seguente implicazione

\begin{equation}
	(x,y) \in F(\R) %
	\implies %
	x^{2} - y^{2} = 1 %
	\implies %
	\left( a, a - \dfrac{1}{a + \sqrt{a^{2} - 1}} \right) \equiv (a,b)
\end{equation}

poiché

\begin{equation}
	a^{2} - \left( a - \dfrac{1}{a + \sqrt{a^{2} - 1}} \right)^{2} = 1
\end{equation}

Anche l'inversa è continua (composizione di proiezione e funzioni continue), dunque $ f $ è un omeomorfismo e $ F $ un embedding topologico.

\paragraph{Immersione}

Per dimostrare che $ F $ sia un'immersione, consideriamo la derivata rispetto a $ t $ di $ F $

\begin{equation}
	\dot{F}(t) = (\sinh(t),\cosh(t)) \neq (0,0) \qcomma \forall t \in \R
\end{equation}

in quanto

\begin{equation}
	\cosh(t) \geqslant 1 \qcomma \forall t \in \R
\end{equation}

Non annullandosi mai la derivata, non esistono punti critici per l'applicazione:

\begin{equation}
	\PC_{F} = \emptyset \implies \PR_{F} = \R
\end{equation}

dunque il differenziale dell'applicazione è suriettivo in ogni punto del dominio $ T_{\R} $, i.e. l'applicazione è un'immersione.

\paragraph{Embedding liscio}

Essendo un embedding topologico e un'immersione, l'applicazione $ F $ è un embedding liscio: questo porta a due conclusioni riguardanti l'immagine dell'applicazione

\begin{gather}
	F(\R) \stackrel{diff}{\simeq} \R \\
	F(\R) \text{ sottovarietà di } \R^{2} \qcomma \operatorname{cod}_{\R^{2}} (F(\R)) = 1
\end{gather}
}

%=======================================================================================

\exer{Composizione e prodotto cartesiano di immersioni}
{exer2-21}
{
Dimostrare che la composizione di immersioni è un’immersione e che il prodotto cartesiano di due immersioni è un’immersione.
}
{
Un'applicazione è definita immersione se il suo differenziale è iniettivo in ogni punto del suo dominio. i.e.

\begin{equation}
	F : N \to M \text{ immersione} %
	\implies %
	F_{*p} : T_{p}(N) \to T_{F(p)}(M) \text{ iniettivo} \qcomma \forall p \in N
\end{equation}

Useremo le due seguenti proposizioni per esprimere l'iniettività:

\begin{gather}
	\E (F_{*p})^{-1} \mid (F_{*p})^{-1} \circ F_{*p} = \bigone_{T_{p}(N)} \\
	F_{*p}(X_{p}) = F_{*p}(Y_{p}) \implies X_{p} = Y_{p}
\end{gather}

le quali valgono per qualsiasi $ p \in N $.

\paragraph{Composizione}

Siano le immersioni

\begin{gather}
	F : N \to M \\
	G : M \to P
\end{gather}

I loro differenziali sono iniettivi

\begin{gather}
	F_{*p} : T_{p}(N) \to T_{F(p)}(M) \\
	G_{*q} : T_{q}(M) \to T_{G(q)}(P)
\end{gather}

rispettivamente per qualsiasi $ p \in N $ e $ q \in M $. \\
Considerando la prima proposizione per esprimere l'iniettività, abbiamo che

\begin{gather}
	\E (F_{*p})^{-1} \mid (F_{*p})^{-1} \circ F_{*p} = \bigone_{T_{p}(N)} \\
	\E (G_{*q})^{-1} \mid (G_{*q})^{-1} \circ G_{*q} = \bigone_{T_{q}(M)}
\end{gather}

Sia la composizione

\begin{equation}
	(G \circ F)_{*p} = G_{*F(p)} \circ F_{*p} : T_{p}(N) \to T_{G(F(p))}(P)
\end{equation}

e consideriamo la seguente espressione:

\begin{align}
	\begin{split}
		((G \circ F)_{*p})^{-1} \circ (G \circ F)_{*p} &= (F_{*p})^{-1} \circ (G_{*F(p)})^{-1} \circ G_{*F(p)} \circ F_{*p} \\
		&= (F_{*p})^{-1} \circ \bigone_{T_{F(p)}(M)} \circ F_{*p} \\
		&= (F_{*p})^{-1} \circ F_{*p} \\
		&= \bigone_{T_{p}(N)}
	\end{split}
\end{align}

Questo significa che esiste un'applicazione che sia l'inversa sinistra della composizione $ (G \circ F)_{*p} $, i.e.

\begin{equation}
	\E ((G \circ F)_{*p})^{-1} \mid ((G \circ F)_{*p})^{-1} \circ (G \circ F)_{*p} = \bigone_{T_{p}(N)}
\end{equation}

da cui otteniamo che il differenziale $ (G \circ F)_{*p} $ è iniettivo per qualsiasi $ p \in N $ e dunque che $ G \circ F $ è un'immersione.

\paragraph{Prodotto cartesiano}

Siano le immersioni

\begin{gather}
	F : N_{1} \to M_{1} \\
	G : N_{2} \to M_{2}
\end{gather}

I loro differenziali sono iniettivi

\begin{gather}
	F_{*p} : T_{p}(N_{1}) \to T_{F(p)}(M_{1}) \\
	G_{*q} : T_{q}(N_{2}) \to T_{G(q)}(M_{2})
\end{gather}

rispettivamente per qualsiasi $ p \in N_{1} $ e $ q \in N_{2} $. \\
Considerando la seconda proposizione per esprimere l'iniettività, abbiamo che

\begin{gather}
	F_{*p}(X_{p}) = F_{*p}(W_{p}) \implies X_{p} = W_{p} \qcomma \forall p \in N_{1} \\
	G_{*q}(Y_{q}) = G_{*q}(Z_{q}) \implies Y_{q} = Z_{q} \qcomma \forall q \in N_{2}
\end{gather}

Sia il prodotto cartesiano

\map{F \times G}
	{N_{1} \times N_{2}}{M_{1} \times M_{2}}
	{(p,q)}{(F(p), G(q))}

e il suo differenziale in $ (p,q) \in N_{1} \times N_{2} $

\map{(F \times G)_{*(p,q)} = F_{*p} \times G_{*q}}
	{T_{p}(N_{1}) \times T_{q}(N_{2})}{T_{F(p)}(M_{1}) \times T_{G(q)}(M_{2})}
	{(X_{p}, Y_{q})}{(F_{*p}(X_{p}), G_{*q}(Y_{q}))}

Vale l'implicazione

\begin{equation}
	(F_{*p}(X_{p}), G_{*q}(Y_{q})) = (F_{*p}(W_{p}), G_{*q}(Z_{q})) %
	\implies %
	(X_{p}, Y_{q}) = (W_{p}, Z_{q}) %
	\qcomma \forall (p,q) \in N_{1} \times N_{2}
\end{equation}

perciò il differenziale $ (F \times G)_{*(p,q)} $ è iniettivo per qualsiasi $ (p,q) \in N_{1} \times N_{2} $ e dunque $ F \times G $ è un'immersione.
}

%=======================================================================================

\exer{Dominio di immersione ristretto a sottovarietà}
{exer2-22}
{
Dimostrare che se $ F : N \to M $ è un'immersione e $ S \subset N $ è una sottovarietà di $ N $ allora $ \eval{F}_{S} : S \to M $ è un'immersione.
}
{
Essendo $ F : N \to M $ un'immersione, il suo differenziale

\begin{equation}
	F_{*p} : T_{p}(N) \to T_{F(p)}(M)
\end{equation}

è iniettivo per qualsiasi $ p \in N $. \\
Per la Proposizione \ref{prop:subman-incl-immersion}, abbiamo che

\begin{equation}
	S \subset N \text{ sottovarietà} %
	\implies %
	i : S \to N \text{ immersione}
\end{equation}

Consideriamo dunque l'uguaglianza

\begin{equation}
	\eval{F}_{S} = F \circ i
\end{equation}

Nell'Esercizio \ref{exer2-21} abbiamo dimostrato che la composizione di immersioni è ancora un'immersione, dunque $ \eval{F}_{S} $ è un'immersione.
}

%=======================================================================================

\exer{Embedding liscio (proiettivo reale)}
{exer2-23}
{
Dimostrare che l'applicazione

\map{F}
	{\S^{2}}{\R^{4}}
	{(x,y,z)}{(x^{2}-y^{2},xy,xz,yz)}

induce un embedding liscio da $ \rp{2} $ a $ \R^{4} $.
}
{
Dall'applicazione vediamo che

\begin{equation}
	F(-x,-y,-z) = F(x,y,z)
\end{equation}

Per questo motivo, possiamo considerare la restrizione del dominio dell'applicazione alla sfera quozientata all'equivalenza antipodale\footnote{%
	Vedi Paragrafo \ref{ss-sec:homeo-rp-qsph}.%
}, i.e. l'applicazione

\map{G}
	{\sfrac{\S^{2}}{\sim_{a}}}{\R^{4}}
	{[(x,y,z)]_{a}}{(x^{2}-y^{2},xy,xz,yz)}

ottenuta tramite la composizione con la proiezione

\map{\pi_{a}}
	{\S^{2}}{\sfrac{\S^{2}}{\sim_{a}}}
	{(x,y,z)}{[(x,y,z)]_{a}}

in quanto

\begin{equation}
	G \doteq F \circ (\pi_{a})^{-1}
\end{equation}

Sappiamo inoltre che la sfera quozientata e il proiettivo reale sono omeomorfi tramite

\map{f}
	{\rp{2}}{\sfrac{\S^{2}}{\sim_{a}}}
	{[(x,y,z)]}{[(x,y,z)]_{a}}

A questo punto possiamo considerare l'applicazione che mappa le classi di equivalenza del proiettivo reale in $ \R^{4} $, i.e.

\map{H}
	{\rp{2}}{\R^{4}}
	{[(x,y,z)]}{(x^{2}-y^{2},xy,xz,yz)}

ottenuta componendo l'omeomorfismo con l'applicazione considerata in precedenza

\begin{equation}
	H \doteq G \circ f = F \circ (\pi_{a})^{-1} \circ f
\end{equation}

Riassumendo tramite uno schema:

\diagr{%
		\S^{2} \arrow[dd, "\pi_{a}"] \arrow[rr, "F"] \&  \& \R^{4}                                  \&  \& {(x,y,z)} \arrow[rr, "F", maps to] \arrow[dd, "\pi_{a}", maps to] \&  \& {(x^{2}-y^{2},xy,xz,yz)}                                      \\
		\&  \&                                         \&  \&                                                                   \&  \&                                                               \\
		\sfrac{\S^{2}}{\sim_{a}} \arrow[rruu, "G"]   \&  \& \rp{2} \arrow[ll, "f"] \arrow[uu, "H"'] \&  \& {[(x,y,z)]_{a}} \arrow[rruu, "G", maps to]                        \&  \& {[(x,y,z)]} \arrow[uu, "H", maps to] \arrow[ll, "f", maps to]
		}

Perché l'applicazione $ H $ sia un embedding liscio è necessario che sia un'immersione e un embedding topologico.

\paragraph{Immersione}

Per dimostrare che $ H $ sia un'immersione, calcoliamone lo jacobiano

\begin{equation}
	J(H)(x,y,z) = \left[ \pdv{H^{i}}{u^{j}} \, (x,y,z) \right] %
	= \bmqty{ %
				2 x & y & z & 0 \\ %
				- 2 y & x & 0 & z \\ %
				0 & 0 & x & y %
				}
\end{equation}

dove abbiamo considerato le seguenti carte e relazioni:

\begin{equation}
	\begin{cases}
		(U, \varphi) = (U; u^{1}, u^{2}, u^{3}) \in \rp{2}, & (U,\varphi) \ni (x,y,z) \\
		(\R^{4}, \id_{\R^{4}}) \in \R^{4} \\
		H^{i} = r^{i} \circ H
	\end{cases}
\end{equation}

Essendo il rango dello jacobiano pari a 2, quindi uguale alla dimensione del proiettivo reale $ \rp{2} $ in ogni punto del dominio\footnote{%
	L'unico punto che rende il rango inferiore a 2 è l'origine di $ \R^{3} $, ma $ (0,0,0) \notin \rp{2} $.%
}, $ H $ è un'immersione.

\paragraph{Embedding topologico}

Perché $ H $ sia un embedding topologico è necessario che induca un omeomorfismo tra il dominio e l'immagine di questo, dunque che l'applicazione

\map{\tilde{H}}
	{\rp{2}}{H(\rp{2}) \subset \R^{4}}
	{[(x,y,z)]}{(x^{2}-y^{2},xy,xz,yz)}

sia una bigezione continua. \\
Siccome per definizione $ \Im(\tilde{H}) = H(\rp{2}) $, l'applicazione $ \tilde{H} $ è suriettiva; essendo inoltre questa immagine una combinazione lineare  di potenze intere (polinomi), l'applicazione è anche continua. Per l'iniettività, dimostriamo la seguente implicazione

\begin{equation}
	(x^{2}-y^{2},xy,xz,yz) = (u^{2}-v^{2},uv,uw,vw) \implies (x,y,z) = (u,v,w)
\end{equation}

Consideriamo dunque dei punti di $ \rp{2} $, per il quale valgono

\begin{equation}
	\begin{cases}
		z > 0 \\
		x^{2} + y^{2} + z^{2} = 1, & \rp{2} \stackrel{omeo}{\simeq} \sfrac{\S^{2}}{\sim_{a}}
	\end{cases}
\end{equation}

perciò

\begin{gather}
	\begin{split}
		\begin{cases}
			x^{2} - y^{2} = u^{2} - v^{2} \\
			xy = uv \\
			xz = uw \\
			yz = vw
		\end{cases} %
		\implies %
		\begin{cases}
			x = \dfrac{uv}{y} \\
			\dfrac{uv}{y} z = uw \\
			yz = vw
		\end{cases} %
		\implies %
		\begin{cases}
			y = \dfrac{vz}{w} \\
			yz = vw
		\end{cases} %
		\implies \\ \\
		\implies %
		z^{2} = w^{2}
		\implies %
		z = w
		\implies %
		(x,y,z) = (u,v,w)
	\end{split}
\end{gather}

dunque $ \tilde{H} $ è una bigezione continua, il che la rende un omeomorfismo e $ H $ un embedding topologico.

\paragraph{Embedding liscio}

Essendo un embedding topologico e un'immersione, l'applicazione $ H $ è un embedding liscio: questo porta a due conclusioni riguardanti l'immagine dell'applicazione

\begin{gather}
	H(\rp{2}) \stackrel{diff}{\simeq} \rp{2} \\
	H(\rp{2}) \text{ sottovarietà di } \R^{4} \qcomma \operatorname{cod}_{\R^{4}} (H(\rp{2})) = 2
\end{gather}
}

%=======================================================================================

\exer{Immersione iniettiva propria come embedding liscio}
{exer2-24}
{
Dimostrare che un'immersione iniettiva e propria è un embedding liscio. Mostrare che esistono embedding lisci che non sono applicazione proprie. \\ \\
Ricordare che un'applicazione continua $ f : X \to Y $ tra spazi topologici è propria se $ f^{-1}(K) $ è compatto in $ X $ per ogni compatto $ K $ di $ Y $.
}
{
Un'immersione, essendo liscia, è continua. Dal Teorema 4.95 da \cite{Lee-top} (pg. 121), possiamo derivare che un'applicazione continua e propria tra varietà differenziabili è chiusa. Dal lemma dell'applicazione chiusa\footnote{%
	Vedi Lemma \ref{lemma:clos-app}.%
}, un'applicazione chiusa e iniettiva tra due spazi localmente compatti e di Hausdorff\footnote{%
	Il lemma dell'applicazione chiusa può essere riformulato sostituendo le condizioni di compattezza e Hausdorff rispettivamente di dominio e codominio dell'applicazione con la condizione che entrambi gli spazi siano localmente compatti e di Hausdorff; quest'ultima condizione è valida per qualsiasi varietà differenziabile.%
} è un embedding topologico. Essendo $ f $ anche un'immersione, è anche un embedding liscio. \\
Se $ f : X \to Y $ è un embedding liscio ma l'immagine $ f(X) \subset Y $ non è chiusa in $ Y $ (i.e. l'applicazione non è chiusa in quanto $ X $ è sia chiuso che aperto in sé stesso), allora $ f $ non è propria. Sia un compatto $ K \subset Y $, l'intersezione $ K \cap f(X) $ non è compatta in $ Y $ in quanto $ f(X) $ non è chiuso in $ Y $: considerando la preimmagine di questa intersezione tramite l'omeomorfismo $ f : X \to f(X) $ (in quanto $ f $ è un embedding liscio) otteniamo

\begin{equation}
	f^{-1}(K \cap f(X)) = f^{-1}(K) \cap X = f^{-1}(K)
\end{equation}

il quale non è compatto poiché un omeomorfismo porta compatti in compatti (per il teorema di Heine-Borel), quindi $ f $ non è propria.
}

%=======================================================================================

\exer{Fibrato tangente di sottovarietà come sottovarietà}
{exer2-25}
{
Sia $ N $ una sottovarietà di una varietà differenziabile $ M $. Dimostrare che $ T(N) $ è una sottovarietà di $ T(M) $.
}
{
Consideriamo la definizione di sottovarietà per il caso in esame:

\begin{gather}
	N \text{ sottovarietà di } M \nonumber \\
	\Downarrow \\
	\forall p \in N, \, \E (U,\varphi) = (U; x^{1}, \dots, x^{n}) \in M \mid \nonumber \\
	\mid (U,\varphi) \ni p \wedge U \cap N = \{ q \in U \mid x^{k+1}(q) = \dots = x^{n}(q) = 0 \} \nonumber
\end{gather}

Perché $ T(N) $ sia sottovarietà di $ T(M) $ è dunque necessario che

\begin{gather}
	T(N) \text{ sottovarietà di } T(M) \nonumber \\
	\Downarrow \\
	\forall v \in T(N), \, \E (T(U),\tilde{\varphi}) \in T(M) \mid \nonumber \\
	\mid (T(U),\tilde{\varphi}) \ni v \wedge T(U) \cap T(N) = \{ w \in T(U) \mid \tilde{\varphi}^{2k+1}(w) = \dots = \tilde{\varphi}^{2n}(w) = 0 \} \nonumber
\end{gather}

Questo è verificato dal fatto che, considerando

\begin{equation}
	\begin{cases}
		\pi(v) = p \\
		v = \displaystyle\sum_{i=1}^{n} c^{i}(v) \eval{\pdv{x^{i}}}_{p}
	\end{cases}
\end{equation}

abbiamo che

\begin{align}
	\begin{split}
		\tilde{\varphi}(v) &= (\varphi \circ \pi, \varphi_{*})(v) \\
		&= \left( \varphi(p), \sum_{i=1}^{n} c^{i}(v) \eval{\pdv{r^{i}}}_{\varphi(p)} \right) \\
		&\equiv \left( \varphi(p), c(v) \right) \\
		&= ( x^{1}(p), \dots, x^{k}(p), 0, \dots, 0, c^{1}(v), \dots, c^{k}(v), 0, \dots, 0 )
	\end{split}
\end{align}

in quanto

\begin{equation}
	T_{p}(N) = \ev{ \eval{\pdv{x^{i}}}_{p} }_{i=1,\dots,k} %
	\implies %
	v = \sum_{i=1}^{k} c^{i}(v) \eval{\pdv{x^{i}}}_{p} \in T_{p}(N)
\end{equation}

Questo prova che $ T(N) $ sia una sottovarietà di $ T(M) $ con $ \operatorname{cod}_{T(M)} = 2(n-k) $.
}

%=======================================================================================

\exer{Parallelizzabilità sfera $ \S^{1} $}
{BONUS2-4}
{
Verificare che la sfera $ \S^{1} $ sia parallelizzabile.
}
{
Per dimostrare la parallelizzabilità della sfera $ \S^{1} $ definita come

\begin{equation}
	\S^{1} = \{ (x,y) \in \R^{2} \mid x^{2} + y^{2} = 1 \} \subset \R^{2}
\end{equation}

consideriamo il campo di vettori liscio in $ X \in \chi(\R^{2}) $ identificato con il punto di $ \R^{2} $

\begin{equation}
	X = - y \pdv{x} + x \pdv{y} \equiv (-y,x)
\end{equation}

Questo campo di vettori è liscio anche nella sfera $ \S^{1} $ perché questa è sottovarietà\footnote{%
	Vedi Lemma \ref{lemma:chi-subman-restr}.%
} di $ \R^{2} $, dunque $ \eval{X}_{\S^{1}} \in \chi(\S^{1}) $. \\
A questo punto, possiamo mostrare che la sfera $ \S^{1} $ sia parallelizzabile, i.e.

\begin{equation}
	\B_{T_{p}(\S^{1})} = \{ X_{p} \} \qcomma \forall p \in \S^{1}
\end{equation}

è una base per lo spazio tangente $ T_{p}(\S^{1}) $: questo perché il numero degli elementi della base $ \B_{T_{p}(\S^{1})} $ è unitario come la dimensione della sfera $ \S^{1} $ e questi appartengono al fibrato tangente della sfera in quanto perpendicolari alla normale in ogni punto della sfera, i.e.

\begin{equation}
	X_{p} \cdot p = (-y,x) \cdot (x,y) = 0 \qcomma \forall p \in \S^{1}
\end{equation}
}

%=======================================================================================

\exer{Varietà orientabili}
{exer2-26}
{
Una varietà differenziabile $ M $ è detta \textit{orientabile} se esiste un atlante di $ M $ rispetto al quale il determinante jacobiano dei cambi di carte è positivo. Dimostrare che:

\begin{enumerate}
	\item $ \rp{3} $ è una varietà orientabile;
	\item il fibrato tangente $ T(M) $ di una varietà differenziabile $ M $ è orientabile.
\end{enumerate}
}
{
\paragraph{Proiettivo reale $ \rp{3} $}

Possiamo ottenere il risultato per $ \rp{3} $ considerando la dimostrazione per $ \rp{n} $ e prendendo $ n = 3 $. \\
Presa un'applicazione $ f \in C^{\infty}(\R^{n+1}) $ dove $ 0 \in \VR_{f} $, allora $ f^{-1}(0) $ è una sottovarietà di $ R^{n+1} $ orientabile; nello specifico, prendendo  l'applicazione liscia

\map{f}
	{\R^{n+1}}{\R}
	{(x^{1},\dots,x^{n+1})}{x^{1} + \cdots + x^{n+1} - 1}

abbiamo che $ \S^{n} = f^{-1}(0) $, dunque $ S^{n} $ è una sottovarietà di $ R^{n+1} $ orientabile. \\
Sappiamo inoltre che il proiettivo reale può essere ricavato dal quoziente

\begin{equation}
	\rp{n} = \dfrac{\S^{n}}{\sim_{a}}
\end{equation}

dove la relazione di equivalenza antipodale $ \sim_{a} $ corrisponde alla condizione

\begin{equation}
	x \sim_{a} y \iff x = \pm y \qcomma x,y \in \S^{n}
\end{equation}

Questa equivalenza può essere rappresentata tramite la riflessione antipodale

\map{a}
	{\R^{n+1}}{\R^{n+1}}
	{(x^{1},\dots,x^{n+1})}{(-x^{1},\dots,-x^{n+1})}

Possiamo riscrivere questa applicazione considerando l'identificazione $ \R^{n+1} = T_{p}(\R^{n+1}) $, dove un elemento di $ T_{p}(\R^{n+1}) $ è rappresentato come un vettore colonna dei coefficienti:

\map{a}
{\R^{n+1}}{\R^{n+1}}
{\bmqty{ x^{1} \\ \vdots \\ x^{n+1} }}{%
										- \bigone_{n+1,n+1} \bmqty{ x^{1} \\ \vdots \\ x^{n+1} } %
										= \bmqty{ \dmat{-1,\ddots,-1} } \bmqty{ x^{1} \\ \vdots \\ x^{n+1} }
										}

Un'applicazione lineare conserva l'orientazione se il determinante della matrice che la identifica è positivo: per il caso della riflessione antipodale

\begin{equation}
	\det(a) \equiv \det( - \bigone_{n+1,n+1} ) = (-1)^{n+1} = %
	\begin{cases}
		+ 1, & n \text{ dispari} \\
		- 1, & n \text{ pari}
	\end{cases}
\end{equation}

A questo punto, per $ \rp{3} $, abbiamo che questo è orientabile in quanto $ n $ dispari.

\paragraph{Fibrato tangente $ T(N) $}

% https://math.stackexchange.com/questions/129514/why-is-the-tangent-bundle-orientable

Siano un punto $ p \in M $ e le carte

\begin{equation}
	\begin{cases}
		(U_{\alpha},\varphi_{\alpha}) = (U_{\alpha},x^{1},\dots,x^{n}) \in M \\
		(U_{\beta},\varphi_{\beta}) = (U_{\beta},y^{1},\dots,y^{n}) \in M \\
		(T(U_{\alpha}),\tilde{\varphi}_{\alpha}) = (T(U_{\alpha}),\xi^{1},\dots,\xi^{2n}) \in T(M) \\
		(T(U_{\beta}),\tilde{\varphi}_{\beta}) = (T(U_{\beta}),\theta^{1},\dots,\theta^{2n}) \in T(M)
	\end{cases}
\end{equation}

dove le applicazioni per il fibrato tangente sono date da

\sbs{0.5}{%
			\map{\tilde{\varphi}_{\alpha}}
				{T(U_{\alpha})}{\varphi(U_{\alpha}) \times \R^{n}}
				{(p,v)}{(\varphi_{\alpha}(p),a(v))}
			}
	{0.5}{%
			\map{\tilde{\varphi}_{\beta}}
				{T(U_{\beta})}{\varphi(U_{\beta}) \times \R^{n}}
				{(p,v)}{(\varphi_{\beta}(p),b(v))}
			}

in quanto possiamo scrivere un vettore nello spazio tangente come

\begin{equation}
	v = \sum_{i=1}^{n} a^{i}(v) \eval{ \pdv{x^{i}} }_{p} %
	= \sum_{i=1}^{n} b^{i}(v) \eval{ \pdv{y^{i}} }_{p} %
	\in T_{p}(U_{\alpha} \cap U_{\beta})
\end{equation}

con

\begin{gather}
	a(v) = (a^{1}(v),\dots,a^{n}(v)) \\
	b(v) = (b^{1}(v),\dots,b^{n}(v))
\end{gather}

Definendo le applicazioni di transizione come

\begin{equation}
	\begin{cases}
		t_{\alpha \beta} \doteq \varphi_{\alpha} \circ \varphi_{\beta}^{-1} \\
		\tilde{t}_{\alpha \beta} \doteq \tilde{\varphi}_{\alpha} \circ \tilde{\varphi}_{\beta}^{-1} 
	\end{cases}
\end{equation}

possiamo esplicitare l'azione di $ \tilde{t}_{\alpha \beta} $ e della sua inversa:

\begin{gather}
	\tilde{t}_{\alpha \beta} (\varphi_{\beta}(p),b(v)) = %
	\left( t_{\alpha \beta}, \sum_{i=1}^{n} b^{i}(v) \, \pdv{(t_{\alpha \beta})^{1}}{r^{i}}, \dots, \sum_{i=1}^{n} b^{i}(v) \, \pdv{(t_{\alpha \beta})^{n}}{r^{i}} \right) (\varphi_{\beta}(p)) \\
	%
	\tilde{t}_{\beta \alpha} (\varphi_{\alpha}(p),a(v)) = %
	\left( t_{\beta \alpha}, \sum_{i=1}^{n} b^{i}(v) \, \pdv{(t_{\beta \alpha})^{1}}{r^{i}}, \dots, \sum_{i=1}^{n} b^{i}(v) \, \pdv{(t_{\beta \alpha})^{n}}{r^{i}} \right) (\varphi_{\alpha}(p))
\end{gather}

Calcoliamo ora lo jacobiano delle applicazioni di transizione\footnote{%
	Calcoliamo solo $ J(\tilde{t}_{\alpha \beta})(\varphi_{\alpha}(p),a(v)) $ in quanto $ J(\tilde{t}_{\beta \alpha})(\varphi_{\beta}(p),b(v)) $ dà un risultato analogo al fine del calcolo del determinante.%
}

\begin{equation}
	J(\tilde{t}_{\alpha \beta})(\varphi_{\alpha}(p),a(v)) = \left[ \pdv{\theta^{i}}{\xi^{j}} \right]_{i,j=1}^{2n} %
	= \bmqty{ %
				\left[ \dpdv{\theta^{i}}{\xi^{j}} \right]_{i,j=1}^{n} & \eval{ \left[ \dpdv{\theta^{i}}{\xi^{j}} \right]_{i=1}^{n} }_{j=n+1}^{2n} \\ \\
				\eval{ \left[ \dpdv{\theta^{i}}{\xi^{j}} \right]_{i=n+1}^{2n} }_{j=1}^{n} & \left[ \dpdv{\theta^{i}}{\xi^{j}} \right]_{i,j=n+1}^{2n} %
			}
\end{equation}

Sappiamo che

\begin{equation}
	(\theta^{1},\dots,\theta^{n}) = \varphi_{\alpha} \left( \varphi_{\beta}^{-1} (\xi^{1},\dots,\xi^{n}) \right)
\end{equation}

non dipendono da $ (\xi^{n+1},\dots,\xi^{2n}) $, i.e.

\begin{equation}
	\eval{ \left[ \dpdv{\theta^{i}}{\xi^{j}} \right]_{i=1}^{n} }_{j=n+1}^{2n} = 0_{n,n}
\end{equation}

Inoltre, la prima e l'ultima matrice di $ J(\tilde{t}_{\alpha \beta})(\varphi_{\alpha}(p),a(v)) $ corrispondono allo jacobiano della matrice di transizione $ t_{\alpha \beta} $:

\begin{gather}
	\left[ \dpdv{\theta^{i}}{\xi^{j}} \right]_{i,j=1}^{n} = J(t_{\alpha \beta})(\varphi_{\beta}(p)) \\
	\nonumber \\
	\left[ \dpdv{\theta^{i}}{\xi^{j}} \right]_{i,j=n+1}^{2n} = J \left( (t_{\alpha \beta})_{*(\xi^{1},\dots,\xi^{n})} \right) (\theta^{1},\dots,\theta^{n}) = J(t_{\alpha \beta})(\varphi_{\beta}(p))
\end{gather}

A questo punto possiamo riscrivere lo jacobiano di $ \tilde{t}_{\alpha \beta} $ come

\begin{equation}
	J(\tilde{t}_{\alpha \beta})(\varphi_{\alpha}(p),a(v)) = %
	\bmqty{ %
			J(t_{\alpha \beta})(\varphi_{\beta}(p)) & 0_{n,n} \\ \\
			\eval{ \left[ \dpdv{\theta^{i}}{\xi^{j}} \right]_{i=n+1}^{2n} }_{j=1}^{n} & J(t_{\alpha \beta})(\varphi_{\beta}(p)) %
			}
\end{equation}

da cui il determinante

\begin{equation}
	\det( J(\tilde{t}_{\alpha \beta})(\varphi_{\alpha}(p),a(v)) ) = \left( \det( J(t_{\alpha \beta})(\varphi_{\beta}(p)) ) \right)^{2} > 0
\end{equation}

Essendo positivo, il fibrato tangente $ T(M) $ è orientabile.
}

%=======================================================================================

\exer{Funtore differenziale su varietà differenziabili}
{exer2-27}
{
Dimostrare che l'applicazione $ \mathcal{F}_{*} $ che associa a ogni varietà differenziabile il suo fibrato tangente e a ogni applicazione $ F : M \to N $ tra varietà differenziabili l'applicazione $ F_{*} : T(M) \to T(N) $ definita come

\begin{equation}
	\mathcal{F}_{*}(p,v) = (F(p), F_{*p}(v)) \qcomma \forall (p,v) \in T(M)
\end{equation}

definisce un funtore covariante dalla categoria delle varietà differenziabili in sé stessa.
}
{
Sia $ \mathfrak{C} $ la categoria delle varietà differenziabili con

\begin{gather}
	\ob(\mathfrak{C}) = \{ \text{collezione di tutte le varietà differenziabili} \} \\
	\mor(M,N) = \{ \text{applicazioni lisce } F : M \to N \} \qcomma \forall M,N \in \ob(\mathfrak{C})
\end{gather}

Scriviamo la coppia di applicazioni $ \mathcal{F}_{*} $ come

\map{\mathcal{F}_{*}}
	{\mathfrak{C}}{\mathfrak{C}}
	{M}{T(M) \\
		F &\mapsto F_{*}}

dove $ F : M \to N $ e l'applicazione $ F_{*} $ è definita come

\map{F_{*}}
	{T(M)}{T(N)}
	{(p,v)}{(F(p), F_{*p}(v))}

A questo punto, $ \mathcal{F}_{*} $ è un funtore covariante se

\begin{equation}
	\begin{cases}
		\mathcal{F}_{*}(M) \in \ob(\mathfrak{C}), & \forall M \in \ob(\mathfrak{C}) \\
		\mathcal{F}_{*}(F) \in \mor(\mathcal{F}_{*}(M),\mathcal{F}_{*}(N)), & \forall F \in \mor(M,N) \\
		\mathcal{F}_{*}(\bigone_{M}) = \bigone_{\mathcal{F}_{*}(M)} \in \mor(\mathcal{F}_{*}(M),\mathcal{F}_{*}(M)), & \forall \bigone_{M} \in \mor(M,M) \\
		\mathcal{F}_{*}(G \circ F) = \mathcal{F}_{*}(G) \circ \mathcal{F}_{*}(F), & \forall F \in \mor(M,N), \forall G \in \mor(N,P)
	\end{cases}
\end{equation}

\paragraph{1)}

Dalla definizione

\begin{equation}
	\mathcal{F}_{*}(M) = T(M) \in \ob(\mathfrak{C})
\end{equation}

in quanto $ T(M) $ è una varietà differenziabile.

\paragraph{2)}

Dalla definizione abbiamo che $ \mathcal{F}_{*}(F) = F_{*} $: per dimostrare che

\begin{equation}
	F_{*} \in \mor(\mathcal{F}_{*}(M),\mathcal{F}_{*}(N)) = \mor(T(M),T(N))
\end{equation}

e quindi che $ F_{*} \in C^{\infty}(T(M)) $, consideriamo il seguente diagramma

\sbs{0.35}{%
			\diagr{%
					T(M) \arrow[rr, "F_{*}"]          \&  \& T(N) \arrow[dd, "\pi"] \\
					\&  \&                        \\
					M \arrow[rr, "F"] \arrow[uu, "X"] \&  \& N                     
					}
			}
	{0.65}{%
			\begin{equation}
				F = \pi \circ F_{*} \circ X
			\end{equation}
			%			
			\begin{align}
				\begin{split}
					(\pi \circ F_{*} \circ X)(p) &= \pi (F_{*} (X_{p})) \\
					&= \pi(F(p), F_{*p}(X_{p})) \\
					&= F(p)
				\end{split}
			\end{align}
			}

Essendo tutte le applicazioni dell'uguaglianza lisce, anche $ F_{*} $ è liscia, i.e.

\begin{equation}
	\begin{cases}
		F \in C^{\infty}(M) \\
		\pi \in C^{\infty}(T(N)) \\
		X \in \chi(M)
	\end{cases} %
	\implies %
	F_{*} \in C^{\infty}(T(M))
\end{equation}

\paragraph{3)}

Sia l'applicazione identità

\map{\bigone_{M}}
	{M}{M}
	{p}{p}

Dalla definizione

\map{\mathcal{F}_{*}(\bigone_{M}) = (\bigone_{M})_{*}}
	{T(M)}{T(M)}
	{(p,v)}{(\bigone_{M}(p), (\bigone_{M})_{*p}(v))}

Siccome vale la relazione\footnote{%
	Vedi Paragrafo \ref{par:funt-prop}.%
}

\begin{equation}
	(\bigone_{M})_{*p} = \bigone_{T_{p}(M)}
\end{equation}

possiamo riscrivere $ \mathcal{F}_{*}(\bigone_{M}) $ come

\map{\mathcal{F}_{*}(\bigone_{M}) = (\bigone_{M})_{*}}
	{T(M)}{T(M)}
	{(p,v)}{ %
			(\bigone_{M}(p), \bigone_{T_{p}(M)}(v)) \\
			&\mapsto (p,v)
			}

dunque

\begin{equation}
	\mathcal{F}_{*}(\bigone_{M}) = \bigone_{\mathcal{F}_{*}(M)}
\end{equation}

\paragraph{4)}

Siano le applicazioni lisce

\begin{equation}
	\begin{cases}
		F : M \to N \\
		G : N \to P
	\end{cases}
\end{equation}

Dalla definizione

\map{\mathcal{F}_{*}(G \circ F) = (G \circ F)_{*}}
	{T(M)}{T(P)}
	{(p,v)}{ %
			((G \circ F)(p), (G \circ F)_{*p}(v)) \\
			&\mapsto ((G \circ F)(p), (G_{*F(p)} \circ F_{*p})(v))
			}

Siccome le applicazioni $ F_{*} $ e $ G_{*} $ sono definite come

\sbs{0.5}{%
			\map{F_{*}}
				{T(M)}{T(N)}
				{(p,v)}{(F(p), F_{*p}(v))}
			}
	{0.5}{%
			\map{G_{*}}
				{T(N)}{T(P)}
				{(q,w)}{(G(q), G_{*q}(w))}
			}

possiamo scrivere la composizione

\map{\mathcal{F}_{*}(G) \circ \mathcal{F}_{*}(F) = G_{*} \circ F_{*}}
	{T(M)}{T(P)}
	{(p,v)}{ %
				G_{*}(F(p), F_{*p}(v)) \\
				&\mapsto (G(F(p)), G_{*F(p)}(F_{*p}(v))) \\
				&\mapsto ((G \circ F)(p), (G_{*F(p)} \circ F_{*p})(v))
				}

dunque

\begin{equation}
	\mathcal{F}_{*}(G \circ F) = \mathcal{F}_{*}(G) \circ \mathcal{F}_{*}(F)
\end{equation}
}

%=======================================================================================

\exer{Commutatore e derivazione di algebra di Lie}
{exer2-28}
{
Una \textit{derivazione} di un'algebra di Lie $ (V,[\cdot,\cdot]) $ su un campo $ \K $ è un'applicazione lineare $ D : V \to V $ tale che

\begin{equation}
	D([Y,Z]) = [DY,Z] + [Y,DZ] \qcomma \forall Y,Z \in V
\end{equation}

Dimostrare che, dato $ X \in V $, l'applicazione

\map{D_{X}}
	{V}{V}
	{Y}{[X,Y]}

è una derivazione.
}
{
Consideriamo l'identità di Jacobi per i commutatori

\begin{equation}
	[X,[Y,Z]] + [Y,[Z,X]] + [Z,[X,Y]] = 0 \qcomma \forall X,Y,Z \in V
\end{equation}

Calcoliamo dunque la seguente espressione:

\begin{align}
	\begin{split}
		[D_{X} Y,Z] + [Y,D_{X} Z] &= [[X,Y],Z] + [Y,[X,Z]] \\
		&= - [Z,[X,Y]] + [Y,-[Z,X]] \\
		&= - [Z,[X,Y]] - [Y,[Z,X]] \\
		&= [X,[Y,Z]] \\
		&= D_{X}([Y,Z])
	\end{split}
\end{align}

dove nel quarto passaggio abbiamo usato l'identità di Jacobi, quindi $ D_{X} $ è una derivazione in quanto

\begin{equation}
	D_{X}([Y,Z]) = [D_{X} Y,Z] + [Y,D_{X} Z] \qcomma \forall Y,Z \in V
\end{equation}
}

%=======================================================================================

\exer{Curva integrale}
{exer2-29}
{
Siano $ M = \R \setminus \{0\} $ e $ X = \pdv*{x} \in \chi(M) $. Trovare la curva integrale di $ X $ massimale che inizia in un generico punto $ p \in \R $.
}
{
Una curva integrale $ c : (a,b) \to \R $ per il campo $ X $ che inizia in $ p $ rispetta le seguenti condizioni

\begin{equation}
	\begin{cases}
		c'(t) = X_{c(t)}, & \forall t \in (a,b) \\
		c(0) = p
	\end{cases}
\end{equation}

Le espressioni della tangente alla curva e del campo di vettori calcolato in $ c(t) $ sono date da

\begin{gather}
	c'(t) = \sum_{i=1}^{n} \dot{c}^{i}(t) \eval{ \pdv{x^{i}} }_{c(t)} = \dot{c}(t) \eval{ \pdv{x} }_{c(t)} \\
	X_{c(t)} = \sum_{i=1}^{n} a^{i}(c(t)) \eval{ \pdv{x^{i}} }_{c(t)} = a(c(t)) \eval{ \pdv{x} }_{c(t)} = \eval{ \pdv{x} }_{c(t)}
\end{gather}

dunque, per ottenere la curva integrale, Possiamo ora calcolare il sistema nel seguente modo:

\begin{equation}
	\begin{cases}
		\dot{c}(t) = 1, & \forall t \in (a,b) \\
		c(0) = p
	\end{cases} %
	\implies %
	c(t) = t + p
\end{equation}

Dunque la curva integrale massimale è la seguente:

\map{c}
	{\R}{M}
	{t}{t + p}
}

%=======================================================================================

\exer{Flusso locale e gruppo di diffeomorfismi a un parametro}
{exer2-30}
{
Trovare il flusso (locale) dei seguenti campi di vettori in $ \chi(\R^{2}) $:

\begin{equation}
	\begin{cases}
		X = x \, \dpdv{x} - y \, \dpdv{y} \\ \\
		Y = x \, \dpdv{x} + y \, \dpdv{y} \\ \\
		Z = \dpdv{x} + y \, \dpdv{y}
	\end{cases}
\end{equation}

Nel caso siano completi, calcolare il loro gruppo di diffeomorfismi a un parametro.
}
{
Per calcolare il flusso (locale) $ F(t,p) $ di un campo di vettori $ X \in \chi(M) $, definiti come

\map{F}
	{(-\varepsilon,\varepsilon) \times W}{V}
	{(t,p)}{F(t,p)}

\begin{equation}
	X = \sum_{i=1}^{n} a^{i} \pdv{x^{i}}
\end{equation}

con $ W,V \subseteq M $, è necessario risolvere il sistema

\begin{equation}
	\begin{cases}
		F'(t,p) = X_{F(t,p)}, & \forall (t,p) \in (-\varepsilon,\varepsilon) \times W \\
		F(0,p) = p
	\end{cases}
\end{equation}

Essendo $ F(\cdot,p) : (-\varepsilon,\varepsilon) \to V $ una curva integrale per $ X $, possiamo riscrivere il sistema come

\begin{equation}
	\begin{cases}
		\dot{F}^{i}(t,p) = a^{i}(F(t,p)), & \forall t \in (-\varepsilon,\varepsilon) \\
		F^{i}(0,p) = p^{i}
	\end{cases}
\end{equation}

Nel caso dei campi considerati:

\begin{equation}
	\begin{cases}
		X = x \, \dpdv{x} - y \, \dpdv{y} \\ \\
		Y = x \, \dpdv{x} + y \, \dpdv{y} \\ \\
		Z = \dpdv{x} + y \, \dpdv{y}
	\end{cases} %
	\implies %
	\begin{cases}
		a = (x,-y) \\
		b = (x,y) \\
		c = (1,y)
	\end{cases}
\end{equation}

Da cui i sistemi

\begin{equation}
	\begin{cases}
		\dot{F_{X}}^{1}(t,p) \doteq \dot{x}(t) = x(t) \\
		\dot{F_{X}}^{2}(t,p) \doteq \dot{y}(t) = - y(t) \\
		{F_{X}}^{1}(0,p) \doteq x(0) = p^{1} \\
		{F_{X}}^{2}(0,p) \doteq y(0) = p^{2}
	\end{cases} %
	\implies %
	\begin{cases}
		x(t) = A e^{t} \\
		y(t) = B e^{-t} \\
		x(0) = p^{1} \\
		y(0) = p^{2}
	\end{cases} %
	\implies %
	F_{X}(t,p) = (p^{1} e^{t}, p^{2} e^{-t})
\end{equation}

\begin{equation}
	\begin{cases}
		\dot{F_{Y}}^{1}(t,p) \doteq \dot{x}(t) = x(t) \\
		\dot{F_{Y}}^{2}(t,p) \doteq \dot{y}(t) = y(t) \\
		{F_{Y}}^{1}(0,p) \doteq x(0) = p^{1} \\
		{F_{Y}}^{2}(0,p) \doteq y(0) = p^{2}
	\end{cases} %
	\implies %
	\begin{cases}
		x(t) = A e^{t} \\
		y(t) = B e^{t} \\
		x(0) = p^{1} \\
		y(0) = p^{2}
	\end{cases} %
	\implies %
	F_{Y}(t,p) = (p^{1} e^{t}, p^{2} e^{t}) = p \, e^{t}
\end{equation}

\begin{equation}
	\begin{cases}
		\dot{F_{Z}}^{1}(t,p) \doteq \dot{x}(t) = 1 \\
		\dot{F_{Z}}^{2}(t,p) \doteq \dot{y}(t) = y(t) \\
		{F_{Z}}^{1}(0,p) \doteq x(0) = p^{1} \\
		{F_{Z}}^{2}(0,p) \doteq y(0) = p^{2}
	\end{cases} %
	\implies %
	\begin{cases}
		x(t) = t + A \\
		y(t) = B e^{t} \\
		x(0) = p^{1} \\
		y(0) = p^{2}
	\end{cases} %
	\implies %
	F_{Z}(t,p) = (p^{1} + t, p^{2} e^{t})
\end{equation}

Essendo tutti i flussi definiti in $ \R \times \R^{2} $, questi sono globali e i campi di vettori associati sono completi. \\
I gruppi di diffeomorfismi a un parametro dei campi di vettori sono dunque

\sbs{0.5}{%
			\map{G_{X}}
				{\R}{\operatorname{Diff}(\R^{2})}
				{t}{F_{X,t}}
			
			\map{G_{Y}}
				{\R}{\operatorname{Diff}(\R^{2})}
				{t}{F_{Y,t}}
			
			\map{G_{Z}}
				{\R}{\operatorname{Diff}(\R^{2})}
				{t}{F_{Z,t}}
		}
	{0.5}{%
			\map{F_{X,t}}
				{\R^{2}}{\R^{2}}
				{p}{(p^{1} e^{t}, p^{2} e^{-t})}
			
			\map{F_{Y,t}}
				{\R^{2}}{\R^{2}}
				{p}{(p^{1} e^{t}, p^{2} e^{t}) = p \, e^{t}}
			
			\map{F_{Z,t}}
				{\R^{2}}{\R^{2}}
				{p}{(p^{1} + t, p^{2} e^{t})}
			}
}

%=======================================================================================

\exer{Campo di vettori non completo}
{exer2-31}
{
Dimostrare che il campo di vettori $ X = \pdv*{x} \in \chi(\R^{2} \setminus \{(0,0)\}) $ non è completo.
}
{
Per calcolare il flusso (locale) $ F(t,p) $ del campo di vettori $ X \in \chi(\R^{2} \setminus \{(0,0)\}) $, definiti come

\map{F}
	{(-\varepsilon,\varepsilon) \times W}{V}
	{(t,p)}{F(t,p)}

\begin{equation}
	X = \sum_{i=1}^{n} a^{i} \pdv{x^{i}} = a^{1} \pdv{x} + a^{2} \pdv{y} \equiv (1,0)
\end{equation}

con $ W,V \subseteq \R^{2} \setminus \{(0,0)\} $, è necessario risolvere il sistema

\begin{equation}
	\begin{cases}
		\dot{F}^{1}(t,p) = a^{1}(F(t,p)) = 1 \\
		\dot{F}^{2}(t,p) = a^{2}(F(t,p)) = 0 \\
		F^{1}(0,p) = p^{1} \\
		F^{2}(0,p) = p^{2}
	\end{cases} %
	\implies %
	\begin{cases}
		F^{1}(t,p) = t + A \\
		F^{2}(t,p) = B \\
		F^{1}(0,p) = p^{1} \\
		F^{2}(0,p) = p^{2}
	\end{cases} %
	\implies %
	F(t,p) = (p^{1} + t, p^{2})
\end{equation}

Possiamo ora considerare due casi per $ p^{2} \neq 0 $ e $ p^{2} = 0 $:

\begin{equation}
	p^{2} \neq 0 %
	\implies %
	F(t,p) : \R \times \R^{2} \setminus \{(0,0)\} \to \R^{2} \setminus \{(0,0)\}
\end{equation}

\begin{equation}
	p^{2} \neq 0 %
	\implies %
	\begin{cases}
		F(t,p) : \R \times (- p^{1}, + \infty) \to \R^{2} \setminus \{(0,0)\}, & p^{1} > 0 \\
		F(t,p) : \R \times (- \infty, - p^{1}) \to \R^{2} \setminus \{(0,0)\}, & p^{1} < 0
	\end{cases}
\end{equation}

Il dominio non è dunque $ \R \times \R^{2} \setminus \{(0,0)\} $ per qualsiasi $ t \in \R $ in quanto

\begin{equation}
	\begin{cases}
		p^{2} = 0 \\
		t = - p^{1}
	\end{cases} %
	\implies %
	F(t,p) = F(- p^{1}; p^{1}, 0) = (0,0) \notin \R^{2} \setminus \{(0,0)\} %
	\qcomma \forall p^{1} \in \R
\end{equation}

perciò il campo non è completo.
}

%=======================================================================================

\exer{Curva integrale costante}
{exer2-32}
{
Siano $ M $ una varietà differenziabile e $ X \in \chi(M) $ un campo di vettori liscio tale che $ X(p) = 0 $ in un punto $ p \in M $. Dimostrare che la curva integrale di $ X $ che inizia in $ p $ è la curva costante $ c(t) = p $.
}
{
Sia la carta

\begin{equation}
	(U, \varphi) = (U; x^{1},\dots,x^{n}) \in M
\end{equation}

Se prendiamo la curva integrale

\map{c}
	{(- \varepsilon, \varepsilon)}{M}
	{t}{p}

con $ c'(t) = X_{c(t)} $, il suo vettore tangente al punto $ c(t) $ sarà dato da

\begin{align}
	\begin{split}
		X_{c(t)} &= c'(t) \\
		&= \sum_{i=1}^{n} \dot{c}^{i}(t) \eval{ \pdv{x^{i}} }_{c(t)} \\
		&\equiv (\dot{c}^{1}(t), \dots, \dot{c}^{n}(t)) \\
		&= \dot{(p^{1}, \dots, p^{n})} \\
		&= (0,\dots,0) \qcomma \forall c(t) \in M
	\end{split}
\end{align}

È quindi necessario che tutte le componenti del vettore tangente alla curva siano nulle sempre perché possano essere nulle in un punto $ p \in M $ perciò, essendo la curva integrale unica, la curva integrale di $ X $ che inizia in $ p $ è la curva costante $ c(t) = p $.
}

%=======================================================================================

\exer{Gruppo di diffeomorfismi a un parametro per campo di vettori nullo}
{exer2-33}
{
Sia $ M $ una varietà differenziabile e $ X \in \chi(M) $ il campo di vettori nullo, i.e. $ X = 0 $. Descrivere il gruppo dei diffeomorfismi a un parametro associato a $ X $.
}
{
Sia la carta

\begin{equation}
	(U, \varphi) = (U; x^{1},\dots,x^{n}) \in M
\end{equation}

Possiamo scrivere il campo di vettori calcolato in $ p $ come

\begin{equation}
	X_{p} = \sum_{i=1}^{n} a^{i}(p) \eval{ \pdv{x^{i}} }_{p} \equiv (0,\dots,0)
\end{equation}

Se prendiamo la curva integrale

\map{F}
	{(- \varepsilon, \varepsilon) \times V}{W}
	{(t,p)}{F_{t}(p)}

con $ V,W \subseteq M $, la sua espressione sarà data dalla soluzione al seguente sistema:

\begin{align}
	\begin{cases}
		F'_{t}(p) = X_{F_{t}(p)}, & \forall (t,p) \in (- \varepsilon, \varepsilon) \times V \\
		F_{0}(p) = p
	\end{cases}
\end{align}

da cui le componenti per $ i = 1,\dots,n $

\begin{equation}
	\begin{cases}
		\dot{F_{t}}^{i}(p) = a^{i}(F_{t}(p)) \\
		{F_{0}}^{i}(p) = p^{i} \\
		i = 1,\dots,n
	\end{cases} %
	\implies %
	F(t,p) = (p^{1},\dots,p^{n}) = p
\end{equation}

dunque $ F = \id_{M} $. \\
Siccome il dominio di $ (t,p) $ può essere esteso a $ \R \times M $, possiamo scrivere il gruppo di diffeomorfismi a un parametro associato al campo di vettori nullo:

\map{G}
	{\R}{\operatorname{Diff}(M)}
	{t}{\id_{M}}
}

%=======================================================================================

\exer{Pushforward di prodotto tra funzione e campo}
{exer2-34}
{
Siano $ F : N \to M $ un diffeomorfismo tra varietà differenziabili, $ X \in \chi(N) $ e $ f \in C^{\infty}(N) $. Dimostrare che

\begin{equation}
	F_{*}(f X) = (f \circ F^{-1}) \, F_{*} X
\end{equation}
}
{
Per dimostrare questa relazione, usiamo la definizione di pushforward, di differenziale, la linearità del differenziale, e il prodotto tra funzione e campo di vettori:

\begin{equation}
	\begin{cases}
		(F_{*} X)_{q} = F_{*F^{-1}(q)} (X_{F^{-1}(q)}), & \forall q \in M \\
		F_{*q} (X_{q}) (g) = X_{q} (g \circ F), & \forall q \in M, \, \forall g \in C^{\infty}(M) \\
		F_{*q} (\alpha X_{q}) = \alpha F_{*q} (X_{q}), & \forall \alpha \in \R \\
		(f X)(p) = f(p) X_{p}, & \forall p \in N 
	\end{cases}
\end{equation}

Considerando $ f : N \to \R $ e un punto qualsiasi $ q \in M $, possiamo scrivere

\begin{align}
	\begin{split}
		F_{*}(f X)(q) &= (F_{*}(f X))_{q} \\
		&= F_{*F^{-1}(q)} ((f X)_{F^{-1}(q)}) \\
		&= F_{*F^{-1}(q)} (f(F^{-1}(q)) \, X_{F^{-1}(q)}) \\
		&= f(F^{-1}(q)) \, F_{*F^{-1}(q)} (X_{F^{-1}(q)}) \\
		&= (f \circ F^{-1})(q) \, F_{*F^{-1}(q)} (X_{F^{-1}(q)}) \\
		&= (f \circ F^{-1})(q) \, (F_{*} X)_{q} \\
		&= ((f \circ F^{-1}) \, F_{*} X)(q)
	\end{split}
\end{align}

in quanto

\begin{equation}
	f(F^{-1}(q)) = (f \circ F^{-1})(q) \in \R
\end{equation}

dunque

\begin{equation}
	F_{*}(f X) = (f \circ F^{-1}) \, F_{*} X
\end{equation}
}


%

\chapter{Exercises: Lie groups and algebras}
%\exer{Prodotto diretto di gruppi di Lie}
{exer3-1}
{
Dimostrare che il prodotto diretto di due gruppi di Lie è un gruppo di Lie.
}
{
Consideriamo i due seguenti gruppi di Lie:

\begin{itemize}
	\item $ (G, \mu, i) $ con operazioni definite come \\
	\sbs{0.45}{%
				\map{\mu}
					{G \times G}{G}
					{(g_{1},g_{2})}{g_{1} \cdot g_{2}}
				}
		{0.45}{%
				\map{i}
					{G}{G}
					{g}{g^{-1}}
				}
	
	\item $ (H, \nu, j) $ con operazioni definite come \\
	\sbs{0.45}{%
				\map{\nu}
					{H \times H}{H}
					{(h_{1},h_{2})}{h_{1} * h_{2}}
				}
		{0.45}{%
				\map{j}
					{H}{H}
					{h}{h^{-1}}
			}	
\end{itemize}

A questo punto definiamo $ (\Omega, \sigma, k) $ come il prodotto diretto dei due gruppi di Lie appena definiti, i.e.

\begin{equation}
	(\Omega, \sigma, k) \doteq (G, \mu, i) \times (H, \nu, j) = (G \times H, \mu \times \nu, i \times j)
\end{equation}

dove le operazioni di $ \Omega $ sono definite come:

\sbs{0.5}{%
			\map{\sigma}
				{\Omega \times \Omega}{\Omega}
				{(s_{1},s_{2})}{s_{1} \star s_{2}}
			}
	{0.5}{%
			\map{k}
				{\Omega}{\Omega}
				{s}{s^{-1}}
			}

Esplicitando i gruppi di Lie sottostanti, possiamo riscrivere queste operazioni come:

\map{\sigma = \mu \times \nu}
	{G \times G \times H \times H}{G \times H}
	{(g_{1},g_{2},h_{1},h_{2})}{(g_{1} \cdot g_{2},h_{1} * h_{2})}

\map{k = i \times j}
	{G \times H}{G \times H}
	{(g,h)}{(g^{-1},h^{-1})}

Il prodotto diretto conserva in $ \Omega $ le proprietà di gruppo algebrico di $ G $ e $ H $. Affinché $ \Omega $ sia un gruppo di Lie, è necessario che le operazioni $ \sigma $ e $ k $ siano lisce: per dimostrare questo, consideriamo le seguenti proiezioni tra varietà

\sbs{0.5}{%
			\map{\pi_{G}}
				{\Omega}{G}
				{s = (g,h)}{g}
			}
	{0.5}{%
			\map{\pi_{H}}
				{\Omega}{H}
				{s = (g,h)}{h}
			}

Da cui possiamo derivare i seguenti diagrammi:

\sbs{0.5}{%
			\diagr{%
					G \times G \arrow[rr, "\mu"]                                                                                         \&  \& G                                                   \\
					\&  \&                                                     \\
					\Omega \times \Omega \arrow[rr, "\sigma"] \arrow[dd, "\pi_{H} \times \pi_{H}"] \arrow[uu, "\pi_{G} \times \pi_{G}"'] \&  \& \Omega \arrow[dd, "\pi_{H}"] \arrow[uu, "\pi_{G}"'] \\
					\&  \&                                                     \\
					H \times H \arrow[rr, "\nu"]                                                                                         \&  \& H                                                  
					}
			}
	{0.5}{%
			\diagr{%
					G \arrow[rr, "i"]                                                   \&  \& G                                                   \\
					\&  \&                                                     \\
					\Omega \arrow[rr, "k"] \arrow[dd, "\pi_{H}"] \arrow[uu, "\pi_{G}"'] \&  \& \Omega \arrow[dd, "\pi_{H}"] \arrow[uu, "\pi_{G}"'] \\
					\&  \&                                                     \\
					G \arrow[rr, "j"]                                                   \&  \& H                                                  
					}
			}

Mediante le proiezioni è possibile scrivere le operazioni del gruppo $ \Omega $ come

\begin{gather}
	\sigma = (\pi_{G})^{-1} \circ \mu \circ (\pi_{G} \times \pi_{G}) %
	= (\pi_{H})^{-1} \circ \nu \circ (\pi_{H} \times \pi_{H}) \\
	k = (\pi_{G})^{-1} \circ i \circ \pi_{G} %
	= (\pi_{H})^{-1} \circ j \circ \pi_{H}
\end{gather}

Siccome le proiezioni e le operazioni dei gruppi $ G $ e $ H $ sono lisce, anche le operazioni di $ \Omega $ sono lisce in quanto composizioni di applicazioni lisce; questo mostra che $ (\Omega, \sigma, k) $ è un gruppo di Lie.\\
Alternativamente, anche il solo prodotto diretto di applicazioni lisce è un'applicazione liscia, dunque non è necessario passare per le proiezioni.
}

%=======================================================================================

\exer{Topologie sul toro}
{exer3-2}
{
Siano la proiezione

\map{\pi}
	{\R^{2}}{\T^{2} = \S^{1} \times \S^{1}}
	{(t,s)}{(e^{2 \pi i t},e^{2 \pi i s})}

l'insieme

\begin{equation}
	L = \{ (t,\alpha t) \in \R^{2} \mid \alpha \in \R \setminus \Q \}
\end{equation}

e la restrizione di $ \pi $ a $ L $, i.e.

\begin{equation}
	f = \eval{\pi}_{L} : L \to \S^{1} \times \S^{1}
\end{equation}

Siano

\begin{itemize}
	\item $ \tau_{f} $ la topologia indotta da $ f $ su $ H = \pi(L) $;
	
	\item $ \tau_{s} $ la topologia indotta dall'inclusione $ H \subset \S^{1} \times \S^{1} $
\end{itemize}

Dimostrare che $ \tau_{s} \subset \tau_{f} $.
}
{
Definiamo le due topologie:

\begin{itemize}
	\item $ \tau_{s} $: $ V $ è aperto in $ H $ se $ V = H \cap U $ con $ U \subset \S^{1} \times \S^{1} $ aperto;
	
	\item $ \tau_{f} $: $ W $ è aperto in $ H $ se $ f^{-1}(W) \subset L $ è aperto in $ L $.
\end{itemize}

Sia un aperto $ f^{-1}(W) \subset L $, la sua immagine $ W \subset \T^{2} $ è aperta per la topologia $ \tau_{f} $ indotta da $ f $ ma non lo è per la topologia $ \tau_{s} $ indotta dall'inclusione. Questo perché un aperto $ V \subset \T^{2} $ della topologia $ \tau_{s} $ è l'intersezione tra $ H $ e una palla aperta $ U = B \subset \R^{2} $ attraverso la proiezione $ \pi $, i.e. $ V = H \cap \pi(B) $, la quale è un'unione infinita di aperti della topologia $ \tau_{f} $. A questo punto, una topologia è inclusa in un'altra se ogni aperto della prima è contenuto in un aperto della seconda: nel caso in esame, abbiamo che
%
\begin{gather}
	\forall \, W \subset \tau_{f}, \, \E V = H \cap \pi(B) \subset \tau_{s} \mid W \subset V %
	\implies %
	\tau_{s} \subset \tau_{f}
\end{gather}

\img{0.9}{img57}
}

%=======================================================================================

\exer{Esponenziale di matrice}
{exer3-3}
{
Sia la matrice

\begin{equation}
	X = \bmqty{ %
				0 & 1 \\
				1 & 0 %
				}
\end{equation}

Dimostrare che

\begin{equation}
	e^{X} = \bmqty{ %
					\cosh(1) & \sinh(1) \\ \\
					\sinh(1) & \cosh(1) %
					}
\end{equation}
}
{
Possiamo vedere che la matrice $ X $ è idempotente, i.e.

\begin{equation}
	X^{2} = \bmqty{ %
					0 & 1 \\
					1 & 0 %
					} %
	\bmqty{ %
			0 & 1 \\
			1 & 0 %
			} %
	= \bmqty{ %
				1 & 0 \\
				0 & 1 %
				} %
	= I
\end{equation}

da cui le relazioni

\begin{equation}
	\begin{cases}
		X^{2k} = I \\
		X^{2k+1} = X
	\end{cases}
	\quad k \in \N \cup \{0\}
\end{equation}

Consideriamo anche le espansioni in serie di Taylor delle due funzioni iperboliche: \\

\sbs{0.5}{%
			\begin{equation}
				\cosh(t) = \sum_{k=0}^{+\infty} \dfrac{t^{2k}}{(2k)!}
			\end{equation}
			}
	{0.5}{%
			\begin{equation}
				\sinh(t) = \sum_{k=0}^{+\infty} \dfrac{t^{2k+1}}{(2k+1)!}
			\end{equation}
			}

A questo punto, possiamo calcolare l'esponenziale di $ X $ mediante la definizione:

\begin{align}
	\begin{split}
		e^{X} &\doteq \sum_{k=0}^{+\infty} \dfrac{X^{k}}{k!} \\ \\
		&= \sum_{k=0}^{+\infty} \left( \dfrac{X^{2k}}{(2k)!} + \dfrac{X^{2k+1}}{(2k+1)!} \right) \\ \\
		&= \sum_{k=0}^{+\infty} \left( \dfrac{(1)^{2k}}{(2k)!} \, I \right) + \sum_{k=0}^{+\infty} \left( \dfrac{(1)^{2k+1}}{(2k+1)!} \, X \right) \\ \\
		&= \sum_{k=0}^{+\infty} \left( \dfrac{(1)^{2k}}{(2k)!} \right) I + \sum_{k=0}^{+\infty} \left( \dfrac{(1)^{2k+1}}{(2k+1)!} \right) X \\ \\
		&= \cosh(1) \, I + \sinh(1) \, X \\ \\
		&= \bmqty{ %
					\cosh(1) & \sinh(1) \\ \\
					\sinh(1) & \cosh(1) %
					}
	\end{split}
\end{align}

dove nel quarto passaggio abbiamo usato la proprietà delle serie convergenti

\begin{equation}
	\sum_{n=0}^{+\infty} a s_{n} = a \sum_{n=0}^{+\infty} s_{n}
\end{equation}

in quanto $ M_{n}(\R) $ è uno spazio di Banach, quindi completo.
}

%=======================================================================================

\exer{Esponenziale di somma di matrici}
{exer3-4}
{
Trovare due matrici $ A $ e $ B $ tali che

\begin{equation}
	e^{A+B} \neq e^{A} e^{B}
\end{equation}
}
{
Prese due matrici $ A,B \in M_{n}(\K) $ con $ \K = \R, \C $, vale la proprietà

\begin{equation}
	e^{A+B} = e^{A} e^{B} \qcomma [A,B] = 0
\end{equation}

Un esempio di matrici che non rispettano questa proprietà può essere:

\sbs{0.5}{%
			\begin{equation}
				A = \bmqty{ %
							0 & 1 \\
							1 & 0 %
							}
			\end{equation}
			}
	{0.5}{%
			\begin{equation}
				B = \bmqty{ %
							0 & 1 \\
							0 & 0 %
							}
			\end{equation}
			}

in quanto

\begin{equation}
	[A,B] = %
	\bmqty{ 0 & 1 \\ 1 & 0 } \bmqty{ 0 & 1 \\ 0 & 0 } - \bmqty{ 0 & 1 \\ 0 & 0 }  \bmqty{ 0 & 1 \\ 1 & 0 } %
	= \bmqty{ -1 & 0 \\ 0 & 1 } %
	\neq 0
\end{equation}
}

%=======================================================================================

\exer{Matrici unitarie come insieme compatto}
{exer3-5}
{
Dimostrare che il gruppo unitario $ U(n) $ è compatto per ogni $ n \geqslant 1 $.
}
{
Per dimostrare che $ U(n) \subset GL_{n}(\C) $ sia compatto, è necessario dimostrare che sia chiuso e limitato in $ GL_{n}(\C) $. \\
È chiuso perché preimmagine di $ I \in GL_{n}(\C) $ tramite l'applicazione continua

\map{F}
	{GL_{n}(\C)}{GL_{n}(\C)}
	{A}{A^{*} A}

Mentre è limitato in quanto le colonne delle matrici di $ U(n) $ hanno norma unitaria e quindi la somma delle colonne è limitata dalla dimensione n delle matrici. \\
Svolgendo il prodotto matriciale nella definizione di $ U(n) $

\begin{equation}
	A^{*} A = %
	\bmqty{ %
			\bar{a}_{11} & \bar{a}_{21} & \cdots & \bar{a}_{n1} \\ %
			\bar{a}_{12} & \bar{a}_{22} & \cdots & \bar{a}_{n2} \\ %
			\vdots & \vdots & \ddots & \vdots \\ %
			\bar{a}_{1n} & \bar{a}_{2n} & \cdots & \bar{a}_{nn}
			}
	\bmqty{ %
			a_{11} & a_{12} & \cdots & a_{1n} \\ %
			a_{21} & a_{22} & \cdots& a_{2n} \\ %
			\vdots & \vdots & \ddots & \vdots \\ %
			a_{n1} & a_{n2} & \cdots & a_{nn}
			}
\end{equation}

se consideriamo solo la prima entrata della matrice prodotto, otteniamo

\begin{equation}
	\abs{a_{11}}^{2} + \abs{a_{21}}^{2} + \cdots + \abs{a_{n1}}^{2} = 1
\end{equation}

e così per il resto degli elementi nella diagonale, mentre tutti gli altri prodotti hermitiani tra il resto delle colonne è nullo. \\
Questo implica che, considerata la norma per matrici complesse

\map{\norm{}}
	{M_{n}(\C)}{\R}
	{X = [x_{ij}]_{i,j=1,\dots,n}}{\left( \sum_{i,j=1}^{n} \abs{x_{ij}}^{2} \right)^{1/2}}

vale per le matrici unitarie

\begin{equation}
	\norm{A} = \sqrt{n} \qcomma \forall A \in U(n)
\end{equation}
}

%=======================================================================================

\exer{Proprietà componente connessa $ G_{0} $ di un gruppo di Lie}
{exer3-6}
{
Siano $ G $ un gruppo di Lie e $ G_{0} $ la componente connessa di $ G $ che contiene $ e $ (elemento neutro di $ G $). Se $ \mu $ e $ i $ denotano rispettivamente la moltiplicazione e l'inversione in $ G $, provare che:

\begin{enumerate}
	\item $ \mu(\{g\} \times G_{0}) \subset G_{0} $ per qualsiasi $ g \in G_{0} $;
	
	\item $ i(G_{0}) \subset G_{0} $;
	
	\item $ G_{0} $ è un sottoinsieme aperto di $ G $;
	
	\item $ G_{0} $ è un sottogruppo di Lie di $ G $.
\end{enumerate}
}
{
\paragraph{1)}

Considerando un elemento $ g \in G $ e la traslazione a sinistra

\map{L_{g}}
	{G}{G}
	{h}{g h}

possiamo scrivere

\begin{equation}
	L_{g}(G_{0}) \doteq \mu(\{g\} \times G_{0})
\end{equation}

Sappiamo che la traslazione a sinistra è un diffeomorfismo\footnote{%
	Vedi Sezione \ref{sec:lie-groups}.%
}, dunque è anche continua: prendendo un qualunque elemento $ g \in G_{0} $, questo è collegato all'identità tramite

\begin{equation}
	L_{g}(e) = e g = g \in G_{0} \qcomma \forall g \in G_{0}
\end{equation}

Questo dimostra che $ L_{g}(G_{0}) $ sia connesso (per archi) e quindi che

\begin{equation}
	L_{g}(G_{0}) = \mu(\{g\} \times G_{0}) \subset G_{0} \qcomma \forall g \in G_{0}
\end{equation}

in quanto $ G_{0} $ è componente connessa di $ G $. \\
L'ultima condizione può essere anche scritta come $ \mu(G_{0} \times G_{0}) \subset G_{0} $.

\paragraph{2)}

Essendo l'inversione $ i : G \to G $ un omeomorfismo (bigezione continua con inversa continua), abbiamo che $ i(G_{0}) $ è connesso in quanto $ G_{0} $ è connesso. Inoltre, abbiamo che l'immagine di $ G_{0} $ tramite l'inversione contiene l'elemento neutro, i.e.

\begin{equation}
	i(G_{0}) \ni e \, \because \, i(e) = e
\end{equation}

A questo punto, siccome $ i(G_{0}) $ è connesso, contiene $ e \in G_{0} $ e $ G_{0} $ è una componente connessa di $ G $, possiamo scrivere $ i(G_{0}) \subset G_{0} $.

\paragraph{3)}

Vedi Proposizione \ref{prop:conn-comp-open}, poiché $ G $ è localmente euclideo in quanto varietà differenziabile.

\paragraph{4)}

Siccome, per il punto precedente, $ G_{0} \subset G $ è aperto in $ G $, la componente connessa $ G_{0} $ è una sottovarietà di $ G $. Tramite la Proposizione \ref{prop:lie-group-cond}, è necessario dunque mostrare che $ G_{0} < G $ perché $ G_{0} $ sia sottogruppo di Lie di $ G $: questo è immediatamente verificato dai primi due punti, i.e.

\begin{equation}
	\begin{cases}
		\mu(G_{0} \times G_{0}) \subset G_{0} \\
		i(G_{0}) \subset G_{0}
	\end{cases} %
	\implies %
	G_{0} < G
\end{equation}

dove moltiplicazione e inversione di $ G $ inducono le operazioni su $ G_{0} $:

\sbs{0.5}{%
			\map{\mu_{0}}
				{G_{0} \times G_{0}}{G_{0}}
				{(g,h)}{g h}
			}
	{0.5}{%
			\map{i_{0}}
				{G_{0}}{G_{0}}
				{g}{g^{-1}}
			}
}

%=======================================================================================

\exer{Diffeomorfismo $ SO(2) \simeq \S^{1} $}
{BONUS3-1}
{
Verificare che $ SO(2) $ sia diffeomorfo a $ \S^{1} $.
}
{
Dalle definizioni degli spazi in esame, possiamo derivare le seguenti definizioni alternative:

\begin{align}
	\begin{split}
		SO(2) &= \{ X \in M_{2}(\R) \mid X^{T} X = I \, \wedge \, \det(X) = 1 \} \subset O(2) \subset GL_{2}(\R) \subset M_{2}(\R) \\
		&= \left\{ X = \bmqty{ a & b \\ -b & a } \in M_{2}(\R) \st \det(X) = a^{2} + b^{2} = 1 \right\} \\
		&= \left\{ \bmqty{ \cos(\theta) & \sin(\theta) \\ -\sin(\theta) & \cos(\theta) } \in M_{2}(\R) \st \theta \in [0,2\pi) \subset \R \right\}
	\end{split}
\end{align}

\begin{align}
	\begin{split}
		\S^{1} &= \{ (x,y) \in \R^{2} \mid x^{2} + y^{2} = 1 \} \subset \R^{2} \\
		&= \{ z \in \C \mid \abs{z} = 1 \} \subset \C = \R^{2} \\
		&= \{ e^{i \theta} \in \C \mid \theta \in [0,2\pi) \subset \R \}
	\end{split}
\end{align}

Consideriamo ora l'applicazione

\map{\varphi}
	{\S^{1}}{SO(2)}
	{e^{i \theta}}{\bmqty{ \cos(\theta) & \sin(\theta) \\ -\sin(\theta) & \cos(\theta) }}

la quale è liscia in quanto composizione delle applicazioni lisce $ \varphi = h \circ g \circ f $ definite come:

\map{f}
	{\S^{1}}{\R^{2}}
	{e^{i \theta}}{(\cos(\theta), \sin(\theta))}

\map{g}
	{\R^{2}}{\R^{4}}
	{(\cos(\theta), \sin(\theta))}{(\cos(\theta), \sin(\theta), -\sin(\theta), \cos(\theta))}

\map{h}
	{\R^{4}}{SO(2)}
	{(\cos(\theta), \sin(\theta), -\sin(\theta), \cos(\theta))}
	{\bmqty{ \cos(\theta) & \sin(\theta) \\ -\sin(\theta) & \cos(\theta) }}

A questo punto, definiamo le applicazioni

\sbs{0.5}{%
			\map{\psi}
				{M_{2}(\R)}{\S^{1}}
				{\bmqty{ a & b \\ c & d }}{a + i b}
			}
	{0.5}{%
			\map{i}
				{SO(2)}{M_{2}(\R)}
				{A}{A}
			}

le quali sono lisce perché rispettivamente $ \psi $ è lineare, e valgono le condizioni\footnote{%
	Vedi Proposizione \ref{prop:subman-incl-immersion}.%
}

\begin{equation}
	SO(2) \subset M_{2}(\R) \text{ sottovarietà\footnotemark} %
	\iff %
	i \text{ immersione} %
	\implies %
	i \in C^{\infty}(SO(2))
\end{equation}
\footnotetext{%
	L'insieme $ SO(n) $ è aperto in $ O(n) $ (vedi Sottosezione \ref{s-sec:o-n-subgroup-lie-glnr}), che è sottovarietà di $ GL_{n}(\R) $, il quale a sua volta è aperto in $ M_{n}(\R) $: questo rende $ SO(n) $ sottovarietà di $ M_{n}(\R) $.%
}

Se componiamo le applicazioni appena definite, otteniamo

\map{\varphi^{-1} \doteq \psi \circ i = \eval{\psi}_{SO(2)}}
	{SO(2)}{\S^{1}}
	{\bmqty{ \cos(\theta) & \sin(\theta) \\ -\sin(\theta) & \cos(\theta) }}
	{\cos(\theta) + i \sin(\theta) = e^{i \theta}}

la quale è liscia perché composizione di applicazioni lisce, e inoltre è inversa di $ \varphi $. Avendo trovato un'applicazione liscia, invertibile, e con inversa liscia tra i due spazi, abbiamo dimostrato che i due sono diffeomorfi, i.e.

\begin{equation}
	\begin{cases}
		\varphi \in C^{\infty}(\S^{1}) \\ \\
		\begin{cases}
			\varphi \circ \varphi^{-1} = \id_{\S^{1}} \\
			\varphi^{-1} \circ \varphi = \id_{SO(2)}
		\end{cases} \\ \\
		\varphi^{-1} \in C^{\infty}(SO(2))
	\end{cases}
	\implies %
	SO(2) \simeq \S^{1}
\end{equation}
}

%=======================================================================================

\exer{Diffeomorfismo $ SU(2) \simeq \S^{3} $}
{BONUS3-2}
{
Verificare che $ SU(2) $ sia diffeomorfo a $ \S^{3} $.
}
{
Dalle definizioni degli spazi in esame, possiamo derivare le seguenti definizioni alternative:

\begin{align}
	\begin{split}
		SU(2) &= \{ A \in M_{2}(\C) \mid A^{*} A = I \, \wedge \, \det(A) = 1 \} \subset U(2) \subset GL_{2}(\C) \subset M_{2}(\C) \\
		&= \left\{ \bmqty{ \alpha & - \bar{\beta} \\ \beta & \bar{\alpha} } \in M_{2}(\C) \st \alpha, \beta \in \C^{2}, \, \det(A) = \abs{\alpha}^{2} + \abs{\beta}^{2} = 1 \right\}
	\end{split}
\end{align}

\begin{align}
	\begin{split}
		\S^{3} &= \{ (a,b,c,d) \in \R^{4} \mid a^{2} + b^{2} + c^{2} + d^{2} = 1 \} \subset \R^{4} \\
		&= \{ (\alpha, \beta) \in \C^{2} \mid \abs{\alpha}^{2} + \abs{\beta}^{2} = 1 \} \subset \C^{2} = \R^{4}
	\end{split}
\end{align}

dove, per legare la prima definizione di $ \S^{3} $ alla seconda, si può considerare l'identificazione

\begin{equation}
	\begin{cases}
		\alpha = a + i b \\
		\beta = c + i d
	\end{cases} %
	\qquad a,b,c,d \in \R
\end{equation}

Analogamente all'esercizio precedente, possiamo considerare l'applicazione liscia, invertibile e con inversa liscia:

\map{\varphi}
	{\S^{3}}{SU(2)}
	{(\alpha, \beta)}{\bmqty{ \alpha & \beta \\ - \bar{\beta} & \bar{\alpha} }}

Questa rende dunque i due spazi diffeomorfi, i.e. $ SU(2) \simeq \S^{3} $.
}

%=======================================================================================

\exer{Differenziale moltiplicazione gruppo di Lie}
{exer3-7}
{
Sia $ G $ un gruppo di Lie con moltiplicazione $ \mu : G \times G \to G $. Dimostrare che

\begin{equation}
	\mu_{*(a,b)}(X_{a},Y_{b}) = (R_{b})_{*a}(X_{a}) + (L_{a})_{*b}(Y_{b}) %
	\qcomma \forall (a,b) \in G \times G, \, \forall X_{a} \in T_{a}(G), \, \forall Y_{b} \in T_{b}(G)
\end{equation}

dove $ L_{a} $ (risp. $ R_{b} $) denota la traslazione a sinistra (risp. a destra) associata ad $ a $ (risp. $ b $).
}
{
Siccome il differenziale è un'applicazione lineare, possiamo scrivere\footnote{%
	Vedi Paragrafo \ref{example:differ-prod-inv-lie}.%
}

\begin{equation}
	\mu_{*(a,b)}(X_{a},Y_{b}) = \mu_{*(a,b)}(X_{a},0) + \mu_{*(a,b)}(0,Y_{b})
\end{equation}

Consideriamo ora le seguenti curve lisce:

\sbs{0.5}{%
			\begin{equation}
				\begin{cases}
					c : (- \varepsilon, \varepsilon) \to G \\
					c(0) = a \\
					c'(0) = X_{a}
				\end{cases}
			\end{equation}
			}
	{0.5}{%
			\begin{equation}
				\begin{cases}
					d : (- \varepsilon, \varepsilon) \to G \\
					d(0) = b \\
					d'(0) = Y_{b}
				\end{cases}
			\end{equation}
	}

\sbs{0.5}{%
			\begin{equation}
				\begin{cases}
					\gamma \doteq (c(t), b) \\
					\gamma(0) = (a,b) \\
					\gamma'(0) = (X_{a}, 0)
				\end{cases}
			\end{equation}
			}
	{0.5}{%
			\begin{equation}
				\begin{cases}
					\eta \doteq (a, d(t)) \\
					\eta(0) = (a,b) \\
					\eta'(0) = (0, Y_{b})
				\end{cases}
			\end{equation}
			}

Queste ci permettono di calcolare i differenziali delle traslazioni a destra e sinistra, e della moltiplicazione:

\begin{gather}
	(R_{b})_{*a}(X_{a}) = \eval{ \dv{t} }_{0} R_{b}(c(t)) %
	= \eval{ \dv{t} }_{0} (c(t) \, b) %
	= c'(0) \, b %
	= X_{a} \, b \\
	%
	(L_{a})_{*b}(Y_{b}) = \eval{ \dv{t} }_{0} L_{a}(d(t)) %
	= \eval{ \dv{t} }_{0} (a \, d(t)) %
	= a \, d'(0) %
	= a \, Y_{b}
\end{gather}

\begin{gather}
	\mu_{*(a,b)}(X_{a}, 0) = \dot{ (\mu \circ \gamma) }(0) %
	= \mu'(c(t), b)(0) %
	= (c(t) \, b)'(0) %
	= c'(0) \, b %
	= X_{a} \, b \\
	%
	\mu_{*(a,b)}(0, Y_{b}) = \dot{ (\mu \circ \eta) }(0) %
	= \mu'(a, d(t))(0) %
	= (a \, d(t))'(0) %
	= a \, d'(0) %
	= a \, Y_{b}
\end{gather}

dunque

\begin{equation}
	\mu_{*(a,b)}(X_{a},Y_{b}) = (R_{b})_{*a}(X_{a}) + (L_{a})_{*b}(Y_{b}) %
	= X_{a} \, b + a \, Y_{b}
\end{equation}
}

%=======================================================================================

\exer{Differenziale inversione gruppo di Lie}
{exer3-8}
{
Sia $ G $ un gruppo di Lie con inversione $ i : G \to G $. Dimostrare che

\begin{equation}
	i_{*a}(Y_{a}) = -(R_{a^{-1}})_{*e}(L_{a^{-1}})_{*a} (Y_{a}) %
	\qcomma \forall a \in G, \, \forall Y_{a} \in T_{a}(G)
\end{equation}
}
{
Per l'inversione, vale la seguente proprietà:

\begin{equation}
	R_{a} \circ i = i \circ L_{a^{-1}} \qcomma \forall a \in G
\end{equation}

Moltiplicando entrambi i membri a sinistra per l'inversa della traslazione a destra, i.e.

\begin{equation}
	(R_{a})^{-1} = R_{a^{-1}} \qcomma \forall a \in G
\end{equation}

otteniamo

\begin{equation}
	i = R_{a^{-1}} \circ i \circ L_{a^{-1}} \qcomma \forall a \in G
\end{equation}

Differenziando quest'espressione in una qualunque elemento $ a \in G $, possiamo scrivere

\begin{align}
	\begin{split}
		i_{*a} &= (R_{a^{-1}} \circ i \circ L_{a^{-1}})_{*a} \\
		&= (R_{a^{-1}})_{*i(e)} \circ i_{*L_{a^{-1}}(a)} \circ (L_{a^{-1}})_{*a} \\
		&= (R_{a^{-1}})_{*e} \circ i_{*e} \circ (L_{a^{-1}})_{*a}
	\end{split}
\end{align}

dove abbiamo utilizzato che

\begin{gather}
	i(e) = e^{-1} = e \\
	L_{a^{-1}}(a) = a^{-1} a = e
\end{gather}

Dal Paragrafo \ref{example:differ-prod-inv-lie}, sappiamo che

\begin{equation}
	i_{*e} (Y_{e}) = - Y_{e}
\end{equation}

possiamo dunque applicare il differenziale a un qualunque vettore dello spazio tangente $ Y_{a} \in T_{a}(G) $, ottenendo

\begin{align}
	\begin{split}
		i_{*a}(Y_{a}) &= (R_{a^{-1}})_{*e} \circ i_{*e} \circ (L_{a^{-1}})_{*a} \\
		&= (- (R_{a^{-1}})_{*e} \circ (L_{a^{-1}})_{*a}) (Y_{a}) \\
		&= - (R_{a^{-1}})_{*e} (L_{a^{-1}})_{*a} (Y_{a})
	\end{split}
\end{align}
}

%=======================================================================================

\exer{Commutatore matriciale per algebre di Lie}
{exer3-9}
{
Verificare che il commutatore tra matrici

\begin{equation}
	[A,B] = AB - BA
\end{equation}

definisce un'algebra di Lie sullo spazio tangente all'identità dei gruppi: $ O(n) $, $ SO(n) $, $ U(n) $, $ SU(n) $, $ SL_{n}(\R) $, $ SL_{n}(\C) $.
}
{
Dal Corollario \ref{cor:incl-alg-lie-subg}, sappiamo che il commutatore dei sottogruppi matriciali di $ GL_{n}(\K) $ è dato dal commutatore standard

\begin{equation}
	[A,B] = AB - BA
\end{equation}

in quanto il commutatore del gruppo lineare induce il commutatore nei suoi sottogruppi tramite il differenziale dell'inclusione. \\
Consideriamo ora i seguenti insiemi

\begin{equation}
	\begin{cases}
		S^{-}_{n}(\K) = \{ X \in M_{n}(\K) \mid X^{T} = - X \} & \text{ (antisimmetriche)} \\
		H^{-}_{n} = \{ X \in M_{n}(\C) \mid X^{*} = - X \} & \text{ (antihermitiane)} \\
		T^{0}_{n}(\K) = \{ X \in M_{n}(\K) \mid \tr(X) = 0 \} & \text{ (traccia nulla)}
	\end{cases} %
	\qquad \K = \R, \C
\end{equation}

i quali permettono di scrivere le algebre di Lie relative ai gruppi di Lie in esame nel seguente modo:

\begin{align}
	O(n) &\longrightarrow \mathfrak{o}(n) = (S^{-}_{n}(\R),[,]) \\
	SO(n) &\longrightarrow \so(n) = (S^{-}_{n}(\R) \cap T^{0}_{n}(\R),[,]) \\
	U(n) &\longrightarrow \mathfrak{u}(n) = (H^{-}_{n},[,]) \\
	SU(n) &\longrightarrow \su(n) = (H^{-}_{n} \cap T^{0}_{n}(\C),[,]) \\
	SL_{n}(\R) &\longrightarrow \mathfrak{sl}(n,\R) = (T^{0}_{n}(\R),[,]) \\
	SL_{n}(\C) &\longrightarrow \mathfrak{sl}(n,\C) = (T^{0}_{n}(\C),[,])
\end{align}

dove l'insieme ambiente delle algebre è dato dallo spazio tangente all'identità del gruppo relativo. \\
Possiamo ora verificare che le condizioni valide per gli elementi dello spazio tangente all'unità dei vari gruppi valgano anche per i commutatori definiti sulla loro algebra. In particolare, dobbiamo verificare che i commutatori di matrici antisimmetriche e di matrici antihermitiane siano rispettivamente antisimmetrici e antihermitiani, e che la traccia di un commutatore di matrici a traccia nulla sia anch'essa nulla:

\begin{gather}
	[X,Y]^{T} = (X Y - Y X)^{T} = Y^{T} X^{T} - X^{T} Y^{T} = Y X - X Y = - [X,Y] %
	\qcomma \forall X,Y \in S^{-}_{n}(\K) \\
	[X,Y]^{*} = (X Y - Y X)^{*} = Y^{*} X^{*} - X^{*} Y^{*} = Y X - X Y = - [X,Y] %
	\qcomma \forall X,Y \in H^{-}_{n} \\
	\tr([X,Y]) = \tr(X Y - Y X) = \tr(X Y) - \tr(Y X) = \tr(X Y) - \tr(X Y) = 0
\end{gather}

dove abbiamo usato la proprietà

\begin{equation}
	\tr(A B) = \tr(B A) \qcomma \forall A,B \in M_{n}(\K)
\end{equation}

A questo punto i commutatori delle algebre in esame appartengono ai rispettivi spazi tangenti all'unità dei gruppi.
}

%=======================================================================================

\exer{Esponenziale per gruppi di Lie}
{exer3-10}
{
Verificare che l'esponenziale di una matrice definisce un'applicazione

\map{e}
	{T_{I}(G)}{G}
	{A}{e^{A}}

per i gruppi di Lie: $ GL_{n}(\R) $, $ GL_{n}(\C) $, $ SL_{n}(\R) $, $ SL_{n}(\C) $, $ O(n) $, $ SO(n) $, $ U(n) $, $ SU(n) $
}
{
Consideriamo i seguenti insiemi

\begin{equation}
	\begin{cases}
		S^{-}_{n}(\K) = \{ A \in M_{n}(\K) \mid A^{T} = - A \} & \text{ (antisimmetriche)} \\
		H^{-}_{n} = \{ A \in M_{n}(\C) \mid A^{*} = - A \} & \text{ (antihermitiane)} \\
		T^{0}_{n}(\K) = \{ A \in M_{n}(\K) \mid \tr(A) = 0 \} & \text{ (traccia nulla)}
	\end{cases} %
	\qquad \K = \R, \C
\end{equation}

i quali permettono di scrivere gli spazi tangenti ai gruppi di Lie in esame nel seguente modo:

\begin{align}
	T_{I}(GL_{n}(\K)) &= M_{n}(\K) \\
	T_{I}(SL_{n}(\K)) &= T^{0}_{n}(\K) \\
	T_{I}(O(n)) &= S^{-}_{n}(\R) \\
	T_{I}(SO(n)) &= S^{-}_{n}(\R) \cap T^{0}_{n}(\R) \\
	T_{I}(U(n)) &= H^{-}_{n} \\
	T_{I}(SU(n)) &= H^{-}_{n} \cap T^{0}_{n}(\C)
\end{align}

Calcoliamo ora il trasposto e il coniugato dell'esponenziale di una matrice $ A \in M_{n}(\K) $ tramite la definizione di esponenziale di matrice:

\begin{gather}
	\left( e^{A} \right)^{T} = \left( \sum_{+ \infty}^{i=1} \dfrac{A^{i}}{i !} \right)^{T} %
	= \sum_{+ \infty}^{i=1} \dfrac{(A^{i})^{T}}{i !} %
	= \sum_{+ \infty}^{i=1} \dfrac{\left( A^{T} \right)^{i}}{i !} %
	= e^{\left( A^{T} \right)} \\
	\nonumber \\
	\left( e^{A} \right)^{*} = \left( \sum_{+ \infty}^{i=1} \dfrac{A^{i}}{i !} \right)^{*} %
	= \sum_{+ \infty}^{i=1} \dfrac{(A^{i})^{*}}{i !} %
	= \sum_{+ \infty}^{i=1} \dfrac{\left( A^{*} \right)^{i}}{i !} %
	= e^{\left( A^{*} \right)}
\end{gather}

Consideriamo infine le seguenti implicazioni:

\begin{gather}
	A \in T^{0}_{n}(\K) %
	\implies \det(e^{A}) = e^{\tr(A)} = e^{0} = 1 %
	\implies e^{A} \in SL_{n}(\K) \\
	%
	A \in M_{n}(\K) %
	\implies \det(e^{A}) = e^{\tr(A)} \neq 0 %
	\implies e^{A} \in GL_{n}(\K) \\
	%
	A \in S^{-}_{n}(\R) %
	\implies e^{A} \left( e^{A} \right)^{T} = e^{A} e^{\left( A^{T} \right)} = e^{A} e^{-A} = e^{0} = I %
	\implies e^{A} \in O(n) \\
	%
	A \in H^{-}_{n} %
	\implies e^{A} \left( e^{A} \right)^{*} = e^{A} e^{\left( A^{*} \right)} = e^{A} e^{-A} = e^{0} = I %
	\implies e^{A} \in U(n)
\end{gather}

dove per la prima abbiamo usato la Proposizione \ref{prop:det-exp-tr} e per la seconda il Corollario \ref{cor:det-exp-gl}. \\
A questo punto, possiamo usare questi ragionamenti per verificare che l'esponenziale di matrice definisca un'applicazione dallo spazio tangente al gruppo matriciale al gruppo stesso:

\begin{gather}
	A \in T_{I}(GL_{n}(\K)) = M_{n}(\K) %
	\implies e^{A} \in GL_{n}(\K) \\
	%
	A \in T_{I}(SL_{n}(\K)) = T^{0}_{n}(\K) %
	\implies e^{A} \in SL_{n}(\K) \\
	%
	A \in T_{I}(O(n)) = S^{-}_{n}(\R) %
	\implies e^{A} \in O(n) \\
	%
	A \in T_{I}(SO(n)) = S^{-}_{n}(\R) \cap T^{0}_{n}(\R) %
	\implies e^{A} \in O(n) \cap SL_{n}(\R) = SO(n) \\
	%
	A \in T_{I}(U(n)) = H^{-}_{n} %
	\implies e^{A} \in U(n) \\
	%
	A \in T_{I}(SU(n)) = H^{-}_{n} \cap T^{0}_{n}(\C) %
	\implies e^{A} \in U(n) \cap SL_{n}(\C) = SU(n)
\end{gather}
}

%=======================================================================================

\exer{Algebra di Lie di prodotto diretto di gruppi di Lie}
{exer3-11}
{
Sia $ G = G_{1} \times \cdots \times G_{s} $ il prodotto diretto di gruppi di Lie. Dimostrare che l'algebra di Lie di $ G $ è isomorfa alla somma diretta delle algebre di Lie dei $ G_{i} $ con $ i=1,\dots,s $.
}
{
% https://math.stackexchange.com/questions/3724626/lieg-times-h-cong-lieg-oplus-lieh

In notazione, dobbiamo dimostrare la seguente proposizione per gruppi di Lie $ G_{i} $

\begin{equation}
	\bigoplus_{i=1}^{s} Lie(G_{i}) \stackrel{iso}{\simeq} Lie \left( \prod_{i=1}^{s} G_{i} \right) %
	\qcomma \forall s \in \N
\end{equation}

Semplifichiamo il problema al caso di due gruppi nel seguente modo:

\begin{equation}
	Lie(G) \oplus Lie(H) \stackrel{iso}{\simeq} Lie(G \times H) %
	\qq{con} %
	H = \prod_{i=1}^{s-1} G_{i}
\end{equation}

L'applicazione

\map{\psi}
	{Lie(G) \oplus Lie(H)}{Lie(G \times H)}
	{(\xi, \eta)}{\xi \oplus \eta}

è un isomorfismo di algebre di Lie se è un omomorfismo di algebre di Lie bigettivo; l'immagine $ \xi \oplus \eta $ è data da

\map{\xi \oplus \eta}
	{G \times H}{T(G \times H)}
	{(g,h)}{(\xi \oplus \eta)_{(g,h)} \in T_{(g,h)}(G \times H)}

definita a sua volta dall'isomorfismo\footnote{%
	Vedi Esercizio \ref{exer2-11}.%
}

\map{f_{(g,h)}}
	{T_{(g,h)}(G \times H)}{T_{g}(G) \times T_{h}(H)}
	{(\xi \oplus \eta)_{(g,h)}}
	{ %
		\left( (\pi_{G})_{*(g,h)} ((\xi \oplus \eta)_{(g,h)}), (\pi_{H})_{*(g,h)} ((\xi \oplus \eta)_{(g,h)}) \right) = (\xi_{g}, \eta_{h})
	}

con le proiezioni e i loro differenziali

\sbs{0.5}{%
			\map{\pi_{G}}
				{G \times H}{G}
				{(g,h)}{g}
			
			\map{(\pi_{G})_{*(g,h)}}
				{T_{(g,h)}(G \times H)}{T_{g}(G)}
				{(\xi \oplus \eta)_{(g,h)}}{\xi_{g}}
			}
	{0.5}{%
			\map{\pi_{H}}
				{G \times H}{H}
				{(g,h)}{h}
			
			\map{(\pi_{H})_{*(g,h)}}
				{T_{(g,h)}(G \times H)}{T_{h}(H)}
				{(\xi \oplus \eta)_{(g,h)}}{\eta_{h}}
			}

Consideriamo ora l'applicazione

\map{\varphi}
	{\mathfrak{X}(G) \oplus \mathfrak{X}(H)}{\mathfrak{X}(G \times H)}
	{(X,Y)}{X \oplus Y}

dove $ X \oplus Y $ è definito analogamente a $ \xi \oplus \eta $ e $ \mathfrak{X}(G) \doteq (\chi(G), [,]) $ è l'algebra di Lie dei campi di vettori lisci su $ G $ con il commutatore tra campi di vettori definito come

\begin{equation}
	[X,Y] = X Y - Y X \qcomma X, Y \in \chi(G)
\end{equation}

da cui l'algebra di Lie

\begin{equation}
	\mathfrak{X}(G) \oplus \mathfrak{X}(H) = (\chi(G) \times \chi(H),[,])
\end{equation}

con il commutatore di quest'ultima algebra definito come

\map{[,]}
	{\chi(G) \times \chi(H) \times \chi(G) \times \chi(H)}{\chi(G) \times \chi(H)}
	{(X_{1},Y_{1},X_{2},Y_{2})}
	{[(X_{1},Y_{1}),(X_{2},Y_{2})] \doteq \left( [X_{1},X_{2}], [Y_{1},Y_{2}] \right)}

Questa applicazione è un omomorfismo di algebre di Lie in quanto

\begin{align}
	\begin{split}
		\varphi \left( [(X_{1},Y_{1}),(X_{2},Y_{2})] \right) &\doteq \varphi \left( [X_{1},X_{2}], [Y_{1},Y_{2}] \right) \\
		&= [X_{1},X_{2}] \oplus [Y_{1},Y_{2}] \\
		&= [X_{1} \oplus Y_{1}, X_{2} \oplus Y_{2}] \\
		&= \left[ \varphi(X_{1}, Y_{1}), \varphi(X_{2}, Y_{2}) \right]
	\end{split}
\end{align}

dunque, essendo anche una bigezione, è un isomorfismo di algebre di Lie, i.e.

\begin{equation}
	\mathfrak{X}(G) \oplus \mathfrak{X}(H) \stackrel{iso}{\simeq} \mathfrak{X}(G \times H)
\end{equation}

A questo punto, siccome

\begin{equation}
	Lie(G) = (T_{e}(G),[,]) \stackrel{iso}{\simeq} (L(G),[,]) < (\chi(G),[,]) = \mathfrak{X}(G) %
	\; \because \; %
	\begin{cases}
		T_{e}(G) \stackrel{iso}{\simeq} L(G) \\
		L(G) \subset \chi(G)
	\end{cases}
\end{equation}

possiamo considerare la restrizione di $ \varphi $ alle algebre di Lie

\begin{equation}
	\psi \doteq \eval{\varphi}_{Lie(G) \oplus Lie(H)} : Lie(G) \oplus Lie(H) \to \mathfrak{X}(G \times H)
\end{equation}

Possiamo riscrivere quest'applicazione come

\map{\psi}
	{Lie(G) \oplus Lie(H)}{Lie(G \times H)}
	{(\xi,\eta)}{\xi \oplus \eta}

in quanto

\begin{equation}
	\psi(Lie(G) \oplus Lie(H)) \subset Lie(G \times H)
\end{equation}

i.e. vale l'espressione

\begin{gather}
	\xi \oplus \eta \in (L(G \times H),[,]) %
	\stackrel{iso}{\simeq} (T_{(e,e)}(G \times H),[,]) %
	= Lie(G \times H) %
	\qcomma \forall (\xi,\eta) \in Lie(G) \oplus Lie(H)
\end{gather}

Per dimostrare questo, ricordiamo la seguente definizione di campi di vettori invarianti a sinistra

\begin{equation}
	(L_{g})_{*h}(X_{h}) = X_{L_{g}(h)} = X_{gh} \qcomma \forall X \in L(G)
\end{equation}

e consideriamo il diagramma\footnote{%
	In generale, i.e. per i campi di vettori non invarianti a sinistra, il differenziale della traslazione a sinistra non lascia i campi di vettori invariati, dunque l'immagine di questa applicazione è rappresentata da un campo di vettori diverso da quello del dominio:
	%
	\begin{equation*}
		\xi'_{gh} \neq (L_{g})_{*h}(\xi_{h}) \qcomma \xi \in T(G)
	\end{equation*}%
}

\diagr{%
	{(\xi \oplus \eta)_{(g_{2}, h_{2})}} \arrow[rrrr, "{(L_{g_{1}} \times L_{h_{1}})_{*(g_{2},h_{2})}}", maps to] \arrow[ddd, "{f_{(g_{2},h_{2})}}"', maps to] \&  \&  \&  \& {(\xi' \oplus \eta')_{(g_{1} g_{2}, h_{1} h_{2})}} \arrow[ddd, "{f_{(g_{1} g_{2}, h_{1} h_{2})}}", maps to] \\
	\&  \&  \&  \&                                                                                                             \\
	\&  \&  \&  \&                                                                                                             \\
	{(\xi_{g_{2}}, \eta_{h_{2}})} \arrow[rrrr, "(L_{g_{1}})_{*g_{2}} \times (L_{h_{1}})_{*h_{2}}"', maps to]                                                  \&  \&  \&  \& {(\xi'_{g_{1} g_{2}}, \eta'_{h_{1} h_{2}})}                                                                
}

la quale commutatività permette di ottenere la seguente uguaglianza

\begin{equation}
	f_{(g_{1} g_{2}, h_{1} h_{2})} \circ (L_{g_{1}} \times L_{h_{1}})_{*(g_{2},h_{2})} \circ (f_{(g,h)})^{-1} = (L_{g_{1}})_{*g_{2}} \times (L_{h_{1}})_{*h_{2}}
\end{equation}

Applicando il differenziale della traslazione a sinistra ai campi di vettori $ \xi \oplus \eta $, otteniamo dunque che questi sono invarianti a sinistra:

\begin{align}
	\begin{split}
		(L_{g_{1}} \times L_{h_{1}})_{*(g_{2},h_{2})} &\left( (\xi \oplus \eta)_{(g_{2},h_{2})} \right) = \left( (L_{g_{1}} \times L_{h_{1}})_{*(g_{2},h_{2})} \circ (f_{(g_{2},h_{2})})^{-1} \right) \left( \xi_{g_{2}}, \eta_{h_{2}} \right) \\
		&= \left( (f_{(g_{1} g_{2}, h_{1} h_{2})})^{-1} \circ f_{(g_{1} g_{2}, h_{1} h_{2})} \circ (L_{g_{1}} \times L_{h_{1}})_{*(g_{2},h_{2})} \circ (f_{(g_{2},h_{2})})^{-1} \right) \left( \xi_{g_{2}}, \eta_{h_{2}} \right) \\
		&= \left( (f_{(g_{1} g_{2}, h_{1} h_{2})})^{-1} \circ \left( (L_{g_{1}})_{*g_{2}} \times (L_{h_{1}})_{*h_{2}} \right) \right) \left( \xi_{g_{2}}, \eta_{h_{2}} \right) \\
		&= (f_{(g_{1} g_{2}, h_{1} h_{2})})^{-1} \left( \left( (L_{g_{1}})_{*g_{2}} \times (L_{h_{1}})_{*h_{2}} \right) \left( \xi_{g_{2}}, \eta_{h_{2}} \right) \right) \\
		&= (f_{(g_{1} g_{2}, h_{1} h_{2})})^{-1} \left( (L_{g_{1}})_{*g_{2}} (\xi_{g_{2}}), (L_{h_{1}})_{*h_{2}} (\eta_{h_{2}}) \right) \\
		&= (f_{(g_{1} g_{2}, h_{1} h_{2})})^{-1} \left( \xi_{g_{1} g_{2}}, \eta_{h_{1} h_{2}} \right) \\
		&= (\xi \oplus \eta)_{(g_{1} g_{2}, h_{1} h_{2})}
	\end{split}
\end{align}

dove abbiamo usato il fatto che i campi di vettori $ \xi $ e $ \eta $ siano singolarmente invarianti a sinistra. \\
A questo punto, l'applicazione $ \psi $ è definita ed è un omomorfismo di algebre di Lie bigettivo (in quanto restrizione dell'isomorfismo $ \phi $), il che implica che $ \psi $ sia un isomorfismo, i.e.

\begin{equation}
	Lie(G) \oplus Lie(H) \stackrel{iso}{\simeq} Lie(G \times H)
\end{equation}

Il ragionamento vale iterativamente per $ H = \displaystyle\prod_{i=1}^{s-1} G_{i} $, da cui

\begin{equation}
	\bigoplus_{i=1}^{s} Lie(G_{i}) \stackrel{iso}{\simeq} Lie \left( \prod_{i=1}^{s} G_{i} \right) %
	\qcomma \forall s \in \N
\end{equation}
}

%=======================================================================================

\exer{Parallelizzabilità dei gruppi di Lie}
{exer3-12}
{
Dimostrare che ogni gruppo di Lie è parallelizzabile.
}
{
Una varietà differenziabile di dimensione $ n $ è parallelizzabile se esistono $ n $ vettori lisci linearmente indipendenti che formino una base per lo spazio tangente in ogni punto, i.e. per un gruppo di Lie $ G $

\begin{equation}
	\E \xi_{1}, \dots, \xi_{n} \in \chi(G) \mid %
	\B_{T_{g}(G)} = \left\{ \xi_{1}, \dots, \xi_{n} \right\} %
	\; \lor \; T_{g}(G) = \ev{(\xi_{i})_{g}}_{i=1,\dots,n} %
	\qcomma \forall g \in G
\end{equation}

Consideriamo dunque una base per lo spazio tangente all'unità

\begin{equation}
	\B_{T_{e}(G)} = \{(\xi_{i})_{e}\}_{i=1,\dots,n}
\end{equation}

Questa è formata da vettori lisci in quanto

\begin{equation}
	T_{e}(G) \stackrel{iso}{\simeq} L(G) \subset \chi(G)
\end{equation}

Essendo legato da un isomorfismo allo spazio tangente all'unità, l'insieme dei campi di vettori invarianti a sinistra possiede anch'esso una base di $ n $ vettori linearmente indipendenti, i.e.

\begin{equation}
	\B_{L(G)} = \left\{ (\hat{\xi}_{i})_{e} \right\}_{i=1,\dots,n} %
	= \{\xi_{i}\}_{i=1,\dots,n}
\end{equation}

A questo punto, applichiamo il differenziale nell'unità della traslazione a sinistra agli elementi della base di $ L(G) $

\begin{equation}
	(L_{g})_{*e} ((\xi_{i})_{e}) = (\xi_{i})_{g} \in T_{g}(G) \qcomma \forall g \in G, \, i=1,\dots,n
\end{equation}

Siccome la traslazione a sinistra è un diffeomorfismo, il suo differenziale è un isomorfismo, dunque porta basi in basi: da questo otteniamo che

\begin{equation}
	\B_{T_{g}(G)} = \{(\xi_{i})_{g}\}_{i=1,\dots,n} \qcomma \forall g \in G
\end{equation}

è una base di $ n $ vettori lisci linearmente indipendenti per lo spazio tangente a ogni punto del gruppo di Lie e dunque che qualsiasi gruppo di Lie è parallelizzabile.

\paragraph{Fibrato tangente triviale (dimostrazione alternativa)}

% https://math.stackexchange.com/questions/308691/let-g-be-a-lie-group-show-that-there-is-a-diffeomorphism-tg-cong-g-times-t

La condizione di parallelizzabilità è equivalente a dire che il fibrato tangente è triviale, i.e. sono isomorfi come spazi vettoriali i seguenti spazi

\begin{equation}
	T(G) \stackrel{iso}{\simeq} G \times T_{g}(G) \qcomma \forall g \in G
\end{equation}

Siccome la traslazione a sinistra è un diffeomorfismo, il suo differenziale è un isomorfismo di spazi vettoriali, i.e.

\begin{equation}
	(L_{g})_{*e} : T_{e}(G) \to T_{g}(G) %
	\implies %
	T_{e}(G) \stackrel{iso}{\simeq} T_{g}(G) \qcomma \forall g \in G
\end{equation}

questo permette di riscrivere la condizione di trivialità del fibrato tangente come

\begin{equation}
	T(G) \stackrel{iso}{\simeq} G \times T_{e}(G)
\end{equation}

Consideriamo quindi l'applicazione

\map{\varphi}
	{G \times T_{e}(G)}{T(G)}
	{(g, \xi)}{(L_{g})_{*}(\xi)}

dal momento che i campi di vettori dello spazio tangente a un gruppo di Lie sono invarianti a sinistra, i.e.

\begin{equation}
	(L_{g})_{*}(\xi) = \xi \qcomma \forall \xi \in T_{e}(G) \stackrel{iso}{\simeq} L(G)
\end{equation}

possiamo scrivere $ \varphi(g, \xi) = \xi $. \\
Perché $ \varphi $ sia un isomorfismo di spazi vettoriali è necessario che sia una bigezione e che sia $ \R $-lineare. \\
Per dimostrare che sia una bigezione, è sufficiente dimostrare che possieda un'inversa destra (i.e. sia suriettiva, perciò $ \Im(\varphi) = T(G) $) in quanto, per il teorema della dimensione e per il fatto che dominio e codominio hanno la stessa dimensione, i.e.

\begin{equation}
	\dim(G \times T_{e}(G)) = \dim(T(G)) = 2 \dim(G)
\end{equation}

da cui

\begin{align}
	\begin{split}
		\dim(\Im(\varphi)) + \dim(\ker(\varphi)) &= \dim(G \times T_{e}(G)) \\
		\cancel{ \dim(T(G)) } + \dim(\ker(\varphi)) &= \cancel{ \dim(G \times T_{e}(G)) } \\
		\dim(\ker(\varphi)) &= 0
	\end{split}
\end{align}

questo mostra che la dimensione del nucleo sia nulla e dunque che sia anche iniettiva; di seguito la composizione dell'inversa destra e la verifica:

\sbs{0.5}{%
			\begin{equation}
				\psi \doteq (\pi_{G} \times i^{-1}) \circ \Delta : T(G) \to G \times T_{e}(G)
			\end{equation}
			
			\map{\pi_{G}}
				{T(G)}{G}
				{\xi}{g}
			
			\centering dove $ (g,\xi) \equiv \xi \in T_{g}(G) $
			}
	{0.5}{%
			\map{i}
				{T_{e}(G)}{T(G)}
				{\xi}{\xi}
			
			\map{\Delta}
				{T(G) \times T(G)}{T(G)}
				{\xi}{(\xi,\xi)}
			}

\begin{gather}
	(\varphi \circ \psi)(\xi) = (\varphi \circ ((\pi_{G} \times i^{-1}) \circ \Delta))(\xi) %
	= \varphi ((\pi_{G} \times i^{-1})(\xi,\xi)) %
	= \varphi(g,\xi) %
	= \xi %
	\qcomma \forall \xi \in T(G) \nonumber \\
	\Downarrow \\
	\varphi \circ \psi = \id_{T(G)} \nonumber
\end{gather}

Per la $ \R $-linearità:

\begin{align}
	\begin{split}
		\varphi(\alpha (g,\xi) + \beta (h,\eta)) &= \varphi((\alpha g,\alpha \xi) + (\beta h,\beta \eta)) \\
		&= \varphi((\alpha g + \beta h,\alpha \xi + \beta \eta)) \\
		&= \alpha \xi + \beta \eta \\
		&= \alpha \varphi(g,\xi) + \beta \varphi(h,\eta)
	\end{split} %
	\qquad %
	\begin{cases}
		\forall (g,\xi),(h,\eta) \in G \times T_{e}(G) \\
		\forall \alpha,\beta \in \R
	\end{cases}
\end{align}

con la moltiplicazione per scalare e la somma standard del dominio e del codominio di $ \varphi $. \\
A questo punto, per il ragionamento precedente, l'applicazione $ \varphi $ è una bigezione $ \R $-lineare, dunque un isomorfismo di spazi vettoriali, i.e. vale per qualunque gruppo di Lie $ G $ il seguente isomorfismo

\begin{equation}
	T(G) \stackrel{iso}{\simeq} G \times T_{e}(G)
\end{equation}

il che rende triviale il fibrato tangente di un gruppo di Lie, equivalente a dimostrare che ogni gruppo di Lie è parallelizzabile. \\
Incidentalmente, come implicazione inversa, otteniamo che ogni spazio non parallelizzabile non può essere dotato della struttura di gruppo di Lie.
}


%

\addcontentsline{toc}{chapter}{Bibliography}

\printbibliography[
	title={Bibliography}
]

\end{document}
