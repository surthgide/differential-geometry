% Simone Iovine's Differential Geometry Notes

\documentclass[12pt]{report}

\usepackage[utf8]{inputenc}
\usepackage[english]{babel}
\usepackage{datetime}
\usepackage{mathtools,amsthm,amssymb}
\usepackage{mathtools}
\usepackage{cases}
\usepackage{centernot}
\usepackage[makeroom]{cancel}
\usepackage{graphics,graphicx}
\graphicspath{ {C:/Users/simon/Desktop/Simo/"geometria differenziale"/images/} }
\usepackage[a4paper,width=170mm,top=25mm,bottom=25mm]{geometry}
\usepackage{fancyhdr}
\usepackage{setspace}
\usepackage{float}
\usepackage[svgnames]{xcolor}
\usepackage{tikz,pgfplots,tikz-3dplot}
%\usepackage{blindtext}
%\usepackage{pgfplots}
\usetikzlibrary{3d,calc,decorations.pathmorphing,patterns}
\usepackage{xfrac}
\usepackage{multirow,multicol}
\usepackage{physics}
\usepackage{xcolor}
\usepackage{tcolorbox}
\usepackage{enumitem}
\usepackage{bbm}
\usepackage[toc]{appendix}
\usepackage{parskip}
\usepackage{tikz-cd}
\usepackage{ifthen}

% symbols definition

\newcommand{\id}{\operatorname{id}}
\newcommand{\bigone}{\mathbbm{1}}
\newcommand{\der}{\operatorname{Der}}
\newcommand{\diag}{\operatorname{diag}}
\newcommand{\supp}{\operatorname{supp}}
\newcommand{\ob}{\operatorname{Ob}}
\newcommand{\mor}{\operatorname{Mor}}
\newcommand{\st}{\, \middle| \,}
\newcommand{\hatapp}{\; \hat{} \;}
\DeclarePairedDelimiter{\ceil}{\lceil}{\rceil}

%\newcommand{\ie}{i.e. \phantom{$ \!\! $}}

\newcommand{\N}{\mathbb{N}}
\newcommand{\Q}{\mathbb{Q}}
\newcommand{\Z}{\mathbb{Z}}
\newcommand{\R}{\mathbb{R}}
\newcommand{\C}{\mathbb{C}}
\newcommand{\K}{\mathbb{K}}
\newcommand{\T}{\mathbb{T}}
\renewcommand{\S}{\mathbb{S}}

\newcommand{\rp}[1]{\R\mathcal{P}^{#1}}
\newcommand{\B}{\mathcal{B}}

\newcommand{\PC}{\mathcal{PC}}
\newcommand{\PR}{\mathcal{PR}}
\newcommand{\VC}{\mathcal{VC}}
\newcommand{\VR}{\mathcal{VR}}

\newcommand{\g}{\mathfrak{g}}
\newcommand{\h}{\mathfrak{h}}
\newcommand{\so}{\mathfrak{so}}
\newcommand{\su}{\mathfrak{su}}

\newcommand{\notimplies}{\centernot\implies}
\newcommand{\E}{\exists \;}
\newcommand{\ps}{\mathcal{P}}

\newcommand{\hal}{\hspace{13px}}

\DeclareDocumentCommand\dpdv{}{\displaystyle\partialderivative}
\DeclareDocumentCommand\ddv{}{\displaystyle\derivative}

% map definition

\newcommand{\map}[5]{
	\begin{align}
		\begin{split}
			#1 : #2 &\to #3 \\
			#4 &\mapsto #5
		\end{split}
	\end{align}
}

% map w/o equation number

\newcommand{\maps}[5]{
	\begin{align*}
		\begin{split}
			#1 : #2 &\to #3 \\
			#4 &\mapsto #5
		\end{split}
	\end{align*}
}

% image definition

\newcommand{\img}[2]{
	\begin{figure}[H]
		\centering
		\includegraphics[width=#1\textwidth,keepaspectratio]{#2}
	\end{figure}
}

% diagram definition

% mind this link for why & -> \&:
% https://tex.stackexchange.com/questions/15093/single-ampersand-used-with-wrong-catcode-error-using-tikz-matrix-in-beamer

\newcommand{\diagr}[1]{
	\begin{figure}[H]
		\centering
		\begin{tikzcd}[ampersand replacement=\&]
			#1
		\end{tikzcd}
	\end{figure}
}

% side-by-side (two neighbouring boxes)

% \sbs{[width left box]}{[content left box]}{[width right box]}{[content right box]}

\newcommand{\sbs}[4]{
	\noindent\begin{minipage}[c]{#1\textwidth}
		#2
	\end{minipage}
	\begin{minipage}[c]{#3\textwidth}
		#4
	\end{minipage}
}

%

% exercise macro

\newcounter{solutions}
\setcounter{solutions}{42} % change this number from 42 to whatever to hide solutions

% \exer{[title]}{[label]}{[exercise text]}{[solution]}

\newcommand{\exer}[4]{
	\section{#1}\label{#2}
	
	\begin{tcolorbox}
		#3
	\end{tcolorbox}
	
	\ifthenelse{\thesolutions = 42}
	{#4}
	{}
	
	%
	
	\newpage
}

%

%\renewcommand{\contentsname}{Indice}

%\makeatletter
%\renewcommand{\@chapapp}{Capitolo}
%\makeatother

% environments names

\newtheorem{theorem}{Theorem}
\newtheorem{corollary}{Corollary}[theorem]
\newtheorem{lemma}[theorem]{Lemma}
\newtheorem*{remark}{Remark}
\newtheorem{definition}{Proposition}[theorem]

%\renewcommand*{\proofname}{Dimostrazione}
\renewcommand\qedsymbol{$\square$}

% \highlight[<colour>]{<stuff>}

\newcommand{\highlight}[2][yellow]{\mathchoice%
	{\colorbox{#1}{$\displaystyle#2$}}%
	{\colorbox{#1}{$\textstyle#2$}}%
	{\colorbox{#1}{$\scriptstyle#2$}}%
	{\colorbox{#1}{$\scriptscriptstyle#2$}}}%

\definecolor{hlc}{RGB}{204,241,202} % hlc = highlight colour

\newcommand{\hl}[1]{%
	\highlight[hlc]{#1}}

% blue links for footsnotes and stuff

\usepackage[
	bookmarksnumbered = true,
	linktocpage = true
]{hyperref}

\hypersetup{
	colorlinks = true,
	linkcolor = blue,
	anchorcolor = blue,
	citecolor = blue,
	filecolor = blue,
	urlcolor = blue
}

% page layout

\pagestyle{fancy}
\renewcommand{\chaptermark}[1]{ \markboth{#1}{} }

\fancyhead{}
\fancyhead[R]{Differential Geometry Notes}
\fancyhead[L]{\leftmark}
\fancyfoot{}
\fancyfoot[L]{Simone Iovine}
\fancyfoot[R]{\thepage}
\renewcommand{\headrulewidth}{0.4pt}
\renewcommand{\footrulewidth}{0.4pt}
\renewcommand{\footnoterule}{\vfill\kern -3pt \hrule width 0.4\columnwidth \kern 2.6pt}

% bibliography

\usepackage[
	backend = bibtex,
	style = nature,
	natbib = true,
	sorting = none,
	autocite = inline
]{biblatex}
% bibliography is sorted in the same order in which the articles are cited

\addbibresource{chapters/biblio} % imports bibliography file

% table of contents

\setcounter{tocdepth}{3}
%\setcounter{secnumdepth}{3}

% title page stuff

\title{\textbf{Differential Geometry Notes}}
\author{Simone Iovine}
\date{\today}

\begin{document}

\maketitle

%

\pagenumbering{roman}

%

\tableofcontents

%

\chapter*{Notes}
\addcontentsline{toc}{chapter}{Notes}
The following notes are a revision of the notes taken during prof. Andrea Loi's online lessons of Differential geometry 2020-2021 (Mathematics dept., Cagliary University). \\
Some definitions are taken from \textit{Introduzione alla Topologia Generale} of Andrea Loi \autocite{loi}. \\
The professor followed the following texts during the course: \textit{Introduction to Smooth Manifolds} di John M. Lee \cite{lee} e \textit{An Introduction to Manifolds} di Loring W. Tu \cite{tu}.

Professor's site: \url{https://loi.unica.it/geomdiff2021.html}

%

\chapter*{Notation}
\addcontentsline{toc}{chapter}{Notation}
\begin{table}[H]
	\sbs{0.5}{%
				\begin{tabular}{|c|p{0.65\linewidth}|}
					\hline
					\textbf{Symbol} & \textbf{Meaning} \\
					\hline
					\hline
					$ = $ & equality \\
					\hline
					$ \equiv $ & identity \\
					\hline
					$ \{\dots\} $ & set elements \\
					\hline
					$ \E $ & exists \\
					\hline
					$ \E ! $ & only one exists \\
					\hline
					$ \forall $ & for all \\
					\hline
					$ \in $ & belongs to \\
					\hline
					$ \implies $ & implies (sufficient) \\
					\hline
					$ \impliedby $ & implied by (necessary) \\
					\hline
					$ \iff $ & if and only if \\
					\hline
					$ \subset $ & is a subset of/included \\
					\hline
					$ \subseteq $ & included or equal \\
					\hline
					$ \supset $ & includes \\
					\hline
					$ \supseteq $ & includes or equal \\
					\hline
					$ \setminus $ & set difference \\
					\hline
					$ \cap $ & intersection \\
					\hline
					$ \cup $ & union \\
					\hline
					$ \emptyset $ & empty set \\
					\hline
					$ \sqcup $ & disjoint union \\
					\hline
					$ \ps(S) $ & power set of $ S $ \\
					\hline
					$ \times $ & direct product \\
					\hline
					$ \oplus $ & direct sum \\
					\hline
					$ \to $ & function/morphism \\
					\hline
					$ \mapsto $ & maps to \\
					\hline
					$ \circ $ & composition \\
					\hline
					$ \eval{f}_{U} $ & $ f $ evaluated in $ U $ \\
					\hline
					$ \id $ & identity \\
					\hline
					$ \therefore $ & therefore \\
					\hline
					$ \because $ & because \\
					\hline
					$ \land $ & logic and \\
					\hline
					$ \lor $ & logic or \\
					\hline
					$ \infty $ & infinity \\
					\hline
					$ \mid $ & such that \\
					\hline
					$ \sim $ & equivalence \\
					\hline
					$ \sfrac{S}{\sim} $ & quotient \\
					\hline
				\end{tabular}
				}
		{0.5}{%
				\begin{tabular}{|c|p{0.65\linewidth}|}
					\hline
					\textbf{Symbol} & \textbf{Meaning} \\
					\hline
					\hline
					$ \stackrel{iso}{\simeq} $ & isomorphism \\
					\hline
					$ \stackrel{omeo}{\simeq} $ & omeomorphism \\
					\hline
					$ \stackrel{diff}{\simeq} $ & diffeomorphism \\
					\hline
					$ \stackrel{omo}{\simeq} $ & omomorphism \\
					\hline
					$ \N $ & natural numbers \\
					\hline
					$ \Z $ & integer numbers \\
					\hline
					$ \Q $ & rational numbers \\
					\hline
					$ \R $ & real numbers \\
					\hline
					$ \C $ & complex numbers \\
					\hline
					$ \K $ & $ \R $ or $ \C $ \\
					\hline
					$ \T^{n} $ & $ n $-dimensional torus \\
					\hline
					$ \S^{n} $ & $ n $-dimensional sphere \\
					\hline
					$ \rp{n} $ & $ n $-dimensional real projective space \\
					\hline
					$ \B $ & base \\
					\hline
					$ \ev{v} $ & spans \\
					\hline
					$ \PC $ &  critical point \\
					\hline
					$ \PR $ &  regular points \\
					\hline
					$ \VC $ & critic values \\
					\hline
					$ \VR $ & regular values \\
					\hline
					$ \g $ & Lie algebra (associated to $ G $) \\
					\hline
					$ \sum_{i=1}^{n} $ & summation from $ 1 $ to $ n $ \\
					\hline
					$ \prod_{i=1}^{n} $ & product from $ 1 $ to $ n $ \\
					\hline
					$ \norm{v} $ & modulo/norm of $ v $ \\
					\hline
					$ \det $ & determinant \\
					\hline
					$ \tr $ & trace \\
					\hline
					$ \sbmqty{ a & b \\ c & d } $ & matrix \\
					\hline
					$ \smdet{ a & b \\ c & d } $ & matrix determinant \\
					\hline
					$ \bigone $ & identity matrix \\
					\hline
					$ \supp $ & support \\
					\hline
					$ \ob $ & objects (category) \\
					\hline
					$ \mor $ & morphisms (category) \\
					\hline
					$ \lceil v \rceil $ & "ceiling" function \\
					\hline
					i.e. & means that (\textit{id est}) \\
					\hline
					e.g. & as an example (\textit{exempli gratia}) \\
					\hline
				\end{tabular}
				}
\end{table}


%

\newpage

%

\pagenumbering{arabic}

%

\chapter{Differential geometry in euclidean spaces}
\input{chapters/eucldiffgeo}

%

\chapter{Differential manifolds}
%\input{chapters/diffman}

%

\chapter{Lie groups and algebras}
%\input{chapters/liegroupsalg}

%

\appendix

\makeatletter
\renewcommand{\@chapapp}{Exercises}
\makeatother

\chapter{Exercises: Differential geometry in euclidean spaces}
%\input{chapters/exercisesA}

%

\chapter{Exercises: Differential manifolds}
%\input{chapters/exercisesB}

%

\chapter{Exercises: Lie groups and algebras}
%\input{chapters/exercisesC}

%

\addcontentsline{toc}{chapter}{Bibliography}

\printbibliography[
	title={Bibliography}
]

\end{document}
