\begin{table}[H]
	\sbs{0.5}{%
				\begin{tabular}{|c|p{0.65\linewidth}|}
					\hline
					\textbf{Simbolo} & \textbf{Significato} \\
					\hline
					\hline
					$ = $ & uguaglianza \\
					\hline
					$ \equiv $ & coincide \\
					\hline
					$ \{\dots\} $ & elementi di insieme \\
					\hline
					$ \E $ & esiste \\
					\hline
					$ \E ! $ & esiste ed è unico \\
					\hline
					$ \forall $ & qualsiasi \\
					\hline
					$ \in $ & appartenente \\
					\hline
					$ \implies $ & implica (sufficiente) \\
					\hline
					$ \impliedby $ & è implicato da (necessario) \\
					\hline
					$ \iff $ & se e solo se \\
					\hline
					$ \subset $ & contenuto \\
					\hline
					$ \subseteq $ & contenuto o uguale \\
					\hline
					$ \supset $ & contiene \\
					\hline
					$ \supseteq $ & contiene o uguale \\
					\hline
					$ \setminus $ & differenza (insiemi) \\
					\hline
					$ \cap $ & intersezione \\
					\hline
					$ \cup $ & unione \\
					\hline
					$ \emptyset $ & insieme vuoto \\
					\hline
					$ \sqcup $ & unione disgiunta \\
					\hline
					$ \ps(S) $ & insieme delle parti di $ S $ \\
					\hline
					$ \times $ & prodotto diretto \\
					\hline
					$ \oplus $ & somma diretta \\
					\hline
					$ \to $ & mappa \\
					\hline
					$ \mapsto $ & associa \\
					\hline
					$ \circ $ & composizione \\
					\hline
					$ \eval{f}_{U} $ & $ f $ valutata in $ U $ \\
					\hline
					$ \id $ & identità \\
					\hline
					$ \therefore $ & quindi \\
					\hline
					$ \because $ & poiché \\
					\hline
					$ \land $ & "e" logico \\
					\hline
					$ \lor $ & "o" logico \\
					\hline
					$ \infty $ & infinito \\
					\hline
					$ \mid $ & tale che \\
					\hline
					$ \sim $ & equivalenza \\
					\hline
					$ \sfrac{S}{\equiv} $ & quoziente \\
					\hline
				\end{tabular}
				}
		{0.5}{%
				\begin{tabular}{|c|p{0.65\linewidth}|}
					\hline
					\textbf{Simbolo} & \textbf{Significato} \\
					\hline
					\hline
					$ \stackrel{iso}{\simeq} $ & isomorfismo \\
					\hline
					$ \stackrel{omeo}{\simeq} $ & omeomorfismo \\
					\hline
					$ \stackrel{diff}{\simeq} $ & diffeomorfismo \\
					\hline
					$ \stackrel{omo}{\simeq} $ & omomorfismo \\
					\hline
					$ \N $ & numeri naturali \\
					\hline
					$ \Z $ & numeri interi \\
					\hline
					$ \Q $ & numeri razionali \\
					\hline
					$ \R $ & numeri reali \\
					\hline
					$ \C $ & numeri complessi \\
					\hline
					$ \K $ & $ \R $ oppure $ \C $ \\
					\hline
					$ \T^{n} $ & toro $ n $-dimensionale \\
					\hline
					$ \S^{n} $ & sfera $ n $-dimensionale \\
					\hline
					$ \rp{n} $ & proiettivo reale $ n $-dimensionale \\
					\hline
					$ \B $ & base \\
					\hline
					$ \ev{v} $ & genera \\
					\hline
					$ \PC $ &  punti critici \\
					\hline
					$ \PR $ &  punti regolari \\
					\hline
					$ \VC $ & valori critici \\
					\hline
					$ \VR $ & valori regolari \\
					\hline
					$ \g $ & algebra di Lie (associata a $ G $) \\
					\hline
					$ \sum_{i=1}^{n} $ & sommatoria da $ 1 $ a $ n $ \\
					\hline
					$ \prod_{i=1}^{n} $ & produttoria da $ 1 $ a $ n $ \\
					\hline
					$ \norm{v} $ & modulo/norma di $ v $ \\
					\hline
					$ \det $ & determinante \\
					\hline
					$ \tr $ & traccia \\
					\hline
					$ \sbmqty{ a & b \\ c & d } $ & matrice \\
					\hline
					$ \smdet{ a & b \\ c & d } $ & determinante di matrice \\
					\hline
					$ \bigone $ & matrice unitaria/identità \\
					\hline
					$ \supp $ & supporto \\
					\hline
					$ \ob $ & oggetti (categoria) \\
					\hline
					$ \mor $ & morfismi (categoria) \\
					\hline
					$ \lceil v \rceil $ & funzione "soffitto" \\
					\hline
					i.e. & cioè (\textit{id est}) \\
					\hline
					e.g. & ad esempio (\textit{exempli gratia}) \\
					\hline
				\end{tabular}
				}
\end{table}
