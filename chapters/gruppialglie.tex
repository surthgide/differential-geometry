%% map definition
%
%\newcommand{\R}{\mathbb{R}}
%\renewcommand{\S}{\mathbb{S}}
%
%\newcommand{\map}[5]{
%	\begin{align}
%		\begin{split}
%			#1 : #2 &\to #3\\
%			#4 &\mapsto #5
%		\end{split}
%	\end{align}
%}
%
%%
%
%% image definition
%
%\newcommand{\img}[2]{
%	\begin{figure}[H]
%		\centering
%		\includegraphics[width=#1\textwidth,keepaspectratio]{#2}
%	\end{figure}
%}
%
%%

\section{Gruppi di Lie}

Un gruppo $ G $ è un \textit{gruppo di Lie} se l'insieme del gruppo è una varietà differenziabile, è un gruppo algebrico e se le sue operazioni

\begin{align}
	\begin{split}
		\mu : G \times G &\to G\\
		(a,b) &\mapsto a b\\\\
		%
		i : G &\to G\\
		a &\mapsto a^{-1}
	\end{split}
\end{align}

sono lisce.\\\\
%
Le traslazioni possono essere a sinistra e a destra:

\begin{itemize}
	\item Dato $ a \in G $, la traslazione a sinistra
	
	\map{L_{a}}{G}{G}{b}{a b}
	
	è un'applicazione liscia, in quanto restrizione di un'applicazione liscia (i.e. $ L_{a} = \mu(a,\cdot) $), ed è inoltre un diffeomorfismo, in quanto $ L_{a}^{-1} = L_{a^{-1}} $ è ancora liscia;
	
	\item Dato $ a \in G $, la traslazione a destra
	
	\map{R_{a}}{G}{G}{b}{b a}
	
	è un'applicazione liscia, in quanto restrizione di un'applicazione liscia (i.e. $ R_{a} = \mu(\cdot,a) $), ed è inoltre un diffeomorfismo, in quanto $ R_{a}^{-1} = R_{a^{-1}} $ è ancora liscia.
\end{itemize}

\subsubsection{\textit{Esempi}}

\paragraph{1. Gruppo lineare $ GL_{n}(\R) $}

Il gruppo delle matrici invertibili

\begin{equation}
	GL_{n}(\R) = \{ A \in M_{n}(\R) \mid \det(A) \neq 0 \} \subset M_{n}(\R) = \R^{n^{2}}
\end{equation}

con la struttura differenziale ereditata da $ M_{n}(\mathbb{R}) = \mathbb{R}^{n^{2}} $ in quanto $ GL_{n}(\mathbb{R}) $ è aperto in questo spazio, è un gruppo di Lie rispetto alla moltiplicazione e l'inversione:

\begin{align}
	\begin{split}
		\mu : GL_{n}(\R) \times GL_{n}(\R) &\to GL_{n}(\R)\\
		(A,B) &\mapsto A B\\\\
		%
		i : GL_{n}(\R) &\to GL_{n}(\R)\\
		A &\mapsto A^{-1}
	\end{split}
\end{align}

le quali sono lisce.

\paragraph{2. Gruppo lineare speciale $ SL_{n}(\R) $}

Il gruppo delle matrici con determinante unitario

\begin{equation}
	SL_{n}(\R) = \{ A \in M_{n}(\R) \mid \det(A) = 1 \} \subset M_{n}(\R) = \R^{n^{2}}
\end{equation}

è un gruppo di Lie rispetto alla moltiplicazione e l'inversione:

\begin{align}
	\begin{split}
		\mu : SL_{n}(\R) \times SL_{n}(\R) &\to SL_{n}(\R)\\
		(A,B) &\mapsto A B\\\\
		%
		i : SL_{n}(\R) &\to SL_{n}(\R)\\
		A &\mapsto A^{-1}
	\end{split}
\end{align}

in quanto sottovarietà di $ GL_{n}(\R) $ (e quindi una varietà) di dimensione $ n^{2}-1 $ e gruppo algebrico rispetto alle sue operazioni, le quali sono lisce.\\
Dimostriamo ora che la moltiplicazione è liscia\footnote{%
	Questa dimostrazione è già stata fatta nell'Esempio \ref{ex-slnr}.%
}: prendiamo la moltiplicazione (liscia) in $ GL_{n}(\R) $

\map{F}{GL_{n}(\R) \times GL_{n}(\R)}{GL_{n}(\R)}{(A,B)}{A B}

e l'inclusione liscia

\begin{equation}
	i \times i : SL_{n}(\R) \times SL_{n}(\R) \to GL_{n}(\R) \times GL_{n}(\R)
\end{equation}

L'applicazione $ G = F \circ (i \times i) $ è dunque liscia perché composizione di applicazione lisce; siccome il prodotto di due matrici con determinante unitario (i.e. in $ SL_{n}(\R) $) è ancora una matrice con determinante unitario, l'immagine di $ G $

\begin{equation}
	G(SL_{n}(\R) \times SL_{n}(\R)) \subset SL_{n}(\R)
\end{equation}

A questo punto, dato che $ SL_{n}(\R) $ è una sottovarietà di $ GL_{n}(\R) $ (ipotesi del teorema che permette di mantenere l'applicazione liscia), possiamo considerare la restrizione del codominio della funzione $ G $

\map{\tilde{G}}{SL_{n}(\R) \times SL_{n}(\R)}{SL_{n}(\R)}{(A,B)}{A B}

la quale è identica a $ \mu $, dunque $ \mu \in C^{\infty}(SL_{n}(\R) \times SL_{n}(\R)) $.\\
Per dimostrare che l'inversione sia liscia, consideriamo l'inversione in $ GL_{n}(\R) $

\map{F}{GL_{n}(\R)}{GL_{n}(\R)}{A}{A^{-1}}

la quale è liscia e, analogamente per la moltiplicazione, definiamo $ G = F \circ i $

\map{G}{SL_{n}(\R)}{GL_{n}(\R)}{A}{A^{-1}}

e dunque la restrizione del suo codominio a $ SL_{n}(\R) $ (sottovarietà di $ GL_{n}(\R) $)

\map{\tilde{G}}{SL_{n}(\R)}{SL_{n}(\R)}{A}{A^{-1}}

funzione che coincide con l'inversione in $ SL_{n}(\R) $, rendendo dunque l'inversione  $ i $ liscia.

\paragraph{3. Gruppo ortogonale $ O(n) $}

Il gruppo delle matrici ortogonali

\begin{equation}
	O(n) = \{ A \in GL_{n}(\R) \mid A^{T} A = I \} \subset GL_{n}(\R)
\end{equation}

è una sottovarietà di $ GL_{n}(\R) $ (dal teorema della preimmagine di un'applicazione di rango costante).\\
Il ragionamento per cui $ O(n) $ sia un gruppo di Lie rispetto alla moltiplicazione e l'inversione

\begin{align}
	\begin{split}
		\mu : O(n) \times O(n) &\to O(n)\\
		(A,B) &\mapsto A B\\\\
		%
		i : O(n) &\to O(n)\\
		A &\mapsto A^{-1}
	\end{split}
\end{align}

è analogo a quello fatto per $ SL_{n}(\R) $.\\
Calcoliamo ora la dimensione di $ O(n) $ come varietà e il suo spazio tangente nell'identità $ T_{I}(O(n)) $. Consideriamo l'insieme delle matrici simmetriche di ordine $ n $

\begin{equation}
	S(n) = \{ B \in M_{n}(\R) \mid B^{T} = B \}
\end{equation}

e l'applicazione liscia

\map{f}{GL_{n}(\R)}{S(n)}{A}{A^{T} A}

dove $ (A^{T} A)^{T} = A^{T} A $ dunque $ A^{T} A \in S(n) $. Dimostrando che $ I \in \mathcal{VR}_{f} $, per il teorema della preimmagine di un valore regolare, avremmo che $ O(n) $ è una sottovarietà di $ GL_{n}(\R) $ con dimensione

\begin{equation}
	\dim(O(n)) = \dim(GL_{n}(\R)) - \dim(S(n))
\end{equation}

dove $ S(n) $ è uno spazio vettoriale su $ \R $ e anche un sottospazio vettoriale di $ M_{n}(\R) $, la cui dimensione è data dalla quantità di condizioni necessarie al fine di determinare una matrice simmetrica, i.e.

\begin{equation}
	\dim(S(n)) = \dfrac{n (n+1)}{2}
\end{equation}

Siccome $ \dim(GL_{n}(\R)) = n^{2} $, otteniamo che

\begin{equation}
	\dim(O(n)) = n^{2} - \dfrac{n (n+1)}{2} = \dfrac{n (n-1)}{2}
\end{equation}

Verifichiamo ora che $ I \in \mathcal{VR}_{f} $: per fare ciò, dobbiamo calcolare il differenziale di $ f $ e controllare che tutti i punti che stanno nell'immagine di $ I $ attraverso $ f_{*} $ siano punti regolari

\begin{equation}
	f_{*A} : T_{A}(GL_{n}(\R)) \to T_{f(A)}(S(n))
\end{equation}

possiamo identificare $ T_{A}(GL_{n}(\R)) = M_{n}(\R) $ e $ T_{f(A)}(S(n)) = S(n) $ (in quanto $ S(n) $ è uno spazio vettoriale), perciò $ f_{*} $ porta matrici in matrici simmetriche. Calcoliamo dunque il differenziale $ f_{*A}(X) $ prendendo una curva che passi per $ A $ in 0 e il cui vettore tangente sia $ B $

\begin{equation}
	\begin{cases}
		c : (-\varepsilon,\varepsilon) \to GL_{n}(\R)\\
		c(0) = A\\
		c'(0) = \dot{c}(0) = X
	\end{cases}
\end{equation}

dove

\begin{equation}
	T_{A}(GL_{n}(\R)) = M_{n}(\R) \implies c'(0) = \dot{c}(0)
\end{equation}

perciò

\begin{align}
	\begin{split}
		f_{*A}(X) &= \left. \dfrac{\operatorname{d}}{\operatorname{dt}} f(c(t)) \right|_{t=0}\\
		&= \left. \dfrac{\operatorname{d}}{\operatorname{dt}} [c(t)^{T} c(t)] \right|_{t=0}\\
		&= \left. [ \dot{c}(t)^{T} c(t) + c(t)^{T} \dot{c}(t) ] \right|_{t=0}\\
		&= X^{T} A + X A^{T}
	\end{split}
\end{align}

i.e.

\map{f_{*A}}{M_{n}(\R)}{S(n)}{X}{X^{T} A + X A^{T}}

Vale la condizione

\begin{equation}
	I \in \mathcal{VR}_{f} %
	\iff%
	f_{*A} : M_{n}(\R) \to S(n) \text{ suriettiva} \qquad \forall A \in f^{-1}(I) = O_{n}
\end{equation}

Perché $ f_{*A} $ sia suriettiva

\begin{equation}
	\forall A \in O(n), \, \forall B \in S(n), \, \exists X \in M_{n}(\R) \mid X^{T} A + X A^{T} = B
\end{equation}

è sufficiente dunque prendere $ X = \sfrac{1}{2} AB $, i.e.

\begin{align}
	\begin{split}
		X^{T} A + X A^{T} &= \left( \dfrac{1}{2} A B \right)^{T} A + \dfrac{1}{2} A^{T} A B\\
		&= \dfrac{1}{2} B^{T} A^{T} A + \dfrac{1}{2} B\\
		&= \dfrac{1}{2} B + \dfrac{1}{2} B\\
		&= B
	\end{split}
\end{align}

dunque $ I \in \mathcal{VR}_{f} $ e $ \dim(O(n)) = n(n-1)/2 $. Da questo ragionamento, otteniamo anche lo spazio tangente a $ O(n) $ in $ I $ poiché, per il teorema della preimmagine di un valore regolare, vale la seguente uguaglianza

\begin{equation}
	T_{A}(f^{-1}(I)) = T_{A}(O(n)) = \ker(f_{*A}), \qquad A \in f^{-1}(I)
\end{equation}

perciò

\begin{equation}
	T_{I}(O(n)) = \ker(f_{*I})
\end{equation}

ma abbiamo che

\begin{equation}
	f_{*I}(X) = X^{T} + X
\end{equation}

perciò

\begin{equation}
	T_{I}(O(n)) = \ker(f_{*I}) = \{ X \in M_{n}(\R) \mid X^{T} = -X \}
\end{equation}

ovvero lo spazio tangente di $ O(n) $ è formato dalle matrici antisimmetriche (spazio vettoriale), il quale ha dimensione esattamente $ \dim(T_{I}(O(n))) = n(n-1)/2 $.

\subsection{Omomorfismi e isomorfismi}

Un \textit{omomorfismo} tra gruppi\footnote{%
	Omomorfismo e omeomorfismo sono due concetti differenti legati a due parti differenti della matematica.%
} è un'applicazione (non necessariamente liscia) che preserva le moltiplicazioni di un gruppo nell'altro.

\begin{remark}
Siano $ F $ un omomorfismo, $ e_{H} \in H $ e $ e_{G} \in G $ gli elementi neutri dei rispettivi gruppi, allora $ F(e_{H}) = e_{G} $.
\end{remark}

Un omomorfismo tra due gruppi di Lie $ H $ e $ G $ è un'applicazione liscia $ F : H \to G $ tale che sia  un omomorfismo tra gruppi, i.e.

\begin{equation}
	F \circ L_{h} = L_{F(h)} \circ F, \qquad \forall h \in H
\end{equation}

in quanto, se $ F $ è un omomorfismo tra gruppi

\begin{align}
	\begin{split}
		F(h k) &= F(h) \, F(k)\\
		(F \circ L_{h})(k) &= (L_{F(h)} \circ F)(k)\\
		F \circ L_{h} &= L_{F(h)} \circ F
	\end{split}
\end{align}

Un \textit{isomorfismo di Lie} è un omomorfismo tra gruppi che sia anche un diffeomorfismo.

\subsection{Sottogruppi di Lie}

qui

\section{Esponenziale di una matrice}

qui

\section{Richiami di algebra lineare}

qui

\section{Traccia, determinante ed esponenziale di una matrice}

qui

\section{Algebra di Lie}

