\section{Unione disgiunta di spazi topologici come spazio topologico}\label{BONUS2-1}

\begin{tcolorbox}
	Verificare che l'unione disgiunta di spazi topologici
	
	\begin{equation}
		A = \bigsqcup_{j \in J} A_{j} \equiv \bigcup_{j \in J} A_{j} \times \{j\}
	\end{equation}
	
	è uno spazio topologico, sapendo che $ U $ è aperto in $ A $ se e solo se $ U \cap A_{j} $ è aperto in $ A_{j} $ per $ \forall j \in J $.
\end{tcolorbox}

Per dimostrare che l'unione disgiunta di spazi topologici $ A $ sia ancora uno spazio topologico dobbiamo dimostrare che:

\begin{itemize}
	\item L'intersezione di due aperti in $ A $ sia ancora aperto in $ A $
	
	\item L'unione di un numero qualsiasi di aperti in $ A $ sia ancora aperto in $ A $
\end{itemize}

Per la prima, prendiamo due aperti $ U $ e $ V $ in $ A $: questo significa che $ U \cap A_{j} $ e $ V \cap A_{j} $ saranno aperti in $ A_{j} $ per $ \forall j \in J $ per la condizione imposta. Siccome gli $ A_{j} $ sono spazi topologici, l'intersezione di due aperti è ancora un aperto, dunque

\begin{equation}
	U \cap V \cap A_{j} \text{ aperto in } A_{j} \qcomma \forall j \in J
\end{equation}

A questo punto, vale l'implicazione inversa, perciò $ U \cap V $ è aperto in $ A $.\\\\
Analogamente, per la seconda, prendiamo una famiglia infinita di aperti $ V_{i} $ (i.e. $ i \in N $) in $ A $: questo significa che $ V_{i} \cap A_{j} $ saranno aperti in $ A_{j} $ per $ \forall j \in J $ per la condizione imposta. Siccome gli $ A_{j} $ sono spazi topologici, l'unione di un numero qualsiasi di aperti è ancora un aperto. Definendo

\begin{equation}
	V \doteq \bigcup_{i=1}^{\infty} V_{i}
\end{equation}

possiamo dire che $ V \cap A_{j} $ è aperto in $ A_{j} $ per $ \forall j \in J $. A questo punto, vale l'implicazione inversa, perciò $ V $ è aperto in $ A $.

\section{Atlante differenziabile per $ \S^{n} $ con $ 2(n+1) $ carte}\label{es2-1}

\begin{tcolorbox}
	Sia $ \S^{n} $ la sfera unitaria in $ \R^{n+1} $. Trovare un atlante differenziabile di $ \S^{n} $ con $ 2(n+1) $ carte.
\end{tcolorbox}

Prendendo un parametro $ k = 1,\dots,n+1 $, definiamo\footnote{%
	La funzione $ \ceil*{x} $ \textit{ceiling} (soffitto) mappa un numero reale nel successivo numero naturale più vicino (i.e. arrotonda per eccesso), e.g. $ \ceil*{\frac{3}{2}} = 2 $.%
} i $ 2(n+1) $ aperti

\begin{equation}
	U_{i} = \left\{ (x^{1},\dots,x^{n+1}) \in \R^{n+1} \st (-1)^{i} x^{\ceil*{\frac{i}{2}}} > 0 \right\} \qcomma k = 1, \dots, 2(n+1)
\end{equation}

per i quali vale $ U_{i} \subset \R^{n+1} $ e che formano un ricoprimento per $ \S^{n} $, i.e.

\begin{equation}
	\S^{n} = \bigcup_{i=1}^{2(n+1)} U_{i}
\end{equation}

Definiamo anche gli aperti per $ j = 1,\dots,n+1 $

\begin{equation}
	D_{j} = \{ (x^{1},\dots,x^{j-1},x^{j+1},\dots,x^{n+1}) \in \R^{n} \times \{0\} = \R^{n} \, \mid \, \norm{x} < 1 \} = B_{1}(0) \simeq \R^{n}
\end{equation}

i quali, in sostanza, sono palle di raggio unitario centrate nell'origine nel piano $ \R^{n} \times \{0\} $, dove

\begin{equation}
	\norm{x} = \sqrt{ \sum_{i=1}^{n} (x^{i})^{2} }
\end{equation}

Infine definiamo le applicazioni lisce

\map{\phi_{i}}%
	{U_{i}}{D_{\ceil*{\frac{i}{2}}}}%
	{(x^{1},\dots,x^{i-1},x^{i},x^{i+1},\dots,x^{n+1})}{(x^{1},\dots,x^{i-1},x^{i+1},\dots,x^{n+1})}

con inverse

\map{\phi_{i}^{-1}}%
	{D_{\ceil*{\frac{i}{2}}}}{U_{i}}%
	{(x^{1},\dots,x^{i-1},x^{i},\dots,x^{n})}{\left( x^{1},\dots,x^{i-1},(-1)^{i} \sqrt{1-\norm{x}^{2}},x^{i},\dots,x^{n} \right)}

La famiglia di coppie $ \mathfrak{U} = \{ (U_{i},\phi_{i}) \}_{i=1,\dots,2(n+1)} $ definisce un atlante differenziabile per $ \S^{n} $ in quanto i cambi di carte sono lisci: la dimostrazione è analoga a quella fatta nell'Esempio \ref{unit-sph} per $ \S^{2} $.

%

\newpage

%

\section{Equivalenza tra strutture differenziabili su $ \S^{n} $}\label{es2-2}

\begin{tcolorbox}
	Dimostrare che la struttura differenziabile su $ \S^{n} $ definita nell’esercizio precedente e quella definita dalle proiezioni
	stereografiche coincidono.
\end{tcolorbox}

Per dimostrare che le due strutture differenziabili coincidano è sufficiente mostrare che le carte di ognuna siano $ C^{\infty} $-compatibili con quelle dell'altra.\\
Considerando le intersezioni:

\begin{align}
	U_{N} \cap U_{i} &= %\\
	\begin{cases}
		U_{i}, & i \neq 2(n+1)\\
		U_{i} \setminus \{ N \}, & i = 2(n+1)
	\end{cases}\\
	%
	\nonumber\\
	%
	U_{S} \cap U_{i} &= %\\
	\begin{cases}
		U_{i}, & i \neq 2(n+1) - 1 = 2 n + 1\\
		U_{i} \setminus \{ S \}, & i = 2n+1
	\end{cases}
\end{align}

possiamo scrivere i cambi di carta

\begin{gather}
	\pi_{N} \circ \phi_{i}^{-1} : \phi_{i}(U_{N} \cap U_{i}) \to \pi_{N}(U_{N} \cap U_{i})\\
	\pi_{S} \circ \phi_{i}^{-1} : \phi_{i}(U_{S} \cap U_{i}) \to \pi_{S}(U_{S} \cap U_{i})\\
	\nonumber\\
	\phi_{i} \circ \pi_{N}^{-1} : \pi_{N}(U_{N} \cap U_{i}) \to \phi_{i}(U_{N} \cap U_{i})\\
	\phi_{i} \circ \pi_{S}^{-1} : \pi_{N}(U_{S} \cap U_{i}) \to \phi_{i}(U_{S} \cap U_{i})
\end{gather}

esplicitamente abbiamo che

\begin{align}
	\begin{split}
		(\pi_{N} \circ \phi_{i}^{-1})(x^{1},\dots,x^{n}) &= \pi_{N} \left( x^{1}, \dots, x^{i-1}, (-1)^{i} \sqrt{1 - \norm{x}^{2}}, x^{i}, \dots, x^{n} \right)\\
		&= \left( \dfrac{2 x^{1}}{1-x^{n}}, \dots, \dfrac{2 x^{i-1}}{1-x^{n}}, \dfrac{2 (-1)^{i} \sqrt{1 - \norm{x}^{2}}}{1-x^{n}}, \dfrac{2 x^{i}}{1-x^{n}}, \dots, \dfrac{2 x^{n-1}}{1-x^{n}} \right)\\\\
		%
		(\phi_{i} \circ \pi_{N}^{-1})(x^{1},\dots,x^{n}) &= \phi_{i} \left( \dfrac{x^{1}}{1 + \norm{x}^{2}}, \dots, \dfrac{x^{n}}{1 + \norm{x}^{2}}, \dfrac{1 - \norm{x}^{2}}{1 + \norm{x}^{2}} \right)\\
		&= \left( \dfrac{x^{1}}{1 + \norm{x}^{2}}, \dots, \dfrac{x^{i-1}}{1 + \norm{x}^{2}}, \dfrac{x^{i+1}}{1 + \norm{x}^{2}}, \dots, \dfrac{x^{n}}{1 + \norm{x}^{2}}, \dfrac{1 - \norm{x}^{2}}{1 + \norm{x}^{2}} \right)
	\end{split}
\end{align}

e analogamente per $ \pi_{S} $: i cambi di carta sono lisci dunque le strutture differenziabili sono equivalenti.

%

\newpage

%

\section{Grassmanniana come spazio topologico connesso e compatto}\label{es2-5}

\begin{tcolorbox}
	Dimostrare che la Grassmanniana $ G(k,n) $ è uno spazio topologico connesso e compatto.
\end{tcolorbox}

\paragraph{Connessione}

Siccome il quoziente preserva la proprietà di connessione, è sufficiente mostrare che lo spazio delle matrici di rango $ k $, i.e. $ F(k,n) $, sia connesso per dimostrare che lo sia $ G(k,n) $.\\
Uno spazio topologico connesso per archi è anche connesso, dunque dimostriamo che per qualunque coppia di matrici di $ F(k,n) $ sia possibile costruire una funzione continua che le colleghi, i.e. preso $ I = [0,1] \subset \R $

\begin{gather}
	F(k,n) \text{ connesso per archi} \nonumber\\
	\Updownarrow\\
	\forall A, B \in F(k,n), \E f : I \to F(k,n) \text{ continua} %
	\, \mid \, f(0) = A \, \wedge \, f(1) = B \nonumber
\end{gather}

Siano due matrici invertibili $ P \in GL_{n}(\R) $ e $ Q \in GL_{k}(\R) $, una qualsiasi matrice $ A \in F(k,n) $ può essere scritta come

\begin{equation}
	A = P J Q \qcomma J \doteq \pmqty{I_{k} \\ 0_{n-k,k}}
\end{equation}

Consideriamo ora le due applicazioni continue seguenti:

\map{P_{t}}
	{I}{GL_{n}(\R)}
	{t}{P_{t}}

dove

\begin{equation}
	\begin{cases}
		P_{t}(0) = P_{0} = P\\
		P_{t}(1) = P_{1} = \diag(1,\dots,1,\det(P))
	\end{cases}
\end{equation}

\map{Q_{t}}
	{I}{GL_{n}(\R)}
	{t}{Q_{t}}

dove

\begin{equation}
	\begin{cases}
		Q_{t}(0) = Q_{0} = Q\\
		Q_{t}(1) = Q_{1} = \diag(1,\dots,1,\det(Q))
	\end{cases}
\end{equation}

Queste sono continue perché ognuna risiede in una delle componenti connesse dell'insieme delle matrici invertibili, i.e. in una delle seguenti due parti

\begin{equation}
	GL_{m}(\R) = \{ A \in GL_{m}(\R) \mid \det(A) > 0 \} \sqcup \{ A \in GL_{m}(\R) \mid \det(A) < 0 \}
\end{equation}

A questo punto, consideriamo l'applicazione

\map{f}
	{I}{F(k,n)}
	{t}{P_{t} J Q_{t}}

dove

\begin{equation}
	\begin{cases}
		f(0) = P_{0} J Q_{0} = A\\
		f(1) = P_{1} J Q_{1} = J
	\end{cases}
\end{equation}

La continuità di questa funzione deriva dalla continuità di $ P_{t} $ e $ Q_{t} $.\\
Collegando ogni matrice di $ F(k,n) $ alla matrice $ J $ (anch'essa in $ F(k,n) $ in quanto di rango $ k $) è possibile collegare queste matrici tra loro, dimostrando che $ F(k,n) $ è connesso e dunque lo è anche la Grassmanniana.

\paragraph{Compattezza}

Possiamo pensare alla Grassmanniana anche come quoziente di una \textit{varietà differenziale di Stiefel}\footnote{%
	Una varietà differenziale di Stiefel $ V_{k}(\R^{n}) $ è l'insieme di tutte le basi ortonormali di dimensione $ k $, i.e. le $ k $-uple di vettori linearmente indipendenti e normalizzati; questa varietà è un sottoinsieme di $ \R^{n} $.%
}

\begin{equation}
	G(k,n) = \dfrac{V_{k}(\R^{n})}{\sim}
\end{equation}

dove due basi di $ V_{k}(\R^{n}) $ sono equivalenti se generano lo stesso $ k $-spazio.\\
Siccome il quoziente preserva anche la proprietà di compattezza, è sufficiente mostrare che la varietà di Stiefel sia connessa per dimostrare che lo sia la Grassmanniana.\\
Sappiamo che se un sottoinsieme di $ \R^{n} $ è chiuso e limitato allora è compatto: dato che $ V_{k}(\R^{n}) \subset \R^{n} $ è chiusa e limitata, allora è compatta e dunque lo è anche la Grassmanniana.

%

\newpage

%

\section{Restrizione di funzione liscia su varietà}\label{es2-7}

\begin{tcolorbox}
	Sia $ \S^{1} $ il cerchio unitario di $ \R^{2} $. Dimostrare che una funzione liscia $ f : \R^{2} \to \R $ si restringe a una funzione liscia $ \eval{f}_{\S^{1}} : \S^{1} \to \R $.
\end{tcolorbox}

Presa una carta $ (U,\phi) $ della struttura differenziabile di $ \R^{2} $, perché $ f $ sia liscia è necessario che sia liscia la composizione

\begin{equation}
	f \circ \phi^{-1} : \phi(U) \to \R
\end{equation}

con $ U \subset \R^{2} $ aperto.\\
Siccome la definizione non dipende dalla carta scelta e dal fatto che un diffeomorfismo

\begin{equation}
	g : U \to g(U) \subset \R^{2}
\end{equation}

definisce una carta della struttura differenziale $ (U,g) $\footnote{%
	Vedi Proposizione \ref{diffeo-map}.%
}, possiamo prendere il diffeomorfismo

\map{g}
	{\R^{2}}{\S^{1}}
	{(x,y)}{\left( \dfrac{x}{\sqrt{x^{2} + y^{2}}},\dfrac{y}{\sqrt{x^{2} + y^{2}}} \right)}

e dunque la carta $ (\S^{1},g) $ della struttura differenziale di $ \R^{2} $ per riscrivere la definizione di funzione liscia, dunque

\begin{equation}
	f \circ g^{-1} : g(\R^{2}) = \S^{1} \to \R
\end{equation}

la quale può essere riscritta come

\begin{equation}
	f_{|\S^{1}} : \S^{1} \to \R
\end{equation}

%

\newpage

%

\section{Inclusione come funzione liscia tra varietà}\label{es2-6}

\begin{tcolorbox}
	Siano $ M $ e $ N $ due varietà differenziabili e $ q_{0} \in N $. Dimostrare che
	
	\map{i_{q_{0}}}
		{M}{M \times N}
		{p}{(p,q_{0})}
	
	è un'applicazione liscia.
\end{tcolorbox}

Prendiamo gli atlanti differenziabili $ \{(U,\phi)\} \in M $ e $ \{(U \times V,\phi \times \psi)\} \in M \times N $: perché $ i_{q_{0}} $ sia liscia questa deve essere continua (lo è in quanto inclusione) e la composizione

\map{(\phi \times \psi) \circ i_{q_{0}} \circ \phi^{-1}}
	{\phi( i_{q_{0}}^{-1}(U \times V) \cap U )}{\R^{m+n}}
	{\phi(p)}{(\phi^{1}(p),\dots,\phi^{m}(p),\psi^{1}(q_{0}),\dots,\psi^{n}(q_{0}))}

deve essere liscia. Per dimostrare che lo sia, dalla Proposizione \ref{map-comp}, basta mostrare che le sue componenti siano lisce: presa la proiezione sulla $ k $-esima componente

\map{r^{k}}
	{\R^{n}}{\R}
	{(x^{1},\dots,x^{n})}{x^{k}}

le componenti della composizione sono date dalla seguente applicazione

\map{r^{k} \circ (\phi \times \psi) \circ i_{q_{0}} \circ \phi^{-1}}
	{\phi( i_{q_{0}}^{-1}(U \times V) \cap U )}{\R}
	{\phi(p)}{%
				\begin{cases}
					\phi^{k}(p), & k \leqslant m\\
					\psi^{k}(q_{0}), & k > m
				\end{cases}%
				}

le quali sono tutte lisce, dunque anche l'inclusione è liscia tra varietà.

%

\newpage

%

\section{Coefficienti campo di vettori}\label{es2-8}

\begin{tcolorbox}
	Siano un punto $ p = (x,y) \in \R^{2} $ e un'applicazione
	
	\map{F}
		{\R^{2}}{\R^{3}}
		{(x,y)}{(x,y,xy)}
	
	Trovare $ a,b,c \in \R $ tali che:
	
	\begin{equation}
		F_{*p} \left( \eval{ \pdv{x} }_{p} \right) = a \eval{ \pdv{u} }_{F(p)} + b \eval{ \pdv{v} }_{F(p)} + c \eval{ \pdv{w} }_{F(p)}
	\end{equation}
\end{tcolorbox}

Per trovare i coefficienti dell'immagine del differenziale ricordiamo che, per una funzione tra varietà differenziabili $ F : N \to M $ con carte

\begin{equation}
	\begin{cases}
		(U,\phi) \in N, & \phi = (x^{1},\dots,x^{n})\\
		(V,\psi) \in M, & \psi = (y^{1},\dots,y^{m})
	\end{cases}
\end{equation}

vale

\begin{equation}
	F_{*p} \left( \eval{\pdv{x^{j}}}_{p} \right) = \sum_{k=1}^{m} \pdv{F^{k}}{x^{j}}\ (p) \left( \eval{\pdv{y^{k}}}_{F(p)} \right) \qcomma j = 1,\dots,n
\end{equation}

dove $ F^{k} = y^{k} \circ F $.\\
Essendo il differenziale un'applicazione lineare, considerando $ X_{p} \in T_{p}(N) $ possiamo anche scrivere

\begin{equation}
	F_{*p} (X_{p}) = A(p) X_{p} = Y_{F(p)} \in T_{F(p)}(M)
\end{equation}

dove

\begin{equation}
	A(p) = \left[ \pdv{F^{k}}{x^{j}}\ (p) \right]
\end{equation}

Per la funzione considerata, abbiamo che

\begin{equation}
	A(x,y) = \bmqty{ 1 & 0 \\ 0 & 1 \\ y & x }
\end{equation}

Siccome possiamo identificare $ T_{p}(\R^{k}) = \R^{k} $, è possibile scrivere

\begin{gather}
	X_{p} = \mu \eval{\pdv{x}}_{p} + \nu \eval{\pdv{y}}_{p} \equiv \bmqty{ \mu \\ \nu }\\
	Y_{F(p)} = a \eval{\pdv{u}}_{F(p)} + b \eval{\pdv{v}}_{F(p)} + c \eval{\pdv{w}}_{F(p)} \equiv \bmqty{ a \\ b \\ c }
\end{gather}

A questo punto

\begin{equation}
	F_{*p} \left( \bmqty{ \mu \\ \nu } \right) = \bmqty{ 1 & 0 \\ 0 & 1 \\ y & x } \bmqty{ \mu \\ \nu } %
	= \bmqty{ a \\ b \\ c } %
	= \bmqty{ \mu \\ \nu \\ y \mu + x \nu }
\end{equation}

perciò abbiamo che

\begin{equation}
	F_{*p} \left( \bmqty{ 1 \\ 0 } \right) = \bmqty{ 1 & 0 \\ 0 & 1 \\ y & x } \bmqty{ 1 \\ 0 } %
	= \bmqty{ 1 \\ 0 \\ y } %
	\equiv \eval{\pdv{u}}_{F(p)} + y \eval{\pdv{w}}_{F(p)}
\end{equation}

i.e. $ (a,b,c) = (1,0,y) $.

\newpage

%

\section{Coefficienti cambio di base}\label{es2-9}

\begin{tcolorbox}
	Siano $ x $ e $ y $ le coordinate standard su $ \R^{2} $ e $ U = \R^{2} \setminus \{(0,0)\} $. In $ U $ le coordinate polari $ (\rho, \theta) $ con $ \rho > 0 $ e $ \theta \in (0,2\pi) $ sono definite come
	
	\begin{equation}
		\begin{cases}
			x = \rho \cos(\theta)\\
			y = \rho \sin(\theta)
		\end{cases}
	\end{equation}
	
	Si scrivano $ \pdv*{\rho} $ e $ \pdv*{\theta} $ in funzione di $ \pdv*{x} $ e $ \pdv*{y} $.
\end{tcolorbox}

L'equazione che lega i vettori di una base dello spazio tangente

\begin{equation}
	T_{p}(\R^{2} \setminus \{(0,0)\}) = \R^{2} \setminus \{(0,0)\}
\end{equation}

a un'altra base è la seguente:

\begin{equation}
	\eval{\pdv{u^{j}}}_{p} = \sum_{k=1}^{2} \pdv{v^{k}}{u^{j}} \, (p) \left( \eval{\pdv{v^{k}}}_{p} \right)
\end{equation}

dove $ u = (\rho,\theta) $ e $ v = (x,y) $.\\
Siccome

\begin{equation}
	\begin{cases}
		x = \rho \cos(\theta)\\
		y = \rho \sin(\theta)
	\end{cases}%
	\implies %
	\begin{cases}
		\rho = \sqrt{x^{2} + y^{2}}\\
		\cos(\theta) = x / \rho\\
		\sin(\theta) =  / \rho
	\end{cases}
\end{equation}

Possiamo scrivere la matrice di trasformazione $ T $ come

\begin{equation}
	T \doteq \left[ \pdv{v^{k}}{u^{j}} \, (p) \right] = \bmqty{ \dpdv{x}{\rho} & \dpdv{x}{\theta} \\\\ \dpdv{y}{\rho} & \dpdv{y}{\theta} } %
	= \bmqty{ \cos(\theta) & - \rho \sin(\theta) \\\\ \sin(\theta) & \rho \cos(\theta) } %
	= \bmqty{ \dfrac{x}{\sqrt{x^{2} + y^{2}}} & - y \\\\ \dfrac{y}{\sqrt{x^{2} + y^{2}}} & x }
\end{equation}

dunque

\begin{gather}
		\eval{\pdv{\rho}}_{p} = \dfrac{1}{\sqrt{x^{2} + y^{2}}} \left( x \eval{\pdv{x}}_{p} + y \eval{\pdv{y}}_{p} \right) %
		\quad \lor \quad %
		\rho \eval{\pdv{\rho}}_{p} = x \eval{\pdv{x}}_{p} + y \eval{\pdv{y}}_{p}\\
		\nonumber\\
		\eval{\pdv{\theta}}_{p} = - y \eval{\pdv{x}}_{p} + x \eval{\pdv{y}}_{p}
\end{gather}

%

\newpage

%

\section{Vettore tangente a una curva}\label{es2-10}

\begin{tcolorbox}
	Sia $ p = (x,y) $ un punto di $ \R^{2} $. Allora
	
	\begin{equation}
		c_{p}(t) = \mqty( \cos(2t) & -\sin(2t) \\\\ \sin(2t) & \cos(2t) ) \mqty(x \\ y)
	\end{equation}
	
	è una curva liscia in $ \R^{2} $ che inizia in $ p $. Calcolare $ c'(0) $.
\end{tcolorbox}

La curva considerata è la seguente:

\map{c}
	{[0, 2 \pi)}{\R^{2}}
	{t}{%
		\bmqty{ \cos(2t) & - \sin(2t) \\\\ \sin(2t) & \cos(2t) } \bmqty{ x \\ y } %
		= \bmqty{ \cos(2t) x - \sin(2t) y \\\\ \sin(2t) x + \cos(2t) y }}

Dalla Proposizione \ref{loc-exp-tan-cur} considerando $ \R^{2} $ come varietà immagine, abbiamo che il vettore tangente alla curva in un punto $ c(t_{0}) $ con $ t_{0} = 0 $ è dato da

\begin{equation}
	c'(0) = \sum_{i=1}^{n} \dot{c}_{i}(0) \eval{ \pdv{r^{i}} }_{c(0)} %
	= \dot{c}_{1}(0) \eval{ \pdv{x} }_{p} + \dot{c}_{2}(0) \eval{ \pdv{y} }_{p}
\end{equation}

in quanto la curva inizia in $ p $, i.e. $ p = c(0) $.

Calcoliamo dunque le componenti del vettore tangente:

\begin{align}
	\begin{split}
		\dot{c}(0) &= \eval{ \dv{t} }_{0} c(t)\\
		&= \eval{ \dv{t} }_{0} \left( \bmqty{ \cos(2t) x - \sin(2t) y \\\\ \sin(2t) x + \cos(2t) y } \right)\\
		&= \eval{ \bmqty{ - 2 \sin(2t) x - 2 \cos(2t) y \\\\ 2 \cos(2t) x - 2 \sin(2t) y } }_{0}\\
		&= \bmqty{ - 2 y \\ 2x }
	\end{split}
\end{align}

da cui

\begin{equation}
	c'(0) = - 2 y \eval{ \pdv{x} }_{0} + 2 x \eval{ \pdv{y} }_{0}
\end{equation}

%

\newpage

%

\section{Isomorfismo prodotto spazi tangenti}\label{es2-11}

\begin{tcolorbox}
	Siano $ N $ e $ M $ varietà differenziabili e $ \pi_{N} : N \times M \to N $ e $ \pi_{M} : N \times M \to M $ le proiezioni naturali. Dimostrare che per $ (p,q) \in N \times M $ l'applicazione
	
	\begin{equation}
		(\pi_{N_{*p}},\pi_{M_{*q}}) : T_{(p,q)}(N \times M) \to T_{p}(N) \times T_{q}(M)
	\end{equation}
	
	è un isomorfismo.
\end{tcolorbox}

% https://math.stackexchange.com/questions/413766/tangent-space-of-product-manifold

Al fine di dimostrare questo isomorfismo, definiamo $ f \doteq (\pi_{N_{*p}},\pi_{M_{*p}}) $ e scriviamo la sua azione su un vettore del dominio

\map{f}
	{T_{(p,q)}(N \times M)}{T_{p}(N) \times T_{q}(M)}
	{X_{(p,q)}}{\pi_{N_{*p}} (X_{(p,q)}), \pi_{M_{*q}} (X_{(p,q)})}
	
Questa applicazione è lineare in quanto composizione di differenziali, i quali sono essi stessi lineari.\\
Consideriamo ora le inclusioni seguenti:

\sbs{0.5}{%
			\map{i_{N}}
				{N}{N \times M}
				{p}{(p,q)}
			}
	{0.5}{%
			\map{i_{M}}
				{M}{N \times M}
				{q}{(p,q)}
			}

le cui azioni sugli spazi sono

\begin{gather}
	i_{N}(N) = N \times \{q\}\\
	i_{M}(M) = \{p\} \times M
\end{gather}

Definiamo inoltre l'applicazione

\map{g}
	{T_{p}(N) \times T_{q}(M)}{T_{(p,q)}(N \times M)}
	{(X_{p}, Y_{q})}{i_{N_{*p}} (X_{p}) + i_{M_{*q}} (Y_{q})}

Consideriamo le seguenti composizioni:

\begin{gather}
	(\pi_{N} \circ i_{N}) (p) = \pi_{N} \circ (i_{N} (p)) = \pi_{N} (p,q) = p\\
	(\pi_{M} \circ i_{M}) (q) = \pi_{M} \circ (i_{M} (q)) = \pi_{M} (p,q) = q
\end{gather}

da cui

\begin{equation}
	\begin{cases}
		\pi_{N} \circ i_{N} = \id_{N}\\
		\pi_{M} \circ i_{M} = \id_{M}
	\end{cases}
\end{equation}

Il differenziale di queste composizioni risulta quindi nell'identità dei relativi spazi tangenti in quanto $ (\id_{N})_{*p} = \id_{T_{p}(N)} $. Le composizioni relative a spazi differenti, i.e. $ \pi_{N} \circ i_{M} $ e $ \pi_{M} \circ i_{N} $, sono delle mappe costanti quindi il loro differenziale è nullo.\\
Calcoliamo ora la seguente composizione delle funzioni $ f $ e $ g $:

\begin{align}
	\begin{split}
		(f \circ g) (X_{p}, Y_{q}) &= f ( i_{N_{*p}} (X_{p}) + i_{M_{*q}} (Y_{q}) )\\
		&= (\pi_{N_{*p}} ( i_{N_{*p}} (X_{p}) + i_{M_{*q}} (Y_{q}) ), \pi_{M_{*q}} ( i_{N_{*p}} (X_{p}) + i_{M_{*q}} (Y_{q}) ))\\
		&= (\pi_{N_{*p}} ( i_{N_{*p}} (X_{p}) ) + \pi_{N_{*p}} ( i_{M_{*q}} (Y_{q}) ), \pi_{M_{*q}} ( i_{N_{*p}} (X_{p})) + \pi_{M_{*q}} ( i_{M_{*q}} (Y_{q}) ))\\
		&= ((\pi_{N} \circ i_{N})_{*p} (X_{p}) + \cancelto{0}{ (\pi_{N} \circ i_{M})_{*q} (Y_{q}) }, \cancelto{0}{ (\pi_{M} \circ i_{N})_{*p} (X_{p}) } + (\pi_{M} \circ i_{M})_{*q} (Y_{q}) )\\
		&=(X_{p}, Y_{q})
	\end{split}
\end{align}

dove i due $ 0 $ appartengono rispettivamente a $ T_{p}(N) $ e $ T_{q}(M) $, dunque

\begin{equation}
	f \circ g = \id_{T_{p}(N) \times T_{q}(M)}
\end{equation}

questo significa che la funzione $ f $ ha un'inversa destra dunque è suriettiva: la suriettività comporta che l'immagine della funzione coincide con il codominio, da cui

\begin{equation}
	\dim(\Im(f)) = \dim(T_{p}(N) \times T_{q}(M))
\end{equation}

Ricordando che la dimensione degli spazi tangenti considerati è la stessa, i.e.

\begin{equation}
	\dim(T_{(p,q)}(N \times M)) = \dim(T_{p}(N) \times T_{q}(M))
\end{equation}

possiamo invocare il teorema della dimensione\footnote{%
	Data un'applicazione lineare $ T : V \to W $, abbiamo che
	
	\begin{equation*}
		\dim(T(V)) + \dim(\ker(T)) = \dim(V)
	\end{equation*}
	
	dove $ \dim(T(V)) \doteq \rank(T) $ viene anche chiamato rango di $ T $.%
}

\begin{align}
	\begin{split}
		\dim(\Im(f)) + \dim(\ker(f)) &= \dim(T_{(p,q)}(N \times M))\\
		\cancel{ \dim(T_{p}(N) \times T_{q}(M)) } + \dim(\ker(f)) &= \cancel{ \dim(T_{(p,q)}(N \times M)) }\\
		\dim(\ker(f)) &= 0
	\end{split}
\end{align}

perciò la funzione $ f = (\pi_{N_{*p}},\pi_{M_{*q}}) $ è anche iniettiva dunque è un isomorfismo, i.e.

\begin{equation}
	T_{(p,q)}(N \times M) \simeq T_{p}(N) \times T_{q}(M)
\end{equation}
 
% -----------------------

% fine pdf 2.1-2.3

% -----------------------

%

\newpage

%

\section{Prodotto di sottovarietà}\label{es2-12}

\begin{tcolorbox}
	Siano $ S $ e $ P $ due sottovarietà di due varietà differenziabili $ N $ e $ M $ rispettivamente. Dimostrare che $ S \times P $ è una sottovarietà di $ N \times M $.
\end{tcolorbox}

Poniamo le dimensioni delle varietà $ N $ e $ M $ pari a $ n $ e $ m $ rispettivamente.\\
Essendo $ S $ e $ P $ sottovarietà rispettivamente di $ N $ e $ M $, possiamo scrivere

\begin{gather}
	S \text{ sottovarietà di } N \nonumber\\
	\Updownarrow\\
	\forall s \in S, \, \E (U,\phi) = (U; x^{1},\dots,x^{n}) \in N \mid (U,\phi) \ni s \, \wedge \nonumber\\
	\wedge \, U \cap S = \{ r \in U \mid x^{k+1}(r) = \dots = x^{n}(r) = 0 \} \nonumber
\end{gather}
%
\begin{gather}
	P \text{ sottovarietà di } M \nonumber\\
	\Updownarrow\\
	\forall p \in P, \, \E (V,\psi) = (V; y^{1},\dots,y^{m}) \in M \mid (V,\psi) \ni p \, \wedge \nonumber\\
	\wedge \, V \cap P = \{ t \in V \mid y^{j+1}(t) = \dots = y^{m}(t) = 0 \} \nonumber
\end{gather}

dove quindi

\begin{equation}
	\begin{cases}
		\dim(S) = n - k\\
		\dim(S) = m - j
	\end{cases} %
	\iff %
	\begin{cases}
		\operatorname{cod}_{N}(S) = k\\
		\operatorname{cod}_{M}(P) = j
	\end{cases}
\end{equation}

Essendo $ N \times M $ ancora una varietà differenziabile, il prodotto delle carte considerate sopra è ancora una carta per la struttura differenziale di $ N \times M $, i.e.

\begin{equation}
	\begin{cases}
		\{(U_{\alpha},\phi_{\alpha})\}_{\alpha \in A} \in N \\
		\{(V_{\beta},\psi_{\beta})\}_{\beta \in B} \in M
	\end{cases} %
	\implies %
	\{(U_{\alpha} \times V_{\beta}, \phi_{\alpha} \times \psi_{\beta})\}_{\alpha \in A, \beta \in B} \in N \times M
\end{equation}

A questo punto, possiamo usare queste carte per definire una carta adattata di $ N \times M $ intorno a un qualsiasi punto $ (s,p) $ relativamente a $ S \times P $, rendendo perciò $ S \times P $ una sottovarietà di $ N \times M $; di seguito la condizione:

\begin{gather}
	\forall (s,p) \in S \times P, \, \E (U \times V, \phi \times \psi) = (U \times V; x^{1},\dots,x^{n}, y^{1},\dots,y^{m}) \in N \times M \mid \nonumber\\
	\begin{cases}
		(U \times V, \phi \times \psi) \ni (s,p)\\
		(U \times V) \cap (S \times P) = \{ (r,t) \in U \times V \mid x^{k+1}(r) = \dots = x^{n}(r) = y^{j+1}(t) = \dots = y^{m}(t) = 0 \}
	\end{cases}
\end{gather}

La dimensione e codimensione della sottovarietà prodotto sono dunque pari a

\begin{gather}
	\dim(S \times P) = (n - k) + (m - j) = (n + m) - (k + j)\\
	\operatorname{cod}_{N \times M}(S \times P) = k + j
\end{gather}

%

\newpage

%

\section{Preimmagine di applicazione come sottovarietà}\label{es2-13}

\begin{tcolorbox}
	Sia l'applicazione
	
	\map{F}
		{\R^{2}}{\R}
		{(x,y)}{x^{2}-6xy+y^{2}}
	
	Trovare i $ c \in \R $ tali che $ F^{-1}(c) $ sia una sottovarietà di $ \R^{2} $.
\end{tcolorbox}

Tramite il teorema della preimmagine\footnote{%
	Vedi Teorema \ref{thm:preimg}.%
}, perché $ F^{-1}(c) $ sia una sottovarietà di $ \R^{2} $ è necessario che $ c \in \VR_{F} \cap \Im(F) $: questa condizione è equivalente a dire che almeno una delle derivate parziali di $ F $ non si annulli se calcolata nei punti che formano la controimmagine di $ c $ attraverso $ F $ (e dunque che tutti questi punti siano punti regolari per $ F $), i.e.

\begin{gather}
	c \in \VR_{F} \cap \Im(F) \nonumber\\
	\Updownarrow \nonumber\\
	\E \pdv{F}{x^{i}} \, (p) \neq 0 \qcomma \forall p \in F^{-1}(c), \, i=1,2\\
	\Updownarrow \nonumber\\
	p \in \PR_{F} \qcomma \forall p \in F^{-1}(c) \nonumber
\end{gather}

dove $ (x^{1},x^{2}) = (x,y) $.\\
Siccome $ \PC_{F} \cap \PR_{F} = \emptyset $, escludendo i punti critici per $ F $ rimangono quelli regolari tra i quali individueremo la controimmagine di $ c $. Per trovare i punti critici, cerchiamo dunque i punti $ p = (x,y) $ che annulla contemporaneamente le derivate di $ F $:

\begin{equation}
	\begin{cases}
		\dpdv{F}{x} \, (x,y) = 2 x - 6 y = 0\\\\
		\dpdv{F}{y} \, (x,y) = - 6 x + 2 y = 0
	\end{cases} %
	\implies %
	(x,y) = (0,0)
\end{equation}

A questo punto possiamo derivare l'insieme dei punti regolari:

\begin{equation}
	\PC_{F} = \{(0,0)\} %
	\implies %
	\PR_{F} = \R^{2} \setminus \{(0,0)\}
\end{equation}

L'unico punto che ha come controimmagine il punti $ (0,0) $ è $ 0 $, i.e. $ F^{-1}(0) = (0,0) $ dunque $ \VR_{F} = \R \setminus \{0\} $: questo permette di scegliere per $ c $ un qualsiasi numero reale non nullo.\\
Il grafico di $ F $ è un paraboloide iperbolico e le controimmagini dei valori regolari di $ F $ sono intersezioni di questo con piani perpendicolari all'asse $ z $ e dunque iperboli.

%

\newpage

%

\section{Sottovarietà tramite condizioni}\label{es2-14}

\begin{tcolorbox}
	Dire se le soluzioni del sistema
	
	\begin{equation}
		\begin{cases}
			x^{3} + y^{3} + z^{3} = 1 \\
			z = xy
		\end{cases}
	\end{equation}
	
	costituiscono una sottovarietà di $ \R^{3} $.
\end{tcolorbox}

La soluzione di questo esercizio segue l'Esempio \ref{ex:subvar-cond}.\\
Il sistema in esame può essere riscritto come l'insieme seguente

\begin{equation}
	S = \{ (x,y,z) \in \R^{3} \mid x^{3} + y^{3} + z^{3} = 1 \, \wedge \, z = xy \}
\end{equation}

Costruiamo ora l'applicazione
%
\map{F}
	{\R^{3}}{\R^{2}}
	{(x,y,z)}{(x^{3} + y^{3} + z^{3} - 1, z - xy)}

la quale porta all'equazione $ S = F^{-1}(0,0) $.\\
Perché $ S $ sia dunque una sottovarietà di $ \R^{3} $ verifichiamo che $ (0,0) \in \VR_{F} $: per fare ciò, dimostriamo che tutti i punti della preimmagine di $ (0,0) $ siano punti regolari, i.e. $ S \cap \PC_{F} = \emptyset $.\\
Siccome $ \PC_{F} \cap \PR_{F} = \emptyset $, cerchiamo i punti critici di $ F $ per escluderli dalla preimmagine: questi sono i punti per cui il differenziale $ F_{*(x,y,z)} $  non è suriettivo dunque, usando lo jacobiano, poniamo le condizioni per cui il rango di $ F $ sia minore del massimo (i.e. 2).

\begin{equation}
	J(F)(x,y,z) = \bmqty{ %
			 	3 x^{2} & 3 y^{2} & 3 z^{2} \\\\
			 	- y 	& - x	  & 1 %
	 		}
\end{equation}

A questo punto, possiamo scrivere le condizioni:

\begin{equation}
	\rank(F_{*(x,y,z)}) < 2 %
	\implies %
	\begin{cases}
		- 3 x^{3} + 3 y^{3} = 0 \\
		3 x^{2} + 3 y z^{2} = 0 \\
		3 y^{2} + 3 x z^{2} = 0
	\end{cases}
\end{equation}

Consideriamo quindi l'intersezione tra i punti critici di $ F $ e l'insieme $ S $:

\begin{equation}
	\begin{cases}
		- 3 x^{3} + 3 y^{3} = 0 \\
		3 x^{2} + 3 y z^{2} = 0 \\
		3 y^{2} + 3 x z^{2} = 0 \\
		x^{3} + y^{3} + z^{3} = 1 \\
		z = xy
	\end{cases}
\end{equation}

Questo sistema non ha soluzioni\footnote{%
	La soluzione del sistema è stata trovata utilizzando un calcolatore online (\href{https://www.wolframalpha.com/input?i=solve+\%7B+-+3+x\%5E3+\%2B+3+y\%5E3+\%3D+0+\%2C+3+x\%5E2+\%2B+3+y+z\%5E2+\%3D+0+\%2C+3+y\%5E2+\%2B+3+x+z\%5E2+\%3D+0+\%2C+x\%5E3+\%2B+y\%5E3+\%2B+z\%5E3+\%3D+1+\%2C+z+\%3D+xy+\%7D}{link a WolframAlpha}).%
}, i.e. $ S \cap \PC_{F} = \emptyset $, dunque $ (0,0) \in \VR_{F} $ e $ S $ è una sottovarietà di $ \R^{3} $ con $ \dim(S) = 1 $ (in quanto le condizioni dell'insieme sono due).

%

\newpage

%

\section{Spazio tangente a sottovarietà}\label{BONUS2-3}

\begin{tcolorbox}
	Sia la sottovarietà di $ \R^{3} $
	
	\begin{equation}
		S = \{ (x,y,z) \in \R^{3} \mid x^{3} + y^{3} + z^{3} = 1 \, \wedge \, x + y + z = 0 \} \subset \R^{3}
	\end{equation}
	
	Calcolare lo spazio tangente $ T_{p}(S) $ con $ p \in S $.
\end{tcolorbox}

Costruiamo l'applicazione

\map{F}
	{\R^{3}}{\R^{2}}
	{(x,y,z)}{(x^{3} + y^{3} + z^{3} - 1, x + y + z)}

tale che $ S = F^{-1}(0,0) $.\\
Tramite il teorema della preimmagine\footnote{%
	Vedi Teorema \ref{thm:preimg}.%
}, sappiamo che

\begin{equation}
	T_{p}(S) = \ker(F_{*p})
\end{equation}

Dunque calcoliamo quali sono i vettori che hanno immagine il vettore nullo tramite il differenziale

\begin{equation}
	F_{*p} : T_{p}(\R^{3}) \to T_{F(p)}(\R^{2})
\end{equation}

Consideriamo una curva liscia $ \gamma (t) = (x(t), y(t), z(t)) $ che rispetta le condizioni:

\begin{equation}
	\begin{cases}
		\gamma (- \varepsilon, \varepsilon) \to S \\
		\gamma (0) = p = (x(0), y(0), z(0)) \\
		\gamma' (0) = (\dot{x}(0), \dot{y}(0), \dot{z}(0)) = X_{p}
	\end{cases}
\end{equation}

e le basi per $ T_{p}(\R^{3}) $ e $ T_{F(p)}(\R^{2}) $

\begin{gather}
	\B_{T_{p}(\R^{3})} = \left\{ \eval{ \pdv{x} }_{p}, \eval{ \pdv{y} }_{p}, \eval{ \pdv{z} }_{p} \right\} \\
	\B_{T_{F(p)}(\R^{2})} = \left\{ \eval{ \pdv{u} }_{F(p)}, \eval{ \pdv{v} }_{F(p)} \right\}
\end{gather}

Tramite questa, calcoliamo l'immagine del differenziale:

\begin{align}
	\begin{split}
		F_{*p}(X_{p}) &= (F \circ \gamma)' (0) \\
		&= ((x(t))^{3} + (y(t))^{3} + (z(t))^{3} - 1, x(t) + y(t) + z(t))' (0) \\
		&= \dot{ ((x(t))^{3} + (y(t))^{3} + (z(t))^{3} - 1) } (0) \eval{ \pdv{u} }_{F(p)} + \dot{ (x(t) + y(t) + z(t)) } (0) \eval{ \pdv{v} }_{F(p)} \\
		&= (3 (x(0))^{2} \dot{x}(0) + 3 (y(0))^{2} \dot{y}(0) + 3 (z(0))^{2} \dot{z}(0)) \eval{ \pdv{u} }_{F(p)} + (\dot{x}(0) + \dot{y}(0) + \dot{z}(0)) \eval{ \pdv{v} }_{F(p)} \\
		&= ( 3 (x(0), y(0), z(0))^{2} \cdot (\dot{x}(0), \dot{y}(0), \dot{z}(0)) \eval{ \pdv{u} }_{F(p)} + ( (1,1,1) \cdot (\dot{x}(0), \dot{y}(0), \dot{z}(0)) ) \eval{ \pdv{v} }_{F(p)} \\
		&= ( 3 p^{2} \cdot X_{p} ) \eval{ \pdv{u} }_{F(p)} + ( (1,1,1) \cdot X_{p} ) \eval{ \pdv{v} }_{F(p)}
	\end{split}
\end{align}

Ponendo

\begin{equation}
	X_{p} = a \eval{ \pdv{x} }_{p} + b \eval{ \pdv{y} }_{p} + c \eval{ \pdv{z} }_{p} %
	\equiv (a,b,c)
\end{equation}

possiamo scrivere

\begin{equation}
	F_{*p}(a,b,c) = ( 3 (a x^{2} + b y^{2} + c z^{2}), a + b + c ) \in T_{F(p)}(\R^{2})
\end{equation}

da cui lo spazio tangente

\begin{equation}
	T_{p}(S) = \left\{ (a,b,c) \in T_{p}(\R^{3}) \st p = (x,y,z) \, \wedge \, %
		\begin{cases}
			a x^{2} + b y^{2} + c z^{2} = 0 \\
			a + b + c = 0
		\end{cases} %
	\right\}
\end{equation}

%

\newpage

%

\section{Sottovarietà e spazio dei polinomi omogenei}\label{es2-15}

\begin{tcolorbox}
	Un polinomio $ F(x_{1},\dots,x_{n}) \in \R[x_{1},\dots,x_{n}] $ è omogeneo di grado $ k $ se è combinazione lineare di monomi $ x_{1}^{i_{1}} \cdots x_{n}^{i_{m}} $ di grado $ k $ tale che $ \sum_{j=1}^{m} i_{j} = k $. Dimostrare che
	
	\begin{equation}
		\sum_{i=1}^{n} x^{i} \, \pdv{F}{x_{i}} = k F
	\end{equation}
	
	Dedurre che $ F^{-1}(c) $ con $ c \neq 0 $ è una sottovarietà di $ \R^{n} $ di dimensione $ n-1 $. Dimostrare inoltre che per $ c,d>0 $ si ha che $ F^{-1}(c) \simeq F^{-1}(d) $ diffeomorfe e lo stesso vale per $ c,d<0 $.\\
	\textit{Suggerimento per la prima parte: usare l'uguaglianza}
	
	\begin{equation}
		F(\lambda x_{1},\dots,\lambda x_{n}) = \lambda^{k} F(x_{1},\dots,x_{n}) \qcomma \forall \lambda \in \R
	\end{equation}
\end{tcolorbox}

\paragraph{Dimostrazione formula}

qui

\paragraph{Sottovarietà}

Perché $ F^{-1}(c) $ con $ c \neq 0 $ sia una sottovarietà, è necessario che l'applicazione $ F $ sia liscia e che $ c \in \VR_{F} \cap \Im(F) $: l'applicazione

\map{F}
	{\R^{n}}{\R[x_{1},\dots,x^{n}]}
	{(x_{1},\dots,x^{n})}{F(x_{1},\dots,x^{n})}

dove $ \R[x_{1},\dots,x^{n}] $ indica lo spazio dei polinomi in $ n $ variabili con coefficienti reali, è liscia in quanto l'immagine è un polinomio; per la seconda condizione, consideriamo le seguenti condizioni equivalenti

\begin{gather}
	c \in \VR_{F} \nonumber \\
	\Updownarrow \nonumber \\
	p \in \PR_{F} \qcomma \forall p \in \R^{n} \\
	\Updownarrow \nonumber \\
	\E \pdv{F}{x^{i}} \, (p) \neq 0 \qcomma i=1,\dots,n \nonumber
\end{gather}

Siccome vale la relazione

\begin{equation}
	\sum_{i=1}^{n} x^{i} \, \pdv{F}{x_{i}} = k F
\end{equation}

esisterà una derivata di $ F $ non nulla per qualsiasi punto che sia controimmagine di un polinomio non nullo: considerando le condizioni

\begin{equation}
	\begin{cases}
		c \notin \R[0,\dots,0]\\
		k \neq 0
	\end{cases} %
	\implies %
	k F(p) \neq 0 \qcomma \forall p \in F^{-1}(c)
\end{equation}

dunque

\begin{equation}
	\begin{cases}
		\displaystyle \sum_{i=1}^{n} x^{i} \, \pdv{F}{x_{i}} \, (p) \neq 0 \\\\
		p = (x^{1},\dots,x^{n}) \neq (0,\dots,0)
	\end{cases} %
	\implies %
	\E \pdv{F}{x^{i}} \, (p) \neq 0 \qcomma i=1,\dots,n
\end{equation}

e perciò $ F^{-1}(c) $ è una sottovarietà di $ \R^{n} $ con $ \dim(F^{-1}(c)) = ??? $

\paragraph{Diffeomorfismi}

qui

%

\newpage

%

\section{Gruppo lineare speciale complesso come sottovarietà}\label{es2-16}

\begin{tcolorbox}
	Dimostrare che
	
	\begin{equation}
		SL_{n}(\C) = \{ A \in M_{n}(\C) \mid \det(A) = 1 \} \subset M_{n}(\C)
	\end{equation}
	
	è una sottovarietà di $ M_{n}(\C) $ con $ \dim(SL_{n}(\C)) = 2n^{2}-2  $
\end{tcolorbox}

Il procedimento è analogo a quello presentato nell'Esempio \ref{es:sl-subman} con le differenze introdotte dal campo dei numeri complessi.\\
In particolare, la funzione usata per considerare la preimmagine del valore $ 1 = 1 + 0 i \in \C $ è la seguente

\map{f}
	{GL_{n}(\C)}{\C}
	{A}{\det(A) = \sum_{i=1}^{n} (-1)^{i+j} \, a_{ij} \, m_{ij}}

dove $ A = (a_{ij}) $ con $ i,j=1,\dots,n $, gli $ m_{ij} = \det(A_{ij}) $ sono i minori di $ A $ e le $ A_{ij} $ sono le sottomatrici ricavate da $ A $ rimuovendo la $ i $-esima riga e la $ j $-esima colonna.\\
Siccome la dimensione del gruppo lineare speciale in campo complesso è doppia rispetto al caso reale e dato che il valore $ 1 $ in $ f^{-1}(1) = SL_{n}(\C) $ è complesso e quindi di dimensione 2 in campo reale (i.e. è come se rappresentasse due condizioni in campo reale), otteniamo che

\begin{gather}
	\begin{cases}
		\dim(GL_{n}(\C)) = 2 \dim(GL_{n}(\R)) = 2 n^{2}\\
		\dim(1) = \dim(1 + 0 i) = 2
	\end{cases} \nonumber \\
	\Downarrow \\ %
	\dim(SL_{n}(\C)) = 2 \dim(SL_{n}(\R)) = 2(n^{2} - 1) = 2 n^{2} - 2 \nonumber
\end{gather}

%

\newpage

%

\section{Punti regolari come aperto del dominio}\label{es2-17}

\begin{tcolorbox}
	Sia $ F : N \to M $ un'applicazione liscia tra varietà differenziabili. Dimostrare che l'insieme $ \PR_{F} $ dei punti regolari di $ F $ è un aperto di $ N $.
\end{tcolorbox}

% soluzione presa da:
% https://math.stackexchange.com/questions/2324431/the-set-of-all-regular-points-of-a-smooth-map-is-open

Siano le dimensioni delle varietà

\begin{equation}
	\begin{cases}
		\dim(N) = n\\
		\dim(M) = m
	\end{cases}
\end{equation}

Se un punto $ p \in N $ è un punto regolare per la funzione $ F $, le seguenti affermazioni sono equivalenti

\begin{itemize}
	\item $ F $ è una sommersione in $ p $
	
	\item $ F_{*p} $ è suriettiva
	
	\item  $ \rank(F_{*p}) = k $ con $ k = \min \{n,m\} $
\end{itemize}

Consideriamo le seguenti carte e condizioni:

\begin{equation}
	\begin{cases}
		(U,\phi) \in N \qcomma (U,\phi) \ni p\\
		(V,\psi) \in M \qcomma (V,\psi) \ni F(p)\\
		F(U) \subseteq V
	\end{cases}
\end{equation}

Consideriamo inoltre l'applicazione continua\footnote{%
	Siano due spazi vettoriali topologici $ V $ e $ W $, una funzione lineare $ L : V \to W $ è continua se $ V $ è finito dimensionale e di Hausdorff; in questo caso, $ G $ è lineare perché un differenziale e dominio e codominio rispettano le condizioni in quanto varietà differenziabili.%
}

\map{G}
	{\phi(U)}{M_{n \times m}(\R)}
	{x}{(\psi \circ F \circ \phi^{-1})_{*x}}

Siccome

\begin{equation}
	G(x) = (\psi \circ F \circ \phi^{-1})_{*x} %
	= \psi_{F(\phi^{-1}(x))} \circ F_{*\phi^{-1}(x)} \circ (\phi^{-1})_{*x}
\end{equation}

se prendiamo $ x = \phi(p) $ con $ p \in \PR_{F} $, abbiamo che

\begin{equation}
	G(\phi(p)) = \psi_{F(p)} \circ F_{*p} \circ (\phi^{-1})_{*\phi(p)}
\end{equation}

dove i differenziali $ \psi_{F(p)} $ e $ (\phi^{-1})_{*\phi(p)} $ sono isomorfismi perché le applicazioni delle carte sono diffeomorfismi e $ F_{*p} $ è suriettiva perché $ p $ è un punto regolare: queste condizioni implicano che $ G $ sia suriettiva in $ \phi(p) $ e dunque che abbia rango massimo, i.e.

\begin{equation}
	\rank(G(\phi(p))) = k
\end{equation}

Questo implica anche che esiste una sottomatrice $ g(\phi(p)) $ di dimensione $ k \times k $ di $ G(\phi(p)) $ che abbia determinante diverso non nullo, i.e.

\begin{equation}
	\E g(\phi(p)) \in M_{k \times k}(\R) \mid \det(g(\phi(p))) \neq 0
\end{equation}

Consideriamo ora l'applicazione continua $ H $ e lo schema riassuntivo delle applicazioni utilizzate:

\sbs{0.4}{%
			\map{H}
				{M_{k \times k}(\R)}{\R}
				{A}{\det(A)}
			}
	{0.6}{%
			\diagr{%
					N \arrow[dd, "\phi"] \arrow[rr, "F"]                               \&  \& M \arrow[dd, "\psi"] \\
					\&  \&                      \\
					\R^{n} \arrow[dd, "G"] \arrow[rr, "\psi \circ F \circ \phi^{-1}"'] \&  \& \R^{m}               \\
					\&  \&                      \\
					M_{n \times m}(\R) \supset M_{k \times k}(\R) \arrow[rr, "H"]      \&  \& \R                  
					}
			}

A questo punto, considerando l'insieme

\begin{equation}
	(H \circ G \circ \phi^{-1}) (\R \setminus \{0\}) \subset N
\end{equation}

questo è un intorno aperto di $ p $ in cui $ G(\phi(p)) $ è suriettiva: prendendo l'unione degli intorni aperti di tutti i punti regolari, otteniamo un aperto che coincide con l'insieme dei punti regolari di $ F $, i.e.

\begin{equation}
	\sum_{p \in \PR_{F}} (H \circ G \circ \phi^{-1}) (\R \setminus \{0\}) = \PR_{F}
\end{equation}

%

\newpage

%

\section{}\label{es2-18}

\begin{tcolorbox}
	Sia $ F : N \to M $ un'applicazione liscia tra varietà differenziabili. Dimostrare che se $ F $ è chiusa allora  l'insieme $ \mathcal{VR}_{F} $ dei valori regolari di $ F $ è un aperto in $ M $.
\end{tcolorbox}

qui

%

\newpage

%

\section{}\label{es2-19}

\begin{tcolorbox}
	Dimostrare che l'applicazione
	
	\begin{align}
		\begin{split}
			F : \R &\to \R^{3}\\
			t &\mapsto (t,t^{2},t^{3})
		\end{split}
	\end{align}
	
	è un embedding liscio e scrivere $ F(\R) $ come zero di funzioni.
\end{tcolorbox}

qui

%

\newpage

%

\section{}\label{es2-20}

\begin{tcolorbox}
	Dimostrare che l'applicazione
	
	\begin{align}
		\begin{split}
			F : \R &\to \R^{3}\\
			t &\mapsto (\cosh(t),\sinh(t))
		\end{split}
	\end{align}
	
	è un embedding liscio e che
	
	\begin{equation}
		F(\R) = \{ (x,y) \in \R^{2} \mid x^{2}-y^{2} = 1 \}
	\end{equation}
\end{tcolorbox}

qui

%

\newpage

%

\section{}\label{es2-21}

\begin{tcolorbox}
	Dimostrare che la composizione di immersioni è un’immersione e che il prodotto cartesiano di due immersioni è un’immersione.
\end{tcolorbox}

qui

%

\newpage

%

\section{}\label{es2-22}

\begin{tcolorbox}
	Dimostrare che se $ F : N \to M $ è un'immersione e $ Z \subset N $ è una sottovarietà di $ N $ allora $ F_{|Z} : Z \to M $ è un'immersione.
\end{tcolorbox}

qui

%

\newpage

%

\section{}\label{es2-23}

\begin{tcolorbox}
	Dimostrare che l'applicazione
	
	\begin{align}
		\begin{split}
			F : \S^{2} &\to \R^{4}\\
			(x,y,z) &\mapsto (x^{2}-y^{2},xy,xz,yz)
		\end{split}
	\end{align}
	
	induce un embedding liscio da $ \rp{2} $ a $ \R^{4} $.
\end{tcolorbox}

qui

%

\newpage

%

\section{}\label{es2-24}

\begin{tcolorbox}
	Dimostrare che un'immersione iniettiva e propria è un embedding liscio. Mostrare che esistono embedding lisci che non sono applicazione proprie.\\\\
	Ricordare che un'applicazione continua $ f : X \to Y $ tra spazi topologici è propria se $ f^{-1}(K) $ è compatto in $ X $ per ogni compatto $ K $ di $ Y $.
\end{tcolorbox}

qui

% -----------------------

% fine pdf 2.4

% -----------------------

%

\newpage

%

\section{}\label{es2-25}

\begin{tcolorbox}
	Sia $ N $ una sottovarietà di una varietà differenziabile $ M $. Dimostrare che $ T(N) $ è una sottovarietà di $ T(M) $.
\end{tcolorbox}

qui

%

\newpage

%

\section{}\label{BONUS2-4}

\begin{tcolorbox}
	Verificare che la sfera $ \S^{1} $ sia parallelizzabile.
\end{tcolorbox}

qui

%

\newpage

%

\section{}\label{es2-26}

\begin{tcolorbox}
	Una varietà differenziabile $ N $ è detta \textit{orientabile} se esiste un atlante di $ N $ rispetto al quale il determinante jacobiano dei cambi di carte è positivo. Dimostrare che:
	
	\begin{enumerate}
		\item $ \rp{3} $ è una varietà orientabile;
		\item il fibrato tangente $ T(N) $ di una varietà differenziabile $ N $ è orientabile.
	\end{enumerate}
\end{tcolorbox}

qui

%

\newpage

%

\section{}\label{es2-27}

\begin{tcolorbox}
	Dimostrare che l'applicazione $ \mathcal{F}_{*} $ che associa a ogni varietà differenziabile il suo fibrato tangente e a ogni applicazione $ F : N \to M $ tra varietà differenziabili il suo differenziale $ F_{*} : T(N) \to T(M) $ definita come
	
	\begin{equation}
		\mathcal{F}_{*}((p,v)) = (F(p), F_{*p}(v))
	\end{equation}
	
	per $ \forall (p,v) \in T(N) $ definisce un funtore covariante dalla categoria delle varietà differenziabili in sé stessa.
\end{tcolorbox}

qui

%

\newpage

%

\section{}\label{es2-28}

\begin{tcolorbox}
	Una \textit{derivazione} di un'algebra di Lie $ (V,[\cdot,\cdot]) $ su un campo $ \K $ è un'applicazione lineare $ D : V \to V $ tale che
	
	\begin{equation}
		D([Y,Z]) = [DY,Z] + [Y,DZ]
	\end{equation}
	
	per $ \forall Y,Z \in V $. Dimostrare che, dato $ X \in V $, l'applicazione
	
	\begin{align}
		\begin{split}
			D_{X} : V &\to V\\
			Y &\mapsto [X,Y]
		\end{split}
	\end{align}
	
	è una derivazione.
\end{tcolorbox}

qui

%

\newpage

%

\section{}\label{es2-29}

\begin{tcolorbox}
	Siano $ M = \R \setminus 0 $ e $ X = \sfrac{\operatorname{d}}{\operatorname{dx}} \in \chi(M) $. Trovare la curva integrale di $ X $ massimale che inizia in un generico punto $ p \in \R $.
\end{tcolorbox}

qui

%

\newpage

%

\section{}\label{es2-30}

\begin{tcolorbox}
	Trovare il flusso (locale) dei seguenti campi di vettori in $ \chi(\R^{2}) $:
	
	\begin{equation}
		\begin{cases}
			X = x \, \dfrac{\partial}{\partial x} - y \, \dfrac{\partial}{\partial y}\\\\
			Y = x \, \dfrac{\partial}{\partial x} + y \, \dfrac{\partial}{\partial y}\\\\
			Z = \dfrac{\partial}{\partial x} + y \, \dfrac{\partial}{\partial y}
		\end{cases}
	\end{equation}
	
	Nel caso siano completi, calcolare il loro gruppo di diffeomorfismi a un parametro.
\end{tcolorbox}

qui

%

\newpage

%

\section{}\label{es2-31}

\begin{tcolorbox}
	Dimostrare che il campo di vettori $ X = \sfrac{\partial}{\partial x} \in \chi(\R^{2} \setminus (0,0)) $ non è completo.
\end{tcolorbox}

qui

%

\newpage

%

\section{}\label{es2-32}

\begin{tcolorbox}
	Sia $ M $ una varietà differenziabile e $ X \in \chi(M) $ tale che $ X(p) = 0 $ in un punto $ p \in M $. Dimostrare che la curva integrale di $ X $ che inizia in $ p $ è la curva costante $ c(t)=p $.
\end{tcolorbox}

\textbf{l 26 m 36}

%

\newpage

%

\section{}\label{es2-33}

\begin{tcolorbox}
	Sia $ M $ una varietà differenziabile e $ X \in \chi(M) $ il campo di vettori nullo, i.e. $ X = 0 $. Descrivere il gruppo dei diffeomorfismi a un parametro associato a $ X $.
\end{tcolorbox}

qui

%

\newpage

%

\section{}\label{es2-34}

\begin{tcolorbox}
	Siano $ F : N \to M $ un diffeomorfismo tra varietà differenziabili, $ X \in \chi(N) $ e $ f \in C^{\infty}(N) $. Dimostrare che
	
	\begin{equation}
		F_{*}(f X) = (f \circ F^{-1}) \, F_{*} X
	\end{equation}
\end{tcolorbox}

qui

