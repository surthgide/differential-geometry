\chapter{Esercizi Capitolo "Geometria differenziale negli spazi euclidei"}

\tocless\section{}\label{es1-1}

\begin{tcolorbox}
	Per ogni numero naturale $ k \in \mathbb{N} $ costruire una funzione $ C^{k}(\mathbb{R}) $ ma non $ C^{k+1}(\mathbb{R}) $.
\end{tcolorbox}

Per la funzione

\map{f_{k}}%
	{\R \times \N}{\R}%
	{(x,k)}{\alpha x^{k(k+2)/(k+1)} + \beta}
	
per $ \forall \alpha,\beta \in \R $, vale $ f_{k} \in C^{k}(\R) $ ma $ f_{k} \notin C^{k+1}(\R) $.

\tocless\section{}\label{es1-2}

\begin{tcolorbox}
	Dimostrare che la funzione
	
	\begin{align}
		\begin{split}
			f : \mathbb{R} & \to \mathbb{R}\\
			x &\mapsto%
				\begin{cases}
					e^{\sfrac{-1}{x^{2}}} & x \neq 0\\
					0 & x = 0
				\end{cases}
		\end{split}
	\end{align}
	
	risulta essere liscia ma non reale analitica.
\end{tcolorbox}

La funzione $ f $ è liscia in quanto, perché lo sia, è necessario che

\begin{equation}
	\pdv[k]{f}{x} = \pdv[k]{x} (0) = 0 \qcomma \forall k \in \N
\end{equation}

e questo è vero poiché

\begin{equation}
	\lim_{x \to 0} \left( \dfrac{e^{\sfrac{-1}{x^{2}}}}{x^{p}} \right) = 0 \qcomma \forall p \in \N %
	\implies%
	\lim_{x \to 0} \left( \pdv[k]{x} \left( e^{\sfrac{-1}{x^{2}}} \right) \right) = 0 \qcomma \forall k \in \N
\end{equation}

La funzione non è però reale analitica perché, in un intervallo aperto qualsiasi di 0 non coincide con il suo sviluppo di Taylor: lo sviluppo di Taylor per la parte dei reali positivi è diversa da 0 per qualsiasi valore di $ x $ non nullo mentre la parte per i reali negativi è identicamente nulla, i.e. preso $ U $ un qualsiasi intorno di 0

\begin{equation}
	\sum_{k=0}^{+\infty} \left( \pdv[k]{x} \left( e^{\sfrac{-1}{x^{2}}} \right) \right) \dfrac{x^{k}}{k!} \neq 0 \qcomma \forall x \in U \setminus \{0\}
\end{equation}

\tocless\section{}\label{es1-3}

\begin{tcolorbox}
	Siano $ a,b,c,d \in \mathbb{R} $ tale che $ a<b $. Dimostrare che i seguenti intervalli sono tutti diffeomorfi tra loro e diffeomorfi a $ \mathbb{R} $:
	
	\begin{equation}
		\begin{cases}
			(a,b)\\
			(c,+\infty)\\
			(-\infty,d)
		\end{cases}
	\end{equation}
\end{tcolorbox}

Consideriamo le applicazioni:

\map{f}%
	{(a,b)}{(0,1)}%
	{x}{\dfrac{x-a}{b-a}}

\map{g}%
	{(0,1)}{(c,+\infty)}%
	{x}{\dfrac{c}{x}}
	
\map{h}%
	{(0,1)}{(-\infty,d)}%
	{x}{\ln(x)-d}
	
\map{i}%
	{(c,+\infty)}{\R}%
	{x}{\ln(x-c)}

queste sono diffeomorfismi in quanto bigezioni lisce con inversa liscia, dunque le loro composizioni sono ancora diffeomorfismi. Le seguenti composizioni delle applicazioni sopraccitate inducono i seguenti diffeomorfismi:

\begin{equation}
	\begin{cases}
		g \circ f \implies (a,b) \simeq (c,+\infty)\\
		h \circ f \implies (a,b) \simeq (-\infty,d)\\
		i \circ g \circ f \implies (a,b) \simeq \R\\
		h \circ g^{-1} \implies (c,+\infty) \simeq (-\infty,d)\\
		i \implies (c,+\infty) \simeq \R\\
		i \circ g \circ h^{-1} \implies (-\infty,d) \simeq \R
	\end{cases}
\end{equation}

\tocless\section{}\label{es1-4}

\begin{tcolorbox}
	Dimostrare che l'applicazione
	
	\map{h}%
		{B_{1}(0)}{\R^{n}}%
		{x}{\left( \dfrac{x^{1}}{\sqrt{1 - \norm{x}^{2}}}, \cdots, \dfrac{x^{n}}{\sqrt{1 - \norm{x}^{2}}} \right)}
	
	definisce un diffeomorfismo tra la palla aperta unitaria centrata nell'origine di $ \R^{n} $ e $ \R^{n} $. Dedurre che la palla aperta di centro $ c \in \R^{n} $ e raggio $ r > 0 $ in $ \mathbb{R}^{n} $ è diffeomorfa a $ \R^{n} $.
\end{tcolorbox}

L'applicazione $ h $ è una bigezione liscia in quanto ogni sua componente è liscia poiché

\begin{equation}
	\pdv[k]{(x^{i})} \left( \dfrac{x^{i}}{\sqrt{ 1-\norm{x}^{2} }} \right) < \infty \qcomma \forall k \in \N, \, \forall x \in B_{1}(0), \, \forall i=1,\dots,n
\end{equation}

La sua inversa

\map{h^{-1}}%
	{\R^{n}}{B_{1}(0)}%
	{x}{\left( \dfrac{x^{1}}{\sqrt{1 + \norm{x}^{2}}}, \cdots, \dfrac{x^{n}}{\sqrt{1 + \norm{x}^{2}}} \right)}
	
è ancora liscia per lo stesso motivo, dunque $ h $ induce il diffeomorfismo $ B_{1}(0) \simeq \R^{n} $.\\
Se consideriamo l'applicazione lineare (dunque liscia con inversa liscia e perciò diffeomorfismo)

\map{g}%
	{B_{r}(c)}{B_{1}(0)}%
	{x}{\dfrac{x-c}{r}}
	
con $ c = (c^{1},\dots,c^{n}) $, e la componiamo con $ h $, otteniamo

\map{f = h \circ g}%
	{B_{r}(c)}{\R^{n}}%
	{x}{\left( \dfrac{\dfrac{x^{1} - c^{1}}{r}}{\sqrt{1 + \norm{\dfrac{x-c}{r}}^{2}}}, \cdots, \dfrac{\dfrac{x^{n} - c^{n}}{r}}{\sqrt{1 + \norm{\dfrac{x-c}{r}}^{2}}} \right)}

L'applicazione $ f $ è un diffeomorfismo in quanto composizione di diffeomorfismo, dunque $ f $ induce il diffeomorfismo $ B_{r}(c) \simeq \R^{n} $.

\tocless\section{}\label{es1-5}

\begin{tcolorbox}
	Sia $ f \in C^{\infty}(\mathbb{R}^{2}) $. Usando il teorema di Taylor con resto, dimostrare che esistono $ g_{11},g_{12},g_{22} \in C^{\infty}(\mathbb{R}^{2}) $ tali che
	
	\begin{equation}
		f(x,y) = f(0,0) + x \, \dfrac{\partial f}{\partial x} ((0,0)) + y \, \dfrac{\partial f}{\partial y} ((0,0)) + x^{2} \, g_{11}(x,y) + x y \, g_{12}(x,y) + y^{2} \, g_{22}(x,y)
	\end{equation}
\end{tcolorbox}

\begin{equation}
	\begin{cases}
		g_{11}(x,y) = \pdv[2]{f}{x} (0,0)\\\\
		g_{12}(x,y) = \pdv{f}{x}{y} (0,0)\\\\
		g_{22}(x,y) = \pdv[2]{f}{y} (0,0)
	\end{cases}
\end{equation}

???

\tocless\section{}\label{es1-6}

\begin{tcolorbox}
	Sia $ f \in C^{\infty}(\mathbb{R}^{2}) $ tale che
	
	\begin{equation}
		f(0,0) = \dfrac{\partial f}{\partial x} (0,0) = \dfrac{\partial f}{\partial y} (0,0) = 0
	\end{equation}

	Sia l'applicazione
	
	\begin{align}
		\begin{split}
			g : \mathbb{R}^{2} & \to \mathbb{R}\\
			(t,u) &\mapsto%
			\begin{cases}
				\dfrac{f(t,tu)}{t} & t \neq 0\\\\
				0 & t = 0
			\end{cases}
		\end{split}
	\end{align}
	
	Dimostrare che $ g \in C^{\infty}(\mathbb{R}^{2}) $.
\end{tcolorbox}

qui

\tocless\section{}\label{es1-7}

\begin{tcolorbox}
	Dimostrare che l'insieme $ C_{p}^{\infty}(\R^{n}) $ dei germi delle funzioni lisce intorno a $ p \in \mathbb{R}^{n} $ con le operazioni di somma e di prodotto definite negli appunti è un'algebra commutativa e unitaria.
\end{tcolorbox}

L'algebra $ A = (C_{p}^{\infty}(\R^{n}),+,\cdot) $ ha le operazioni definite come segue:

\map{+}%
	{A \times A}{A}%
	{([(f,U)],[(g,V)])}{[(f+g,U \cap V)]}
	
\map{\cdot}%
	{A \times A}{A}%
	{([(f,U)],[(g,V)])}{[(f g,U \cap V)]}
	
per $ \forall [(f,U)],[(g,V)] \in C_{p}^{\infty}(\R^{n}) $.\\
La commutatività può essere verificata tramite i seguenti passaggi:

\begin{align}
	\begin{split}
		[(f,U)] + [(g,V)] &= [(f+g,U \cap V)]\\
		&= [(g+f,V \cap U)]\\
		&= [(g,V)] + [(f,U)]\\\\
		%
		[(f,U)] \cdot [(g,V)] &= [(f \cdot g,U \cap V)]\\
		&= [(g \cdot f,V \cap U)]\\
		&= [(g,V)] \cdot [(f,U)]
	\end{split}
\end{align}

L'unitarietà, i.e.

\begin{equation}
	\exists \, e = [(1,U)] \in C_{p}^{\infty}(\R^{n}) \mid [(f,U)] \cdot e = e \cdot [(f,U)] = [(f,U)] \qcomma \forall [(f,U)] \in C_{p}^{\infty}(\R^{n})
\end{equation}

può essere verificata tramite i seguenti passaggi:

\begin{align}
	\begin{split}
		[(f,U)] \cdot [(1,U)] &= [(f \cdot 1,U \cap U)]\\
		&= [(1 \cdot f,U \cap U)]\\
		&= [(1,U)] \cdot [(f,U)]\\
		&= [(f,U)]
	\end{split}
\end{align}

\tocless\section{}\label{es1-8}

\begin{tcolorbox}
	Dimostrare che l'insieme $ Der_{p}(C_{p}^{\infty}(\mathbb{R}^{n})) $ delle derivazioni puntuali con le operazioni definite negli appunti è uno spazio vettoriale su $ \mathbb{R} $.
\end{tcolorbox}

qui

\tocless\section{}\label{es1-9}

\begin{tcolorbox}
	Dimostrare che i campi di vettori lisci $ \chi(U) $ su un aperto $ U \subset \mathbb{R}^{n} $ con le operazioni definite negli appunti è uno spazio vettoriale su $ \mathbb{R} $ e un $ C^{\infty} $-modulo.
\end{tcolorbox}

qui

\tocless\section{}\label{es1-10}

\begin{tcolorbox}
	Sia $ A $ un'algebra su un campo $ \mathbb{K} $. Dimostrare che le operazioni
	
	\begin{equation}
		\begin{cases}
			(D_{1}+D_{2})(a) = D_{1}(a) + D_{2}(a)\\
			(\lambda D)(a) = \lambda D(a)
		\end{cases}
	\end{equation}
	
	per $ \forall \lambda \in \mathbb{K} $ e $ \forall D_{1},D_{2},D \in Der(A) $ dotano $ Der(A) $ della struttura di spazio vettoriale su $ \mathbb{K} $.
\end{tcolorbox}

qui

\tocless\section{}\label{es1-11}

\begin{tcolorbox}
	Siano $ D_{1} $ e $ D_{2} $ due derivazioni di un'algebra $ A $ su un campo $ \mathbb{K} $, i.e. $ D_{1},D_{2} \in Der(A) $. Mostrare che $ D_{1} \circ D_{2} $ non è necessariamente una derivazione di $ A $ mentre
	
	\begin{equation}
		D_{1} \circ D_{2} - D_{2} \circ D_{1} \in Der(A)
	\end{equation}
\end{tcolorbox}

qui

\chapter{Esercizi Capitolo "Varietà differenziabili"}

\tocless\section{}\label{BONUS2-1}

\begin{tcolorbox}
	Verificare che l'unione disgiunta di spazi topologici
	
	\begin{align}
		\begin{split}
			A =& \, \coprod_{j \in J} A_{j}\\
			=& \, \bigcup_{j \in J} A_{j} \times \{j\}
		\end{split}
	\end{align}
	
	è uno spazio topologico, sapendo che $ U $ è aperto in $ A $ se e solo se $ U \cap A_{j} $ è aperto in $ A_{j} $ per $ \forall j \in J $.
\end{tcolorbox}

qui

\tocless\section{}\label{es2-1}

\begin{tcolorbox}
	Sia $ \mathbb{S}^{n} $ la sfera unitaria in $ \mathbb{R}^{n+1} $. Trovare un atlante differenziabile di $ \mathbb{S}^{n} $ con $ 2(n+1) $ carte.

\end{tcolorbox}

qui

\tocless\section{}\label{es2-2}

\begin{tcolorbox}
	Dimostrare che la strutture differenziabili su $ \mathbb{S}^{n} $ definite dall’esercizio precedente e dalle proiezioni
	stereografiche coincidono.
\end{tcolorbox}

qui

\tocless\section{}\label{es2-3}

\begin{tcolorbox}
	Sia $ S $ uno spazio topologico e $ \sim $ una relazione d'evalenza aperta su $ S $. Dimostrare che lo spazio quoziente $ \sfrac{S}{\sim} $ è $ T_{2} $ se e solo se
	
	\begin{equation}
		R = \{ (x,y) \in S \times S \mid x \sim y \}
	\end{equation}
	
	è un sottoinsieme chiuso di $ S \times S $.
\end{tcolorbox}

Vedi Dimostrazione \ref{solved-es2-3}.

\tocless\section{}\label{es2-4}

\begin{tcolorbox}
	Sia $ S $ uno spazio topologico $ N_{2} $ e $ \sim $ una relazione d'evalenza aperta su $ S $. Dimostrare che lo spazio quoziente $ \sfrac{S}{\sim} $ è $ N_{2} $.
\end{tcolorbox}

Vedi Dimostrazione \ref{solved-es2-4}.

\tocless\section{}\label{es2-5}

\begin{tcolorbox}
	Dimostrare che la Grassmanniana $ G(k,n) $ è uno spazio topologico connesso e compatto.
\end{tcolorbox}

qui

\tocless\section{}\label{es2-6}

\begin{tcolorbox}
	Siano $ M $ e $ N $ due varietà differenziabili e $ q_{0} \in N $. Dimostrare che
	
	\begin{align}
		\begin{split}
			i_{q_{0}} : N &\to M\\
			p &\mapsto (p,q_{0})
		\end{split}
	\end{align}
	
	è un'applicazione liscia.
\end{tcolorbox}

qui

\tocless\section{}\label{es2-7}

\begin{tcolorbox}
	Sia $ \mathbb{S}^{1} $ il cerchio unitario di $ \mathbb{R}^{2} $. Dimostrare che una funzione liscia $ f : \mathbb{R}^{2} \to \mathbb{R} $ si restringe ad una funzione liscia $ f_{|\mathbb{S}^{1}} : \mathbb{S}^{1} \to \mathbb{R} $.
\end{tcolorbox}

qui

\tocless\section{}\label{BONUS2-2}

\begin{tcolorbox}
	Verificare che $ Der_{p}(C_{p}^{\infty}(N)) $ sia uno spazio vettoriale su $ \mathbb{R} $.
\end{tcolorbox}

qui

\tocless\section{}\label{es2-8}

\begin{tcolorbox}
	Sia un'applicazione
	
	\begin{align}
		\begin{split}
			F : \mathbb{R}^{2} &\to \mathbb{R}^{3}\\
			(x,y) &\mapsto (x,y,xy)
		\end{split}
	\end{align}
	
	e $ p = (x,y) \in \mathbb{R}^{2} $. Trovare $ a,b,c \in \mathbb{R} $ tali che:
	
	\begin{equation}
		F_{*p} \left( \left. \dfrac{\partial}{\partial x} \right|_{p} \right) = a \left. \dfrac{\partial}{\partial u} \right|_{F(p)} + b \left. \dfrac{\partial}{\partial v} \right|_{F(p)} + c \left. \dfrac{\partial}{\partial w} \right|_{F(p)}
	\end{equation}
\end{tcolorbox}

qui

\tocless\section{}\label{es2-9}

\begin{tcolorbox}
	Siano $ x $ e $ y $ le coordinate standard su $ \mathbb{R}^{2} $ e $ U = \mathbb{R}^{2} \setminus \{(0,0)\} $. In $ U $ le coordinate polari $ (\rho, \theta) $ con $ \rho > 0 $ e $ \theta \in (0,2\pi) $ sono definite come
	
	\begin{equation}
		\begin{cases}
			x = \rho \cos(\theta)\\
			y = \rho \sin(\theta)
		\end{cases}
	\end{equation}
	
	Si scrivano $ \sfrac{\partial}{\partial \rho} $ e $ \sfrac{\partial}{\partial \theta} $ in funzione di $ \sfrac{\partial}{\partial x} $ e $ \sfrac{\partial}{\partial y} $.
\end{tcolorbox}

qui

\tocless\section{}\label{es2-10}

\begin{tcolorbox}
	Sia $ p = (x,y) $ un punto di $ \mathbb{R}^{2} $. Allora
	
	\begin{equation}
		c_{p}(t) = \begin{pmatrix} \cos(2t) & -\sin(2t) \\\\ \sin(2t) & \cos(2t) \end{pmatrix} \begin{pmatrix} x \\\\ y \end{pmatrix}
	\end{equation}
	
	è una curva liscia in $ \mathbb{R}^{2} $ che inizia in $ p $. Calcolare $ c'(0) $.
\end{tcolorbox}

qui

\tocless\section{}\label{es2-11}

\begin{tcolorbox}
	Siano $ N $ e $ M $ varietà differenziabili e $ \pi_{N} : N \times M \to N $ e $ \pi_{M} : N \times M \to M $ le proiezioni naturali. Dimostrare che per $ (p,q) \in N \times M $ l'applicazione
	
	\begin{equation}
		(\pi_{N_{*p}},\pi_{M_{*p}}) : T_{(p,q)}(N \times M) \to T_{p}(N) \times T_{q}(M)
	\end{equation}
	
	è un isomorfismo.
\end{tcolorbox}

qui

% -----------------------

% fine pdf 2.1-2.3

% -----------------------

\tocless\section{}\label{es2-12}

\begin{tcolorbox}
	Siano $ S_{1} $ e $ S_{2} $ due sottovarietà di due varietà differenziabili $ M_{1} $ e $ M_{2} $ rispettivamente. Dimostrare che $ S_{1} \times S_{2} $ è una sottovarietà di $ M_{1} \times M_{2} $.
\end{tcolorbox}

qui

\tocless\section{}\label{es2-13}

\begin{tcolorbox}
	Sia l'applicazione
	
	\begin{align}
		\begin{split}
			F : \mathbb{R}^{2} &\to \mathbb{R}\\
			(x,y) &\mapsto x^{2}-6xy+y^{2}
		\end{split}
	\end{align}

	Trovare i $ c \in \mathbb{R} $ tali che $ F^{-1}(c) $ sia una sottovarietà di $ \mathbb{R}^{2} $.
\end{tcolorbox}

qui

\tocless\section{}\label{es2-14}

\begin{tcolorbox}
	Dire se le soluzione del sistema
	
	\begin{equation}
		\begin{cases}
			x^{3} + y^{3} + z^{3} = 1\\
			z = xy
		\end{cases}
	\end{equation}

	costituiscono una sottovarietà di $ \mathbb{R}^{3} $.
\end{tcolorbox}

qui

\tocless\section{}\label{BONUS2-3}

\begin{tcolorbox}
	Sia la sottovarietà di $ \mathbb{R}^{3} $
	
	\begin{equation}
		S = \{ (x,y,z) \in \mathbb{R}^{3} \mid x^{3} + y^{3} + z^{3} = 1 \wedge x+y+z=0 \} \subset \mathbb{R}^{3}
	\end{equation}
	
	Calcolare lo spazio tangente $ T_{p}(S) $ con $ p \in S $.
\end{tcolorbox}

qui

\tocless\section{}\label{es2-15}

\begin{tcolorbox}
	Un polinomio $ F(x_{0},\dots,x_{n}) \in \mathbb{R}[x_{0},\dots,x_{n}] $ è omogeneo di grado $ k $ se è combinazione lineare di monomi $ x_{0}^{i_{1}} \cdots x_{n}^{i_{m}} $ di grado $ k $ tale che $ \sum_{j=1}^{m} i_{j} = k $. Dimostrare che
	
	\begin{equation}
		\sum_{i=0}^{n} x^{i} \, \dfrac{\partial F}{\partial x^{i}} = k F
	\end{equation}

	Dedurre che $ F^{-1}(c) $ con $ c \neq 0 $ è una sottovarietà di $ \mathbb{R}^{n} $ di dimensione $ n-1 $. Dimostrare inoltre che per $ c,d>0 $ si ha che $ F^{-1}(c) \simeq F^{-1}(d) $ diffeomorfe e lo stesso vale per $ c,d<0 $.\\\\
	Suggerimento per la prima parte: usare l'uguaglianza
	
	\begin{equation}
		F(\lambda x_{0},\dots,\lambda x_{n}) = \lambda^{k} F(x_{0},\dots,x_{n})
	\end{equation}
	
	valida per ogni $ \lambda \in \mathbb{R} $.
\end{tcolorbox}

qui

\tocless\section{}\label{es2-16}

\begin{tcolorbox}
	Dimostrare che
	
	\begin{equation}
		\mathbb{SL}_{n}(\mathbb{C}) = \{ A \in M_{n}(\mathbb{C}) \mid \det(A) \neq 0 \} \subset M_{n}(\mathbb{C})
	\end{equation}

	è una sottovarietà di $ M_{n}(\mathbb{C}) $ con $ \dim(\mathbb{SL}_{n}(\mathbb{C})) = 2n^{2}-2  $
\end{tcolorbox}

qui

\tocless\section{}\label{es2-17}

\begin{tcolorbox}
	Sia $ F : N \to M $ un'applicazione liscia tra varietà differenziabili. Dimostrare che l'insieme $ \mathcal{PR}_{F} $ dei punti regolari di $ F $ è un aperto di $ N $.
\end{tcolorbox}

qui

\tocless\section{}\label{es2-18}

\begin{tcolorbox}
		Sia $ F : N \to M $ un'applicazione liscia tra varietà differenziabili. Dimostrare che se $ F $ è chiusa allora  l'insieme $ \mathcal{VR}_{F} $ dei valori regolari di $ F $ è un aperto in $ M $.
\end{tcolorbox}

qui

\tocless\section{}\label{es2-19}

\begin{tcolorbox}
	Dimostrare che l'applicazione
	
	\begin{align}
		\begin{split}
			F : \mathbb{R} &\to \mathbb{R}^{3}\\
			t &\mapsto (t,t^{2},t^{3})
		\end{split}
	\end{align}

	è un embedding liscio e scrivere $ F(\mathbb{R}) $ come zero di funzioni.
\end{tcolorbox}

qui

\tocless\section{}\label{es2-20}

\begin{tcolorbox}
	Dimostrare che l'applicazione
	
	\begin{align}
		\begin{split}
			F : \mathbb{R} &\to \mathbb{R}^{3}\\
			t &\mapsto (\cosh(t),\sinh(t))
		\end{split}
	\end{align}
	
	è un embedding liscio e che
	
	\begin{equation}
		F(\mathbb{R}) = \{ (x,y) \in \mathbb{R}^{2} \mid x^{2}-y^{2} = 1 \}
	\end{equation}
\end{tcolorbox}

qui

\tocless\section{}\label{es2-21}

\begin{tcolorbox}
	Dimostrare che la composizione di immersioni è un’immersione e che il prodotto cartesiano di due immersioni è un’immersione.
\end{tcolorbox}



\tocless\section{}\label{es2-22}

\begin{tcolorbox}
	Dimostrare che se $ F : N \to M $ è un'immersione e $ Z \subset N $ è una sottovarietà di $ N $ allora $ F_{|Z} : Z \to M $ è un'immersione.
\end{tcolorbox}

qui

\tocless\section{}\label{es2-23}

\begin{tcolorbox}
	Dimostrare che l'applicazione
	
	\begin{align}
		\begin{split}
			F : \mathbb{S}^{2} &\to \mathbb{R}^{4}\\
			(x,y,z) &\mapsto (x^{2}-y^{2},xy,xz,yz)
		\end{split}
	\end{align}
	
	induce un embedding liscio da $ \mathcal{RP}^{2} $ a $ \mathbb{R}^{4} $.
\end{tcolorbox}

qui

\tocless\section{}\label{es2-24}

\begin{tcolorbox}
	Dimostrare che un'immersione iniettiva e propria è un embedding liscio. Mostrare che esistono embedding lisci che non sono applicazione proprie.\\\\
	Ricordare che un'applicazione continua $ f : X \to Y $ tra spazi topologici è propria se $ f^{-1}(K) $ è compatto in $ X $ per ogni compatto $ K $ di $ Y $.
\end{tcolorbox}

qui

% -----------------------

% fine pdf 2.4

% -----------------------

\tocless\section{}\label{es2-25}

\begin{tcolorbox}
	Sia $ N $ una sottovarietà di una varietà differenziabile $ M $. Dimostrare che $ T(N) $ è una sottovarietà di $ T(M) $.
\end{tcolorbox}

qui

\tocless\section{}\label{BONUS2-4}

\begin{tcolorbox}
	Verificare che la sfera $ \mathbb{S}^{1} $ sia parallelizzabile.
\end{tcolorbox}

qui

\tocless\section{}\label{es2-26}

\begin{tcolorbox}
	Una varietà differenziabile $ N $ è detta \textit{orientabile} se esiste un atlante di $ N $ rispetto al quale il determinante jacobiano dei cambi di carte è positivo. Dimostrare che:
	
	\begin{enumerate}
		\item $ \mathcal{RP}^{3} $ è una varietà orientabile;
		\item il fibrato tangente $ T(N) $ di una varietà differenziabile $ N $ è orientabile.
	\end{enumerate}
\end{tcolorbox}

qui

\tocless\section{}\label{es2-27}

\begin{tcolorbox}
	Dimostrare che l'applicazione $ \mathcal{F}_{*} $ che associa a ogni varietà differenziabile il suo fibrato tangente e a ogni applicazione $ F : N \to M $ tra varietà differenziabili il suo differenziale $ F_{*} : T(N) \to T(M) $ definita come
	
	\begin{equation}
		\mathcal{F}_{*}((p,v)) = (F(p), F_{*p}(v))
	\end{equation}

	per $ \forall (p,v) \in T(N) $ definisce un funtore covariante dalla categoria delle varietà differenziabili in sé stessa.
\end{tcolorbox}

qui

\tocless\section{}\label{es2-28}

\begin{tcolorbox}
	Una \textit{derivazione} di un'algebra di Lie $ (V,[\cdot,\cdot]) $ su un campo $ \mathbb{K} $ è un'applicazione lineare $ D : V \to V $ tale che
	
	\begin{equation}
		D([Y,Z]) = [DY,Z] + [Y,DZ]
	\end{equation}

	per $ \forall Y,Z \in V $. Dimostrare che, dato $ X \in V $, l'applicazione
	
	\begin{align}
		\begin{split}
			D_{X} : V &\to V\\
			Y &\mapsto [X,Y]
		\end{split}
	\end{align}

	è una derivazione.
\end{tcolorbox}

qui

\tocless\section{}\label{es2-29}

\begin{tcolorbox}
	Siano $ M = \mathbb{R} \setminus 0 $ e $ X = \sfrac{\operatorname{d}}{\operatorname{dx}} \in \chi(M) $. Trovare la curva integrale di $ X $ massimale che inizia in un generico punto $ p \in \mathbb{R} $.
\end{tcolorbox}

qui

\tocless\section{}\label{es2-30}

\begin{tcolorbox}
	Trovare il flusso (locale) dei seguenti campi di vettori in $ \chi(\mathbb{R}^{2}) $:
	
	\begin{equation}
		\begin{cases}
			X = x \, \dfrac{\partial}{\partial x} - y \, \dfrac{\partial}{\partial y}\\\\
			Y = x \, \dfrac{\partial}{\partial x} + y \, \dfrac{\partial}{\partial y}\\\\
			Z = \dfrac{\partial}{\partial x} + y \, \dfrac{\partial}{\partial y}
		\end{cases}
	\end{equation}

	Nel caso siano completi, calcolare il loro gruppo di diffeomorfismi ad un parametro.
\end{tcolorbox}

qui

\tocless\section{}\label{es2-31}

\begin{tcolorbox}
	Dimostrare che il campo di vettori $ X = \sfrac{\partial}{\partial x} \in \chi(\mathbb{R}^{2} \setminus (0,0)) $ non è completo.
\end{tcolorbox}

qui

\tocless\section{}\label{es2-32}

\begin{tcolorbox}
	Sia $ M $ una varietà differenziabile e $ X \in \chi(M) $ tale che $ X(p) = 0 $ in un punto $ p \in M $. Dimostrare che la curva integrale di $ X $ che inizia in $ p $ è la curva costante $ c(t)=p $.
\end{tcolorbox}

\textbf{l 26 m 36}

\tocless\section{}\label{es2-33}

\begin{tcolorbox}
	Sia $ M $ una varietà differenziabile e $ X \in \chi(M) $ il campo di vettori nullo, i.e. $ X = 0 $. Descrivere il gruppo dei diffeomorfismi ad un parametro associato a $ X $.
\end{tcolorbox}

qui

\tocless\section{}\label{es2-34}

\begin{tcolorbox}
	Siano $ F : N \to M $ un diffeomorfismo tra varietà differenziabili, $ X \in \chi(N) $ e $ f \in C^{\infty}(N) $. Dimostrare che
	
	\begin{equation}
		F_{*}(f X) = (f \circ F^{-1}) \, F_{*} X
	\end{equation}
\end{tcolorbox}

qui

\chapter{Esercizi Capitolo "Gruppi e algebre di Lie"}

\tocless\section{}\label{es3-1}

\begin{tcolorbox}
	Dimostrare che il prodotto diretto di due gruppi di Lie è un gruppo di Lie.
\end{tcolorbox}

qui

\tocless\section{}\label{es3-2}

\begin{tcolorbox}
	Siano la proiezione
	
	\begin{align}
		\begin{split}
			\pi : \mathbb{R}^{2} &\to \mathbb{S}^{1} \times \mathbb{S}^{1}\\
			(t,s) &\mapsto (e^{2 \pi i t},e^{2 \pi i s})
		\end{split}
	\end{align}

	l'insieme
	
	\begin{equation}
		L = \{ (t,\alpha t) \in \mathbb{R}^{2} \mid \alpha \in \mathbb{R} \setminus \mathbb{Q} \}
	\end{equation}

	e la restrizione di $ \pi $ ad $ L $, i.e. $ f = \pi_{|L} : L \to \mathbb{S}^{1} \times \mathbb{S}^{1} $. Siano
	
	\begin{itemize}
		\item $ \tau_{f} $ la topologia indotta da $ f $ su $ H = \pi(L) $;
		
		\item $ \tau_{s} $ la topologia indotta dall'inclusione $ H \subset \mathbb{S}^{1} \times \mathbb{S}^{1} $
	\end{itemize}

	Dimostrare che $ \tau_{s} \subset \tau_{f} $.
\end{tcolorbox}

qui

\tocless\section{}\label{es3-3}

\begin{tcolorbox}
	Sia la matrice
	
	\begin{equation}
		X = \begin{pmatrix} 0 & 1 \\\\ 1 & 0 \end{pmatrix}
	\end{equation}

	Dimostrare che
	
	\begin{equation}
		e^{X} = \begin{pmatrix} \cosh(1) & \sinh(1) \\\\ \sinh(1) & \cosh(1) \end{pmatrix}
	\end{equation}
\end{tcolorbox}

qui

\tocless\section{}\label{es3-4}

\begin{tcolorbox}
	Trovare due matrici $ A $ e $ B $ tali che
	
	\begin{equation}
		e^{A+B} \neq e^{A} e^{B}
	\end{equation}
\end{tcolorbox}

qui

\tocless\section{}\label{es3-5}

\begin{tcolorbox}
	Dimostrare che il gruppo unitario $ U(n) $ è compatto per ogni $ n \geqslant 1 $.
\end{tcolorbox}

qui

\tocless\section{}\label{es3-6}

\begin{tcolorbox}
	Siano $ G $ un gruppo di Lie e $ G_{0} $ la componente connessa di $ G $ che contiene $ e $ (elemento neutro di $ G $). Se $ \mu $ e $ i $ denotano rispettivamente la moltiplicazione e l'inversione in $ G $, provare che:
	
	\begin{enumerate}
		\item $ \mu(\{x\} \times G_{0}) \subset G_{0} $ per $ \forall x \in G $;
		
		\item $ i(G_{0}) \subset G_{0} $;
		
		\item $ G_{0} $ è un sottoinsieme aperto di $ G $;
		
		\item $ G_{0} $ è un sottogruppo di Lie di $ G $
	\end{enumerate}
\end{tcolorbox}

qui

\tocless\section{}\label{BONUS3-1}

\begin{tcolorbox}
	Verificare che $ SO(2) $ sia diffeomorfo a $ \S^{1} $.
\end{tcolorbox}

qui

\tocless\section{}\label{BONUS3-2}

\begin{tcolorbox}
	Verificare che $ SU(2) $ sia diffeomorfo a $ \S^{3} $.
\end{tcolorbox}

qui

\tocless\section{}\label{es3-7}

\begin{tcolorbox}
	Sia $ G $ un gruppo di Lie con moltiplicazione $ \mu : G \times G \to G $. Dimostrare che
	
	\begin{equation}
		\mu_{*(a,b)}(X_{a},Y_{b}) = (R_{b})_{*a}(X_{a}) + (L_{a})_{*b}(Y_{b}) \qcomma \forall (a,b) \in G \times G, \, \forall X_{a} \in T_{a}(G), \, \forall Y_{b} \in T_{b}(G)
	\end{equation}

	dove $ L_{a} $ (risp. $ R_{a} $) denota la traslazione a sinistra (risp. a destra) associata ad $ a $ (risp. $ b $).
\end{tcolorbox}

qui

\tocless\section{}\label{es3-8}

\begin{tcolorbox}
		Sia $ G $ un gruppo di Lie con inversione $ i : G \to G $. Dimostrare che
	
	\begin{equation}
		i_{*a}(Y_{a}) = -(R_{a^{-1}})_{*e}(L_{a^{-1}})_{*e} (Y_{a}) \qcomma \forall a \in G, \, \forall Y_{a} \in T_{a}(G)
	\end{equation}
\end{tcolorbox}

qui

\tocless\section{}\label{es3-9}

\begin{tcolorbox}
	Verificare che il commutatore tra matrici
	
	\begin{equation}
		[A,B] = AB - BA
	\end{equation}

	definisce un'algebra di Lie sullo spazio tangente all'identità dei gruppi:
	
	\begin{equation}
		\begin{cases}
			O(n)\\
			SO(n)\\
			U(n)\\
			SU(n)\\
			SL_{n}(\mathbb{R})\\
			SL_{n}(\mathbb{C})
		\end{cases}
	\end{equation}
\end{tcolorbox}

qui

\tocless\section{}\label{es3-10}

\begin{tcolorbox}
	Verificare che l'esponenziale di una matrice definisce un'applicazione
	
	\begin{align}
		\begin{split}
			e : T_{I_{n}}(G) &\to G\\
			A &\mapsto e^{A}
		\end{split}
	\end{align}

	per i gruppi di Lie
	
	\begin{equation}
		\begin{cases}
			GL_{n}(\mathbb{R})\\
			GL_{n}(\mathbb{C})\\
			O(n)\\
			SO(n)\\
			U(n)\\
			SU(n)\\
			SL_{n}(\mathbb{R})\\
			SL_{n}(\mathbb{C})
		\end{cases}
	\end{equation}
\end{tcolorbox}

qui

\tocless\section{}\label{es3-11}

\begin{tcolorbox}
	Sia $ G = G_{1} \times \cdots \times G_{s} $ il prodotto diretto di gruppi di Lie. Dimostrare che l'algebra di Lie di $ G $ è isomorfa alla somma diretta delle algebre di Lie dei $ G_{i} $ con $ i=1,\dots,n $.
\end{tcolorbox}

qui

\tocless\section{}\label{es3-12}

\begin{tcolorbox}
	Dimostrare che ogni gruppo di Lie è parallelizzabile.
\end{tcolorbox}

qui
