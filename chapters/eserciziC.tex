\section{}\label{es3-1}

\begin{tcolorbox}
	Dimostrare che il prodotto diretto di due gruppi di Lie è un gruppo di Lie.
\end{tcolorbox}

qui

%

\newpage

%

\section{}\label{es3-2}

\begin{tcolorbox}
	Siano la proiezione
	
	\begin{align}
		\begin{split}
			\pi : \R^{2} &\to \S^{1} \times \S^{1}\\
			(t,s) &\mapsto (e^{2 \pi i t},e^{2 \pi i s})
		\end{split}
	\end{align}
	
	l'insieme
	
	\begin{equation}
		L = \{ (t,\alpha t) \in \R^{2} \mid \alpha \in \R \setminus \Q \}
	\end{equation}
	
	e la restrizione di $ \pi $ a $ L $, i.e. $ f = \pi_{|L} : L \to \S^{1} \times \S^{1} $. Siano
	
	\begin{itemize}
		\item $ \tau_{f} $ la topologia indotta da $ f $ su $ H = \pi(L) $;
		
		\item $ \tau_{s} $ la topologia indotta dall'inclusione $ H \subset \S^{1} \times \S^{1} $
	\end{itemize}
	
	Dimostrare che $ \tau_{s} \subset \tau_{f} $.
\end{tcolorbox}

qui

%

\newpage

%

\section{}\label{es3-3}

\begin{tcolorbox}
	Sia la matrice
	
	\begin{equation}
		X = \begin{pmatrix} 0 & 1 \\\\ 1 & 0 \end{pmatrix}
	\end{equation}
	
	Dimostrare che
	
	\begin{equation}
		e^{X} = \begin{pmatrix} \cosh(1) & \sinh(1) \\\\ \sinh(1) & \cosh(1) \end{pmatrix}
	\end{equation}
\end{tcolorbox}

qui

%

\newpage

%

\section{}\label{es3-4}

\begin{tcolorbox}
	Trovare due matrici $ A $ e $ B $ tali che
	
	\begin{equation}
		e^{A+B} \neq e^{A} e^{B}
	\end{equation}
\end{tcolorbox}

qui

%

\newpage

%

\section{}\label{es3-5}

\begin{tcolorbox}
	Dimostrare che il gruppo unitario $ U(n) $ è compatto per ogni $ n \geqslant 1 $.
\end{tcolorbox}

qui

%

\newpage

%

\section{}\label{es3-6}

\begin{tcolorbox}
	Siano $ G $ un gruppo di Lie e $ G_{0} $ la componente connessa di $ G $ che contiene $ e $ (elemento neutro di $ G $). Se $ \mu $ e $ i $ denotano rispettivamente la moltiplicazione e l'inversione in $ G $, provare che:
	
	\begin{enumerate}
		\item $ \mu(\{x\} \times G_{0}) \subset G_{0} $ per $ \forall x \in G $;
		
		\item $ i(G_{0}) \subset G_{0} $;
		
		\item $ G_{0} $ è un sottoinsieme aperto di $ G $;
		
		\item $ G_{0} $ è un sottogruppo di Lie di $ G $
	\end{enumerate}
\end{tcolorbox}

qui

%

\newpage

%

\section{}\label{BONUS3-1}

\begin{tcolorbox}
	Verificare che $ SO(2) $ sia diffeomorfo a $ \S^{1} $.
\end{tcolorbox}

qui

%

\newpage

%

\section{}\label{BONUS3-2}

\begin{tcolorbox}
	Verificare che $ SU(2) $ sia diffeomorfo a $ \S^{3} $.
\end{tcolorbox}

qui

%

\newpage

%

\section{}\label{es3-7}

\begin{tcolorbox}
	Sia $ G $ un gruppo di Lie con moltiplicazione $ \mu : G \times G \to G $. Dimostrare che
	
	\begin{equation}
		\mu_{*(a,b)}(X_{a},Y_{b}) = (R_{b})_{*a}(X_{a}) + (L_{a})_{*b}(Y_{b}) \qcomma \forall (a,b) \in G \times G, \, \forall X_{a} \in T_{a}(G), \, \forall Y_{b} \in T_{b}(G)
	\end{equation}
	
	dove $ L_{a} $ (risp. $ R_{a} $) denota la traslazione a sinistra (risp. a destra) associata ad $ a $ (risp. $ b $).
\end{tcolorbox}

qui

%

\newpage

%

\section{}\label{es3-8}

\begin{tcolorbox}
	Sia $ G $ un gruppo di Lie con inversione $ i : G \to G $. Dimostrare che
	
	\begin{equation}
		i_{*a}(Y_{a}) = -(R_{a^{-1}})_{*e}(L_{a^{-1}})_{*e} (Y_{a}) \qcomma \forall a \in G, \, \forall Y_{a} \in T_{a}(G)
	\end{equation}
\end{tcolorbox}

qui

%

\newpage

%

\section{}\label{es3-9}

\begin{tcolorbox}
	Verificare che il commutatore tra matrici
	
	\begin{equation}
		[A,B] = AB - BA
	\end{equation}
	
	definisce un'algebra di Lie sullo spazio tangente all'identità dei gruppi:
	
	\begin{equation}
		\begin{cases}
			O(n)\\
			SO(n)\\
			U(n)\\
			SU(n)\\
			SL_{n}(\R)\\
			SL_{n}(\C)
		\end{cases}
	\end{equation}
\end{tcolorbox}

qui

%

\newpage

%

\section{}\label{es3-10}

\begin{tcolorbox}
	Verificare che l'esponenziale di una matrice definisce un'applicazione
	
	\begin{align}
		\begin{split}
			e : T_{I_{n}}(G) &\to G\\
			A &\mapsto e^{A}
		\end{split}
	\end{align}
	
	per i gruppi di Lie
	
	\begin{equation}
		\begin{cases}
			GL_{n}(\R)\\
			GL_{n}(\C)\\
			O(n)\\
			SO(n)\\
			U(n)\\
			SU(n)\\
			SL_{n}(\R)\\
			SL_{n}(\C)
		\end{cases}
	\end{equation}
\end{tcolorbox}

qui

%

\newpage

%

\section{}\label{es3-11}

\begin{tcolorbox}
	Sia $ G = G_{1} \times \cdots \times G_{s} $ il prodotto diretto di gruppi di Lie. Dimostrare che l'algebra di Lie di $ G $ è isomorfa alla somma diretta delle algebre di Lie dei $ G_{i} $ con $ i=1,\dots,n $.
\end{tcolorbox}

qui

%

\newpage

%

\section{}\label{es3-12}

\begin{tcolorbox}
	Dimostrare che ogni gruppo di Lie è parallelizzabile.
\end{tcolorbox}

qui
