\section{Prodotto diretto di gruppi di Lie}\label{es3-1}

\begin{tcolorbox}
	Dimostrare che il prodotto diretto di due gruppi di Lie è un gruppo di Lie.
\end{tcolorbox}

Consideriamo i due seguenti gruppi di Lie:

\begin{itemize}
	\item $ (G, \mu, i) $ con operazioni definite come\\
		\sbs{0.45}{%
					\map{\mu}
						{G \times G}{G}
						{(g_{1},g_{2})}{g_{1} \cdot g_{2}}
					}
			{0.45}{%
					\map{i}
						{G}{G}
						{g}{g^{-1}}
					}
		
	\item $ (H, \nu, j) $ con operazioni definite come\\
		\sbs{0.45}{%
					\map{\nu}
					{H \times H}{H}
					{(h_{1},h_{2})}{h_{1} * h_{2}}
					}
			{0.45}{%
					\map{j}
						{H}{H}
						{h}{h^{-1}}
					}	
\end{itemize}

A questo punto definiamo $ (\Omega, \sigma, k) $ il prodotto diretto dei due gruppi di Lie appena definiti, i.e.

\begin{equation}
	(\Omega, \sigma, k) \doteq (G, \mu, i) \times (H, \nu, j) = (G \times H, \mu \times \nu, i \times j)
\end{equation}

dove le operazioni di $ \Omega $ sono definite come:

\sbs{0.5}{%
			\map{\sigma}
				{\Omega \times \Omega}{\Omega}
				{(s_{1},s_{2})}{s_{1} \star s_{2}}
			}
	{0.5}{%
			\map{k}
				{\Omega}{\Omega}
				{s}{s^{-1}}
			}

Esplicitando i gruppi di Lie sottostanti, possiamo riscrivere queste operazioni come:

\map{\sigma = \mu \times \nu}
	{G \times G \times H \times H}{G \times H}
	{(g_{1},g_{2},h_{1},h_{2})}{(g_{1} \cdot g_{2},h_{1} * h_{2})}

\map{k = i \times j}
	{G \times H}{G \times H}
	{(g,h)}{(g^{-1},h^{-1})}

Il prodotto diretto conserva in $ \Omega $ le proprietà di gruppo algebrico di $ G $ e $ H $. Affinché $ \Omega $ sia un gruppo di Lie, è necessario che le operazioni $ \sigma $ e $ k $ siano lisce: per dimostrare questo, consideriamo le seguenti proiezioni tra varietà

\sbs{0.5}{%
			\map{\pi_{G}}
				{\Omega}{G}
				{s = (g,h)}{g}
			}
	{0.5}{%
			\map{\pi_{H}}
				{\Omega}{H}
				{s = (g,h)}{h}
			
			}
		
Da cui possiamo derivare i seguenti diagrammi:

\sbs{0.5}{%
			\diagr{%
					G \times G \arrow[rr, "\mu"]                                                                                         \&  \& G                                                   \\
					\&  \&                                                     \\
					\Omega \times \Omega \arrow[rr, "\sigma"] \arrow[dd, "\pi_{H} \times \pi_{H}"] \arrow[uu, "\pi_{G} \times \pi_{G}"'] \&  \& \Omega \arrow[dd, "\pi_{H}"] \arrow[uu, "\pi_{G}"'] \\
					\&  \&                                                     \\
					H \times H \arrow[rr, "\nu"]                                                                                         \&  \& H                                                  
					}
			}
	{0.5}{%
			\diagr{%
					G \arrow[rr, "i"]                                                   \&  \& G                                                   \\
					\&  \&                                                     \\
					\Omega \arrow[rr, "k"] \arrow[dd, "\pi_{H}"] \arrow[uu, "\pi_{G}"'] \&  \& \Omega \arrow[dd, "\pi_{H}"] \arrow[uu, "\pi_{G}"'] \\
					\&  \&                                                     \\
					G \arrow[rr, "j"]                                                   \&  \& H                                                  
					}
			}

Mediante le proiezioni è possibile scrivere le operazioni del gruppo $ \Omega $ come

\begin{gather}
		\sigma = (\pi_{G})^{-1} \circ \mu \circ (\pi_{G} \times \pi_{G}) %
		= (\pi_{H})^{-1} \circ \nu \circ (\pi_{H} \times \pi_{H})\\
		k = (\pi_{G})^{-1} \circ i \circ \pi_{G} %
		= (\pi_{H})^{-1} \circ j \circ \pi_{H}
\end{gather}

Siccome le proiezioni e le operazioni dei gruppi $ G $ e $ H $ sono liscie, anche le operazioni di $ \Omega $ sono lisce in quanto composizioni di applicazioni lisce; questo mostra che $ (\Omega, \sigma, k) $ è un gruppo di Lie.

%

\newpage

%

\section{Topologie sul toro}\label{es3-2}

\begin{tcolorbox}
	Siano la proiezione
	
	\map{\pi}
		{\R^{2}}{\T^{2} = \S^{1} \times \S^{1}}
		{(t,s)}{(e^{2 \pi i t},e^{2 \pi i s})}
	
	l'insieme
	
	\begin{equation}
		L = \{ (t,\alpha t) \in \R^{2} \mid \alpha \in \R \setminus \Q \}
	\end{equation}
	
	e la restrizione di $ \pi $ a $ L $, i.e.
	
	\begin{equation}
		f = \eval{\pi}_{L} : L \to \S^{1} \times \S^{1}
	\end{equation}

	Siano
	
	\begin{itemize}
		\item $ \tau_{f} $ la topologia indotta da $ f $ su $ H = \pi(L) $;
		
		\item $ \tau_{s} $ la topologia indotta dall'inclusione $ H \subset \S^{1} \times \S^{1} $
	\end{itemize}
	
	Dimostrare che $ \tau_{s} \subset \tau_{f} $.
\end{tcolorbox}

Definiamo le due topologie:

\begin{itemize}
	\item $ \tau_{s} $: $ W $ è aperto in $ H $ se $ W = U \cap H $ con $ U \subset \S^{1} \times \S^{1} $ aperto;
	
	\item $ \tau_{f} $: $ W $ è aperto in $ H $ se $ f^{-1}(W) \subset L $ è aperto in $ L $.
\end{itemize}

Sappiamo che $ f $ è un'immersione iniettiva\footnote{%
	Vedi Esempio \ref{ex:embed-torus}.%
} (non un embedding).

%

\newpage

%

\section{Esponenziale di matrice}\label{es3-3}

\begin{tcolorbox}
	Sia la matrice
	
	\begin{equation}
		X = \bmqty{ %
					0 & 1 \\
					1 & 0 %
					}
	\end{equation}
	
	Dimostrare che
	
	\begin{equation}
		e^{X} = \bmqty{ %
						\cosh(1) & \sinh(1) \\\\
						\sinh(1) & \cosh(1) %
						}
	\end{equation}
\end{tcolorbox}

Possiamo vedere che la matrice $ X $ è idempotente, i.e.

\begin{equation}
	X^{2} = \bmqty{ %
					0 & 1 \\
					1 & 0 %
					} %
			\bmqty{ %
					0 & 1 \\
					1 & 0 %
					} %
	= \bmqty{ %
				1 & 0 \\
				0 & 1 %
				} %
	= I
\end{equation}

da cui le relazioni

\begin{equation}
	\begin{cases}
		X^{2k} = I \\
		X^{2k+1} = X
	\end{cases}
	\quad k \in \N \cup \{0\}
\end{equation}

Consideriamo anche le espansioni in serie di Taylor delle seguenti due funzioni iperboliche:\\

\sbs{0.5}{%
			\begin{equation}
				\cosh(t) = \sum_{k=0}^{+\infty} \dfrac{t^{2k}}{(2k)!}
			\end{equation}
			}
	{0.5}{%
			\begin{equation}
				\sinh(t) = \sum_{k=0}^{+\infty} \dfrac{t^{2k+1}}{(2k+1)!}
			\end{equation}
			}

A questo punto, possiamo calcolare l'esponenziale di $ X $ mediante la definizione:

\begin{align}
	\begin{split}
		e^{X} &\doteq \sum_{k=0}^{+\infty} \dfrac{X^{k}}{k!} \\\\
		&= \sum_{k=0}^{+\infty} \left( \dfrac{X^{2k}}{(2k)!} + \dfrac{X^{2k+1}}{(2k+1)!} \right) \\\\
		&= \sum_{k=0}^{+\infty} \left( \dfrac{(1)^{2k}}{(2k)!} \, I \right) + \sum_{k=0}^{+\infty} \left( \dfrac{(1)^{2k+1}}{(2k+1)!} \, X \right) \\\\
		&= \sum_{k=0}^{+\infty} \left( \dfrac{(1)^{2k}}{(2k)!} \right) I + \sum_{k=0}^{+\infty} \left( \dfrac{(1)^{2k+1}}{(2k+1)!} \right) X \\\\
		&= \cosh(1) \, I + \sinh(1) \, X \\\\
		&= \bmqty{ %
					\cosh(1) & \sinh(1) \\\\
					\sinh(1) & \cosh(1) %
					}
	\end{split}
\end{align}

dove nel quarto passaggio abbiamo usato la proprietà delle serie convergenti

\begin{equation}
	\sum_{n=0}^{+\infty} a s_{n} = a \sum_{n=0}^{+\infty} s_{n}
\end{equation}

in quanto $ M_{n}(\R) $ è uno spazio di Banach, quindi completo.

%

\newpage

%

\section{}\label{es3-4}

\begin{tcolorbox}
	Trovare due matrici $ A $ e $ B $ tali che
	
	\begin{equation}
		e^{A+B} \neq e^{A} e^{B}
	\end{equation}
\end{tcolorbox}

qui

%

\newpage

%

\section{}\label{es3-5}

\begin{tcolorbox}
	Dimostrare che il gruppo unitario $ U(n) $ è compatto per ogni $ n \geqslant 1 $.
\end{tcolorbox}

qui

%

\newpage

%

\section{}\label{es3-6}

\begin{tcolorbox}
	Siano $ G $ un gruppo di Lie e $ G_{0} $ la componente connessa di $ G $ che contiene $ e $ (elemento neutro di $ G $). Se $ \mu $ e $ i $ denotano rispettivamente la moltiplicazione e l'inversione in $ G $, provare che:
	
	\begin{enumerate}
		\item $ \mu(\{x\} \times G_{0}) \subset G_{0} $ per $ \forall x \in G $;
		
		\item $ i(G_{0}) \subset G_{0} $;
		
		\item $ G_{0} $ è un sottoinsieme aperto di $ G $;
		
		\item $ G_{0} $ è un sottogruppo di Lie di $ G $
	\end{enumerate}
\end{tcolorbox}

qui

%

\newpage

%

\section{}\label{BONUS3-1}

\begin{tcolorbox}
	Verificare che $ SO(2) $ sia diffeomorfo a $ \S^{1} $.
\end{tcolorbox}

qui

%

\newpage

%

\section{}\label{BONUS3-2}

\begin{tcolorbox}
	Verificare che $ SU(2) $ sia diffeomorfo a $ \S^{3} $.
\end{tcolorbox}

qui

%

\newpage

%

\section{}\label{es3-7}

\begin{tcolorbox}
	Sia $ G $ un gruppo di Lie con moltiplicazione $ \mu : G \times G \to G $. Dimostrare che
	
	\begin{equation}
		\mu_{*(a,b)}(X_{a},Y_{b}) = (R_{b})_{*a}(X_{a}) + (L_{a})_{*b}(Y_{b}) \qcomma \forall (a,b) \in G \times G, \, \forall X_{a} \in T_{a}(G), \, \forall Y_{b} \in T_{b}(G)
	\end{equation}
	
	dove $ L_{a} $ (risp. $ R_{a} $) denota la traslazione a sinistra (risp. a destra) associata ad $ a $ (risp. $ b $).
\end{tcolorbox}

qui

%

\newpage

%

\section{}\label{es3-8}

\begin{tcolorbox}
	Sia $ G $ un gruppo di Lie con inversione $ i : G \to G $. Dimostrare che
	
	\begin{equation}
		i_{*a}(Y_{a}) = -(R_{a^{-1}})_{*e}(L_{a^{-1}})_{*e} (Y_{a}) \qcomma \forall a \in G, \, \forall Y_{a} \in T_{a}(G)
	\end{equation}
\end{tcolorbox}

qui

%

\newpage

%

\section{}\label{es3-9}

\begin{tcolorbox}
	Verificare che il commutatore tra matrici
	
	\begin{equation}
		[A,B] = AB - BA
	\end{equation}
	
	definisce un'algebra di Lie sullo spazio tangente all'identità dei gruppi:
	
	\begin{equation}
		\begin{cases}
			O(n)\\
			SO(n)\\
			U(n)\\
			SU(n)\\
			SL_{n}(\R)\\
			SL_{n}(\C)
		\end{cases}
	\end{equation}
\end{tcolorbox}

qui

%

\newpage

%

\section{}\label{es3-10}

\begin{tcolorbox}
	Verificare che l'esponenziale di una matrice definisce un'applicazione
	
	\begin{align}
		\begin{split}
			e : T_{I_{n}}(G) &\to G\\
			A &\mapsto e^{A}
		\end{split}
	\end{align}
	
	per i gruppi di Lie
	
	\begin{equation}
		\begin{cases}
			GL_{n}(\R)\\
			GL_{n}(\C)\\
			O(n)\\
			SO(n)\\
			U(n)\\
			SU(n)\\
			SL_{n}(\R)\\
			SL_{n}(\C)
		\end{cases}
	\end{equation}
\end{tcolorbox}

qui

%

\newpage

%

\section{}\label{es3-11}

\begin{tcolorbox}
	Sia $ G = G_{1} \times \cdots \times G_{s} $ il prodotto diretto di gruppi di Lie. Dimostrare che l'algebra di Lie di $ G $ è isomorfa alla somma diretta delle algebre di Lie dei $ G_{i} $ con $ i=1,\dots,n $.
\end{tcolorbox}

qui

%

\newpage

%

\section{}\label{es3-12}

\begin{tcolorbox}
	Dimostrare che ogni gruppo di Lie è parallelizzabile.
\end{tcolorbox}

qui
